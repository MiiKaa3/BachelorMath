\documentclass{report}

\input{../../latex_template/preamble}
%From M275 "Topology" at SJSU
\newcommand{\id}{\mathrm{id}}
\newcommand{\taking}[1]{\xrightarrow{#1}}
\newcommand{\inv}{^{-1}}

%From M170 "Introduction to Graph Theory" at SJSU
\DeclareMathOperator{\diam}{diam}
\DeclareMathOperator{\ord}{ord}
\newcommand{\defeq}{\overset{\mathrm{def}}{=}}

%From the USAMO .tex files
\newcommand{\ts}{\textsuperscript}
\newcommand{\dg}{^\circ}
\newcommand{\ii}{\item}

% % From Math 55 and Math 145 at Harvard
% \newenvironment{subproof}[1][Proof]{%
% \begin{proof}[#1] \renewcommand{\qedsymbol}{$\blacksquare$}}%
% {\end{proof}}

\newcommand{\liff}{\leftrightarrow}
\newcommand{\lthen}{\rightarrow}
\newcommand{\opname}{\operatorname}
\newcommand{\surjto}{\twoheadrightarrow}
\newcommand{\injto}{\hookrightarrow}
\newcommand{\On}{\mathrm{On}} % ordinals
\DeclareMathOperator{\img}{im} % Image
\DeclareMathOperator{\Img}{Im} % Image
\DeclareMathOperator{\coker}{coker} % Cokernel
\DeclareMathOperator{\Coker}{Coker} % Cokernel
\DeclareMathOperator{\Ker}{Ker} % Kernel
\DeclareMathOperator{\rank}{rank}
\DeclareMathOperator{\Spec}{Spec} % spectrum
\DeclareMathOperator{\Tr}{Tr} % trace
\DeclareMathOperator{\pr}{pr} % projection
\DeclareMathOperator{\ext}{ext} % extension
\DeclareMathOperator{\pred}{pred} % predecessor
\DeclareMathOperator{\dom}{dom} % domain
\DeclareMathOperator{\ran}{ran} % range
\DeclareMathOperator{\Hom}{Hom} % homomorphism
\DeclareMathOperator{\Mor}{Mor} % morphisms
\DeclareMathOperator{\End}{End} % endomorphism

\newcommand{\eps}{\epsilon}
\newcommand{\veps}{\varepsilon}
\newcommand{\ol}{\overline}
\newcommand{\ul}{\underline}
\newcommand{\wt}{\widetilde}
\newcommand{\wh}{\widehat}
\newcommand{\vocab}[1]{\textbf{\color{blue} #1}}
\providecommand{\half}{\frac{1}{2}}
\newcommand{\dang}{\measuredangle} %% Directed angle
\newcommand{\ray}[1]{\overrightarrow{#1}}
\newcommand{\seg}[1]{\overline{#1}}
\newcommand{\arc}[1]{\wideparen{#1}}
\DeclareMathOperator{\cis}{cis}
\DeclareMathOperator*{\lcm}{lcm}
\DeclareMathOperator*{\argmin}{arg min}
\DeclareMathOperator*{\argmax}{arg max}
\newcommand{\cycsum}{\sum_{\mathrm{cyc}}}
\newcommand{\symsum}{\sum_{\mathrm{sym}}}
\newcommand{\cycprod}{\prod_{\mathrm{cyc}}}
\newcommand{\symprod}{\prod_{\mathrm{sym}}}
\newcommand{\Qed}{\begin{flushright}\qed\end{flushright}}
\newcommand{\parinn}{\setlength{\parindent}{1cm}}
\newcommand{\parinf}{\setlength{\parindent}{0cm}}
% \newcommand{\norm}{\|\cdot\|}
\newcommand{\inorm}{\norm_{\infty}}
\newcommand{\opensets}{\{V_{\alpha}\}_{\alpha\in I}}
\newcommand{\oset}{V_{\alpha}}
\newcommand{\opset}[1]{V_{\alpha_{#1}}}
\newcommand{\lub}{\text{lub}}
\newcommand{\del}[2]{\frac{\partial #1}{\partial #2}}
\newcommand{\Del}[3]{\frac{\partial^{#1} #2}{\partial^{#1} #3}}
\newcommand{\deld}[2]{\dfrac{\partial #1}{\partial #2}}
\newcommand{\Deld}[3]{\dfrac{\partial^{#1} #2}{\partial^{#1} #3}}
\newcommand{\lm}{\lambda}
\newcommand{\uin}{\mathbin{\rotatebox[origin=c]{90}{$\in$}}}
\newcommand{\usubset}{\mathbin{\rotatebox[origin=c]{90}{$\subset$}}}
\newcommand{\lt}{\left}
\newcommand{\rt}{\right}
\newcommand{\bs}[1]{\boldsymbol{#1}}
\newcommand{\exs}{\exists}
\newcommand{\st}{\strut}
\newcommand{\dps}[1]{\displaystyle{#1}}

\newcommand{\sol}{\setlength{\parindent}{0cm}\textbf{\textit{Solution:}}\setlength{\parindent}{1cm} }
\newcommand{\solve}[1]{\setlength{\parindent}{0cm}\textbf{\textit{Solution: }}\setlength{\parindent}{1cm}#1 \Qed}

\DeclareMathOperator{\sech}{sech}
\DeclareMathOperator{\csch}{csch}
\DeclareMathOperator{\arcsec}{arcsec}
\DeclareMathOperator{\arccsc}{arccsc}
\DeclareMathOperator{\arccot}{arccot}
\DeclareMathOperator{\arsinh}{arsinh}
\DeclareMathOperator{\arcosh}{arcosh}
\DeclareMathOperator{\artanh}{artanh}
\DeclareMathOperator{\arcsch}{arcsch}
\DeclareMathOperator{\arsech}{arsech}
\DeclareMathOperator{\arcoth}{arcoth}

\newcommand{\sinx}{\sin x}          \newcommand{\arcsinx}{\arcsin x}    
\newcommand{\cosx}{\cos x}          \newcommand{\arccosx}{\arccosx}
\newcommand{\tanx}{\tan x}          \newcommand{\arctanx}{\arctan x}
\newcommand{\cscx}{\csc x}          \newcommand{\arccscx}{\arccsc x}
\newcommand{\secx}{\sec x}          \newcommand{\arcsecx}{\arcsec x}
\newcommand{\cotx}{\cot x}          \newcommand{\arccotx}{\arccot x}
\newcommand{\sinhx}{\sinh x}          \newcommand{\arsinhx}{\arsinh x}
\newcommand{\coshx}{\cosh x}          \newcommand{\arcoshx}{\arcosh x}
\newcommand{\tanhx}{\tanh x}          \newcommand{\artanhx}{\artanh x}
\newcommand{\cschx}{\csch x}          \newcommand{\arcschx}{\arcsch x}
\newcommand{\sechx}{\sech x}          \newcommand{\arsechx}{\arsech x}
\newcommand{\cothx}{\coth x}          \newcommand{\arcothx}{\arcoth x}
\newcommand{\lnx}{\ln x}
\newcommand{\expx}{\exp x}

\newcommand{\bba}{\mathbb{A}}   \newcommand{\bbn}{\mathbb{N}}
\newcommand{\bbb}{\mathbb{B}}   \newcommand{\bbo}{\mathbb{O}}
\newcommand{\bbc}{\mathbb{C}}   \newcommand{\bbp}{\mathbb{P}}
\newcommand{\bbd}{\mathbb{D}}   \newcommand{\bbq}{\mathbb{Q}}
\newcommand{\bbe}{\mathbb{E}}   \newcommand{\bbr}{\mathbb{R}}
\newcommand{\bbf}{\mathbb{F}}   \newcommand{\bbs}{\mathbb{S}}
\newcommand{\bbg}{\mathbb{G}}   \newcommand{\bbt}{\mathbb{T}}
\newcommand{\bbh}{\mathbb{H}}   \newcommand{\bbu}{\mathbb{U}}
\newcommand{\bbi}{\mathbb{I}}    \newcommand{\bbv}{\mathbb{V}}
\newcommand{\bbj}{\mathbb{J}}   \newcommand{\bbw}{\mathbb{W}}
\newcommand{\bbk}{\mathbb{K}}   \newcommand{\bbx}{\mathbb{X}}
\newcommand{\bbl}{\mathbb{L}}    \newcommand{\bby}{\mathbb{Y}}
\newcommand{\bbm}{\mathbb{M}}   \newcommand{\bbz}{\mathbb{Z}}

\newcommand{\lb}{\left(}
\newcommand{\rb}{\right)}
\newcommand{\lbr}{\left\lbrace}
\newcommand{\rbr}{\right\rbrace}
\newcommand{\lsb}{\left[}
\newcommand{\rsb}{\right]}
\newcommand{\suchthat}{\medspace\middle|\medspace}
\newcommand{\bracks}[1]{\lb #1 \rb}
\newcommand{\braces}[1]{\lbr #1 \rbr}
\newcommand{\sqbracks}[1]{\lsb #1 \rsb}

\renewcommand{\floor}[1]{\lfloor #1 \rfloor}
\renewcommand{\ceil}[1]{\lceil #1 \rceil}

\newcommand{\cd}{\cdot}
\newcommand{\tf}{\therefore}
\newcommand{\Let}{\text{Let }}
\newcommand{\Given}{\text{Given }}
\newcommand{\Suppose}{\text{Suppose }}
\newcommand{\WeSee}{\text{We see }}
\newcommand{\So}{\text{So }}

\newcommand{\QED}{\hfill \qed}

\renewcommand{\dd}[1]{\frac{d}{d#1}}
\newcommand{\dyd}[2][y]{\frac{d#1}{d#2}}

\newcommand{\ddx}{\dd{x}}       \newcommand{\ddxsq}{\dyd[^2]{x^2}}
\newcommand{\ddy}{\dd{y}}       \newcommand{\ddysq}{\dyd[^2]{y^2}}
\newcommand{\ddu}{\dd{u}}       \newcommand{\ddusq}{\dyd[^2]{u^2}}
\newcommand{\ddv}{\dd{v}}       \newcommand{\ddvsq}{\dyd[^2]{v^2}}

\newcommand{\dydx}{\dyd{x}}     \newcommand{\dydxsq}{\dyd[^2y]{x^2}}
\newcommand{\dfdx}{\dyd[f]{x}}  \newcommand{\dfdxsq}{\dyd[^2f]{x^2}}
\newcommand{\dudx}{\dyd[u]{x}}  \newcommand{\dudxsq}{\dyd[^2u]{x^2}}
\newcommand{\dvdx}{\dyd[v]{x}}  \newcommand{\dvdxsq}{\dyd[^2v]{x^2}}

% Mathfrak primes
\newcommand{\km}{\mathfrak{m}}
\newcommand{\kp}{\mathfrak{p}}
\newcommand{\kq}{\mathfrak{q}}

%---------------------------------------
% Blackboard Math Fonts :-
%---------------------------------------
\newcommand{\bba}{\mathbb{A}}   \newcommand{\bbn}{\mathbb{N}}
\newcommand{\bbb}{\mathbb{B}}   \newcommand{\bbo}{\mathbb{O}}
\newcommand{\bbc}{\mathbb{C}}   \newcommand{\bbp}{\mathbb{P}}
\newcommand{\bbd}{\mathbb{D}}   \newcommand{\bbq}{\mathbb{Q}}
\newcommand{\bbe}{\mathbb{E}}   \newcommand{\bbr}{\mathbb{R}}
\newcommand{\bbf}{\mathbb{F}}   \newcommand{\bbs}{\mathbb{S}}
\newcommand{\bbg}{\mathbb{G}}   \newcommand{\bbt}{\mathbb{T}}
\newcommand{\bbh}{\mathbb{H}}   \newcommand{\bbu}{\mathbb{U}}
\newcommand{\bbi}{\mathbb{I}}   \newcommand{\bbv}{\mathbb{V}}
\newcommand{\bbj}{\mathbb{J}}   \newcommand{\bbw}{\mathbb{W}}
\newcommand{\bbk}{\mathbb{K}}   \newcommand{\bbx}{\mathbb{X}}
\newcommand{\bbl}{\mathbb{L}}   \newcommand{\bby}{\mathbb{Y}}
\newcommand{\bbm}{\mathbb{M}}   \newcommand{\bbz}{\mathbb{Z}}

%---------------------------------------
% Roman Math Fonts :-
%---------------------------------------
\newcommand{\rma}{\mathrm{A}}   \newcommand{\rmn}{\mathrm{N}}
\newcommand{\rmb}{\mathrm{B}}   \newcommand{\rmo}{\mathrm{O}}
\newcommand{\rmc}{\mathrm{C}}   \newcommand{\rmp}{\mathrm{P}}
\newcommand{\rmd}{\mathrm{D}}   \newcommand{\rmq}{\mathrm{Q}}
\newcommand{\rme}{\mathrm{E}}   \newcommand{\rmr}{\mathrm{R}}
\newcommand{\rmf}{\mathrm{F}}   \newcommand{\rms}{\mathrm{S}}
\newcommand{\rmg}{\mathrm{G}}   \newcommand{\rmt}{\mathrm{T}}
\newcommand{\rmh}{\mathrm{H}}   \newcommand{\rmu}{\mathrm{U}}
\newcommand{\rmi}{\mathrm{I}}   \newcommand{\rmv}{\mathrm{V}}
\newcommand{\rmj}{\mathrm{J}}   \newcommand{\rmw}{\mathrm{W}}
\newcommand{\rmk}{\mathrm{K}}   \newcommand{\rmx}{\mathrm{X}}
\newcommand{\rml}{\mathrm{L}}   \newcommand{\rmy}{\mathrm{Y}}
\newcommand{\rmm}{\mathrm{M}}   \newcommand{\rmz}{\mathrm{Z}}

%---------------------------------------
% Calligraphic Math Fonts :-
%---------------------------------------
\newcommand{\cla}{\mathcal{A}}   \newcommand{\cln}{\mathcal{N}}
\newcommand{\clb}{\mathcal{B}}   \newcommand{\clo}{\mathcal{O}}
\newcommand{\clc}{\mathcal{C}}   \newcommand{\clp}{\mathcal{P}}
\newcommand{\cld}{\mathcal{D}}   \newcommand{\clq}{\mathcal{Q}}
\newcommand{\cle}{\mathcal{E}}   \newcommand{\clr}{\mathcal{R}}
\newcommand{\clf}{\mathcal{F}}   \newcommand{\cls}{\mathcal{S}}
\newcommand{\clg}{\mathcal{G}}   \newcommand{\clt}{\mathcal{T}}
\newcommand{\clh}{\mathcal{H}}   \newcommand{\clu}{\mathcal{U}}
\newcommand{\cli}{\mathcal{I}}   \newcommand{\clv}{\mathcal{V}}
\newcommand{\clj}{\mathcal{J}}   \newcommand{\clw}{\mathcal{W}}
\newcommand{\clk}{\mathcal{K}}   \newcommand{\clx}{\mathcal{X}}
\newcommand{\cll}{\mathcal{L}}   \newcommand{\cly}{\mathcal{Y}}
\newcommand{\calm}{\mathcal{M}}  \newcommand{\clz}{\mathcal{Z}}

%---------------------------------------
% Fraktur  Math Fonts :-
%---------------------------------------
\newcommand{\fka}{\mathfrak{A}}   \newcommand{\fkn}{\mathfrak{N}}
\newcommand{\fkb}{\mathfrak{B}}   \newcommand{\fko}{\mathfrak{O}}
\newcommand{\fkc}{\mathfrak{C}}   \newcommand{\fkp}{\mathfrak{P}}
\newcommand{\fkd}{\mathfrak{D}}   \newcommand{\fkq}{\mathfrak{Q}}
\newcommand{\fke}{\mathfrak{E}}   \newcommand{\fkr}{\mathfrak{R}}
\newcommand{\fkf}{\mathfrak{F}}   \newcommand{\fks}{\mathfrak{S}}
\newcommand{\fkg}{\mathfrak{G}}   \newcommand{\fkt}{\mathfrak{T}}
\newcommand{\fkh}{\mathfrak{H}}   \newcommand{\fku}{\mathfrak{U}}
\newcommand{\fki}{\mathfrak{I}}   \newcommand{\fkv}{\mathfrak{V}}
\newcommand{\fkj}{\mathfrak{J}}   \newcommand{\fkw}{\mathfrak{W}}
\newcommand{\fkk}{\mathfrak{K}}   \newcommand{\fkx}{\mathfrak{X}}
\newcommand{\fkl}{\mathfrak{L}}   \newcommand{\fky}{\mathfrak{Y}}
\newcommand{\fkm}{\mathfrak{M}}   \newcommand{\fkz}{\mathfrak{Z}}


\title{\Huge{MATH1061}\\Discrete Mathematics I}
\author{\huge{Michael Kasumagic, s4430266}}
\date{\huge{Semester 2, 2024}}

\begin{document}

\maketitle
\newpage% or \cleardoublepage
% \pdfbookmark[<level>]{<title>}{<dest>}
\pdfbookmark[section]{\contentsname}{toc}
\tableofcontents
\pagebreak

\chapter{Week 1}
\section{Lecture 1}
This course will run a little differently. Prior to every lecture, we must work through a set of pre-lecture problems. The goal of timetabled lectures is to discuss and learn from solving problems.

\subsection*{What is in this course?}
\subsubsection*{Logic and set theory, methods of proof}
Modern mathematics uses the language of set theory and the notation of logic.
$$
((P \land \lnot Q) \lor (P\land Q)) \land Q \equiv P \land Q
$$
We will learn to read and analyse this. Historical, there was a big shift in recent history, there was a big effort to define and axiom-itise everything, such that math itself is defined rigorously. Symbolic logic is the basis for many areas of computer science. It helps us formulate mathematical ideas and proofs effectively and cobbrectly! \\

\dfn{G\"odel's Incompleteness Theorem (1931)}{There exists true statements which we can not prove!}

\subsubsection*{Number Theory}
\ex{$1 + \cdots + 100$}{
	A young Gauss had to add up all the numbers from 1 to 100 in primary school. What did he do?
	\begin{gather*}
		\begin{array}{ccccccccccccc}
			& & 1 & + & 2 & + & \cdots & + & 98 & + & 99 & + & 100 \\	
			100 & + & 99 & + & 98 & + & \cdots & + & 2 & + & 1 & & \\	\hline
			100 & + & 100 & + & 100 & + & \cdots & + & 100 & + & 100 & + & 100   
		\end{array} \\
			\intertext{So...} \\
		1 + \cdots + 100 = \frac{101\cd100}{2} = 5050
	\end{gather*}
}
This generalises to $\forall n \in \bbn$. Two leaps of faith are needed though!
\begin{itemize}
	\item The dots: We introduce the notation to deal with them.
	\item The equality of two equations invoving dots. We will use induction to deal with this!
\end{itemize}

\subsubsection*{Graph Theory}
\ex{The K\"onigsberg Bridge Problem}{Find a route through the city which crosses each of seven bridges exactly once, and returns you to your start location.}
This is provably impossible! But how can we rigorously prove this? Euler solve this problem in 1935 and in doing so invented graph theory. We'll learn how eventually... :p

\subsubsection*{Counting and Probability}
Both fundamental and beautifully applicable. We introduce the pigeonhole principle as a introduction to ``counting.''
\ex{The pigeonhole principle}{If you have $n$ pigeons sitting in $k$ pigeonholes, if $n>k$, then at least of the pigeonholes contains at least 2 pigeons.}

\qs{}{If you have socks of three different colours in your drawer, what is the minimum number of socks you need to pull out to guarantee a matching pair?}
\sol \#socks $\equiv$ \#pigeons and \#colours $\equiv$ \#holes. If \#socks $>$ \#colors, a double must occur. Therefore, we need a minimum of 4 socks to guarantee a match.

\qs{True or False?}{In every group of five people, there are two people who have the same number of friends within the group.}
\sol True! \#people $\equiv$ \#pigeons and \#friends $\equiv$ \#holes. There are 5 possible values for the amount of friends one could have, $\braces{0, 1, 2, 3, 4}$, but you can never have an individual with 0 friends, and 4 friends in the same group. So there are 5 people, and 4 possible \#friend values (think ``holes.'') Therefore, by pigeonhole principle, the statement is true!

\qs{True or False?}{A plane is coloured blue and red. Is it possible to find exactly two points the same colour exactly one unit apart?}
\nt{We will answer this on Wednesday!}

\subsubsection*{Recursion}
\ex{The Tower of Hanoi}{Given: a tower of 8 discs in decreasing size on one of three pegs. \\Problem: transfer the entire towert to one of the other pegs. \\Rule 1: Move only one disc at a time. \\Rule 2: Never move a larger disc onto a smaller disk.}
\begin{enumerate}
	\item Is there a solution?
	\item What's the minimal number of moves necessary and sufficient for the task?
\end{enumerate}
A key idea is to generalise! What if there are $n$ discs? Let $T_n$ be the minimal number of moves, then trivially $T_0 = 0$, $T_1 = 1$, $T_2 = 3$, so what is $T_3 = ?$. Is there a pattern? The winning strategy is
\begin{enumerate}
	\item Move the $n-1$ smallest discs from peg $A$ to $B$.
	\item Move the big disc from $A$ to $C$.
	\item Move $n-!$ smallest discs from $B$ to $C$
\end{enumerate}
By induction we show that
$$
	T_n = 2T_{n-1} + 1.
$$
So $T_3 = 7$, $T_4 = 15$, $T_5 = 31$, $T_6 = 63$. Remarkably, this is one less then the sqaure numbers! We will prove this fact by induction later in the course.
\nt{On Wednesday we start proberly.\\ Read: \vspace{-10pt}\begin{center}Pages 23-36 (Epp, 4th) or \\ Pages 37-50 (Epp, 5th).\end{center}
Watch the first video on UQ Extend, try the first quiz before Wednesday's lecture!}

\section{Lecture 2}
\dfn{Statement or Proposition}{A sentence that is either true or false but not both.}
\ex{}{
	Statements:
	\begin{itemize}
		\item The number 6 is a number.
		\item $\pi>3$
		\item Euler was born in 1707.
	\end{itemize}
	Not statements:
	\begin{itemize}
		\item How are you? (This is a question.)
		\item Stop! (This is a command.)
		\item She likes math. (``She'' is not well defined.)
		\item $x^2 = 2x - 1$ ($x$ is not well defined.)
	\end{itemize}
}

\dfn{Negation}{
	Let $p$ be a statement. The negation is of $p$ is denoted $\lnot p$ or $\varlnot p$ and is read ``not $p$.'' It is defined as in the following truth table: 
	\begin{center}\begin{tabular}{|c||c|}
		\hline
		$p$ & $\lnot p$ \\ \hline
		T & F \\
		F & T \\ \hline
	\end{tabular}\end{center}
}

\dfn{Conjuction}{
	Let $p$ and $q$ be statements. The conjuction of $p$ and $q$ is denoted $p \land q$ and is read ``$p$ and $q$.'' It is defined as in the following truth table: 
	\begin{center}\begin{tabular}{|cc||c|}
		\hline
		$p$ & $q$ & $p\land q$ \\ \hline
		T & T	& T	\\
		T & F	& F	\\ 
		F	& T	& F	\\
		F & F	& F	\\\hline
	\end{tabular}\end{center}
}

\dfn{Disjunction}{
	Let $p$ and $q$ be statements. The disjunction of $p$ and $q$ is denoted $p \lor q$ and is read ``$p$ or $q$.'' It is defined as in the following truth table: 
	\begin{center}\begin{tabular}{|cc||c|}
		\hline
		$p$ & $q$ & $p\lor q$ \\ \hline
		T & T	& T	\\
		T & F	& T	\\ 
		F	& T	& T	\\
		F & F	& F	\\\hline
	\end{tabular}\end{center}
}

\dfn{Logical Equivalence}{
	Two statements, $p$ and $q$ are said to be logically equivalent if have identical truth values for every possible combination of truth values for their statement variables. This is denoted $p\equiv q$.
}
\ex{}{
	$$\lnot(\lnot p)\equiv p.$$
	\begin{center}
		\begin{tabular}{|cc||c|}
			\hline
			$p$ & $\lnot p$ & $\lnot(\lnot p)$ \\ \hline
			T & F & T \\
			F & T & F \\ \hline
		\end{tabular}
	\end{center}
	Consider $P = \lnot(p\land q)$, $Q = \lnot p \land \lnot q$ and $R = \lnot p\lor \lnot q$.
	\begin{center}
		\begin{tabular}{|cc||cc||ccc|}
			\hline
			$p$	& $q$	& $\lnot p$	& $\lnot q$ & $P$ & $Q$ & $R$ \\ \hline
			T	& T	& F	& F	& F	& F	& F	\\
			T	& F	& F	& T	& T	& F	& T	\\
			F	& T	& T	& F	& T	& F	& T	\\
			F	& F	& T	& T	& T	& T	& T	\\ \hline
		\end{tabular}
	\end{center}
	$\tf P\equiv R\not\equiv Q.$
}

\dfn{Contradictions and Tautologies}{
	A contradiction has truth values of false for every possible combination of its statement's truth values, and is denoted $c$ or $\bot$. A tautology has truth values of true for every possible combination of its statement's truth values, and is denoted $t$ or $\top$.
}
\ex{}{
	\begin{multicols}{2}
		\begin{center}
			\begin{tabular}{|cc||c|}
				\hline
				$p$ & $\lnot p$ & $p \land \lnot p$ \\ \hline
				T & F & F \\
				F & T & F \\ \hline
			\end{tabular}
		\end{center}
		$$\tf p \land \lnot p \equiv \top $$
		\begin{center}
			\begin{tabular}{|cc||c|}
				\hline
				$p$ & $\lnot p$ & $p\lor \lnot p$ \\ \hline
				T & F & T \\
				F & T & T \\ \hline
			\end{tabular}
		\end{center}
		$$\tf p \lor \lnot p \equiv \bot $$
	\end{multicols}
}

\subsection*{Important Laws of Logical Equivalence!}
\begin{multicols}{2}
	\subsubsection*{De Morgan's Law}
	\begin{align*}
		\lnot(p\land q) \equiv \lnot p \lor \lnot q \\
		\lnot(p\lor q) \equiv \lnot p \land \lnot q 
	\end{align*}
	
	\subsubsection*{Commutativity}
	\begin{align*}
		p \land q \equiv q\land p \\
		p \lor q \equiv q \lor p
	\end{align*}
	
	\subsubsection*{Associativity}
	\begin{align*}
		p \land (q\land r) \equiv (p\land q) \land r \\
		p \lor (q\lor r) \equiv (p\lor q) \lor r	
	\end{align*}
	
	\subsubsection*{Distributivity}
	\begin{align*}
		p \land (q \lor r) \equiv (p\land q) \lor (p \land r) \\
		p \lor (q \land r) \equiv (p\lor q) \land (p \lor r)
	\end{align*}
	
	\subsubsection*{Double Negative}
	$$
		\lnot(\lnot p) \equiv p
	$$
	
	\subsubsection*{Idempotent}
	\begin{align*}
		p \land p \equiv p \\
		p \lor p \equiv p 
	\end{align*}
	
	\subsubsection*{Absorbtion}
	\begin{align*}
		p \lor (p \land q) \equiv p \\
		p \land (p \lor q) \equiv p 
	\end{align*}
	
	\subsubsection*{Identity Laws}
	\begin{align*}
		p \land \top \equiv p \\
		p \lor \bot \equiv p	
	\end{align*}
	
	\subsubsection*{Domination}
	\begin{align*}
		p \lor \top \equiv \top \\
		p \land \bot \equiv \bot
	\end{align*}
	
	\subsubsection*{Negation Laws}
	\begin{align*}
		p \lor \lnot p \equiv \top \\
		p \land \lnot p \equiv \bot
	\end{align*}
	
	\subsubsection*{Negations}
	\begin{align*}
		\lnot\top\equiv\bot \\
		\lnot\bot\equiv\top
	\end{align*}
\end{multicols}

\ex{}{
	Prove that $((p\land\lnot q)\lor(p\land q))\land q \equiv p\land q.$
	\begin{align*}
		((p\land\lnot q)\lor(p\land q))\land q &\equiv (p\land(\lnot q\lor q))\land q \tag*{\text{(Distributivity)}} \\
			&\equiv (p\land\top)\land q \tag*{\text{(Negation Law)}} \\
			&\equiv p\land q \tag*{\text{(Identity)}} \\
			\tag*{\qed}
	\end{align*}
}

\subsection*{Questions}

\qs{}{
	Which of the following are statements?
	\begin{enumerate}[label=(\alph*)]
		\item "Is it going to rain tomobbrow?"
		\item "She is happy."
		\item "23 July 2024 is a Tuesday"
		\item $x = 5y + 2$
		\item $65 < 2$
	\end{enumerate}
}
\sol (a) No, a question. (b) No, "she" undefined. (c) Yes. (d) No, $x$, $y$ undefined. (e) Yes.

\qs{}{
	Let $p$, $q$, $r$ be statements.
	\begin{itemize}
		\item $p$ = ``it is cold.''
		\item $q$ = ``it is snowing.''
		\item $r$ = ``it is sunny.''
	\end{itemize}
	Translate these to symbols:
	\begin{enumerate}[label=(\alph*)]
		\item ``It is not cold but it is snowing.''
		\item ``It is neither snowing nor cold, but it is sunny.''
	\end{enumerate}
	Translate these to English:
	\begin{enumerate}[label=(\alph*),start=3]
		\item $\lnot p\land q$
		\item $(p\land q)\lor r$
	\end{enumerate}
}
\sol (a) $\lnot p\land q$ (b) $\lnot p \land \lnot q \land r$ (c) ``It is not cold but it is snowing'' (d) ``It is either snowing and cold, or sunny, or it's both.''

\qs{}{Construct the truth table for $(p\land \lnot q)\lor (q\land r)$}
\sol \begin{center}
	\begin{tabular}{|ccc||c||cc||c|}
		\hline
		$p$ & $q$ & $r$ & $\lnot q$ & $p\land\lnot q$ & $q\land r$ & $(p\land\lnot q)\lor(q\land r)$ \\ \hline
		T & T & T & F & F & T & T \\
		T & T & F & F & F & F & F \\
		T & F & T & T & T & F & T \\
		T & F & F & T & T & F & T \\
		F & T & T & F & F & T & T \\
		F & T & F & F & F & F & F \\
		F & F & T & T & F & F & F \\
		F & F & F & T & F & F & F \\ \hline
	\end{tabular}
\end{center}

\qs{}{Using De Morgan's Law, write down a statement which is logically equivalent to the negation of ``5 is even and 6 is even.''}
\sol ``5 is even and 6 is even.'' $\equiv p \land q$. The solution we want is the negation, $\lnot(p\land q)$, which, by De Morgan's Law is the same as $\lnot p \lor \lnot q$ which in English is ``5 is odd or 6 is odd.''

\qs{}{
	Show that
	$$\lnot((\lnot p\land q) \lor (\lnot p \land \lnot q)) \equiv p$$
	using a truth table, and by laws of logical equivalence.
}
\begin{center}
	\begin{tabular}{|cc||cc||cc||c||c|}
		\hline
		$p$ & $q$ & $\lnot p$ & $\lnot q$ & $\lnot p\land q$ & $\lnot p\land\lnot q$ & $(\lnot p\land q) \lor (\lnot p\land\lnot q)$ & $\lnot((\lnot p\land q) \lor (\lnot p\land\lnot q))$ \\ \hline
		T & T & F & F & F & F & F & T \\
		T & F & F & T & F & F & F & T \\
		F & T & T & F & T & F & T & F \\
		F & F & T & T & F & T & T & F \\ \hline
	\end{tabular}
\end{center}
$\tf\lnot((\lnot p\land q) \lor (\lnot p \land \lnot q)) \equiv p$ by exhaustion.
\begin{align*}
	\lnot((\lnot p\land q) \lor (\lnot p \land \lnot q)) &\equiv \lnot(\lnot p\land q) \land \lnot(\lnot p \land \lnot q) \tag*{(\text{De Morgan's Law})}\\
		&\equiv (p\lor \lnot q) \land (p \lor q) \tag*{(\text{De Morgan's Law})} \\
		&\equiv p \lor (q \land\lnot q) \tag*{(Distributivity)} \\
		&\equiv p \lor \top \tag*{(Negation Law)} \\
		&\equiv p \tag*{(Identity)} \\
		\tag*{\qed}
\end{align*}
$\tf\lnot((\lnot p\land q) \lor (\lnot p \land \lnot q)) \equiv p$ by logical equivalence.

\section{Lecture 3}
\dfn{Conditional Statement}{
	Let $p$ and $q$ be statement variables. The conditional form $p$ to $q$ is denoted $p\lthen q$, and read as ``if $p$, then $q$,'' or ``$p$ implies $q$.'' It is defined by the following truth table
	\begin{center}
		\begin{tabular}{|cc||c|}
			\hline
			$p$	&$q$	&$p\lthen q$ \\ \hline
			T & T & T \\
			T & F & F \\
			F & T & T \\
			F & F & T \\ \hline
		\end{tabular}
	\end{center}
	$p$ is called the hypothesis.\\
	$q$ is called the conclusion.
}
\ex{}{
	Suppose I make you the following promise:
	\begin{center}
		``If you do your homework then you get a chocolate.''
	\end{center}
	\begin{enumerate}[label=(\alph*)]
		\item You do not do your homework and you get a chocolate.
		\item You do your homework and you get a chocolate.
		\item You do your homework and you do not get a chocolate.
		\item You do not do your homework and you do not get a chocolate.
	\end{enumerate}
	I only lied in scenario (c), which corresponds with $(p,q)=(F,T)$.
}

\nt{
	$$ p\lthen q \equiv \lnot p \lor q $$
	\begin{center}
		\begin{tabular}{|cc||c||cc|}
			\hline
			$p$ &$q$ &$\lnot p$ &$\lnot p \lor q$ &$p\lthen q$ \\ \hline
			T & T & F & T & T \\ 
			T & F & F & F & F \\
			F & T & T & T & T \\
			F & F & T & T & F \\ \hline 
		\end{tabular}
	\end{center}
}

\dfn{Contrapositive}{
	The contrapositive of $p\lthen q$ is $\lnot q\lthen\lnot p$.
	\begin{center}
		\begin{tabular}{|cc||cc||cc|}
			\hline
			$p$ & $q$ & $\lnot p$ & $\lnot q$ & $p\lthen q$ & $\lnot q\lthen \lnot p$ \\ \hline
			T & T & F & F & T & T \\
			T & F & F & T & F & F \\
			F & T & T & F & T & T \\
			F & F & T & T & T & T \\ \hline
		\end{tabular}
	\end{center}
	$$
		p\lthen q \equiv \lnot q \lthen \lnot p
	$$
}
\ex{}{
	The contrapositive of 
	\begin{center}
		``If you do your homework then you get a chocolate.''
	\end{center}
	Is the equivalent
	\begin{center}
		``If you did not get a chocolate then you did not finish your homework.''
	\end{center}
}

\subsection*{Negation of the Conditional Statement}
The negation of $p\lthen q$ is given by $p\land \lnot q$ and can be proved logcically.
\begin{align*}
	p\lthen p &\equiv \lnot p \lor q \\
	\lnot(p\lthen p) &\equiv \lnot (\lnot p \lor q) \\
		&\equiv \lnot (\lnot p) \land \lnot q \\
		&\equiv p \land \lnot q 
\end{align*}

\ex{}{
	The negation of
	\begin{center}
		``If today is Monday, then tomorrow is my birthday''
	\end{center}
	Is
	\begin{center}
		``Today is Monday but tomorrow is not my birthday.''
	\end{center}
}

\dfn{Biconditional Statement}{
	Let $p$ and $q$ be statement variables. The biconditional statement of $p$ and $q$, denoted $p\liff q$, and read ``$p$ if and only if $q$'' is defined by the following truth table
	\begin{center}
		\begin{tabular}{|cc||c|}
			\hline
			$p$	& $q$	& $p\liff q$ \\ \hline
			T & T & T \\
			T & F & F \\
			F & T & F \\
			T & T & T \\ \hline
		\end{tabular}
	\end{center}
	$$p \liff q \equiv (p\lthen q) \land (q\lthen p)$$
}

\subsection*{Questions}
\qs{}{
	Which of the following sentences have the same meaning as ``If I am worried then I did not sleep''?
	\begin{enumerate}[label=(\alph*)]
		\item if I am worried then I do not sleep.
		\item if I am not worried then I do sleep.
		\item If I do not sleep then I am worried.
		\item I am worried and I do sleep.
		\item If I do sleep then I am not worried.
		\item I am worried or I do not sleep
		\item I do not sleep or I am not worried.
	\end{enumerate}
}
\sol 
\begin{itemize}
	\item Original: $p\lthen \lnot q$
	\item (a): $p \lthen \lnot q$, equivalent.
	\item (b): $\lnot p \lthen q$, not equivalent.
	\item (c): $\lnot q \lthen p$, not equivalent.
	\item (d): $p\land q$, not equivalent.
	\item (e): $q \lthen\lnot p$, equivalent, the contrapositive.
	\item (f): $p \lor \lnot q$, not equivalent.
	\item (g): $\lnot q \lor \lnot p$, equivalent, logically equivalent.
\end{itemize} 

\qs{}{
	Express the operations $\lor$, $\lthen$, and $\liff$ using only $\lnot$ and $\land$.
}
\sol
\begin{align*}
	p \lor q &\equiv \lnot(\lnot (p \lor q)) \\
		&\equiv \lnot (\lnot p \land \lnot q) \\
	p \lthen q &\equiv \lnot p\lor q \\
		&\equiv \lnot (\lnot (\lnot p\lor q))\\
		&\equiv \lnot (p\land \lnot q)\\
	p \liff q &\equiv (p\lthen q) \land (q\lthen p) \\
		&\equiv (\lnot p\lor q) \land (\lnot q\lor p) \\
		&\equiv \lnot(\lnot(\lnot p\lor q)) \land \lnot(\lnot(\lnot q\lor p)) \\
		&\equiv \lnot(p\land \lnot q) \land \lnot(q\land \lnot p)
\end{align*}

\qs{Challenge}{
	Consider the NAND operation $p\lnand q \equiv \lnot(p\land q)$ can you express $\land$, $\lor$, $\lnot$, and $\lthen$ using only $\lnand$ operations? Can you express using only $\lnot$ and $\lxor$?
}
\sol Generating expressions using only NANDs:
\begin{align*}
	\lnot p &\equiv \lnot(p \land p) \\
		&\equiv p\lnand p \\
	p\land q &\equiv (p\land q)\land(p\land q) \\
		&\equiv \lnot (\lnot(p\land q)\land\lnot(p\land q)) \\
		&\equiv \lnot((p\lnand q)\land (p\lnand q)) \\
		&\equiv (p\lnand q)\lnand (p\lnand q) \\
	p\lor q &\equiv \lnot(\lnot(p\lor q)) \\
		&\equiv \lnot(\lnot p\land \lnot q) \\
		&\equiv \lnot p\lnand\lnot q \\
		&\equiv (p\lnand p)\lnand(q\lnand q) \\
	p \lthen q &\equiv \lnot p\lor q \\
		&\equiv \lnot (\lnot (\lnot p\lor q)) \\
		&\equiv \lnot (p\land \lnot q) \\
		&\equiv p\lnand\lnot q \\
		&\equiv p\lnand(q\lnand q)
\end{align*}
These sick fucks had me testing and observing truth tables for two hours. I was suspicious at times, but I assumed there must be a solution\dots\\
There is not. No matter how many XOR and NOT operations you apply, ultimately, you will always have 2 Falses and 2 Trues, or 4 Falses, or 4 Trues. \textbf{AHHHHHHHH}.

\qs{}{
	Show that $\lnot(p\lthen q) \not\equiv \lnot p\lthen \lnot q$.
}
\sol By counterexample. Suppose $p=\true\And q=\true$
\begin{align*}
\lnot(p\lthen q) &\equiv \lnot(\true\lthen\true) \\
	&\equiv \lnot(\true) \\
	&\equiv \false \\
\lnot p\lthen\lnot q &\equiv \lnot\true\lthen\lnot\true \\
	&\equiv \false\lthen\false \\
	&\equiv \true \not\equiv\lnot(p\lthen q) \tag*{\qed}
\end{align*}

\qs{}{
	Which of the following sentences have the opposite truth valuse as ``If I am worried then I did not sleep''?
	\begin{enumerate}[label=(\alph*)]
		\item if I am worried then I do not sleep.
		\item if I am not worried then I do sleep.
		\item If I do not sleep then I am worried.
		\item I am worried and I do sleep.
		\item If I do sleep then I am not worried.
		\item I am worried or I do not sleep
		\item I do not sleep or I am not worried.
	\end{enumerate}
}
\sol 
\begin{itemize}
	\item Original: $p\lthen \lnot q$
	\item (a): $p \lthen \lnot q$, No, equivalent statement.
	\item (b): $\lnot p \lthen q$, No, True 3/4 times, same as our statement. Not possible to be opposite.
	\item (c): $\lnot q \lthen p$, No, True 3/4 times.
	\item (d): $p\land q$, Yes! $p\lthen\lnot q \equiv \lnot p\lor \lnot q\equiv \lnot(p\land q)$, exactly the opposite.
	\item (e): $q \lthen\lnot p$, No, equivalent statement.
	\item (f): $p \lor \lnot q$, No, True 3/4 times.
	\item (g): $\lnot q \lor \lnot p$, No, equivalent statement.
\end{itemize} 

\qs{}{
	Show that $$p\lthen(q\lor r)\equiv (p\land\lnot q)\lthen r.$$
}
\sol
\begin{align*}
	p\lthen (q\lor r) &\equiv \lnot p \lor (q\lor r) \\
		&\equiv (\lnot p \lor q)\lor r \\
		&\equiv \lnot(\lnot(\lnot p \lor q))\lor r \\
		&\equiv \lnot(p \land\lnot q)\lor r \\
		&\equiv (p\land q)\lthen r
\end{align*}

\qs{}{
	Let $n$ be a positive integer. Find conditions that are:
	\begin{enumerate}[label=(\alph*)]
		\item necessary, but not sufficent for $n$ to be a multiple of 10.
		\item sufficient but not necessary for $n$ to be divisible by 10.
		\item necessary and sufficient for $n$ to be divisible by 10.
	\end{enumerate}
}
\sol\begin{enumerate}[label=(\alph*)]
	\item $n$ is a multiple of 10 $\lthen$ $n$ is necessarily even.
	\item $n$ is a multiple of 50 $\lthen$ sufficient to conclude that $n$ is divisible by 10.
	\item $n$'s last digit is a 0 $\liff$ $n$ is divisible by 10.
\end{enumerate}

\chapter{Week 2}
\section{Lecture 4}

\end{document}
