\documentclass{report}

%%%%%%%%%%%%%%%%%%%%%%%%%%%%%%%%%
% PACKAGE IMPORTS
%%%%%%%%%%%%%%%%%%%%%%%%%%%%%%%%%
\usepackage[tmargin=2cm,rmargin=1in,lmargin=1in,margin=0.85in,bmargin=2cm,footskip=.2in]{geometry}
\usepackage[none]{hyphenat}
\usepackage{amsmath,amsfonts,amsthm,amssymb,mathtools}
\allowdisplaybreaks
\usepackage{undertilde}
\usepackage{xfrac}
\usepackage[makeroom]{cancel}
\usepackage{mathtools}
\usepackage{bookmark}
\usepackage{enumitem}
\usepackage{kbordermatrix}
\renewcommand{\kbldelim}{(} % Change left delimiter to (
\renewcommand{\kbrdelim}{)} % Change right delimiter to )
\usepackage{hyperref,theoremref}
\hypersetup{
	pdftitle={Assignment},
	colorlinks=true, linkcolor=doc!90,
	bookmarksnumbered=true,
	bookmarksopen=true
}
\usepackage[most,many,breakable]{tcolorbox}
\usepackage{xcolor}
\usepackage{varwidth}
\usepackage{varwidth}
\usepackage{etoolbox}
%\usepackage{authblk}
\usepackage{nameref}
\usepackage{multicol,array}
\usepackage{tikz-cd}
\usepackage[ruled,vlined,linesnumbered]{algorithm2e}
\usepackage{comment} % enables the use of multi-line comments (\ifx \fi) 
\usepackage{import}
\usepackage{xifthen}
\usepackage{pdfpages}
\usepackage{svg}
\usepackage{transparent}
\usepackage{pgfplots}
\pgfplotsset{compat=1.18}
\usetikzlibrary{calc}
\usetikzlibrary{graphs}
\usetikzlibrary{graphs.standard}
% \usetikzlibrary{graphdrawing}

\newcommand\mycommfont[1]{\footnotesize\ttfamily\textcolor{blue}{#1}}
\SetCommentSty{mycommfont}
\newcommand{\incfig}[1]{%
    \def\svgwidth{\columnwidth}
    \import{./figures/}{#1.pdf_tex}
}


\usepackage{tikzsymbols}
% \renewcommand\qedsymbol{$\Laughey$}

\definecolor{commentgreen}{RGB}{2,112,10}
%%
%% Julia definition (c) 2014 Jubobs
%%
\lstdefinelanguage{Julia}%
  {morekeywords={abstract,break,case,catch,const,continue,do,else,elseif,%
      end,export,false,for,function,immutable,import,importall,if,in,%
      macro,module,otherwise,quote,return,switch,true,try,type,typealias,%
      using,while},%
   sensitive=true,%
   alsoother={$},%
   morecomment=[l]\#,%
   morecomment=[n]{\#=}{=\#},%
   morestring=[s]{"}{"},%
   morestring=[m]{'}{'},%
}[keywords,comments,strings]%

\lstset{%
    language        	= Julia,
    basicstyle      	= \ttfamily,
    keywordstyle    	= \bfseries\color{blue},
    stringstyle     	= \color{magenta},
    commentstyle    	= \color{commentgreen},
    showstringspaces	= false,
		numbers						= left,
		tabsize						= 4,
}

\definecolor{stringyellow}{RGB}{227, 78, 48}
%% 
%% Shamelessly stolen from Vivi on Stackoverflow
%% https://tex.stackexchange.com/questions/75116/what-can-i-use-to-typeset-matlab-code-in-my-document
%%
\lstset{language=Matlab,%
    %basicstyle=\color{red},
    breaklines=true,%
    morekeywords={matlab2tikz},
		morekeywords={subtitle}
    keywordstyle=\color{blue},%
    morekeywords=[2]{1}, keywordstyle=[2]{\color{black}},
    identifierstyle=\color{black},%
    stringstyle=\color{stringyellow},
    commentstyle=\color{commentgreen},%
    showstringspaces=false,%without this there will be a symbol in the places where there is a space
    numbers=left,%
		firstnumber=1,
    % numberstyle={\tiny \color{black}},% size of the numbers
    % numbersep=9pt, % this defines how far the numbers are from the text
    emph=[1]{for,end,break},emphstyle=[1]\color{red}, %some words to emphasise
    %emph=[2]{word1,word2}, emphstyle=[2]{style},    
}

%% 
%% Shamelessly stolen from egreg on Stackoverflow
%% https://tex.stackexchange.com/questions/280681/how-to-have-multiple-lines-of-intertext-within-align-environment
%%
\newlength{\normalparindent}
\AtBeginDocument{\setlength{\normalparindent}{\parindent}}
\newcommand{\longintertext}[1]{%
  \intertext{%
    \parbox{\linewidth}{%
      \setlength{\parindent}{\normalparindent}
      \noindent#1%
    }%
  }%
}

%\usepackage{import}
%\usepackage{xifthen}
%\usepackage{pdfpages}
%\usepackage{transparent}

%%%%%%%%%%%%%%%%%%%%%%%%%%%%%%
% SELF MADE COLORS
%%%%%%%%%%%%%%%%%%%%%%%%%%%%%%
\definecolor{myg}{RGB}{56, 140, 70}
\definecolor{myb}{RGB}{45, 111, 177}
\definecolor{myr}{RGB}{199, 68, 64}
\definecolor{mytheorembg}{HTML}{F2F2F9}
\definecolor{mytheoremfr}{HTML}{00007B}
\definecolor{mylenmabg}{HTML}{FFFAF8}
\definecolor{mylenmafr}{HTML}{983b0f}
\definecolor{mypropbg}{HTML}{f2fbfc}
\definecolor{mypropfr}{HTML}{191971}
\definecolor{myexamplebg}{HTML}{F2FBF8}
\definecolor{myexamplefr}{HTML}{88D6D1}
\definecolor{myexampleti}{HTML}{2A7F7F}
\definecolor{mydefinitbg}{HTML}{E5E5FF}
\definecolor{mydefinitfr}{HTML}{3F3FA3}
\definecolor{notesgreen}{RGB}{0,162,0}
\definecolor{myp}{RGB}{197, 92, 212}
\definecolor{mygr}{HTML}{2C3338}
\definecolor{myred}{RGB}{127,0,0}
\definecolor{myyellow}{RGB}{169,121,69}
\definecolor{myexercisebg}{HTML}{F2FBF8}
\definecolor{myexercisefg}{HTML}{88D6D1}

%%%%%%%%%%%%%%%%%%%%%%%%%%%%
% TCOLORBOX SETUPS
%%%%%%%%%%%%%%%%%%%%%%%%%%%%
\setlength{\parindent}{0pt}

%================================
% THEOREM BOX
%================================
\tcbuselibrary{theorems,skins,hooks}
\newtcbtheorem[number within=section]{Theorem}{Theorem}
{%
	enhanced,
	breakable,
	colback = mytheorembg,
	frame hidden,
	boxrule = 0sp,
	borderline west = {2pt}{0pt}{mytheoremfr},
	sharp corners,
	detach title,
	before upper = \tcbtitle\par\smallskip,
	coltitle = mytheoremfr,
	fonttitle = \bfseries\sffamily,
	description font = \mdseries,
	separator sign none,
	segmentation style={solid, mytheoremfr},
}
{th}

\tcbuselibrary{theorems,skins,hooks}
\newtcbtheorem[number within=chapter]{theorem}{Theorem}
{%
	enhanced,
	breakable,
	colback = mytheorembg,
	frame hidden,
	boxrule = 0sp,
	borderline west = {2pt}{0pt}{mytheoremfr},
	sharp corners,
	detach title,
	before upper = \tcbtitle\par\smallskip,
	coltitle = mytheoremfr,
	fonttitle = \bfseries\sffamily,
	description font = \mdseries,
	separator sign none,
	segmentation style={solid, mytheoremfr},
}
{th}


\tcbuselibrary{theorems,skins,hooks}
\newtcolorbox{Theoremcon}
{%
	enhanced
	,breakable
	,colback = mytheorembg
	,frame hidden
	,boxrule = 0sp
	,borderline west = {2pt}{0pt}{mytheoremfr}
	,sharp corners
	,description font = \mdseries
	,separator sign none
}

%================================
% Corollery
%================================
\tcbuselibrary{theorems,skins,hooks}
\newtcbtheorem[number within=section]{Corollary}{Corollary}
{%
	enhanced
	,breakable
	,colback = myp!10
	,frame hidden
	,boxrule = 0sp
	,borderline west = {2pt}{0pt}{myp!85!black}
	,sharp corners
	,detach title
	,before upper = \tcbtitle\par\smallskip
	,coltitle = myp!85!black
	,fonttitle = \bfseries\sffamily
	,description font = \mdseries
	,separator sign none
	,segmentation style={solid, myp!85!black}
}
{th}
\tcbuselibrary{theorems,skins,hooks}
\newtcbtheorem[number within=chapter]{corollary}{Corollary}
{%
	enhanced
	,breakable
	,colback = myp!10
	,frame hidden
	,boxrule = 0sp
	,borderline west = {2pt}{0pt}{myp!85!black}
	,sharp corners
	,detach title
	,before upper = \tcbtitle\par\smallskip
	,coltitle = myp!85!black
	,fonttitle = \bfseries\sffamily
	,description font = \mdseries
	,separator sign none
	,segmentation style={solid, myp!85!black}
}
{th}

%================================
% LENMA
%================================
\tcbuselibrary{theorems,skins,hooks}
\newtcbtheorem[number within=section]{Lenma}{Lenma}
{%
	enhanced,
	breakable,
	colback = mylenmabg,
	frame hidden,
	boxrule = 0sp,
	borderline west = {2pt}{0pt}{mylenmafr},
	sharp corners,
	detach title,
	before upper = \tcbtitle\par\smallskip,
	coltitle = mylenmafr,
	fonttitle = \bfseries\sffamily,
	description font = \mdseries,
	separator sign none,
	segmentation style={solid, mylenmafr},
}
{th}

\tcbuselibrary{theorems,skins,hooks}
\newtcbtheorem[number within=chapter]{lenma}{Lenma}
{%
	enhanced,
	breakable,
	colback = mylenmabg,
	frame hidden,
	boxrule = 0sp,
	borderline west = {2pt}{0pt}{mylenmafr},
	sharp corners,
	detach title,
	before upper = \tcbtitle\par\smallskip,
	coltitle = mylenmafr,
	fonttitle = \bfseries\sffamily,
	description font = \mdseries,
	separator sign none,
	segmentation style={solid, mylenmafr},
}
{th}

%================================
% PROPOSITION
%================================
\tcbuselibrary{theorems,skins,hooks}
\newtcbtheorem[number within=section]{Prop}{Proposition}
{%
	enhanced,
	breakable,
	colback = mypropbg,
	frame hidden,
	boxrule = 0sp,
	borderline west = {2pt}{0pt}{mypropfr},
	sharp corners,
	detach title,
	before upper = \tcbtitle\par\smallskip,
	coltitle = mypropfr,
	fonttitle = \bfseries\sffamily,
	description font = \mdseries,
	separator sign none,
	segmentation style={solid, mypropfr},
}
{th}

\tcbuselibrary{theorems,skins,hooks}
\newtcbtheorem[number within=chapter]{prop}{Proposition}
{%
	enhanced,
	breakable,
	colback = mypropbg,
	frame hidden,
	boxrule = 0sp,
	borderline west = {2pt}{0pt}{mypropfr},
	sharp corners,
	detach title,
	before upper = \tcbtitle\par\smallskip,
	coltitle = mypropfr,
	fonttitle = \bfseries\sffamily,
	description font = \mdseries,
	separator sign none,
	segmentation style={solid, mypropfr},
}
{th}

%================================
% CLAIM
%================================
\tcbuselibrary{theorems,skins,hooks}
\newtcbtheorem[number within=section]{claim}{Claim}
{%
	enhanced
	,breakable
	,colback = myg!10
	,frame hidden
	,boxrule = 0sp
	,borderline west = {2pt}{0pt}{myg}
	,sharp corners
	,detach title
	,before upper = \tcbtitle\par\smallskip
	,coltitle = myg!85!black
	,fonttitle = \bfseries\sffamily
	,description font = \mdseries
	,separator sign none
	,segmentation style={solid, myg!85!black}
}
{th}

%================================
% Exercise
%================================
\tcbuselibrary{theorems,skins,hooks}
\newtcbtheorem[number within=section]{Exercise}{Exercise}
{%
	enhanced,
	breakable,
	colback = myexercisebg,
	frame hidden,
	boxrule = 0sp,
	borderline west = {2pt}{0pt}{myexercisefg},
	sharp corners,
	detach title,
	before upper = \tcbtitle\par\smallskip,
	coltitle = myexercisefg,
	fonttitle = \bfseries\sffamily,
	description font = \mdseries,
	separator sign none,
	segmentation style={solid, myexercisefg},
}
{th}

\tcbuselibrary{theorems,skins,hooks}
\newtcbtheorem[number within=chapter]{exercise}{Exercise}
{%
	enhanced,
	breakable,
	colback = myexercisebg,
	frame hidden,
	boxrule = 0sp,
	borderline west = {2pt}{0pt}{myexercisefg},
	sharp corners,
	detach title,
	before upper = \tcbtitle\par\smallskip,
	coltitle = myexercisefg,
	fonttitle = \bfseries\sffamily,
	description font = \mdseries,
	separator sign none,
	segmentation style={solid, myexercisefg},
}
{th}

%================================
% EXAMPLE BOX
%================================
\newtcbtheorem[number within=section]{Example}{Example}
{%
	colback = myexamplebg
	,breakable
	,colframe = myexamplefr
	,coltitle = myexampleti
	,boxrule = 1pt
	,sharp corners
	,detach title
	,before upper=\tcbtitle\par\smallskip
	,fonttitle = \bfseries
	,description font = \mdseries
	,separator sign none
	,description delimiters parenthesis
}
{ex}

\newtcbtheorem[number within=chapter]{example}{Example}
{%
	colback = myexamplebg
	,breakable
	,colframe = myexamplefr
	,coltitle = myexampleti
	,boxrule = 1pt
	,sharp corners
	,detach title
	,before upper=\tcbtitle\par\smallskip
	,fonttitle = \bfseries
	,description font = \mdseries
	,separator sign none
	,description delimiters parenthesis
}
{ex}

%================================
% DEFINITION BOX
%================================
\newtcbtheorem[number within=section]{Definition}{Definition}{enhanced,
	before skip=2mm,after skip=2mm, colback=red!5,colframe=red!80!black,boxrule=0.5mm,
	attach boxed title to top left={xshift=1cm,yshift*=1mm-\tcboxedtitleheight}, varwidth boxed title*=-3cm,
	boxed title style={frame code={
					\path[fill=tcbcolback]
					([yshift=-1mm,xshift=-1mm]frame.north west)
					arc[start angle=0,end angle=180,radius=1mm]
					([yshift=-1mm,xshift=1mm]frame.north east)
					arc[start angle=180,end angle=0,radius=1mm];
					\path[left color=tcbcolback!60!black,right color=tcbcolback!60!black,
						middle color=tcbcolback!80!black]
					([xshift=-2mm]frame.north west) -- ([xshift=2mm]frame.north east)
					[rounded corners=1mm]-- ([xshift=1mm,yshift=-1mm]frame.north east)
					-- (frame.south east) -- (frame.south west)
					-- ([xshift=-1mm,yshift=-1mm]frame.north west)
					[sharp corners]-- cycle;
				},interior engine=empty,
		},
	fonttitle=\bfseries,
	title={#2},#1}{def}
\newtcbtheorem[number within=chapter]{definition}{Definition}{enhanced,
	before skip=2mm,after skip=2mm, colback=red!5,colframe=red!80!black,boxrule=0.5mm,
	attach boxed title to top left={xshift=1cm,yshift*=1mm-\tcboxedtitleheight}, varwidth boxed title*=-3cm,
	boxed title style={frame code={
					\path[fill=tcbcolback]
					([yshift=-1mm,xshift=-1mm]frame.north west)
					arc[start angle=0,end angle=180,radius=1mm]
					([yshift=-1mm,xshift=1mm]frame.north east)
					arc[start angle=180,end angle=0,radius=1mm];
					\path[left color=tcbcolback!60!black,right color=tcbcolback!60!black,
						middle color=tcbcolback!80!black]
					([xshift=-2mm]frame.north west) -- ([xshift=2mm]frame.north east)
					[rounded corners=1mm]-- ([xshift=1mm,yshift=-1mm]frame.north east)
					-- (frame.south east) -- (frame.south west)
					-- ([xshift=-1mm,yshift=-1mm]frame.north west)
					[sharp corners]-- cycle;
				},interior engine=empty,
		},
	fonttitle=\bfseries,
	title={#2},#1}{def}

%================================
% Solution BOX
%================================
\makeatletter
\newtcbtheorem{question}{Question}{enhanced,
	breakable,
	colback=white,
	colframe=myb!80!black,
	attach boxed title to top left={yshift*=-\tcboxedtitleheight},
	fonttitle=\bfseries,
	title={#2},
	boxed title size=title,
	boxed title style={%
			sharp corners,
			rounded corners=northwest,
			colback=tcbcolframe,
			boxrule=0pt,
		},
	underlay boxed title={%
			\path[fill=tcbcolframe] (title.south west)--(title.south east)
			to[out=0, in=180] ([xshift=5mm]title.east)--
			(title.center-|frame.east)
			[rounded corners=\kvtcb@arc] |-
			(frame.north) -| cycle;
		},
	#1
}{def}
\makeatother

%================================
% SOLUTION BOX
%================================
\makeatletter
\newtcolorbox{solution}{enhanced,
	breakable,
	colback=white,
	colframe=myg!80!black,
	attach boxed title to top left={yshift*=-\tcboxedtitleheight},
	title=Solution,
	boxed title size=title,
	boxed title style={%
			sharp corners,
			rounded corners=northwest,
			colback=tcbcolframe,
			boxrule=0pt,
		},
	underlay boxed title={%
			\path[fill=tcbcolframe] (title.south west)--(title.south east)
			to[out=0, in=180] ([xshift=5mm]title.east)--
			(title.center-|frame.east)
			[rounded corners=\kvtcb@arc] |-
			(frame.north) -| cycle;
		},
}
\makeatother

%================================
% Question BOX
%================================
\makeatletter
\newtcbtheorem{qstion}{Question}{enhanced,
	breakable,
	colback=white,
	colframe=mygr,
	attach boxed title to top left={yshift*=-\tcboxedtitleheight},
	fonttitle=\bfseries,
	title={#2},
	boxed title size=title,
	boxed title style={%
			sharp corners,
			rounded corners=northwest,
			colback=tcbcolframe,
			boxrule=0pt,
		},
	underlay boxed title={%
			\path[fill=tcbcolframe] (title.south west)--(title.south east)
			to[out=0, in=180] ([xshift=5mm]title.east)--
			(title.center-|frame.east)
			[rounded corners=\kvtcb@arc] |-
			(frame.north) -| cycle;
		},
	#1
}{def}
\makeatother

\newtcbtheorem[number within=chapter]{wconc}{Wrong Concept}{
	breakable,
	enhanced,
	colback=white,
	colframe=myr,
	arc=0pt,
	outer arc=0pt,
	fonttitle=\bfseries\sffamily\large,
	colbacktitle=myr,
	attach boxed title to top left={},
	boxed title style={
			enhanced,
			skin=enhancedfirst jigsaw,
			arc=3pt,
			bottom=0pt,
			interior style={fill=myr}
		},
	#1
}{def}

%================================
% NOTE BOX
%================================
\usetikzlibrary{arrows,calc,shadows.blur}
\tcbuselibrary{skins}
\newtcolorbox{note}[1][]{%
	enhanced jigsaw,
	colback=gray!20!white,%
	colframe=gray!80!black,
	size=small,
	boxrule=1pt,
	title=\textbf{Note:-},
	halign title=flush center,
	coltitle=black,
	breakable,
	drop shadow=black!50!white,
	attach boxed title to top left={xshift=1cm,yshift=-\tcboxedtitleheight/2,yshifttext=-\tcboxedtitleheight/2},
	minipage boxed title=1.5cm,
	boxed title style={%
			colback=white,
			size=fbox,
			boxrule=1pt,
			boxsep=2pt,
			underlay={%
					\coordinate (dotA) at ($(interior.west) + (-0.5pt,0)$);
					\coordinate (dotB) at ($(interior.east) + (0.5pt,0)$);
					\begin{scope}
						\clip (interior.north west) rectangle ([xshift=3ex]interior.east);
						\filldraw [white, blur shadow={shadow opacity=60, shadow yshift=-.75ex}, rounded corners=2pt] (interior.north west) rectangle (interior.south east);
					\end{scope}
					\begin{scope}[gray!80!black]
						\fill (dotA) circle (2pt);
						\fill (dotB) circle (2pt);
					\end{scope}
				},
		},
	#1,
}

%%%%%%%%%%%%%%%%%%%%%%%%%%%%%%
% SELF MADE COMMANDS
%%%%%%%%%%%%%%%%%%%%%%%%%%%%%%
\newcommand{\thm}[2]{\begin{Theorem}{#1}{}#2\end{Theorem}}
\newcommand{\cor}[2]{\begin{Corollary}{#1}{}#2\end{Corollary}}
\newcommand{\mlenma}[2]{\begin{Lenma}{#1}{}#2\end{Lenma}}
\newcommand{\mprop}[2]{\begin{Prop}{#1}{}#2\end{Prop}}
\newcommand{\clm}[3]{\begin{claim}{#1}{#2}#3\end{claim}}
\newcommand{\wc}[2]{\begin{wconc}{#1}{}\setlength{\parindent}{1cm}#2\end{wconc}}
\newcommand{\thmcon}[1]{\begin{Theoremcon}{#1}\end{Theoremcon}}
\newcommand{\ex}[2]{\begin{Example}{#1}{}#2\end{Example}}
\newcommand{\dfn}[2]{\begin{Definition}[colbacktitle=red!75!black]{#1}{}#2\end{Definition}}
\newcommand{\dfnc}[2]{\begin{definition}[colbacktitle=red!75!black]{#1}{}#2\end{definition}}
\newcommand{\qs}[2]{\begin{question}{#1}{}#2\end{question}}
\newcommand{\pf}[2]{\begin{myproof}[#1]#2\end{myproof}}
\newcommand{\nt}[1]{\begin{note}#1\end{note}}

\newcommand*\circled[1]{\tikz[baseline=(char.base)]{
		\node[shape=circle,draw,inner sep=1pt] (char) {#1};}}
\newcommand\getcurrentref[1]{%
	\ifnumequal{\value{#1}}{0}
	{??}
	{\the\value{#1}}%
}
\newcommand{\getCurrentSectionNumber}{\getcurrentref{section}}
\newenvironment{myproof}[1][\proofname]{%
	\proof[\bfseries #1: ]%
}{\endproof}

\newcommand{\mclm}[2]{\begin{myclaim}[#1]#2\end{myclaim}}
\newenvironment{myclaim}[1][\claimname]{\proof[\bfseries #1: ]}{}

\newcounter{mylabelcounter}

\makeatletter
\newcommand{\setword}[2]{%
	\phantomsection
	#1\def\@currentlabel{\unexpanded{#1}}\label{#2}%
}
\makeatother

\tikzset{
	symbol/.style={
			draw=none,
			every to/.append style={
					edge node={node [sloped, allow upside down, auto=false]{$#1$}}}
		}
}

% deliminators
\DeclarePairedDelimiter{\abs}{\lvert}{\rvert}
\DeclarePairedDelimiter{\norm}{\lVert}{\rVert}

\DeclarePairedDelimiter{\ceil}{\lceil}{\rceil}
\DeclarePairedDelimiter{\floor}{\lfloor}{\rfloor}
\DeclarePairedDelimiter{\round}{\lfloor}{\rceil}

\newsavebox\diffdbox
\newcommand{\slantedromand}{{\mathpalette\makesl{d}}}
\newcommand{\makesl}[2]{%
\begingroup
\sbox{\diffdbox}{$\mathsurround=0pt#1\mathrm{#2}$}%
\pdfsave
\pdfsetmatrix{1 0 0.2 1}%
\rlap{\usebox{\diffdbox}}%
\pdfrestore
\hskip\wd\diffdbox
\endgroup
}
\newcommand{\dd}[1][]{\ensuremath{\mathop{}\!\ifstrempty{#1}{%
\slantedromand\@ifnextchar^{\hspace{0.2ex}}{\hspace{0.1ex}}}%
{\slantedromand\hspace{0.2ex}^{#1}}}}
\ProvideDocumentCommand\dv{o m g}{%
  \ensuremath{%
    \IfValueTF{#3}{%
      \IfNoValueTF{#1}{%
        \frac{\dd #2}{\dd #3}%
      }{%
        \frac{\dd^{#1} #2}{\dd #3^{#1}}%
      }%
    }{%
      \IfNoValueTF{#1}{%
        \frac{\dd}{\dd #2}%
      }{%
        \frac{\dd^{#1}}{\dd #2^{#1}}%
      }%
    }%
  }%
}
\providecommand*{\pdv}[3][]{\frac{\partial^{#1}#2}{\partial#3^{#1}}}
%  - others
\DeclareMathOperator{\Lap}{\mathcal{L}}
\DeclareMathOperator{\Var}{Var} % varience
\DeclareMathOperator{\Cov}{Cov} % covarience
\DeclareMathOperator{\E}{E} % expected

% Since the amsthm package isn't loaded

% I dot not prefer the slanted \leq ;P
% % I prefer the slanted \leq
% \let\oldleq\leq % save them in case they're every wanted
% \let\oldgeq\geq
% \renewcommand{\leq}{\leqslant}
% \renewcommand{\geq}{\geqslant}

% % redefine matrix env to allow for alignment, use r as default
% \renewcommand*\env@matrix[1][r]{\hskip -\arraycolsep
%     \let\@ifnextchar\new@ifnextchar
%     \array{*\c@MaxMatrixCols #1}}

%\usepackage{framed}
%\usepackage{titletoc}
%\usepackage{etoolbox}
%\usepackage{lmodern}

%\patchcmd{\tableofcontents}{\contentsname}{\sffamily\contentsname}{}{}

%\renewenvironment{leftbar}
%{\def\FrameCommand{\hspace{6em}%
%		{\color{myyellow}\vrule width 2pt depth 6pt}\hspace{1em}}%
%	\MakeFramed{\parshape 1 0cm \dimexpr\textwidth-6em\relax\FrameRestore}\vskip2pt%
%}
%{\endMakeFramed}

%\titlecontents{chapter}
%[0em]{\vspace*{2\baselineskip}}
%{\parbox{4.5em}{%
%		\hfill\Huge\sffamily\bfseries\color{myred}\thecontentspage}%
%	\vspace*{-2.3\baselineskip}\leftbar\textsc{\small\chaptername~\thecontentslabel}\\\sffamily}
%{}{\endleftbar}
%\titlecontents{section}
%[8.4em]
%{\sffamily\contentslabel{3em}}{}{}
%{\hspace{0.5em}\nobreak\itshape\color{myred}\contentspage}
%\titlecontents{subsection}
%[8.4em]
%{\sffamily\contentslabel{3em}}{}{}  
%{\hspace{0.5em}\nobreak\itshape\color{myred}\contentspage}

%%%%%%%%%%%%%%%%%%%%%%%%%%%%%%%%%%%%%%%%%%%
% TABLE OF CONTENTS
%%%%%%%%%%%%%%%%%%%%%%%%%%%%%%%%%%%%%%%%%%%
\usepackage{tikz}
\definecolor{doc}{RGB}{0,60,110}
\usepackage{titletoc}
\contentsmargin{0cm}
\titlecontents{chapter}[3.7pc]
{\addvspace{30pt}%
	\begin{tikzpicture}[remember picture, overlay]%
		\draw[fill=doc!60,draw=doc!60] (-7,-.1) rectangle (-0.9,.5);%
		\pgftext[left,x=-3.5cm,y=0.2cm]{\color{white}\Large\sc\bfseries Chapter\ \thecontentslabel};%
	\end{tikzpicture}\color{doc!60}\large\sc\bfseries}%
{}
{}
{\;\titlerule\;\large\sc\bfseries Page \thecontentspage
	\begin{tikzpicture}[remember picture, overlay]
		\draw[fill=doc!60,draw=doc!60] (2pt,0) rectangle (4,0.1pt);
	\end{tikzpicture}}%
\titlecontents{section}[3.7pc]
{\addvspace{2pt}}
{\contentslabel[\thecontentslabel]{2pc}}
{}
{\hfill\small \thecontentspage}
[]
\titlecontents*{subsection}[3.7pc]
{\addvspace{-1pt}\small}
{}
{}
{\ --- \small\thecontentspage}
[ \textbullet\ ][]

\makeatletter
\renewcommand{\tableofcontents}{%
	\chapter*{%
	  \vspace*{-20\p@}%
	  \begin{tikzpicture}[remember picture, overlay]%
		  \pgftext[right,x=15cm,y=0.2cm]{\color{doc!60}\Huge\sc\bfseries \contentsname};%
		  \draw[fill=doc!60,draw=doc!60] (13,-.75) rectangle (20,1);%
		  \clip (13,-.75) rectangle (20,1);
		  \pgftext[right,x=15cm,y=0.2cm]{\color{white}\Huge\sc\bfseries \contentsname};%
	  \end{tikzpicture}}%
	\@starttoc{toc}}
\makeatother

\newcommand{\inv}{^{-1}}
\newcommand{\opname}{\operatorname}
\newcommand{\surjto}{\twoheadrightarrow}
% \newcommand{\injto}{\hookrightarrow}
\newcommand{\injto}{\rightarrowtail}
\newcommand{\bijto}{\leftrightarrow}

\newcommand{\liff}{\leftrightarrow}
\newcommand{\notliff}{\mathrel{\ooalign{$\leftrightarrow$\cr\hidewidth$/$\hidewidth}}}
\newcommand{\lthen}{\rightarrow}
\let\varlnot\lnot
\newcommand{\ordsim}{\mathord{\sim}}
\renewcommand{\lnot}{\ordsim}
\newcommand{\lxor}{\oplus}
\newcommand{\lnand}{\barwedge}
\newcommand{\divs}{\mathrel{\mid}}
\newcommand{\ndivs}{\mathrel{\nmid}}
\def\contra{\tikz[baseline, x=0.22em, y=0.22em, line width=0.032em]\draw (0,2.83)--(2.83,0) (0.71,3.54)--(3.54,0.71) (0,0.71)--(2.83,3.54) (0.71,0)--(3.54,2.83);}

\newcommand{\On}{\mathrm{On}} % ordinals
\DeclareMathOperator{\img}{im} % Image
\DeclareMathOperator{\Img}{Im} % Image
\DeclareMathOperator{\coker}{coker} % Cokernel
\DeclareMathOperator{\Coker}{Coker} % Cokernel
\DeclareMathOperator{\Ker}{Ker} % Kernel
\DeclareMathOperator{\rank}{rank}
\DeclareMathOperator{\Spec}{Spec} % spectrum
\DeclareMathOperator{\Tr}{Tr} % trace
\DeclareMathOperator{\pr}{pr} % projection
\DeclareMathOperator{\ext}{ext} % extension
\DeclareMathOperator{\pred}{pred} % predecessor
\DeclareMathOperator{\dom}{dom} % domain
\DeclareMathOperator{\ran}{ran} % range
\DeclareMathOperator{\Hom}{Hom} % homomorphism
\DeclareMathOperator{\Mor}{Mor} % morphisms
\DeclareMathOperator{\End}{End} % endomorphism
\DeclareMathOperator{\Span}{span}
\newcommand{\Mod}{\mathbin{\mathrm{mod}}}

\newcommand{\eps}{\epsilon}
\newcommand{\veps}{\varepsilon}
\newcommand{\ol}{\overline}
\newcommand{\ul}{\underline}
\newcommand{\wt}{\widetilde}
\newcommand{\wh}{\widehat}
\newcommand{\ut}{\utilde}
\newcommand{\unit}[1]{\ut{\hat{#1}}}
\newcommand{\emp}{\varnothing}

\newcommand{\vocab}[1]{\textbf{\color{blue} #1}}
\providecommand{\half}{\frac{1}{2}}
\newcommand{\dang}{\measuredangle} %% Directed angle
\newcommand{\ray}[1]{\overrightarrow{#1}}
\newcommand{\seg}[1]{\overline{#1}}
\newcommand{\arc}[1]{\wideparen{#1}}
\DeclareMathOperator{\cis}{cis}
\DeclareMathOperator*{\lcm}{lcm}
\DeclareMathOperator*{\argmin}{arg min}
\DeclareMathOperator*{\argmax}{arg max}
\newcommand{\cycsum}{\sum_{\mathrm{cyc}}}
\newcommand{\symsum}{\sum_{\mathrm{sym}}}
\newcommand{\cycprod}{\prod_{\mathrm{cyc}}}
\newcommand{\symprod}{\prod_{\mathrm{sym}}}
\newcommand{\parinn}{\setlength{\parindent}{1cm}}
\newcommand{\parinf}{\setlength{\parindent}{0cm}}
% \newcommand{\norm}{\|\cdot\|}
\newcommand{\inorm}{\norm_{\infty}}
\newcommand{\opensets}{\{V_{\alpha}\}_{\alpha\in I}}
\newcommand{\oset}{V_{\alpha}}
\newcommand{\opset}[1]{V_{\alpha_{#1}}}
\newcommand{\lub}{\text{lub}}
\newcommand{\lm}{\lambda}
\newcommand{\uin}{\mathbin{\rotatebox[origin=c]{90}{$\in$}}}
\newcommand{\usubset}{\mathbin{\rotatebox[origin=c]{90}{$\subset$}}}
\newcommand{\lt}{\left}
\newcommand{\rt}{\right}
\newcommand{\bs}[1]{\boldsymbol{#1}}
\newcommand{\exs}{\exists}
\newcommand{\st}{\strut}
\newcommand{\dps}[1]{\displaystyle{#1}}

\newcommand{\sol}{\textbf{\textit{Solution:}} }
\newcommand{\solve}[1]{\textbf{\textit{Solution: }} #1 \qed}
% \newcommand{\proof}{\underline{\textit{proof:}}\\}

\DeclareMathOperator{\sech}{sech}
\DeclareMathOperator{\csch}{csch}
\DeclareMathOperator{\arcsec}{arcsec}
\DeclareMathOperator{\arccsc}{arccsc}
\DeclareMathOperator{\arccot}{arccot}
\DeclareMathOperator{\arsinh}{arsinh}
\DeclareMathOperator{\arcosh}{arcosh}
\DeclareMathOperator{\artanh}{artanh}
\DeclareMathOperator{\arcsch}{arcsch}
\DeclareMathOperator{\arsech}{arsech}
\DeclareMathOperator{\arcoth}{arcoth}

\newcommand{\sinx}{\sin x}          \newcommand{\arcsinx}{\arcsin x}    
\newcommand{\cosx}{\cos x}          \newcommand{\arccosx}{\arccosx}
\newcommand{\tanx}{\tan x}          \newcommand{\arctanx}{\arctan x}
\newcommand{\cscx}{\csc x}          \newcommand{\arccscx}{\arccsc x}
\newcommand{\secx}{\sec x}          \newcommand{\arcsecx}{\arcsec x}
\newcommand{\cotx}{\cot x}          \newcommand{\arccotx}{\arccot x}
\newcommand{\sinhx}{\sinh x}          \newcommand{\arsinhx}{\arsinh x}
\newcommand{\coshx}{\cosh x}          \newcommand{\arcoshx}{\arcosh x}
\newcommand{\tanhx}{\tanh x}          \newcommand{\artanhx}{\artanh x}
\newcommand{\cschx}{\csch x}          \newcommand{\arcschx}{\arcsch x}
\newcommand{\sechx}{\sech x}          \newcommand{\arsechx}{\arsech x}
\newcommand{\cothx}{\coth x}          \newcommand{\arcothx}{\arcoth x}
\newcommand{\lnx}{\ln x}
\newcommand{\expx}{\exp x}

\newcommand{\Theom}{\textbf{Theorem. }}
\newcommand{\Lemma}{\textbf{Lemma. }}
\newcommand{\Corol}{\textbf{Corollary. }}
\newcommand{\Remar}{\textit{Remark. }}
\newcommand{\Defin}[1]{\textbf{Definition} (#1).}
\newcommand{\Claim}{\textbf{Claim. }}
\newcommand{\Propo}{\textbf{Proposition. }}

\newcommand{\lb}{\left(}
\newcommand{\rb}{\right)}
\newcommand{\lbr}{\left\lbrace}
\newcommand{\rbr}{\right\rbrace}
\newcommand{\lsb}{\left[}
\newcommand{\rsb}{\right]}
\newcommand{\bracks}[1]{\lb #1 \rb}
\newcommand{\braces}[1]{\lbr #1 \rbr}
\newcommand{\suchthat}{\medspace\middle|\medspace}
\newcommand{\sqbracks}[1]{\lsb #1 \rsb}
\renewcommand{\abs}[1]{\left| #1 \right|}
\newcommand{\Mag}[1]{\left|\left| #1 \right|\right|}
\renewcommand{\floor}[1]{\left\lfloor #1 \right\rfloor}
\renewcommand{\ceil}[1]{\left\lceil #1 \right\rceil}

\newcommand{\cd}{\cdot}
\newcommand{\tf}{\therefore}
\newcommand{\Let}{\text{Let }}
\newcommand{\Given}{\text{Given }}
% \newcommand{\and}{\text{and }}
\newcommand{\Substitute}{\text{Substitute }}
\newcommand{\Suppose}{\text{Suppose }}
\newcommand{\WeSee}{\text{We see }}
\newcommand{\So}{\text{So }}
\newcommand{\Then}{\text{Then }}
\newcommand{\Choose}{\text{Choose }}
\newcommand{\Take}{\text{Take }}
\newcommand{\false}{\text{False}}
\newcommand{\true}{\text{True}}

\newcommand{\QED}{\hfill \qed}
\newcommand{\CONTRA}{\hfill \contra}

\newcommand{\ihat}{\hat{\imath}}
\newcommand{\jhat}{\hat{\jmath}}
\newcommand{\khat}{\hat{k}}

\newcommand{\grad}{\nabla}
\newcommand{\D}{\Delta}
\renewcommand{\d}{\mathrm{d}}

\renewcommand{\dd}[1]{\frac{\d}{\d #1}}
\newcommand{\dyd}[2][y]{\frac{\d #1}{\d #2}}

\newcommand{\ddx}{\dd{x}}       \newcommand{\ddxsq}{\dyd[^2]{x^2}}
\newcommand{\ddy}{\dd{y}}       \newcommand{\ddysq}{\dyd[^2]{y^2}}
\newcommand{\ddu}{\dd{u}}       \newcommand{\ddusq}{\dyd[^2]{u^2}}
\newcommand{\ddv}{\dd{v}}       \newcommand{\ddvsq}{\dyd[^2]{v^2}}

\newcommand{\dydx}{\dyd{x}}     \newcommand{\dydxsq}{\dyd[^2y]{x^2}}
\newcommand{\dfdx}{\dyd[f]{x}}  \newcommand{\dfdxsq}{\dyd[^2f]{x^2}}
\newcommand{\dudx}{\dyd[u]{x}}  \newcommand{\dudxsq}{\dyd[^2u]{x^2}}
\newcommand{\dvdx}{\dyd[v]{x}}  \newcommand{\dvdxsq}{\dyd[^2v]{x^2}}

\newcommand{\del}[2]{\frac{\partial #1}{\partial #2}}
\newcommand{\Del}[3]{\frac{\partial^{#1} #2}{\partial #3^{#1}}}
\newcommand{\deld}[2]{\dfrac{\partial #1}{\partial #2}}
\newcommand{\Deld}[3]{\dfrac{\partial^{#1} #2}{\partial #3^{#1}}}

\newcommand{\argument}[2]{
  \begin{array}{rll}
    #1
    \cline{2-2}
    \therefore & #2 
  \end{array}
}
% Mathfrak primes
\newcommand{\km}{\mathfrak m}
\newcommand{\kp}{\mathfrak p}
\newcommand{\kq}{\mathfrak q}

%---------------------------------------
% Blackboard Math Fonts :-
%---------------------------------------
\newcommand{\bba}{\mathbb{A}}   \newcommand{\bbn}{\mathbb{N}}
\newcommand{\bbb}{\mathbb{B}}   \newcommand{\bbo}{\mathbb{O}}
\newcommand{\bbc}{\mathbb{C}}   \newcommand{\bbp}{\mathbb{P}}
\newcommand{\bbd}{\mathbb{D}}   \newcommand{\bbq}{\mathbb{Q}}
\newcommand{\bbe}{\mathbb{E}}   \newcommand{\bbr}{\mathbb{R}}
\newcommand{\bbf}{\mathbb{F}}   \newcommand{\bbs}{\mathbb{S}}
\newcommand{\bbg}{\mathbb{G}}   \newcommand{\bbt}{\mathbb{T}}
\newcommand{\bbh}{\mathbb{H}}   \newcommand{\bbu}{\mathbb{U}}
\newcommand{\bbi}{\mathbb{I}}   \newcommand{\bbv}{\mathbb{V}}
\newcommand{\bbj}{\mathbb{J}}   \newcommand{\bbw}{\mathbb{W}}
\newcommand{\bbk}{\mathbb{K}}   \newcommand{\bbx}{\mathbb{X}}
\newcommand{\bbl}{\mathbb{L}}   \newcommand{\bby}{\mathbb{Y}}
\newcommand{\bbm}{\mathbb{M}}   \newcommand{\bbz}{\mathbb{Z}}

%---------------------------------------
% Roman Math Fonts :-
%---------------------------------------
\newcommand{\rma}{\mathrm{A}}   \newcommand{\rmn}{\mathrm{N}}
\newcommand{\rmb}{\mathrm{B}}   \newcommand{\rmo}{\mathrm{O}}
\newcommand{\rmc}{\mathrm{C}}   \newcommand{\rmp}{\mathrm{P}}
\newcommand{\rmd}{\mathrm{D}}   \newcommand{\rmq}{\mathrm{Q}}
\newcommand{\rme}{\mathrm{E}}   \newcommand{\rmr}{\mathrm{R}}
\newcommand{\rmf}{\mathrm{F}}   \newcommand{\rms}{\mathrm{S}}
\newcommand{\rmg}{\mathrm{G}}   \newcommand{\rmt}{\mathrm{T}}
\newcommand{\rmh}{\mathrm{H}}   \newcommand{\rmu}{\mathrm{U}}
\newcommand{\rmi}{\mathrm{I}}   \newcommand{\rmv}{\mathrm{V}}
\newcommand{\rmj}{\mathrm{J}}   \newcommand{\rmw}{\mathrm{W}}
\newcommand{\rmk}{\mathrm{K}}   \newcommand{\rmx}{\mathrm{X}}
\newcommand{\rml}{\mathrm{L}}   \newcommand{\rmy}{\mathrm{Y}}
\newcommand{\rmm}{\mathrm{M}}   \newcommand{\rmz}{\mathrm{Z}}

%---------------------------------------
% Calligraphic Math Fonts :-
%---------------------------------------
\newcommand{\cla}{\mathcal{A}}   \newcommand{\cln}{\mathcal{N}}
\newcommand{\clb}{\mathcal{B}}   \newcommand{\clo}{\mathcal{O}}
\newcommand{\clc}{\mathcal{C}}   \newcommand{\clp}{\mathcal{P}}
\newcommand{\cld}{\mathcal{D}}   \newcommand{\clq}{\mathcal{Q}}
\newcommand{\cle}{\mathcal{E}}   \newcommand{\clr}{\mathcal{R}}
\newcommand{\clf}{\mathcal{F}}   \newcommand{\cls}{\mathcal{S}}
\newcommand{\clg}{\mathcal{G}}   \newcommand{\clt}{\mathcal{T}}
\newcommand{\clh}{\mathcal{H}}   \newcommand{\clu}{\mathcal{U}}
\newcommand{\cli}{\mathcal{I}}   \newcommand{\clv}{\mathcal{V}}
\newcommand{\clj}{\mathcal{J}}   \newcommand{\clw}{\mathcal{W}}
\newcommand{\clk}{\mathcal{K}}   \newcommand{\clx}{\mathcal{X}}
\newcommand{\cll}{\mathcal{L}}   \newcommand{\cly}{\mathcal{Y}}
\newcommand{\calm}{\mathcal{M}}  \newcommand{\clz}{\mathcal{Z}}

%---------------------------------------
% Fraktur  Math Fonts :-
%---------------------------------------
\newcommand{\fka}{\mathfrak{A}}   \newcommand{\fkn}{\mathfrak{N}}
\newcommand{\fkb}{\mathfrak{B}}   \newcommand{\fko}{\mathfrak{O}}
\newcommand{\fkc}{\mathfrak{C}}   \newcommand{\fkp}{\mathfrak{P}}
\newcommand{\fkd}{\mathfrak{D}}   \newcommand{\fkq}{\mathfrak{Q}}
\newcommand{\fke}{\mathfrak{E}}   \newcommand{\fkr}{\mathfrak{R}}
\newcommand{\fkf}{\mathfrak{F}}   \newcommand{\fks}{\mathfrak{S}}
\newcommand{\fkg}{\mathfrak{G}}   \newcommand{\fkt}{\mathfrak{T}}
\newcommand{\fkh}{\mathfrak{H}}   \newcommand{\fku}{\mathfrak{U}}
\newcommand{\fki}{\mathfrak{I}}   \newcommand{\fkv}{\mathfrak{V}}
\newcommand{\fkj}{\mathfrak{J}}   \newcommand{\fkw}{\mathfrak{W}}
\newcommand{\fkk}{\mathfrak{K}}   \newcommand{\fkx}{\mathfrak{X}}
\newcommand{\fkl}{\mathfrak{L}}   \newcommand{\fky}{\mathfrak{Y}}
\newcommand{\fkm}{\mathfrak{M}}   \newcommand{\fkz}{\mathfrak{Z}}


\title{\Huge{MATH1061}\\Discrete Mathematics I}
\author{\huge{Michael Kasumagic, s4430266}}
\date{\huge{Semester 2, 2024}}

\begin{document}

\maketitle
\newpage% or \cleardoublepage
% \pdfbookmark[<level>]{<title>}{<dest>}
\pdfbookmark[section]{\contentsname}{toc}
\tableofcontents
\pagebreak

\chapter{Week 1}
\section{Lecture 1}
This course will run a little differently. Prior to every lecture, we must work through a set of pre-lecture problems. The goal of timetabled lectures is to discuss and learn from solving problems.

\subsection*{What is in this course?}
\subsubsection*{Logic and set theory, methods of proof}
Modern mathematics uses the language of set theory and the notation of logic.
$$
((P \land \lnot Q) \lor (P\land Q)) \land Q \equiv P \land Q
$$
We will learn to read and analyse this. Historical, there was a big shift in recent history, there was a big effort to define and axiom-itise everything, such that math itself is defined rigorously. Symbolic logic is the basis for many areas of computer science. It helps us formulate mathematical ideas and proofs effectively and cobbrectly! \\

\dfn{G\"odel's Incompleteness Theorem (1931)}{There exists true statements which we can not prove!}

\subsubsection*{Number Theory}
\ex{$1 + \cdots + 100$}{
	A young Gauss had to add up all the numbers from 1 to 100 in primary school. What did he do?
	\begin{gather*}
		\begin{array}{ccccccccccccc}
			& & 1 & + & 2 & + & \cdots & + & 98 & + & 99 & + & 100 \\	
			100 & + & 99 & + & 98 & + & \cdots & + & 2 & + & 1 & & \\	\hline
			100 & + & 100 & + & 100 & + & \cdots & + & 100 & + & 100 & + & 100   
		\end{array} \\
			\intertext{So...} \\
		1 + \cdots + 100 = \frac{101\cd100}{2} = 5050
	\end{gather*}
}
This generalises to $\forall n \in \bbn$. Two leaps of faith are needed though!
\begin{itemize}
	\item The dots: We introduce the notation to deal with them.
	\item The equality of two equations invoving dots. We will use induction to deal with this!
\end{itemize}

\subsubsection*{Graph Theory}
\ex{The K\"onigsberg Bridge Problem}{Find a route through the city which crosses each of seven bridges exactly once, and returns you to your start location.}
This is provably impossible! But how can we rigorously prove this? Euler solve this problem in 1935 and in doing so invented graph theory. We'll learn how eventually... :p

\subsubsection*{Counting and Probability}
Both fundamental and beautifully applicable. We introduce the pigeonhole principle as a introduction to ``counting.''
\ex{The pigeonhole principle}{If you have $n$ pigeons sitting in $k$ pigeonholes, if $n>k$, then at least of the pigeonholes contains at least 2 pigeons.}

\qs{}{If you have socks of three different colours in your drawer, what is the minimum number of socks you need to pull out to guarantee a matching pair?}
\sol \#socks $\equiv$ \#pigeons and \#colours $\equiv$ \#holes. If \#socks $>$ \#colors, a double must occur. Therefore, we need a minimum of 4 socks to guarantee a match.

\qs{True or False?}{In every group of five people, there are two people who have the same number of friends within the group.}
\sol True! \#people $\equiv$ \#pigeons and \#friends $\equiv$ \#holes. There are 5 possible values for the amount of friends one could have, $\braces{0, 1, 2, 3, 4}$, but you can never have an individual with 0 friends, and 4 friends in the same group. So there are 5 people, and 4 possible \#friend values (think ``holes.'') Therefore, by pigeonhole principle, the statement is true!

\qs{True or False?}{A plane is coloured blue and red. Is it possible to find exactly two points the same colour exactly one unit apart?}
\nt{We will answer this on Wednesday!}

\subsubsection*{Recursion}
\ex{The Tower of Hanoi}{Given: a tower of 8 discs in decreasing size on one of three pegs. \\Problem: transfer the entire towert to one of the other pegs. \\Rule 1: Move only one disc at a time. \\Rule 2: Never move a larger disc onto a smaller disk.}
\begin{enumerate}
	\item Is there a solution?
	\item What's the minimal number of moves necessary and sufficient for the task?
\end{enumerate}
A key idea is to generalise! What if there are $n$ discs? Let $T_n$ be the minimal number of moves, then trivially $T_0 = 0$, $T_1 = 1$, $T_2 = 3$, so what is $T_3 = ?$. Is there a pattern? The winning strategy is
\begin{enumerate}
	\item Move the $n-1$ smallest discs from peg $A$ to $B$.
	\item Move the big disc from $A$ to $C$.
	\item Move $n-!$ smallest discs from $B$ to $C$
\end{enumerate}
By induction we show that
$$
	T_n = 2T_{n-1} + 1.
$$
So $T_3 = 7$, $T_4 = 15$, $T_5 = 31$, $T_6 = 63$. Remarkably, this is one less then the sqaure numbers! We will prove this fact by induction later in the course.
\nt{On Wednesday we start proberly.\\ Read: \vspace{-10pt}\begin{center}Pages 23-36 (Epp, 4th) or \\ Pages 37-50 (Epp, 5th).\end{center}
Watch the first video on UQ Extend, try the first quiz before Wednesday's lecture!}

\section{Lecture 2}
\dfn{Statement or Proposition}{A sentence that is either true or false but not both.}
\ex{}{
	Statements:
	\begin{itemize}
		\item The number 6 is a number.
		\item $\pi>3$
		\item Euler was born in 1707.
	\end{itemize}
	Not statements:
	\begin{itemize}
		\item How are you? (This is a question.)
		\item Stop! (This is a command.)
		\item She likes math. (``She'' is not well defined.)
		\item $x^2 = 2x - 1$ ($x$ is not well defined.)
	\end{itemize}
}

\dfn{Negation}{
	Let $p$ be a statement. The negation is of $p$ is denoted $\lnot p$ or $\varlnot p$ and is read ``not $p$.'' It is defined as in the following truth table: 
	\begin{center}\begin{tabular}{|c||c|}
		\hline
		$p$ & $\lnot p$ \\ \hline
		T & F \\
		F & T \\ \hline
	\end{tabular}\end{center}
}

\dfn{Conjuction}{
	Let $p$ and $q$ be statements. The conjuction of $p$ and $q$ is denoted $p \land q$ and is read ``$p$ and $q$.'' It is defined as in the following truth table: 
	\begin{center}\begin{tabular}{|cc||c|}
		\hline
		$p$ & $q$ & $p\land q$ \\ \hline
		T & T	& T	\\
		T & F	& F	\\ 
		F	& T	& F	\\
		F & F	& F	\\\hline
	\end{tabular}\end{center}
}

\dfn{Disjunction}{
	Let $p$ and $q$ be statements. The disjunction of $p$ and $q$ is denoted $p \lor q$ and is read ``$p$ or $q$.'' It is defined as in the following truth table: 
	\begin{center}\begin{tabular}{|cc||c|}
		\hline
		$p$ & $q$ & $p\lor q$ \\ \hline
		T & T	& T	\\
		T & F	& T	\\ 
		F	& T	& T	\\
		F & F	& F	\\\hline
	\end{tabular}\end{center}
}

\dfn{Logical Equivalence}{
	Two statements, $p$ and $q$ are said to be logically equivalent if have identical truth values for every possible combination of truth values for their statement variables. This is denoted $p\equiv q$.
}
\ex{}{
	$$\lnot(\lnot p)\equiv p.$$
	\begin{center}
		\begin{tabular}{|cc||c|}
			\hline
			$p$ & $\lnot p$ & $\lnot(\lnot p)$ \\ \hline
			T & F & T \\
			F & T & F \\ \hline
		\end{tabular}
	\end{center}
	Consider $P = \lnot(p\land q)$, $Q = \lnot p \land \lnot q$ and $R = \lnot p\lor \lnot q$.
	\begin{center}
		\begin{tabular}{|cc||cc||ccc|}
			\hline
			$p$	& $q$	& $\lnot p$	& $\lnot q$ & $P$ & $Q$ & $R$ \\ \hline
			T	& T	& F	& F	& F	& F	& F	\\
			T	& F	& F	& T	& T	& F	& T	\\
			F	& T	& T	& F	& T	& F	& T	\\
			F	& F	& T	& T	& T	& T	& T	\\ \hline
		\end{tabular}
	\end{center}
	$\tf P\equiv R\not\equiv Q.$
}

\dfn{Contradictions and Tautologies}{
	A contradiction has truth values of false for every possible combination of its statement's truth values, and is denoted $c$ or $\bot$. A tautology has truth values of true for every possible combination of its statement's truth values, and is denoted $t$ or $\top$.
}
\ex{}{
	\begin{multicols}{2}
		\begin{center}
			\begin{tabular}{|cc||c|}
				\hline
				$p$ & $\lnot p$ & $p \land \lnot p$ \\ \hline
				T & F & F \\
				F & T & F \\ \hline
			\end{tabular}
		\end{center}
		$$\tf p \land \lnot p \equiv \top $$
		\begin{center}
			\begin{tabular}{|cc||c|}
				\hline
				$p$ & $\lnot p$ & $p\lor \lnot p$ \\ \hline
				T & F & T \\
				F & T & T \\ \hline
			\end{tabular}
		\end{center}
		$$\tf p \lor \lnot p \equiv \bot $$
	\end{multicols}
}

\subsection*{Important Laws of Logical Equivalence!}
\begin{multicols}{2}
	\subsubsection*{De Morgan's Law}
	\begin{align*}
		\lnot(p\land q) \equiv \lnot p \lor \lnot q \\
		\lnot(p\lor q) \equiv \lnot p \land \lnot q 
	\end{align*}
	
	\subsubsection*{Commutativity}
	\begin{align*}
		p \land q \equiv q\land p \\
		p \lor q \equiv q \lor p
	\end{align*}
	
	\subsubsection*{Associativity}
	\begin{align*}
		p \land (q\land r) \equiv (p\land q) \land r \\
		p \lor (q\lor r) \equiv (p\lor q) \lor r	
	\end{align*}
	
	\subsubsection*{Distributivity}
	\begin{align*}
		p \land (q \lor r) \equiv (p\land q) \lor (p \land r) \\
		p \lor (q \land r) \equiv (p\lor q) \land (p \lor r)
	\end{align*}
	
	\subsubsection*{Double Negative}
	$$
		\lnot(\lnot p) \equiv p
	$$
	
	\subsubsection*{Idempotent}
	\begin{align*}
		p \land p \equiv p \\
		p \lor p \equiv p 
	\end{align*}
	
	\subsubsection*{Absorbtion}
	\begin{align*}
		p \lor (p \land q) \equiv p \\
		p \land (p \lor q) \equiv p 
	\end{align*}
	
	\subsubsection*{Identity Laws}
	\begin{align*}
		p \land \top \equiv p \\
		p \lor \bot \equiv p	
	\end{align*}
	
	\subsubsection*{Domination}
	\begin{align*}
		p \lor \top \equiv \top \\
		p \land \bot \equiv \bot
	\end{align*}
	
	\subsubsection*{Negation Laws}
	\begin{align*}
		p \lor \lnot p \equiv \top \\
		p \land \lnot p \equiv \bot
	\end{align*}
	
	\subsubsection*{Negations}
	\begin{align*}
		\lnot\top\equiv\bot \\
		\lnot\bot\equiv\top
	\end{align*}
\end{multicols}

\ex{}{
	Prove that $((p\land\lnot q)\lor(p\land q))\land q \equiv p\land q.$
	\begin{align*}
		((p\land\lnot q)\lor(p\land q))\land q &\equiv (p\land(\lnot q\lor q))\land q \tag*{\text{(Distributivity)}} \\
			&\equiv (p\land\top)\land q \tag*{\text{(Negation Law)}} \\
			&\equiv p\land q \tag*{\text{(Identity)}} \\
			\tag*{\qed}
	\end{align*}
}

\subsection*{Questions}

\qs{}{
	Which of the following are statements?
	\begin{enumerate}[label=(\alph*)]
		\item "Is it going to rain tomobbrow?"
		\item "She is happy."
		\item "23 July 2024 is a Tuesday"
		\item $x = 5y + 2$
		\item $65 < 2$
	\end{enumerate}
}
\sol (a) No, a question. (b) No, "she" undefined. (c) Yes. (d) No, $x$, $y$ undefined. (e) Yes.

\qs{}{
	Let $p$, $q$, $r$ be statements.
	\begin{itemize}
		\item $p$ = ``it is cold.''
		\item $q$ = ``it is snowing.''
		\item $r$ = ``it is sunny.''
	\end{itemize}
	Translate these to symbols:
	\begin{enumerate}[label=(\alph*)]
		\item ``It is not cold but it is snowing.''
		\item ``It is neither snowing nor cold, but it is sunny.''
	\end{enumerate}
	Translate these to English:
	\begin{enumerate}[label=(\alph*),start=3]
		\item $\lnot p\land q$
		\item $(p\land q)\lor r$
	\end{enumerate}
}
\sol (a) $\lnot p\land q$ (b) $\lnot p \land \lnot q \land r$ (c) ``It is not cold but it is snowing'' (d) ``It is either snowing and cold, or sunny, or it's both.''

\qs{}{Construct the truth table for $(p\land \lnot q)\lor (q\land r)$}
\sol \begin{center}
	\begin{tabular}{|ccc||c||cc||c|}
		\hline
		$p$ & $q$ & $r$ & $\lnot q$ & $p\land\lnot q$ & $q\land r$ & $(p\land\lnot q)\lor(q\land r)$ \\ \hline
		T & T & T & F & F & T & T \\
		T & T & F & F & F & F & F \\
		T & F & T & T & T & F & T \\
		T & F & F & T & T & F & T \\
		F & T & T & F & F & T & T \\
		F & T & F & F & F & F & F \\
		F & F & T & T & F & F & F \\
		F & F & F & T & F & F & F \\ \hline
	\end{tabular}
\end{center}

\qs{}{Using De Morgan's Law, write down a statement which is logically equivalent to the negation of ``5 is even and 6 is even.''}
\sol ``5 is even and 6 is even.'' $\equiv p \land q$. The solution we want is the negation, $\lnot(p\land q)$, which, by De Morgan's Law is the same as $\lnot p \lor \lnot q$ which in English is ``5 is odd or 6 is odd.''

\qs{}{
	Show that
	$$\lnot((\lnot p\land q) \lor (\lnot p \land \lnot q)) \equiv p$$
	using a truth table, and by laws of logical equivalence.
}
\begin{center}
	\begin{tabular}{|cc||cc||cc||c||c|}
		\hline
		$p$ & $q$ & $\lnot p$ & $\lnot q$ & $\lnot p\land q$ & $\lnot p\land\lnot q$ & $(\lnot p\land q) \lor (\lnot p\land\lnot q)$ & $\lnot((\lnot p\land q) \lor (\lnot p\land\lnot q))$ \\ \hline
		T & T & F & F & F & F & F & T \\
		T & F & F & T & F & F & F & T \\
		F & T & T & F & T & F & T & F \\
		F & F & T & T & F & T & T & F \\ \hline
	\end{tabular}
\end{center}
$\tf\lnot((\lnot p\land q) \lor (\lnot p \land \lnot q)) \equiv p$ by exhaustion.
\begin{align*}
	\lnot((\lnot p\land q) \lor (\lnot p \land \lnot q)) &\equiv \lnot(\lnot p\land q) \land \lnot(\lnot p \land \lnot q) \tag*{(\text{De Morgan's Law})}\\
		&\equiv (p\lor \lnot q) \land (p \lor q) \tag*{(\text{De Morgan's Law})} \\
		&\equiv p \lor (q \land\lnot q) \tag*{(Distributivity)} \\
		&\equiv p \lor \top \tag*{(Negation Law)} \\
		&\equiv p \tag*{(Identity)} \\
		\tag*{\qed}
\end{align*}
$\tf\lnot((\lnot p\land q) \lor (\lnot p \land \lnot q)) \equiv p$ by logical equivalence.

\section{Lecture 3}
\dfn{Conditional Statement}{
	Let $p$ and $q$ be statement variables. The conditional form $p$ to $q$ is denoted $p\lthen q$, and read as ``if $p$, then $q$,'' or ``$p$ implies $q$.'' It is defined by the following truth table
	\begin{center}
		\begin{tabular}{|cc||c|}
			\hline
			$p$	&$q$	&$p\lthen q$ \\ \hline
			T & T & T \\
			T & F & F \\
			F & T & T \\
			F & F & T \\ \hline
		\end{tabular}
	\end{center}
	$p$ is called the hypothesis.\\
	$q$ is called the conclusion.
}
\ex{}{
	Suppose I make you the following promise:
	\begin{center}
		``If you do your homework then you get a chocolate.''
	\end{center}
	\begin{enumerate}[label=(\alph*)]
		\item You do not do your homework and you get a chocolate.
		\item You do your homework and you get a chocolate.
		\item You do your homework and you do not get a chocolate.
		\item You do not do your homework and you do not get a chocolate.
	\end{enumerate}
	I only lied in scenario (c), which corresponds with $(p,q)=(F,T)$.
}

\nt{
	$$ p\lthen q \equiv \lnot p \lor q $$
	\begin{center}
		\begin{tabular}{|cc||c||cc|}
			\hline
			$p$ &$q$ &$\lnot p$ &$\lnot p \lor q$ &$p\lthen q$ \\ \hline
			T & T & F & T & T \\ 
			T & F & F & F & F \\
			F & T & T & T & T \\
			F & F & T & T & F \\ \hline 
		\end{tabular}
	\end{center}
}

\dfn{Contrapositive}{
	The contrapositive of $p\lthen q$ is $\lnot q\lthen\lnot p$.
	\begin{center}
		\begin{tabular}{|cc||cc||cc|}
			\hline
			$p$ & $q$ & $\lnot p$ & $\lnot q$ & $p\lthen q$ & $\lnot q\lthen \lnot p$ \\ \hline
			T & T & F & F & T & T \\
			T & F & F & T & F & F \\
			F & T & T & F & T & T \\
			F & F & T & T & T & T \\ \hline
		\end{tabular}
	\end{center}
	$$
		p\lthen q \equiv \lnot q \lthen \lnot p
	$$
}
\ex{}{
	The contrapositive of 
	\begin{center}
		``If you do your homework then you get a chocolate.''
	\end{center}
	Is the equivalent
	\begin{center}
		``If you did not get a chocolate then you did not finish your homework.''
	\end{center}
}

\subsection*{Negation of the Conditional Statement}
The negation of $p\lthen q$ is given by $p\land \lnot q$ and can be proved logcically.
\begin{align*}
	p\lthen p &\equiv \lnot p \lor q \\
	\lnot(p\lthen p) &\equiv \lnot (\lnot p \lor q) \\
		&\equiv \lnot (\lnot p) \land \lnot q \\
		&\equiv p \land \lnot q 
\end{align*}

\ex{}{
	The negation of
	\begin{center}
		``If today is Monday, then tomorrow is my birthday''
	\end{center}
	Is
	\begin{center}
		``Today is Monday but tomorrow is not my birthday.''
	\end{center}
}

\dfn{Biconditional Statement}{
	Let $p$ and $q$ be statement variables. The biconditional statement of $p$ and $q$, denoted $p\liff q$, and read ``$p$ if and only if $q$'' is defined by the following truth table
	\begin{center}
		\begin{tabular}{|cc||c|}
			\hline
			$p$	& $q$	& $p\liff q$ \\ \hline
			T & T & T \\
			T & F & F \\
			F & T & F \\
			T & T & T \\ \hline
		\end{tabular}
	\end{center}
	$$p \liff q \equiv (p\lthen q) \land (q\lthen p)$$
}

\subsection*{Questions}
\qs{}{
	Which of the following sentences have the same meaning as ``If I am worried then I did not sleep''?
	\begin{enumerate}[label=(\alph*)]
		\item if I am worried then I do not sleep.
		\item if I am not worried then I do sleep.
		\item If I do not sleep then I am worried.
		\item I am worried and I do sleep.
		\item If I do sleep then I am not worried.
		\item I am worried or I do not sleep
		\item I do not sleep or I am not worried.
	\end{enumerate}
}
\sol 
\begin{itemize}
	\item Original: $p\lthen \lnot q$
	\item (a): $p \lthen \lnot q$, equivalent.
	\item (b): $\lnot p \lthen q$, not equivalent.
	\item (c): $\lnot q \lthen p$, not equivalent.
	\item (d): $p\land q$, not equivalent.
	\item (e): $q \lthen\lnot p$, equivalent, the contrapositive.
	\item (f): $p \lor \lnot q$, not equivalent.
	\item (g): $\lnot q \lor \lnot p$, equivalent, logically equivalent.
\end{itemize} 

\qs{}{
	Express the operations $\lor$, $\lthen$, and $\liff$ using only $\lnot$ and $\land$.
}
\sol
\begin{align*}
	p \lor q &\equiv \lnot(\lnot (p \lor q)) \\
		&\equiv \lnot (\lnot p \land \lnot q) \\
	p \lthen q &\equiv \lnot p\lor q \\
		&\equiv \lnot (\lnot (\lnot p\lor q))\\
		&\equiv \lnot (p\land \lnot q)\\
	p \liff q &\equiv (p\lthen q) \land (q\lthen p) \\
		&\equiv (\lnot p\lor q) \land (\lnot q\lor p) \\
		&\equiv \lnot(\lnot(\lnot p\lor q)) \land \lnot(\lnot(\lnot q\lor p)) \\
		&\equiv \lnot(p\land \lnot q) \land \lnot(q\land \lnot p)
\end{align*}

\qs{Challenge}{
	Consider the NAND operation $p\lnand q \equiv \lnot(p\land q)$ can you express $\land$, $\lor$, $\lnot$, and $\lthen$ using only $\lnand$ operations? Can you express using only $\lnot$ and $\lxor$?
}
\sol Generating expressions using only NANDs:
\begin{align*}
	\lnot p &\equiv \lnot(p \land p) \\
		&\equiv p\lnand p \\
	p\land q &\equiv (p\land q)\land(p\land q) \\
		&\equiv \lnot (\lnot(p\land q)\land\lnot(p\land q)) \\
		&\equiv \lnot((p\lnand q)\land (p\lnand q)) \\
		&\equiv (p\lnand q)\lnand (p\lnand q) \\
	p\lor q &\equiv \lnot(\lnot(p\lor q)) \\
		&\equiv \lnot(\lnot p\land \lnot q) \\
		&\equiv \lnot p\lnand\lnot q \\
		&\equiv (p\lnand p)\lnand(q\lnand q) \\
	p \lthen q &\equiv \lnot p\lor q \\
		&\equiv \lnot (\lnot (\lnot p\lor q)) \\
		&\equiv \lnot (p\land \lnot q) \\
		&\equiv p\lnand\lnot q \\
		&\equiv p\lnand(q\lnand q)
\end{align*}
These sick fucks had me testing and observing truth tables for two hours. I was suspicious at times, but I assumed there must be a solution\dots\\
There is not. No matter how many XOR and NOT operations you apply, ultimately, you will always have 2 Falses and 2 Trues, or 4 Falses, or 4 Trues. \textbf{AHHHHHHHH}.

\qs{}{
	Show that $\lnot(p\lthen q) \not\equiv \lnot p\lthen \lnot q$.
}
\sol By counterexample. Suppose $p=\true\And q=\true$
\begin{align*}
\lnot(p\lthen q) &\equiv \lnot(\true\lthen\true) \\
	&\equiv \lnot(\true) \\
	&\equiv \false \\
\lnot p\lthen\lnot q &\equiv \lnot\true\lthen\lnot\true \\
	&\equiv \false\lthen\false \\
	&\equiv \true \not\equiv\lnot(p\lthen q) \tag*{\qed}
\end{align*}

\qs{}{
	Which of the following sentences have the opposite truth valuse as ``If I am worried then I did not sleep''?
	\begin{enumerate}[label=(\alph*)]
		\item if I am worried then I do not sleep.
		\item if I am not worried then I do sleep.
		\item If I do not sleep then I am worried.
		\item I am worried and I do sleep.
		\item If I do sleep then I am not worried.
		\item I am worried or I do not sleep
		\item I do not sleep or I am not worried.
	\end{enumerate}
}
\sol 
\begin{itemize}
	\item Original: $p\lthen \lnot q$
	\item (a): $p \lthen \lnot q$, No, equivalent statement.
	\item (b): $\lnot p \lthen q$, No, True 3/4 times, same as our statement. Not possible to be opposite.
	\item (c): $\lnot q \lthen p$, No, True 3/4 times.
	\item (d): $p\land q$, Yes! $p\lthen\lnot q \equiv \lnot p\lor \lnot q\equiv \lnot(p\land q)$, exactly the opposite.
	\item (e): $q \lthen\lnot p$, No, equivalent statement.
	\item (f): $p \lor \lnot q$, No, True 3/4 times.
	\item (g): $\lnot q \lor \lnot p$, No, equivalent statement.
\end{itemize} 

\qs{}{
	Show that $$p\lthen(q\lor r)\equiv (p\land\lnot q)\lthen r.$$
}
\sol
\begin{align*}
	p\lthen (q\lor r) &\equiv \lnot p \lor (q\lor r) \\
		&\equiv (\lnot p \lor q)\lor r \\
		&\equiv \lnot(\lnot(\lnot p \lor q))\lor r \\
		&\equiv \lnot(p \land\lnot q)\lor r \\
		&\equiv (p\land q)\lthen r
\end{align*}

\qs{}{
	Let $n$ be a positive integer. Find conditions that are:
	\begin{enumerate}[label=(\alph*)]
		\item necessary, but not sufficent for $n$ to be a multiple of 10.
		\item sufficient but not necessary for $n$ to be divisible by 10.
		\item necessary and sufficient for $n$ to be divisible by 10.
	\end{enumerate}
}
\sol\begin{enumerate}[label=(\alph*)]
	\item $n$ is a multiple of 10 $\lthen$ $n$ is necessarily even.
	\item $n$ is a multiple of 50 $\lthen$ sufficient to conclude that $n$ is divisible by 10.
	\item $n$'s last digit is a 0 $\liff$ $n$ is divisible by 10.
\end{enumerate}

\chapter{Week 2}
\section{Lecture 4}
\dfn{Argument}{
	Given a collection of statements, $p_1, p_2, \dots, p_n$ (called premises), and another statement $q$ (called the conclusion), an arugment is the assertion that the conjuction of the premises implies the conclusion. This is often represented
	$$
		\argument{
			p_1 & \dots & \\
			p_2 & \dots & \\
			\multicolumn{1}{c}{\vdots} & & \\
			p_n & \dots & \\
		}{
			q &
		}
	$$
	An argument is \textbf{valid} if whenever all the premises are true, the conclusion is true. Mathematically,
	$$
		\bigwedge_{i=1}^{n} \bracks{p_i} \lthen q \equiv \top
	$$
	An argument is \textbf{invalid} if there exists a configuration, such that all the premises are true, but the conclusion is false.
	$$
		\bigwedge_{i=1}^{n} \bracks{p_i} \lthen q \not\equiv \top
	$$
}
\ex{}{
	$$
		\argument{
			p_1 & \text{If it is raining, then there are clouds.} & \\
			p_2 & \text{It is raining.} & \\
		}{\text{There are clouds.}&}
	$$
	$$
		\text{Which is an argument of form:}
		\argument{
			& p\lthen q & \\
			& p & \\
		}{q &}
	$$ 
	This is a valid argument! As long as $p_1$ and $p_2$ are true, $q$ is necessarily true.
	$$
		\argument{
			p_1 & \text{If it is raining then there are clouds.} & \\
			p_2 & \text{There are clouds.} & \\
		}{\text{It is raining. }&}
	$$
	$$
		\text{Which is an argument of form:}
		\argument{
			& p\lthen q & \\
			& q & \\
		}{p &}
	$$ 
	This is an invalid argument! Because the conclusion doesn't follow from the premises. For example, if $p_1$ and $p_2$ were true, $q$ still may be false.
}

\subsection*{Rules of Inference }
These are common argument forms.
\subsubsection*{Modus Ponens}
$$
	\argument{
		& p\lthen q & \\
		& p & \\
	}{q&}
$$
$$
	\begin{array}{|cc|cc|c|}
		\hline
		p & q & \text{Premise 1: } p\lthen q & \text{Premise 2: } p & \text{Conclusion: }  q \\ \hline
		T & T & T & T & T \\
		T & F & F & T & F \\
		F & T & T & F & T \\
		F & F & T & F & T \\ 
		\hline
	\end{array}
$$
Pay special attention to row 1. This is the only row in which every premise is true. When every primise is true, the conclusion is always true. Therefore this is a valid argument form. Every argument form can be proven with a truth table in this manner.

\begin{multicols}{2}
	\subsubsection*{Modus Tellens}
	$$
		\argument{&p\lthen q & \\ &\lnot q & \\}{\lnot p &}
	$$
	
	\subsubsection*{Generalisation}
	$$
		\argument{&p & \\}{p\lor q &}
	$$
	$$
		\argument{&q & \\}{q\lor p &}
	$$
	
	\subsubsection*{Specalisation}
	$$
		\argument{&p\land q & \\}{p &}
	$$
	$$
		\argument{&p\land q & \\}{q &}
	$$

	\subsubsection*{Conjuction}
	$$
		\argument{&p & \\ &q & \\}{p\land q &}
	$$
	
	\subsubsection*{Elimination}
	$$
		\argument{&p\lor q & \\&\lnot q & \\}{p &}
	$$
	$$
		\argument{&p\lor q & \\&\lnot p & \\}{q &}
	$$

	\subsubsection*{Transitivity}
	$$
		\argument{&p\lthen q & \\&q\lthen r & \\}{p\lthen r &}
	$$

	\subsubsection*{Proof by Division into Cases}
	$$
		\argument{&p\lor q & \\&p\lthen r & \\&q\lthen r & \\}{r &}
	$$

	\subsubsection*{Contradiction}
	$$
		\argument{&\lnot p\lthen \bot & \\}{p &}
	$$
\end{multicols}

\ex{Valid or invalid}{
	Is the following argument valid?
	$$
		\argument{
			1. & p \lthen \lnot r & \\
			2. & r \lor \lnot q & \\
			3. & q & \\
		}{\lnot p &}
	$$
	We might be tempted to use a truth table, but it'll have an unreaonsable, $\bracks{2^3}$, amount of rows! We can use our rules of inference to figure this out.
	$$
		\argument{
			1. & p \lthen \lnot r 	& \\
			2. & r \lor \lnot q 		& \\
			3. & q 									& \\
			4. & \lnot q \lor r 		& (\text{2. by Commutativity})\\
			5. & p \lthen r 				& (\text{4. by Logical Equivalence})\\
			6. & r 									& (\text{3. and 5. by Modus Ponens})\\
			7. & \lnot(\lnot r) 		& (\text{6. by Double Negative})\\
		}{\lnot p & (\text{1. and 7. by Modus Tellens})}
	$$
	Therefore the argument is valid!
}

\subsection*{Searching for Invalidity}
Another method for checking validity may be to look for truth values which make the premises true, but the conclusion false. If we can find such an example, we can prove that the arugment is invalid. If it is impossible to do this, then the argument is valid.

\ex{}{
	Consider the argument
	$$
		\argument{&p\lthen q &\\ &q &\\}{p &}
	$$
	Since $p$ is the conclusion, take it to be false.\\
	Since $q$ is a premise, take it to be true.\\
	The premise $p\lthen q$ is therefore $\false\lthen\true\equiv\true$.\\
	Therfore all our premises are true.\\
	But wait! Our conclusion was set to false!\\
	Therefore, there and is called the existential quantifier.\\
	
	Let $Q(x)$ be a preda configuration of truth values, namely $(p,q)=(\false,\true)$, such that all the premises are true, but the conclusion is false.\\
	Therefore, the argument is invalid.
}

\ex{}{
	Consider the argument
	$$
		\argument{&p\lthen\lnot r &\\ &r\lor\lnot q &\\ &q &\\}{\lnot p &}
	$$
	Let's suppose the argument is invalid.\\
	Then, our conclusion $\lnot p$, is false.\\
	Then $p$ is true.\\
	Since $q$ is a premise, take it to be true.\\
	Consider the premise $p\lthen\lnot r$. Since $p$ is true, $\lnot r$ must also be true, such that the premise is true.\\
	Then, $r$ is false.\\
	Consider the premise $r\lor\lnot q$. Substituting, we see $\false\lor \false\equiv\false$.\\
	Therefore, it is impossible for us to configure $(p,q,r)$ such that all the premises are true, and the conclusion is false.\\
	Therefore the argument is not invalid.\\
	Therefore the argument is valid.
}

\subsection*{Questions}
\qs{}{
	Write the following arguments symbolically
	\begin{itemize}
		\item If wages are raised, then buying increases.
		\item If there is a depression, then buying does not increase.
		\item Therefore, there is not a depression, or wages are not raised.
	\end{itemize}
	Decide whether this argument is valid, using three methods:
	\begin{enumerate}[label=(\alph*)]
		\item a truth table
		\item rules of inference
		\item configuring truth values
	\end{enumerate}
}
\sol 
\begin{gather*}
	\Let p = \text{``Wages are raised.''} \\
	\Let q = \text{``There is a depression.''} \\
	\Let r = \text{``Buying increases.''} \\
	\argument{
		1. & p\lthen r & \\ 
		2. & q\lthen\lnot r & \\
	}{\lnot p\lor\lnot q &}
\end{gather*}

\sol (a)
\begin{center}
	\begin{tabular}{|ccc||cc||c|}
		\hline
		\multicolumn{3}{|c||}{Variables} & \multicolumn{2}{c||}{Premises} & \multicolumn{1}{c|}{Conclusion} \\
		$p$ & $q$ & $r$ & $p\lthen r$ & $q\lthen\lnot r$ & $\lnot p\lor\lnot q$ \\ \hline
		T & T & T & T & F & F \\
		T & T & F & F & T & F \\ \cline{4-6}
		T & F & T & T & T & T \\ \cline{4-6}
		T & F & F & F & T & T \\
		F & T & T & T & F & T \\ \cline{4-6}
		F & T & F & T & T & T \\ \cline{4-6}
		F & F & T & T & T & T \\ \cline{4-6}
		F & F & F & T & T & T \\ \cline{4-6} \hline
	\end{tabular}
\end{center}
Consider rows 3, 6, 7, and 8. Whenever all the premises are true, the conclusion is true. Therefore the argument is valid. 
\begin{flushright}$\qed$\\\end{flushright}

\sol (b)
\begin{gather*}
	\argument{
		1. & p\lthen r 			 			& \text{} \\
		2. & q\lthen\lnot r 			& \text{} \\
		3. & r \lthen\lnot q 			& (\text{Contrapositive of 2.}) \\
		4. & p \lthen\lnot q 			& (\text{Transitivity of 1. and 3.}) \\ 
		5. & \lnot p \lor\lnot q	& (\text{Expansion of }\lthen) \\
	}{\lnot p\lor\lnot q & \text{}}				 
\end{gather*}
Therefore, by using laws of inference, we've proven that the conclusion follows from the premises. Therefore the argument is valid.
\begin{flushright}$\qed$\\\end{flushright}

\sol (c)
\begin{gather*}
	\longintertext{Suppose the argument is invalid\\
		Then, the premises are all true, and the conclusion is false.\\
		So, $\lnot p \lor \lnot q$ is False.}
	\argument{
		1. & p\lthen r 													& \text{} \\
		2. & q\lthen\lnot r 										& \text{} \\		
		3. & \lnot (p\land q) \text{ is False} 	& \text{(De Morgan's Law)} \\
		4. & p\land q \text{ is True} 					& \text{(Follows from 3.)} \\							
		5. & p \text{ is True} 									& \text{(Specalisation)} \\			
		6. & q \text{ is True} 									& \text{(Specalisation)} \\			
		7. & r \text{ is True} 									& \text{(Follows from 1., given 5.)} \\			
		8. & r \text{ is False}\qquad \contra 	& \text{(Follows from 2., given 6.)} \\
	}{\lnot p\lor\lnot q &}
\end{gather*}
We've identified a contradiction! If we assume invalidity, we see contradictions arise. Which means our original assumption was incorrect. Which means the argument is valid.
\begin{flushright}$\qed$\\\end{flushright}

\qs{}{
	Is the following argument valid? Again, use all three methods.
	$$
		\argument{
			1. & p\lthen q &\\
			2. & p\lor r &\\
			3. & p\lor\lnot r &\\
		}{q &}
	$$
}
\sol (a)
\begin{center}
	\begin{tabular}{|ccc||ccc||c|l}
		\cline{1-7}
		\multicolumn{3}{|c||}{Variables} & \multicolumn{3}{c||}{Premises} & \multicolumn{1}{c|}{Conclusion} & \\ 
		$p$ & $q$ & $r$ & $p\lthen q$ & $p\lor r$ & $p\lor\lnot r$ & $q$ \\ \cline{1-7}
		T & T & T & T & T & T & T & $\leftarrow$\\
		T & T & F & T & T & T & T & $\leftarrow$\\
		T & F & T & F & T & T & F & \\
		T & F & F & F & T & T & F & \\
		F & T & T & T & T & F & T & \\
		F & T & F & T & F & T & T & \\
		F & F & T & T & T & F & F & \\
		F & F & F & T & F & T & F & \\ \cline{1-7} 
	\end{tabular}
\end{center}
Observing rows 1, and 2, we can see that when all the premises are True, the conclusion is True. Therefore the argument is valid.
\begin{flushright}$\qed$\end{flushright}

\sol (b)
$$
\argument{
	1. & p\lthen q &\\
	2. & p\lor r &\\
	3. & p\lor\lnot r &\\
	4. & \lnot r\lor p & (\text{Commutativity of 3.}) \\
	5. & r \lthen p	& (\text{Logical Equivalence of 4.}) \\
	6. & r \lthen q & (\text{Transitivity of 5. and 1.}) \\
	7. & q & (\text{Division of Cases of 2. by 1., 6.}) \\
}{q &}
$$
Therefore, we've shown using rules of inference that the Argument is logically valid.
\begin{flushright}$\qed$\end{flushright}

\sol (c) \\
Suppose the argument is invalid\\
Then, the premises are all true, and the conclusion is false.\\
So, $q$ is False.
$$
	\argument{
		1. & p\lthen q &\\
		2. & p\lor r &\\
		3. & p\lor\lnot r &\\
		4. & p \text{ is False } & (\text{We know from 1. given } q) \\
		5. & r \text{ is True } & (\text{We know from 2. given } p) \\
		6. & r \text{ is False }\qquad\contra & (\text{We know from 3. given } p) \\
	}{q &}
$$
We've identified a contradiction! If we assume invalidity, we see contradictions arise. Which means our original assumption was incorrect. Which means the argument is valid.

\qs{}{
	Is the following argument valid?
	$$
		\argument{
			1. & p\lthen q & \\
			2. & q\lthen r & \\
			3. & \lnot p\lor\lnot q & \\
		}{r &}
	$$
}	
\sol I'll use rules of inference, since it's my weakest solution method.
\begin{gather*}
	\argument{
		1. & p\lthen q & \\
		2. & q\lthen r & \\
		3. & \lnot p\lor\lnot q & \\
		4. & p\lthen \lnot q\qquad\contra & (\text{Collection of 1.}\lor\text{ to }\lthen) \\
	}{r &}
\end{gather*}
This is a contradiction, because $p$ cannot similateously imply $q$ and $\lnot q$, no matter what truth value it takes. Therefore the argument is invalid.
\begin{flushright}$\qed$\end{flushright}

\qs{}{
	Determine whether or not the following argument is valid, using all three methods.
	\begin{itemize}
		\item If new messages are queued, then the filesystem is locked.
		\item The filesystem is not locked if and only if the system is functioning normally.
		\item New messages will not be sent to the message buffer only if they are queued.
		\item New messages will not be sent to the message buffer.
		\item Therefore, the system is functioning normally.
	\end{itemize}
}
\sol \begin{gather*}
	\longintertext{Let $p$ be the statement ``New messages are queued.''\\
	Let $q$ be the statement ``The filesystem is locked.''\\
	Let $r$ be the statement ``New messages are sent to the message buffer.''\\
	Let $s$ be the statement ``The system is functioning normally.''\\
	Then the argument can be written symbolically}
	\argument{
		1. & p\lthen q &\\
		2. & \lnot q\liff s &\\
		3. & p \lthen \lnot r &\\
		4. & \lnot r &\\
	}{s&\\}
	\longintertext{Let's check this arguments validity with a truth table.}
	\begin{array}{|cccc||cccc||c|l}
		\cline{1-9}
		\multicolumn{4}{|c||}{\text{Variables}} & \multicolumn{4}{c||}{\text{Premises}} & \multicolumn{1}{c|}{\text{Conclusion}} \\
		p & q & r & s & p\lthen q & \lnot q\liff s & p \lthen \lnot r & \lnot r & s \\ \cline{1-9}
		T & T & T & T & T & F & F & F & T & \\
		T & T & T & F & T & T & F & F & F & \\
		T & T & F & T & T & F & T & T & T & \\
		T & T & F & F & T & T & T & T & F & \leftarrow \\
		T & F & T & T & F & T & F & F & T & \\
		T & F & T & F & F & F & F & F & F & \\
		T & F & F & T & F & T & T & T & T & \\
		T & F & F & F & F & F & T & T & F & \\
		F & T & T & T & T & F & T & F & T & \\
		F & T & T & F & T & T & T & F & F & \\
		F & T & F & T & T & F & T & T & T & \\
		F & T & F & F & T & T & T & T & F & \leftarrow \\
		F & F & T & T & T & T & T & F & T & \\
		F & F & T & F & T & F & T & F & F & \\
		F & F & F & T & T & T & T & T & T & \\
		F & F & F & F & T & F & T & T & F & \\ \cline{1-9}
	\end{array}
	\longintertext{Observe rows 4 and 12, where in all the premises are true, but the conclusion is false. Therefore, the argument is invalid.\\
	Now let's prove this invalidity using rules of inference.}
	\argument{
		1. & p\lthen q &\\
		2. & \lnot q\liff s &\\
		3. & p\lthen \lnot r &\\
		4. & \lnot r &\\
		5. & s\lthen\lnot q & (\text{Falls from biconditional 2.}) \\
		6. & q\lthen\lnot s & (\text{Contrapositve of 5.}) \\
		7. & p\lthen\lnot s & (\text{Transitivity of 1. to 6.})\\
		8. & \lnot p\lor q & (\text{Expansion of 1.}\lthen)\\ 
		9. & \lnot p\lor\lnot r & (\text{Expansion of 3.}\lthen)\\
	}{s&\\}
	\longintertext{Therefore, we can see that the argument is invalid.\\
	Finally, lets check in validity by searching for values for the variables.\\
	Let's assume the argument is valid. All the premises are true, and the conclusion is false.\\
	$s$ is false.}
	\argument{
		1. & p\lthen q &\\
		2. & \lnot q\liff s &\\
		3. & p \lthen \lnot r &\\
		4. & \lnot r &\\
		5. & s\lthen \lnot q & (\text{Falls from biconditional 2.})\\
		6. & \lnot q \text{ is False} & (\text{Follows from 5.})\\
		7. & q \text{ is True} & (\text{Follows from 6.})\\
		8. & p \text{ is True} & (\text{Follows from 1. given }q)\\
		9. & r \text{ is False} & (\text{Negation of 4.})\\
	}{s&\\}
\end{gather*}


\section{Lecture 5}
\dfn{Predicate}{
	A predicate is a sentence which contains finitely many variables, and which becomes a statement if the varaibles are given specific values.\\

	The \textbf{domain} of each variable in a predicate is the set of all possible values that may be assigned to it.

	Predicates are commonly denoted with an upper case letter followed by a list of finitely many varaibles within brackets, $P(x)$, $Q(x)$, $R(x)$.
}
\ex{}{
	Given some variables $x, y, a,b,c \in \bbz$, here are some example predicates:
		\begin{itemize}
			\item $x$ is even.
			\item $x\leq y$.
			\item $a$ divides $b$ and $b$ divides $c$.
		\end{itemize}
	The following are not predicates:
	\begin{itemize}
		\item Divide by 2.
		\item Is $x$ an integer?
	\end{itemize}
}

\dfn{Truth Set}{
	The truth set of a predicate is the set of all values in the variables' domains, such that when a value from those domains are assigned to those variables, the predicate is evaluated as true. 
}
\ex{}{
	Let $P(x)$ be the predicate $x|5$, and $\dom x= \bbn$.\\

	The truth set of $P(x)$ is $\braces{-5, -1, 1, 5}$, because these are all the numbers in the domain which divide 5.
}

\subsection*{Common Domains}
\begin{itemize}
	\item The integers: $\bbz = \braces{\dots,-3,-2,-1,0,1,2,3,\dots}$
	\item The positive integers: $\bbz^{+} = \braces{1, 2, 3, \dots}$
	\item The nonnegative integers: $\bbz^{\geq0}=\braces{0,1,2,3,\dots}$
	\item The natural numbers: $\bbn=\braces{1,2,3,\dots}$
	\item The rational numbers: $\bbq=\braces{\frac{a}{b}\suchthat a,b\in\bbz\land b\neq0}$
	\item The real numbers: The entire number line.
\end{itemize}
\nt{The real numbers have a rigorous definition, but it is outside the scope of this introducctory course.}

\subsection*{The Universal Quantifier}
The symbol $\forall$ denotes ``for all'' and is called the universal quantifier.\\

Let $Q(x)$ be a predicate and $\dom x = D$.\\
The statement 
$$
	\forall x\in D, Q(x)
$$
is true if and only if $Q(x)$ is true for every single element in $D$.\\
It is false if and only if $Q(x)$ is false for at least one element in $D$.

\ex{}{
	Let $Q(x)$ be the predicate $x\leq x^2$, and $\dom x=\bbz$. The statement $\forall x\in\bbz, Q(x)$ can be expressed in the following equivalent ways:
	\begin{itemize}
		\item $\forall x\in\bbz, x\leq x^2$
		\item For all $x\in\bbz, x\leq x^2$
		\item Every integer is less then or equal to its sqaure.
	\end{itemize}

	Are the following statements true or false?
	$$
		\forall x\in\bbz, x\in\bbr
	$$
	True. $\because\bbz\subseteq\bbr$.

	$$
		\forall y\in\bbq, y^2\geq1
	$$
	False. Counterexample, let $y=\frac{1}{2}$. Then $\bracks{\frac{1}{2}}^2 = \frac{1}{4} < 1$. Take any $y< 1$. It's square is less then 1. 
}

\subsection*{The Existential Quantifier}
The symbol $\exists$ denotes ``there exists'' and is called the existential quantifier.\\

Let $Q(x)$ be a predicate and $\dom x = D$.\\
The statement 
$$
	\exists x\in D: Q(x)
$$
is true if and only if $Q(x)$ is true for at least a single element in $D$.\\
It is false if and only if $Q(x)$ is false for every single element in $D$.

\ex{}{
	Let $Q(x)$ be the predicate $x^2=4$, and $\dom x=\bbz$. The statement $\exists x\in D: Q(x)$ can be expressed in the following equivalent ways:
	\begin{itemize}
		\item $\exists x\in\bbz \text{ such that } x^2 = 4$
		\item There exists an integer $x$ such that $x^2 = 4$
		\item There is some integer whose square is 4.
	\end{itemize}

	Are the following true or false?
	$$
		\exists x\in\bbr: x^2=1 \land x<0
	$$
	Note that $P(x)$ is the conjuction of two other predicates.\\
	This is true. Take $x=-1\in\bbr$.\\
	Then $-1^2=1$ and $-1<0$.

	$$
		\exists x\in\braces{2, 4, 6}: x^2 = 9.
	$$
	False. We can prove this by exhaustion.\\
	$2^2 = 4 \neq 9\quad 4^2=16\neq 9\quad 6^2=36\neq9$.\\
	Therefore, there is no $x$ in the domain such that the predicate is satisfied.
}

\subsection*{Universal Conditional Statements}
One of the most important statement forms in mathematics:
$$
	\forall x \in D, P(x)\lthen Q(x)
$$
\ex{}{
	The universal conditional statement
	$$
		\forall x\in\bbr, x>3 \lthen x^2 > 9
	$$
	Can be equivalently expressed
	\begin{itemize}
		\item For every real number, $x$, if $x>3$, then $x^2>9$.
		\item Whenever a real number is greater then 3, its square is greater then 9.
		\item The squares of real numbers greater then 3, are greater then 9.
	\end{itemize}
}

\subsection*{Questions}
\qs{}{example}

\section{Lecture 6}
\subsection*{Negations of Quantified Statements}
\subsubsection*{Negating the Universal Quantifier}
Consider the universally quantified statement
$$
	\forall x\in D, Q(x).
$$
The negation of this statement is logically equivalent to
$$
	\exists x\in D: \lnot Q(x).
$$
$\forall$ negates to $\exists$, and the predicate $Q(x)$ negates to $\lnot Q(x)$.

\ex{}{
	Consider the statement
	$$
		\forall x\in\bbz, x \text{ is prime.}
	$$
	The negation of this statement is
	$$
		\exists x\in\bbz: x \text{ is not prime.} 
	$$
	Naturally, and to maintain logical equivalence, the original statement, in this case, evaluates to False, while its negation evaluates to True.\\ 

	Now, Consider the statement
	$$
		\text{All integers are odd or even.}
	$$
	This can be written mathematically as
	$$
		\forall x\in\bbz, x\equiv0\ (\mathrm{mod}\ 2) \lor x\equiv1\ (\mathrm{mod}\ 2).
	$$
	And it's negation is
	$$
		\exists x\in\bbz: x\not\equiv0\ (\mathrm{mod}\ 2) \land x\not\equiv1\ (\mathrm{mod}\ 2). 
	$$
	Which when brought back into the English language, is read
	$$
		\text{There is an integer which is not even and not odd.}
	$$
	Clearly, the original statement evaluates to True, and its negation to False.
}

\subsubsection*{Negating the Existential Quantifier}
Now let's consider the existentially quantified statement
$$
	\exists x\in D: P(x).
$$
The negation of this statement Is
$$
	\forall x\in D: \lnot P(x).
$$
$\exists$ negates to $\forall$, and the predicate $P(x)$ negates to $\lnot P(x)$.

\ex{}{
	Consider the statement
	$$
		\text{There is a pink elephant.}
	$$
	Its negation is 
	$$
		\text{Every elephant is not pink.}
	$$
	
	${}^{}$\\Consider the statement 
	$$
		\exists x\in\bbq:x\in\bbz
	$$
	Its negation is
	$$
		\forall x\in\bbq, x\notin\bbz
	$$
	Again, we can tell that the original statement is true, and its negation is false. \\

	A couple more examples\dots Let's consider the statement
	$$
		\text{Some rabbit has white fur.}
	$$
	Note that this is existentially quantified, so its negation will be universally quantified,
	$$
		\text{No rabbit has white fur.}
	$$
	Finally, consider the statement
	$$
		\text{Every UQ student is happy.}
	$$
	This time, note that this statement is universally quantified, so its negation will be existentially quantified,
	$$
		\text{There is a UQ student who is not happy.}
	$$
	We're getting the hang of this!
}

\subsubsection*{Negating the Universal Conditional Quantifier}
Finally, we consider the statement
$$
	\forall x\in D, P(x)\lthen Q(x).
$$
Using laws of logical equivalence, and what we've just learned, we can easily conclude that the negation of this statement is
$$
	\exists x\in D: \lnot(P(x)\lthen Q(x)) \equiv \exists x\in D: P(x)\land\lnot Q(x)
$$
ultimately, the $\forall$ still negates to $\exists$, and if you consider the composite statement $R(x) = P(x)\lthen Q(x)$, then $\lnot R(x)\equiv \lnot(P(x)\lthen Q(x))\equiv \lnot(\lnot P(x)\lor Q(x)) \equiv P(x)\land\lnot Q(x)$.

\ex{}{
	Let's negate some more statements!
	\begin{gather*}
		\begin{aligned}
			A &= \forall x\in\bbz, x\geq1\lthen x\in\bbn &\text{(True)}\\
			\tf\lnot A &= \exists x\in\bbz: x\geq1\land x\notin\bbn &\text{(False)}\\
			B &= \forall x\in\bbz, \bracks{3\divs x} \lthen \bracks{6\divs x} &\text{(False)}\\
			\tf\lnot B &= \exists x\in\bbz: \bracks{3\divs x}\land \bracks{6\ndivs x} &\text{(True)}\\
			C&= \text{``If a rabbit has white fur, then it has long ears''} &\text{(False)}\\
			\tf\lnot C&= \text{``There is a rabbit with white fur and short ears''} &\text{(True)}\\
			D&= \text{``All parks that have grass, have playgrounds.''} &\text{(False)}\\
			\tf\lnot D&= \text{``Some park has grass and not playground.''} &\text{(True)}\\
		\end{aligned}
	\end{gather*}
}

\subsection*{Statements with Multiple Quantifiers}
Some predicates, for instance $x\leq y$, involve more then one varaible. In such a case, we use the notation $P(x,y)$ to denote such a predicate. Such predicates often appear with more than one quantifier. For example, consider the statement
$$
	\exists x\in\bbn: \forall y\in\bbn, x\leq y.
$$
We would read this as ``There exists a natural number which is smaller than all natural numbers.'' or ``There is a smallest natural number.'' 
\nt{``Such that'', :,  always pairs with the existential quantifier!}

\noindent It's negation would be
$$
	\forall y\in\bbn, \exists x\in\bbn: x\leq y
$$
which is read, ``Every natural numbers has some other number which is less then or equal to it.''

\subsubsection*{Establishing the Truth, when given Multiple Quantifiers}
Suppose we want to prove the given the statement
$$
	\forall x\in D, \exists y\in E: P(x,y).
$$
To prove this, we must allow someone to pick any element in $D$ they want, and we must any element in $E$ which makes $P(x,y)$ true.
\ex{}{
	Let's prove the statement
	\begin{gather*}
		\forall x\in\bbz, \exists y\in\bbz: x + y = 0. \\
		\forall x\in\bbz,\\
		\Choose y=-x,\\
		\Then x+y = x - x = 0. \tag*{\qed}
	\end{gather*}
}
\noindent No matter what value for $x$ is chosen, I choose $y=-x$, and the predicate $P(x,y)$ always evaluates to true. Because we can do this for all integers $x$, we know that this statement is true.\\

\noindent Now lets suppose we have the statement
$$
	\exists x\in D: \forall y\in E, P(x,y).
$$
To prove this statement, we need to find one particular $x\in D$ which makes $P(x,y)$ true, no matter what selection is made for $y\in E$.
\ex{}{
	Let's prove the statement
	\begin{gather*}
		\exists x\in\bbn: \forall y\in\bbn, x\leq y. \\
		\Take x=1. \\
		\text{Now, } \forall y\in\bbz, 1\leq y. \tag*{\qed}
	\end{gather*}
}

\subsubsection*{Negations of Statements with Multiple Quantifiers}
Consider the statement
$$
	\forall x\in D, \exists y\in E: P(x,y).
$$
This statement will negate to
$$
	\exists x\in D:\forall y\in E, \lnot P(x,y).
$$
Similarly, consider the statement
$$
	\exists x\in D: \forall y\in E, P(x,y).
$$
This statement will negate to
$$
	\forall x\in D, \exists y\in E: \not P(x,y).
$$
Again, note, that the such that always pairs with the existential quantifier. Let's look at some examples now\dots
\ex{}{
	\begin{gather*}
		\begin{aligned}
			A &= \forall x\in\bbz, \exists y\in\bbz:x+y=0 &\text{(True)}\\
			\tf\lnot A &= \exists x\in\bbz: \forall y\in\bbz, x+y\neq0 &\text{(False)}\\
			B &= \exists x\in\bbr: \forall y\in\bbr, \abs{x} \leq \abs{y} &\text{(False)}\\
			\tf\lnot B &= \forall x\in\bbr, \exists y\in\bbr: \abs{y}<\abs{x} &\text{(True)}\\
		\end{aligned}
	\end{gather*}
}

\subsection*{Questions}
\qs{}{example}

\chapter{Week 3}
\section{Lecture 7}

\end{document}
