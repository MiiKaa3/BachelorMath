\documentclass{report}

\input{../../../latex_template/preamble}
%From M275 "Topology" at SJSU
\newcommand{\id}{\mathrm{id}}
\newcommand{\taking}[1]{\xrightarrow{#1}}
\newcommand{\inv}{^{-1}}

%From M170 "Introduction to Graph Theory" at SJSU
\DeclareMathOperator{\diam}{diam}
\DeclareMathOperator{\ord}{ord}
\newcommand{\defeq}{\overset{\mathrm{def}}{=}}

%From the USAMO .tex files
\newcommand{\ts}{\textsuperscript}
\newcommand{\dg}{^\circ}
\newcommand{\ii}{\item}

% % From Math 55 and Math 145 at Harvard
% \newenvironment{subproof}[1][Proof]{%
% \begin{proof}[#1] \renewcommand{\qedsymbol}{$\blacksquare$}}%
% {\end{proof}}

\newcommand{\liff}{\leftrightarrow}
\newcommand{\lthen}{\rightarrow}
\newcommand{\opname}{\operatorname}
\newcommand{\surjto}{\twoheadrightarrow}
\newcommand{\injto}{\hookrightarrow}
\newcommand{\On}{\mathrm{On}} % ordinals
\DeclareMathOperator{\img}{im} % Image
\DeclareMathOperator{\Img}{Im} % Image
\DeclareMathOperator{\coker}{coker} % Cokernel
\DeclareMathOperator{\Coker}{Coker} % Cokernel
\DeclareMathOperator{\Ker}{Ker} % Kernel
\DeclareMathOperator{\rank}{rank}
\DeclareMathOperator{\Spec}{Spec} % spectrum
\DeclareMathOperator{\Tr}{Tr} % trace
\DeclareMathOperator{\pr}{pr} % projection
\DeclareMathOperator{\ext}{ext} % extension
\DeclareMathOperator{\pred}{pred} % predecessor
\DeclareMathOperator{\dom}{dom} % domain
\DeclareMathOperator{\ran}{ran} % range
\DeclareMathOperator{\Hom}{Hom} % homomorphism
\DeclareMathOperator{\Mor}{Mor} % morphisms
\DeclareMathOperator{\End}{End} % endomorphism

\newcommand{\eps}{\epsilon}
\newcommand{\veps}{\varepsilon}
\newcommand{\ol}{\overline}
\newcommand{\ul}{\underline}
\newcommand{\wt}{\widetilde}
\newcommand{\wh}{\widehat}
\newcommand{\vocab}[1]{\textbf{\color{blue} #1}}
\providecommand{\half}{\frac{1}{2}}
\newcommand{\dang}{\measuredangle} %% Directed angle
\newcommand{\ray}[1]{\overrightarrow{#1}}
\newcommand{\seg}[1]{\overline{#1}}
\newcommand{\arc}[1]{\wideparen{#1}}
\DeclareMathOperator{\cis}{cis}
\DeclareMathOperator*{\lcm}{lcm}
\DeclareMathOperator*{\argmin}{arg min}
\DeclareMathOperator*{\argmax}{arg max}
\newcommand{\cycsum}{\sum_{\mathrm{cyc}}}
\newcommand{\symsum}{\sum_{\mathrm{sym}}}
\newcommand{\cycprod}{\prod_{\mathrm{cyc}}}
\newcommand{\symprod}{\prod_{\mathrm{sym}}}
\newcommand{\Qed}{\begin{flushright}\qed\end{flushright}}
\newcommand{\parinn}{\setlength{\parindent}{1cm}}
\newcommand{\parinf}{\setlength{\parindent}{0cm}}
% \newcommand{\norm}{\|\cdot\|}
\newcommand{\inorm}{\norm_{\infty}}
\newcommand{\opensets}{\{V_{\alpha}\}_{\alpha\in I}}
\newcommand{\oset}{V_{\alpha}}
\newcommand{\opset}[1]{V_{\alpha_{#1}}}
\newcommand{\lub}{\text{lub}}
\newcommand{\del}[2]{\frac{\partial #1}{\partial #2}}
\newcommand{\Del}[3]{\frac{\partial^{#1} #2}{\partial^{#1} #3}}
\newcommand{\deld}[2]{\dfrac{\partial #1}{\partial #2}}
\newcommand{\Deld}[3]{\dfrac{\partial^{#1} #2}{\partial^{#1} #3}}
\newcommand{\lm}{\lambda}
\newcommand{\uin}{\mathbin{\rotatebox[origin=c]{90}{$\in$}}}
\newcommand{\usubset}{\mathbin{\rotatebox[origin=c]{90}{$\subset$}}}
\newcommand{\lt}{\left}
\newcommand{\rt}{\right}
\newcommand{\bs}[1]{\boldsymbol{#1}}
\newcommand{\exs}{\exists}
\newcommand{\st}{\strut}
\newcommand{\dps}[1]{\displaystyle{#1}}

\newcommand{\sol}{\setlength{\parindent}{0cm}\textbf{\textit{Solution:}}\setlength{\parindent}{1cm} }
\newcommand{\solve}[1]{\setlength{\parindent}{0cm}\textbf{\textit{Solution: }}\setlength{\parindent}{1cm}#1 \Qed}

\DeclareMathOperator{\sech}{sech}
\DeclareMathOperator{\csch}{csch}
\DeclareMathOperator{\arcsec}{arcsec}
\DeclareMathOperator{\arccsc}{arccsc}
\DeclareMathOperator{\arccot}{arccot}
\DeclareMathOperator{\arsinh}{arsinh}
\DeclareMathOperator{\arcosh}{arcosh}
\DeclareMathOperator{\artanh}{artanh}
\DeclareMathOperator{\arcsch}{arcsch}
\DeclareMathOperator{\arsech}{arsech}
\DeclareMathOperator{\arcoth}{arcoth}

\newcommand{\sinx}{\sin x}          \newcommand{\arcsinx}{\arcsin x}    
\newcommand{\cosx}{\cos x}          \newcommand{\arccosx}{\arccosx}
\newcommand{\tanx}{\tan x}          \newcommand{\arctanx}{\arctan x}
\newcommand{\cscx}{\csc x}          \newcommand{\arccscx}{\arccsc x}
\newcommand{\secx}{\sec x}          \newcommand{\arcsecx}{\arcsec x}
\newcommand{\cotx}{\cot x}          \newcommand{\arccotx}{\arccot x}
\newcommand{\sinhx}{\sinh x}          \newcommand{\arsinhx}{\arsinh x}
\newcommand{\coshx}{\cosh x}          \newcommand{\arcoshx}{\arcosh x}
\newcommand{\tanhx}{\tanh x}          \newcommand{\artanhx}{\artanh x}
\newcommand{\cschx}{\csch x}          \newcommand{\arcschx}{\arcsch x}
\newcommand{\sechx}{\sech x}          \newcommand{\arsechx}{\arsech x}
\newcommand{\cothx}{\coth x}          \newcommand{\arcothx}{\arcoth x}
\newcommand{\lnx}{\ln x}
\newcommand{\expx}{\exp x}

\newcommand{\bba}{\mathbb{A}}   \newcommand{\bbn}{\mathbb{N}}
\newcommand{\bbb}{\mathbb{B}}   \newcommand{\bbo}{\mathbb{O}}
\newcommand{\bbc}{\mathbb{C}}   \newcommand{\bbp}{\mathbb{P}}
\newcommand{\bbd}{\mathbb{D}}   \newcommand{\bbq}{\mathbb{Q}}
\newcommand{\bbe}{\mathbb{E}}   \newcommand{\bbr}{\mathbb{R}}
\newcommand{\bbf}{\mathbb{F}}   \newcommand{\bbs}{\mathbb{S}}
\newcommand{\bbg}{\mathbb{G}}   \newcommand{\bbt}{\mathbb{T}}
\newcommand{\bbh}{\mathbb{H}}   \newcommand{\bbu}{\mathbb{U}}
\newcommand{\bbi}{\mathbb{I}}    \newcommand{\bbv}{\mathbb{V}}
\newcommand{\bbj}{\mathbb{J}}   \newcommand{\bbw}{\mathbb{W}}
\newcommand{\bbk}{\mathbb{K}}   \newcommand{\bbx}{\mathbb{X}}
\newcommand{\bbl}{\mathbb{L}}    \newcommand{\bby}{\mathbb{Y}}
\newcommand{\bbm}{\mathbb{M}}   \newcommand{\bbz}{\mathbb{Z}}

\newcommand{\lb}{\left(}
\newcommand{\rb}{\right)}
\newcommand{\lbr}{\left\lbrace}
\newcommand{\rbr}{\right\rbrace}
\newcommand{\lsb}{\left[}
\newcommand{\rsb}{\right]}
\newcommand{\suchthat}{\medspace\middle|\medspace}
\newcommand{\bracks}[1]{\lb #1 \rb}
\newcommand{\braces}[1]{\lbr #1 \rbr}
\newcommand{\sqbracks}[1]{\lsb #1 \rsb}

\renewcommand{\floor}[1]{\lfloor #1 \rfloor}
\renewcommand{\ceil}[1]{\lceil #1 \rceil}

\newcommand{\cd}{\cdot}
\newcommand{\tf}{\therefore}
\newcommand{\Let}{\text{Let }}
\newcommand{\Given}{\text{Given }}
\newcommand{\Suppose}{\text{Suppose }}
\newcommand{\WeSee}{\text{We see }}
\newcommand{\So}{\text{So }}

\newcommand{\QED}{\hfill \qed}

\renewcommand{\dd}[1]{\frac{d}{d#1}}
\newcommand{\dyd}[2][y]{\frac{d#1}{d#2}}

\newcommand{\ddx}{\dd{x}}       \newcommand{\ddxsq}{\dyd[^2]{x^2}}
\newcommand{\ddy}{\dd{y}}       \newcommand{\ddysq}{\dyd[^2]{y^2}}
\newcommand{\ddu}{\dd{u}}       \newcommand{\ddusq}{\dyd[^2]{u^2}}
\newcommand{\ddv}{\dd{v}}       \newcommand{\ddvsq}{\dyd[^2]{v^2}}

\newcommand{\dydx}{\dyd{x}}     \newcommand{\dydxsq}{\dyd[^2y]{x^2}}
\newcommand{\dfdx}{\dyd[f]{x}}  \newcommand{\dfdxsq}{\dyd[^2f]{x^2}}
\newcommand{\dudx}{\dyd[u]{x}}  \newcommand{\dudxsq}{\dyd[^2u]{x^2}}
\newcommand{\dvdx}{\dyd[v]{x}}  \newcommand{\dvdxsq}{\dyd[^2v]{x^2}}

% Mathfrak primes
\newcommand{\km}{\mathfrak{m}}
\newcommand{\kp}{\mathfrak{p}}
\newcommand{\kq}{\mathfrak{q}}

%---------------------------------------
% Blackboard Math Fonts :-
%---------------------------------------
\newcommand{\bba}{\mathbb{A}}   \newcommand{\bbn}{\mathbb{N}}
\newcommand{\bbb}{\mathbb{B}}   \newcommand{\bbo}{\mathbb{O}}
\newcommand{\bbc}{\mathbb{C}}   \newcommand{\bbp}{\mathbb{P}}
\newcommand{\bbd}{\mathbb{D}}   \newcommand{\bbq}{\mathbb{Q}}
\newcommand{\bbe}{\mathbb{E}}   \newcommand{\bbr}{\mathbb{R}}
\newcommand{\bbf}{\mathbb{F}}   \newcommand{\bbs}{\mathbb{S}}
\newcommand{\bbg}{\mathbb{G}}   \newcommand{\bbt}{\mathbb{T}}
\newcommand{\bbh}{\mathbb{H}}   \newcommand{\bbu}{\mathbb{U}}
\newcommand{\bbi}{\mathbb{I}}   \newcommand{\bbv}{\mathbb{V}}
\newcommand{\bbj}{\mathbb{J}}   \newcommand{\bbw}{\mathbb{W}}
\newcommand{\bbk}{\mathbb{K}}   \newcommand{\bbx}{\mathbb{X}}
\newcommand{\bbl}{\mathbb{L}}   \newcommand{\bby}{\mathbb{Y}}
\newcommand{\bbm}{\mathbb{M}}   \newcommand{\bbz}{\mathbb{Z}}

%---------------------------------------
% Roman Math Fonts :-
%---------------------------------------
\newcommand{\rma}{\mathrm{A}}   \newcommand{\rmn}{\mathrm{N}}
\newcommand{\rmb}{\mathrm{B}}   \newcommand{\rmo}{\mathrm{O}}
\newcommand{\rmc}{\mathrm{C}}   \newcommand{\rmp}{\mathrm{P}}
\newcommand{\rmd}{\mathrm{D}}   \newcommand{\rmq}{\mathrm{Q}}
\newcommand{\rme}{\mathrm{E}}   \newcommand{\rmr}{\mathrm{R}}
\newcommand{\rmf}{\mathrm{F}}   \newcommand{\rms}{\mathrm{S}}
\newcommand{\rmg}{\mathrm{G}}   \newcommand{\rmt}{\mathrm{T}}
\newcommand{\rmh}{\mathrm{H}}   \newcommand{\rmu}{\mathrm{U}}
\newcommand{\rmi}{\mathrm{I}}   \newcommand{\rmv}{\mathrm{V}}
\newcommand{\rmj}{\mathrm{J}}   \newcommand{\rmw}{\mathrm{W}}
\newcommand{\rmk}{\mathrm{K}}   \newcommand{\rmx}{\mathrm{X}}
\newcommand{\rml}{\mathrm{L}}   \newcommand{\rmy}{\mathrm{Y}}
\newcommand{\rmm}{\mathrm{M}}   \newcommand{\rmz}{\mathrm{Z}}

%---------------------------------------
% Calligraphic Math Fonts :-
%---------------------------------------
\newcommand{\cla}{\mathcal{A}}   \newcommand{\cln}{\mathcal{N}}
\newcommand{\clb}{\mathcal{B}}   \newcommand{\clo}{\mathcal{O}}
\newcommand{\clc}{\mathcal{C}}   \newcommand{\clp}{\mathcal{P}}
\newcommand{\cld}{\mathcal{D}}   \newcommand{\clq}{\mathcal{Q}}
\newcommand{\cle}{\mathcal{E}}   \newcommand{\clr}{\mathcal{R}}
\newcommand{\clf}{\mathcal{F}}   \newcommand{\cls}{\mathcal{S}}
\newcommand{\clg}{\mathcal{G}}   \newcommand{\clt}{\mathcal{T}}
\newcommand{\clh}{\mathcal{H}}   \newcommand{\clu}{\mathcal{U}}
\newcommand{\cli}{\mathcal{I}}   \newcommand{\clv}{\mathcal{V}}
\newcommand{\clj}{\mathcal{J}}   \newcommand{\clw}{\mathcal{W}}
\newcommand{\clk}{\mathcal{K}}   \newcommand{\clx}{\mathcal{X}}
\newcommand{\cll}{\mathcal{L}}   \newcommand{\cly}{\mathcal{Y}}
\newcommand{\calm}{\mathcal{M}}  \newcommand{\clz}{\mathcal{Z}}

%---------------------------------------
% Fraktur  Math Fonts :-
%---------------------------------------
\newcommand{\fka}{\mathfrak{A}}   \newcommand{\fkn}{\mathfrak{N}}
\newcommand{\fkb}{\mathfrak{B}}   \newcommand{\fko}{\mathfrak{O}}
\newcommand{\fkc}{\mathfrak{C}}   \newcommand{\fkp}{\mathfrak{P}}
\newcommand{\fkd}{\mathfrak{D}}   \newcommand{\fkq}{\mathfrak{Q}}
\newcommand{\fke}{\mathfrak{E}}   \newcommand{\fkr}{\mathfrak{R}}
\newcommand{\fkf}{\mathfrak{F}}   \newcommand{\fks}{\mathfrak{S}}
\newcommand{\fkg}{\mathfrak{G}}   \newcommand{\fkt}{\mathfrak{T}}
\newcommand{\fkh}{\mathfrak{H}}   \newcommand{\fku}{\mathfrak{U}}
\newcommand{\fki}{\mathfrak{I}}   \newcommand{\fkv}{\mathfrak{V}}
\newcommand{\fkj}{\mathfrak{J}}   \newcommand{\fkw}{\mathfrak{W}}
\newcommand{\fkk}{\mathfrak{K}}   \newcommand{\fkx}{\mathfrak{X}}
\newcommand{\fkl}{\mathfrak{L}}   \newcommand{\fky}{\mathfrak{Y}}
\newcommand{\fkm}{\mathfrak{M}}   \newcommand{\fkz}{\mathfrak{Z}}


\title{\Huge{MATH1061}\\Discrete Mathematics I}
\author{\huge{Michael Kasumagic, s4430266}}
\date{\huge{Semester 2, 2024}}

\begin{document}
\chapter*{MATH1061 Tutorial Week 2}\vspace{-1.35cm}
\section*{Michael Kasumagic, s4430266}

\qs{1.1}{
	\begin{enumerate}[label=(\roman*)] 
		\item Suppose you know that $(\sim p \wedge q) \vee p$ is false. What can you conclude about the truth values of each of the two variables?
		\item Suppose you know that $(p \vee q) \wedge \sim r$ is true. What can you conclude about the truth values of each of the three variables?
		\item Suppose you know that $(\sim p \wedge \sim q) \wedge r$ is false. What can you conclude about the truth values of each of the three variables?
	\end{enumerate}
}
\sol (i) \begin{align*}
	(\lnot p\land q) \lor p &\equiv \false \\
	\Rightarrow \lnot p \land q &\equiv \false \\
	\land p &\equiv \false \\
	\Rightarrow \lnot p &\equiv \true \\
	\Rightarrow q &\equiv \false
\end{align*}
$\tf p\equiv\false$ and $q\equiv\false$. \\

\sol (ii) \begin{align*}
	(p \lor q) \land \lnot r &\equiv \true \\
	\Rightarrow \lnot r &\equiv \true \\
	\Rightarrow r &\equiv \false \\
	\Rightarrow p \lor q &\equiv \true \\
	\text{Case 1}\Rightarrow p \equiv \true &\lor q \equiv \false \\
	\text{Case 2}\Rightarrow p \equiv \false &\lor q \equiv \true \\
	\text{Case 3}\Rightarrow p \equiv \true &\lor q \equiv \true
\end{align*}
$\tf (p,q,r)\in\braces{(\true, \false, \false), (\false, \true, \false), (\true, \true, \false)}$ \\

\sol (iii) \begin{align*}
	\text{If } (\lnot p \land \lnot q) \land r &\equiv \true \\
	\Rightarrow r &\equiv \true \\
	\lnot p \equiv \true &\Rightarrow p \equiv\false \\
	\lnot q \equiv \true &\Rightarrow q \equiv\false
\end{align*}
$\tf(\lnot p \land \lnot q) \land r \equiv \true \iff (p,q,r)\in\braces{(\false,\false,\true)}$. \\
$\tf(\lnot p \land \lnot q) \land r \equiv \false \iff (p,q,r)\notin\braces{(\false,\false,\true)}$. \\
$\tf(\lnot p \land \lnot q) \land r \equiv \false \iff (p,q,r)\in U \setminus \braces{(\false,\false,\true)}$. 

\qs{1.2}{
	For each of the following, write down a truth table for the statement, and determine whether the statement is a tautology, a contradiction, or neither.
	\begin{enumerate}[label=(\roman*)] 
		\item $((p\land q)\lor(q\land r))\lor\lnot q$
		\item $(\lnot p \lor q)\lor(p\land\lnot q)$
	\end{enumerate}
}
\sol (i)
\begin{center}
	\begin{tabular}{|ccc||cc||cc||c|}
		\hline 
		$p$ & $q$ & $r$ & $p\land q$ & $q\land r$ & $(p\land q)\lor(q\land r)$ & $\lnot q$ & $((p\land q)\lor(q\land r))\lor\lnot q$ \\ \hline
		T	& T	& T	& T	& T	& T	& F	&	T \\
		T	& T	&	F & T	& F	& T	& F	&	T \\
		T	& F	&	T & F	& F	& F	& T	&	T \\
		T	& F	&	F & F	& F	& F	& T	&	T \\
		F	& T	&	T & F	& T	& T	& F	&	T \\
		F	& T	&	F & F	& F	& F	& F	&	F \\
		F	& F	&	T & F	& F	& F	& T	&	T \\
		F	& F	&	F & F	& F	& F	& T	&	T \\ \hline 
	\end{tabular}
\end{center}
The statement is neither a contradiction nor a tautology. \\

\sol (ii)
\begin{center}
	\begin{tabular}{|cc||cc||cc||c|}
		\hline 
		$p$ & $q$ & $\lnot p$ & $\lnot p \lor q$ & $\lnot q$ & $p\land\lnot q$ & $(\lnot p\lor q) \lor (p\land\lnot q)$ \\ \hline
		T	& T	& F	& T	& F	& F	& T	\\
		T	& F &	F	& F	& T	& T	& T	\\
		F	& T &	T	& T	& F	& F	& T	\\
		F	& F &	T	& T	& T	& F	& T	\\ \hline 
	\end{tabular}
\end{center}
Since the statement is always true, it is a tautology.

\qs{1.3}{
	\begin{enumerate}[label=(\roman*)] 
		\item Use a truth table to show that $(p \lor q) \land\lnot p \equiv q \land \lnot p$.
		\item Use a truth table to show that $(p \lxor q) \land r \equiv (p \land r) \lxor (q \land r)$.
		\item Use the laws of logical equivalence and the fact that $p \lxor q \equiv (p \lor q) \land \lnot (p \land q)$, to show that $(p \lxor q) \land r \equiv (p \land r) \lxor (q \land r)$.
	\end{enumerate}
}
\sol (i)
\begin{center}
	\begin{tabular}{|cc||c||c||cc|}
		\hline
		$p$	&$q$ &$\lnot p$ &$p\lor q$ &$(p\lor q)\land \lnot p$ &$q\land\lnot p$ \\ \hline
		T & T & F & T & F & F \\ 
		T & F & F & T & F & F \\ 
		F & T & T & T & T & T \\ 
		F & F & T & F & F & F \\ \hline
	\end{tabular}
\end{center}

\sol (ii)
\begin{center}
	\begin{tabular}{|ccc||c||cc||cc|}
		\hline
		$p$ &$q$	&$r$	&$p\lxor q$ &$p\land r$ &$q\land r$ &$(p\lxor q)\land r$ &$(p\land r)\lxor(q\land r)$ \\ \hline
		T & T & T & F & T & T & F & F \\
		T & T & F & F & F & F & F & F \\
		T & F & T & T & T & F & T & T \\
		T & F & F & T & F & F & F & F \\
		F & T & T & T & F & T & T & T \\
		F & T & F & T & F & F & F & F \\
		F & F & T & F & F & F & F & F \\
		F & F & F & F & F & F & F & F \\ \hline
	\end{tabular}	
\end{center}

\sol (iii)
\begin{align*}
	(p\lxor q) \land r &\equiv ((p \lor q) \land \lnot (p \land q)) \land r \\ 
		&\equiv (p \lor q) \land (\lnot p \lor\lnot q) \land r \\ 
		&\equiv ((r\land p) \lor (r\land q)) \land (\lnot p \lor\lnot q) \\ 
		&\equiv (((r\land p) \lor (r\land q)) \land \lnot p) \lor (((r\land p) \lor (r\land q)) \land \lnot q) \\ 
		&\equiv (r\land p\land \lnot p) \lor (r\land q\land \lnot p) \lor (r\land p\land \lnot q) \lor (r\land q\land \lnot q) \\
		&\equiv \bot \lor (r\land q\land \lnot p) \lor (r\land p\land \lnot q) \lor \bot \\
		&\equiv r\land ((q\land \lnot p) \lor (p\land \lnot q)) \\
		&\equiv r\land ((q\land \lnot p) \lor p) \land ((q\land \lnot p) \lor\lnot q) \\
	GOAL &\equiv ((p\land r) \lor (q\land r)) \land \lnot((p\land r)\land(q\land r)) \\
		&\equiv ((p\land r) \lor (q\land r)) \land \lnot(p\land q \land r) \\
		&\equiv (r\land(p\lor q)) \land \lnot(p\land q \land r) \\
		&\equiv r\land(p\lor q) \land \lnot(p\land q \land r) \\
		&\equiv r\land(p\lor q) \land(\lnot p\lor\lnot q \lor\lnot r) \\
\end{align*}
AHHHHHHHHHHHHHHHHHHHHHH THIS IS HARD.

\qs{1.4}{
	\begin{enumerate}[label=(\roman*)] 
		\item Use the laws of logical equivalence to show that $p\land q\equiv\lnot(\lnot p\lor\lnot q)$.
		\item Use the laws of logical equivalence to show that $\lnot(p\lor\lnot q)\lor(\lnot p\land\lnot q)\equiv \lnot p$.
	\end{enumerate}
}
\sol (i)
\begin{align*}
	p\land q &\equiv \lnot (\lnot (p\land q)) \\
		&\equiv \lnot (\lnot p \lor \lnot q)
\end{align*}

\sol (ii)
\begin{align*}
	\lnot (p\lor \lnot q) \lor (\lnot p\land \lnot q) &\equiv (\lnot p\land q) \lor (\lnot p\land \lnot q) \\
		&= \lnot p \land (q\lor \lnot q) \\
		&= \lnot p \land \top \\
		&= \lnot p
\end{align*}

\qs{1.5}{
	Using De Morgan's Law, rewrite these sentences retaining equivalence.
	\begin{enumerate}[label=(\roman*)]
		\item t is not true that I am studying Computer Science and I am studying Engineering.
		\item I am not going to the movies this weekend or I am not going swimming this weekend.
	\end{enumerate} 
}
\sol (i) It is not true that I am studying Computer Science and I am studying Engineering. $\equiv \lnot (p \land q) \equiv \lnot p \lor \lnot q\equiv$ I am not studying CS or I am not studying engineering. \\

\sol (ii) I am not going to the movies this weekend or I am not going swimming this weekend. $\equiv \lnot p \lor \lnot q \equiv \lnot(p\land q)\equiv$ I am not going to the movies and swimming this weekend.

\qs{1.6}{
	For each of the following, write down a truth table for the statement, and determine whether the statement is a tautology, a contradiction, or neither.
	\begin{enumerate}[label=(\roman*)]
		\item $(\lnot p\land (p\lthen q)) \lthen \lnot q$
		\item $(p\lthen(q\lor r)) \liff ((p\land\lnot q)\lthen r)$
	\end{enumerate}
}
\sol (i)
For fun and practice I will prove this by logical equivalence.
\begin{align*}
	(\lnot p\land (p\lthen q)) \lthen \lnot q &\equiv (\lnot p\land (\lnot p \lor q)) \lthen \lnot q \\
		&\equiv \lnot (\lnot p\land (\lnot p \lor q)) \lor \lnot q \\
		&\equiv \lnot (\lnot p\land (\lnot p \lor q) \land q) \\
		&\equiv \lnot ((\lnot p\land q) \land (\lnot p \lor q)) \\
		&\equiv \lnot ((\lnot p\land q \land \lnot p) \lor (\lnot p\land q \land q)) \\
		&\equiv \lnot ((\lnot p\land q) \lor (\lnot p\land q)) \\
		&\equiv \lnot (\lnot p\land (q \lor q)) \\
		&\equiv \lnot (\lnot p\land q) \\
		&\equiv p\lor \lnot q 
\end{align*}
Which is not a contradiction nor a tautology. Now we will answer the question.
\begin{center}
	\begin{tabular}{|cc||cc||c||c||c|}
		\hline
			$p$ &$q$ &$\lnot p$ &$\lnot q$ &$p\lthen q$ &$\lnot p\land(p\lthen q)$ &$(\lnot p\land(p\lthen q))\lthen\lnot q$ \\ \hline
			T & T & F & F & T & F & T \\
			T & F & F & T & F & F & T \\
			F & T & T & F & T & T & F \\
			F & F & T & T & T & T & T \\ \hline
	\end{tabular}
\end{center}

\sol (ii)
\begin{center}
	\begin{tabular}{|ccc||c||cc||ccc||}
		\hline
			$p$ &$q$ &$r$ &$\lnot q$ &$q\lor r$ &$p\land\lnot q$ &$p\lthen(q\lor r)$& $\liff$ &$(p\land\lnot q)\lthen r$ \\ \hline
			T & T & T & F & T & F & T & T  & T \\
			T & T & F & F & T & F & T & T  & T \\
			T & F & T & T & T & T & T & T  & T \\
			T & F & F & T & F & T & F & T  & F \\
			F & T & T & F & T & F & T & T  & T \\
			F & T & F & F & T & F & T & T  & T \\
			F & F & T & T & T & F & T & T  & T \\
			F & F & F & T & F & F & T & T  & T \\ \hline 
	\end{tabular}
\end{center}
Therefore the statement in question is a tautology.

\qs{1.7}{
	Write each of the following statements in the form ``if.. then...''.
	\begin{enumerate}[label=(\roman*)]
		\item A sufficient condition for the warranty to be good is that you bought the computer less than a
	year ago.
		\item Jane gets seasick whenever she is on a boat.
	\end{enumerate}
	Now negate the following two statements.
	\begin{enumerate}[label=(\roman*), start=3]
		\item If it rains, then Sue takes her umbrella.
		\item The cakes burn if the oven temperature is too high.
	\end{enumerate}
}
\sol (i) If you bought the computer less then the warranty is good. \\

\sol (ii) If Jane is on a boat, then she gets seasick. \\

\sol (iii) It rains, and Sue does not take her umbrella. \\

\sol (iv) The oven temperture is too high, and the cake does not burn.

\end{document}
