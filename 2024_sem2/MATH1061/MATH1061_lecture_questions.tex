\documentclass{report}

\input{../../latex_template/preamble}
%From M275 "Topology" at SJSU
\newcommand{\id}{\mathrm{id}}
\newcommand{\taking}[1]{\xrightarrow{#1}}
\newcommand{\inv}{^{-1}}

%From M170 "Introduction to Graph Theory" at SJSU
\DeclareMathOperator{\diam}{diam}
\DeclareMathOperator{\ord}{ord}
\newcommand{\defeq}{\overset{\mathrm{def}}{=}}

%From the USAMO .tex files
\newcommand{\ts}{\textsuperscript}
\newcommand{\dg}{^\circ}
\newcommand{\ii}{\item}

% % From Math 55 and Math 145 at Harvard
% \newenvironment{subproof}[1][Proof]{%
% \begin{proof}[#1] \renewcommand{\qedsymbol}{$\blacksquare$}}%
% {\end{proof}}

\newcommand{\liff}{\leftrightarrow}
\newcommand{\lthen}{\rightarrow}
\newcommand{\opname}{\operatorname}
\newcommand{\surjto}{\twoheadrightarrow}
\newcommand{\injto}{\hookrightarrow}
\newcommand{\On}{\mathrm{On}} % ordinals
\DeclareMathOperator{\img}{im} % Image
\DeclareMathOperator{\Img}{Im} % Image
\DeclareMathOperator{\coker}{coker} % Cokernel
\DeclareMathOperator{\Coker}{Coker} % Cokernel
\DeclareMathOperator{\Ker}{Ker} % Kernel
\DeclareMathOperator{\rank}{rank}
\DeclareMathOperator{\Spec}{Spec} % spectrum
\DeclareMathOperator{\Tr}{Tr} % trace
\DeclareMathOperator{\pr}{pr} % projection
\DeclareMathOperator{\ext}{ext} % extension
\DeclareMathOperator{\pred}{pred} % predecessor
\DeclareMathOperator{\dom}{dom} % domain
\DeclareMathOperator{\ran}{ran} % range
\DeclareMathOperator{\Hom}{Hom} % homomorphism
\DeclareMathOperator{\Mor}{Mor} % morphisms
\DeclareMathOperator{\End}{End} % endomorphism

\newcommand{\eps}{\epsilon}
\newcommand{\veps}{\varepsilon}
\newcommand{\ol}{\overline}
\newcommand{\ul}{\underline}
\newcommand{\wt}{\widetilde}
\newcommand{\wh}{\widehat}
\newcommand{\vocab}[1]{\textbf{\color{blue} #1}}
\providecommand{\half}{\frac{1}{2}}
\newcommand{\dang}{\measuredangle} %% Directed angle
\newcommand{\ray}[1]{\overrightarrow{#1}}
\newcommand{\seg}[1]{\overline{#1}}
\newcommand{\arc}[1]{\wideparen{#1}}
\DeclareMathOperator{\cis}{cis}
\DeclareMathOperator*{\lcm}{lcm}
\DeclareMathOperator*{\argmin}{arg min}
\DeclareMathOperator*{\argmax}{arg max}
\newcommand{\cycsum}{\sum_{\mathrm{cyc}}}
\newcommand{\symsum}{\sum_{\mathrm{sym}}}
\newcommand{\cycprod}{\prod_{\mathrm{cyc}}}
\newcommand{\symprod}{\prod_{\mathrm{sym}}}
\newcommand{\Qed}{\begin{flushright}\qed\end{flushright}}
\newcommand{\parinn}{\setlength{\parindent}{1cm}}
\newcommand{\parinf}{\setlength{\parindent}{0cm}}
% \newcommand{\norm}{\|\cdot\|}
\newcommand{\inorm}{\norm_{\infty}}
\newcommand{\opensets}{\{V_{\alpha}\}_{\alpha\in I}}
\newcommand{\oset}{V_{\alpha}}
\newcommand{\opset}[1]{V_{\alpha_{#1}}}
\newcommand{\lub}{\text{lub}}
\newcommand{\del}[2]{\frac{\partial #1}{\partial #2}}
\newcommand{\Del}[3]{\frac{\partial^{#1} #2}{\partial^{#1} #3}}
\newcommand{\deld}[2]{\dfrac{\partial #1}{\partial #2}}
\newcommand{\Deld}[3]{\dfrac{\partial^{#1} #2}{\partial^{#1} #3}}
\newcommand{\lm}{\lambda}
\newcommand{\uin}{\mathbin{\rotatebox[origin=c]{90}{$\in$}}}
\newcommand{\usubset}{\mathbin{\rotatebox[origin=c]{90}{$\subset$}}}
\newcommand{\lt}{\left}
\newcommand{\rt}{\right}
\newcommand{\bs}[1]{\boldsymbol{#1}}
\newcommand{\exs}{\exists}
\newcommand{\st}{\strut}
\newcommand{\dps}[1]{\displaystyle{#1}}

\newcommand{\sol}{\setlength{\parindent}{0cm}\textbf{\textit{Solution:}}\setlength{\parindent}{1cm} }
\newcommand{\solve}[1]{\setlength{\parindent}{0cm}\textbf{\textit{Solution: }}\setlength{\parindent}{1cm}#1 \Qed}

\DeclareMathOperator{\sech}{sech}
\DeclareMathOperator{\csch}{csch}
\DeclareMathOperator{\arcsec}{arcsec}
\DeclareMathOperator{\arccsc}{arccsc}
\DeclareMathOperator{\arccot}{arccot}
\DeclareMathOperator{\arsinh}{arsinh}
\DeclareMathOperator{\arcosh}{arcosh}
\DeclareMathOperator{\artanh}{artanh}
\DeclareMathOperator{\arcsch}{arcsch}
\DeclareMathOperator{\arsech}{arsech}
\DeclareMathOperator{\arcoth}{arcoth}

\newcommand{\sinx}{\sin x}          \newcommand{\arcsinx}{\arcsin x}    
\newcommand{\cosx}{\cos x}          \newcommand{\arccosx}{\arccosx}
\newcommand{\tanx}{\tan x}          \newcommand{\arctanx}{\arctan x}
\newcommand{\cscx}{\csc x}          \newcommand{\arccscx}{\arccsc x}
\newcommand{\secx}{\sec x}          \newcommand{\arcsecx}{\arcsec x}
\newcommand{\cotx}{\cot x}          \newcommand{\arccotx}{\arccot x}
\newcommand{\sinhx}{\sinh x}          \newcommand{\arsinhx}{\arsinh x}
\newcommand{\coshx}{\cosh x}          \newcommand{\arcoshx}{\arcosh x}
\newcommand{\tanhx}{\tanh x}          \newcommand{\artanhx}{\artanh x}
\newcommand{\cschx}{\csch x}          \newcommand{\arcschx}{\arcsch x}
\newcommand{\sechx}{\sech x}          \newcommand{\arsechx}{\arsech x}
\newcommand{\cothx}{\coth x}          \newcommand{\arcothx}{\arcoth x}
\newcommand{\lnx}{\ln x}
\newcommand{\expx}{\exp x}

\newcommand{\bba}{\mathbb{A}}   \newcommand{\bbn}{\mathbb{N}}
\newcommand{\bbb}{\mathbb{B}}   \newcommand{\bbo}{\mathbb{O}}
\newcommand{\bbc}{\mathbb{C}}   \newcommand{\bbp}{\mathbb{P}}
\newcommand{\bbd}{\mathbb{D}}   \newcommand{\bbq}{\mathbb{Q}}
\newcommand{\bbe}{\mathbb{E}}   \newcommand{\bbr}{\mathbb{R}}
\newcommand{\bbf}{\mathbb{F}}   \newcommand{\bbs}{\mathbb{S}}
\newcommand{\bbg}{\mathbb{G}}   \newcommand{\bbt}{\mathbb{T}}
\newcommand{\bbh}{\mathbb{H}}   \newcommand{\bbu}{\mathbb{U}}
\newcommand{\bbi}{\mathbb{I}}    \newcommand{\bbv}{\mathbb{V}}
\newcommand{\bbj}{\mathbb{J}}   \newcommand{\bbw}{\mathbb{W}}
\newcommand{\bbk}{\mathbb{K}}   \newcommand{\bbx}{\mathbb{X}}
\newcommand{\bbl}{\mathbb{L}}    \newcommand{\bby}{\mathbb{Y}}
\newcommand{\bbm}{\mathbb{M}}   \newcommand{\bbz}{\mathbb{Z}}

\newcommand{\lb}{\left(}
\newcommand{\rb}{\right)}
\newcommand{\lbr}{\left\lbrace}
\newcommand{\rbr}{\right\rbrace}
\newcommand{\lsb}{\left[}
\newcommand{\rsb}{\right]}
\newcommand{\suchthat}{\medspace\middle|\medspace}
\newcommand{\bracks}[1]{\lb #1 \rb}
\newcommand{\braces}[1]{\lbr #1 \rbr}
\newcommand{\sqbracks}[1]{\lsb #1 \rsb}

\renewcommand{\floor}[1]{\lfloor #1 \rfloor}
\renewcommand{\ceil}[1]{\lceil #1 \rceil}

\newcommand{\cd}{\cdot}
\newcommand{\tf}{\therefore}
\newcommand{\Let}{\text{Let }}
\newcommand{\Given}{\text{Given }}
\newcommand{\Suppose}{\text{Suppose }}
\newcommand{\WeSee}{\text{We see }}
\newcommand{\So}{\text{So }}

\newcommand{\QED}{\hfill \qed}

\renewcommand{\dd}[1]{\frac{d}{d#1}}
\newcommand{\dyd}[2][y]{\frac{d#1}{d#2}}

\newcommand{\ddx}{\dd{x}}       \newcommand{\ddxsq}{\dyd[^2]{x^2}}
\newcommand{\ddy}{\dd{y}}       \newcommand{\ddysq}{\dyd[^2]{y^2}}
\newcommand{\ddu}{\dd{u}}       \newcommand{\ddusq}{\dyd[^2]{u^2}}
\newcommand{\ddv}{\dd{v}}       \newcommand{\ddvsq}{\dyd[^2]{v^2}}

\newcommand{\dydx}{\dyd{x}}     \newcommand{\dydxsq}{\dyd[^2y]{x^2}}
\newcommand{\dfdx}{\dyd[f]{x}}  \newcommand{\dfdxsq}{\dyd[^2f]{x^2}}
\newcommand{\dudx}{\dyd[u]{x}}  \newcommand{\dudxsq}{\dyd[^2u]{x^2}}
\newcommand{\dvdx}{\dyd[v]{x}}  \newcommand{\dvdxsq}{\dyd[^2v]{x^2}}

% Mathfrak primes
\newcommand{\km}{\mathfrak{m}}
\newcommand{\kp}{\mathfrak{p}}
\newcommand{\kq}{\mathfrak{q}}

%---------------------------------------
% Blackboard Math Fonts :-
%---------------------------------------
\newcommand{\bba}{\mathbb{A}}   \newcommand{\bbn}{\mathbb{N}}
\newcommand{\bbb}{\mathbb{B}}   \newcommand{\bbo}{\mathbb{O}}
\newcommand{\bbc}{\mathbb{C}}   \newcommand{\bbp}{\mathbb{P}}
\newcommand{\bbd}{\mathbb{D}}   \newcommand{\bbq}{\mathbb{Q}}
\newcommand{\bbe}{\mathbb{E}}   \newcommand{\bbr}{\mathbb{R}}
\newcommand{\bbf}{\mathbb{F}}   \newcommand{\bbs}{\mathbb{S}}
\newcommand{\bbg}{\mathbb{G}}   \newcommand{\bbt}{\mathbb{T}}
\newcommand{\bbh}{\mathbb{H}}   \newcommand{\bbu}{\mathbb{U}}
\newcommand{\bbi}{\mathbb{I}}   \newcommand{\bbv}{\mathbb{V}}
\newcommand{\bbj}{\mathbb{J}}   \newcommand{\bbw}{\mathbb{W}}
\newcommand{\bbk}{\mathbb{K}}   \newcommand{\bbx}{\mathbb{X}}
\newcommand{\bbl}{\mathbb{L}}   \newcommand{\bby}{\mathbb{Y}}
\newcommand{\bbm}{\mathbb{M}}   \newcommand{\bbz}{\mathbb{Z}}

%---------------------------------------
% Roman Math Fonts :-
%---------------------------------------
\newcommand{\rma}{\mathrm{A}}   \newcommand{\rmn}{\mathrm{N}}
\newcommand{\rmb}{\mathrm{B}}   \newcommand{\rmo}{\mathrm{O}}
\newcommand{\rmc}{\mathrm{C}}   \newcommand{\rmp}{\mathrm{P}}
\newcommand{\rmd}{\mathrm{D}}   \newcommand{\rmq}{\mathrm{Q}}
\newcommand{\rme}{\mathrm{E}}   \newcommand{\rmr}{\mathrm{R}}
\newcommand{\rmf}{\mathrm{F}}   \newcommand{\rms}{\mathrm{S}}
\newcommand{\rmg}{\mathrm{G}}   \newcommand{\rmt}{\mathrm{T}}
\newcommand{\rmh}{\mathrm{H}}   \newcommand{\rmu}{\mathrm{U}}
\newcommand{\rmi}{\mathrm{I}}   \newcommand{\rmv}{\mathrm{V}}
\newcommand{\rmj}{\mathrm{J}}   \newcommand{\rmw}{\mathrm{W}}
\newcommand{\rmk}{\mathrm{K}}   \newcommand{\rmx}{\mathrm{X}}
\newcommand{\rml}{\mathrm{L}}   \newcommand{\rmy}{\mathrm{Y}}
\newcommand{\rmm}{\mathrm{M}}   \newcommand{\rmz}{\mathrm{Z}}

%---------------------------------------
% Calligraphic Math Fonts :-
%---------------------------------------
\newcommand{\cla}{\mathcal{A}}   \newcommand{\cln}{\mathcal{N}}
\newcommand{\clb}{\mathcal{B}}   \newcommand{\clo}{\mathcal{O}}
\newcommand{\clc}{\mathcal{C}}   \newcommand{\clp}{\mathcal{P}}
\newcommand{\cld}{\mathcal{D}}   \newcommand{\clq}{\mathcal{Q}}
\newcommand{\cle}{\mathcal{E}}   \newcommand{\clr}{\mathcal{R}}
\newcommand{\clf}{\mathcal{F}}   \newcommand{\cls}{\mathcal{S}}
\newcommand{\clg}{\mathcal{G}}   \newcommand{\clt}{\mathcal{T}}
\newcommand{\clh}{\mathcal{H}}   \newcommand{\clu}{\mathcal{U}}
\newcommand{\cli}{\mathcal{I}}   \newcommand{\clv}{\mathcal{V}}
\newcommand{\clj}{\mathcal{J}}   \newcommand{\clw}{\mathcal{W}}
\newcommand{\clk}{\mathcal{K}}   \newcommand{\clx}{\mathcal{X}}
\newcommand{\cll}{\mathcal{L}}   \newcommand{\cly}{\mathcal{Y}}
\newcommand{\calm}{\mathcal{M}}  \newcommand{\clz}{\mathcal{Z}}

%---------------------------------------
% Fraktur  Math Fonts :-
%---------------------------------------
\newcommand{\fka}{\mathfrak{A}}   \newcommand{\fkn}{\mathfrak{N}}
\newcommand{\fkb}{\mathfrak{B}}   \newcommand{\fko}{\mathfrak{O}}
\newcommand{\fkc}{\mathfrak{C}}   \newcommand{\fkp}{\mathfrak{P}}
\newcommand{\fkd}{\mathfrak{D}}   \newcommand{\fkq}{\mathfrak{Q}}
\newcommand{\fke}{\mathfrak{E}}   \newcommand{\fkr}{\mathfrak{R}}
\newcommand{\fkf}{\mathfrak{F}}   \newcommand{\fks}{\mathfrak{S}}
\newcommand{\fkg}{\mathfrak{G}}   \newcommand{\fkt}{\mathfrak{T}}
\newcommand{\fkh}{\mathfrak{H}}   \newcommand{\fku}{\mathfrak{U}}
\newcommand{\fki}{\mathfrak{I}}   \newcommand{\fkv}{\mathfrak{V}}
\newcommand{\fkj}{\mathfrak{J}}   \newcommand{\fkw}{\mathfrak{W}}
\newcommand{\fkk}{\mathfrak{K}}   \newcommand{\fkx}{\mathfrak{X}}
\newcommand{\fkl}{\mathfrak{L}}   \newcommand{\fky}{\mathfrak{Y}}
\newcommand{\fkm}{\mathfrak{M}}   \newcommand{\fkz}{\mathfrak{Z}}


\title{\Huge{MATH1061}\\Discrete Mathematics I\\\huge{In-Lecture Questions}}
\author{\huge{Michael Kasumagic, s4430266}}
\date{\huge{Semester 2, 2024}}

\begin{document}

\maketitle
\newpage% or \cleardoublepage
% \pdfbookmark[<level>]{<title>}{<dest>}
\pdfbookmark[section]{\contentsname}{toc}
\tableofcontents
\pagebreak

\chapter{Week 1}
\section{Lecture 2}
\subsection*{Questions}
\qs{}{
	Which of the following are statements?
	\begin{enumerate}[label=(\alph*)]
		\item "Is it going to rain tomobbrow?"
		\item "She is happy."
		\item "23 July 2024 is a Tuesday"
		\item $x = 5y + 2$
		\item $65 < 2$
	\end{enumerate}
}
\sol (a) No, a question. (b) No, "she" undefined. (c) Yes. (d) No, $x$, $y$ undefined. (e) Yes.

\qs{}{
	Let $p$, $q$, $r$ be statements.
	\begin{itemize}
		\item $p$ = ``it is cold.''
		\item $q$ = ``it is snowing.''
		\item $r$ = ``it is sunny.''
	\end{itemize}
	Translate these to symbols:
	\begin{enumerate}[label=(\alph*)]
		\item ``It is not cold but it is snowing.''
		\item ``It is neither snowing nor cold, but it is sunny.''
	\end{enumerate}
	Translate these to English:
	\begin{enumerate}[label=(\alph*),start=3]
		\item $\lnot p\land q$
		\item $(p\land q)\lor r$
	\end{enumerate}
}
\sol (a) $\lnot p\land q$ (b) $\lnot p \land \lnot q \land r$ (c) ``It is not cold but it is snowing'' (d) ``It is either snowing and cold, or sunny, or it's both.''

\qs{}{Construct the truth table for $(p\land \lnot q)\lor (q\land r)$}
\sol \begin{center}
	\begin{tabular}{|ccc||c||cc||c|}
		\hline
		$p$ & $q$ & $r$ & $\lnot q$ & $p\land\lnot q$ & $q\land r$ & $(p\land\lnot q)\lor(q\land r)$ \\ \hline
		T & T & T & F & F & T & T \\
		T & T & F & F & F & F & F \\
		T & F & T & T & T & F & T \\
		T & F & F & T & T & F & T \\
		F & T & T & F & F & T & T \\
		F & T & F & F & F & F & F \\
		F & F & T & T & F & F & F \\
		F & F & F & T & F & F & F \\ \hline
	\end{tabular}
\end{center}

\qs{}{Using De Morgan's Law, write down a statement which is logically equivalent to the negation of ``5 is even and 6 is even.''}
\sol ``5 is even and 6 is even.'' $\equiv p \land q$. The solution we want is the negation, $\lnot(p\land q)$, which, by De Morgan's Law is the same as $\lnot p \lor \lnot q$ which in English is ``5 is odd or 6 is odd.''

\qs{}{
	Show that
	$$\lnot((\lnot p\land q) \lor (\lnot p \land \lnot q)) \equiv p$$
	using a truth table, and by laws of logical equivalence.
}
\begin{center}
	\begin{tabular}{|cc||cc||cc||c||c|}
		\hline
		$p$ & $q$ & $\lnot p$ & $\lnot q$ & $\lnot p\land q$ & $\lnot p\land\lnot q$ & $(\lnot p\land q) \lor (\lnot p\land\lnot q)$ & $\lnot((\lnot p\land q) \lor (\lnot p\land\lnot q))$ \\ \hline
		T & T & F & F & F & F & F & T \\
		T & F & F & T & F & F & F & T \\
		F & T & T & F & T & F & T & F \\
		F & F & T & T & F & T & T & F \\ \hline
	\end{tabular}
\end{center}
$\tf\lnot((\lnot p\land q) \lor (\lnot p \land \lnot q)) \equiv p$ by exhaustion.
\begin{align*}
	\lnot((\lnot p\land q) \lor (\lnot p \land \lnot q)) &\equiv \lnot(\lnot p\land q) \land \lnot(\lnot p \land \lnot q) \tag*{(\text{De Morgan's Law})}\\
		&\equiv (p\lor \lnot q) \land (p \lor q) \tag*{(\text{De Morgan's Law})} \\
		&\equiv p \lor (q \land\lnot q) \tag*{(Distributivity)} \\
		&\equiv p \lor \top \tag*{(Negation Law)} \\
		&\equiv p \tag*{(Identity)} \\
		\tag*{\qed}
\end{align*}
$\tf\lnot((\lnot p\land q) \lor (\lnot p \land \lnot q)) \equiv p$ by logical equivalence.

\newpage
\section{Lecture 3}
\subsection*{Questions}
\qs{}{
	Which of the following sentences have the same meaning as ``If I am worried then I did not sleep''?
	\begin{enumerate}[label=(\alph*)]
		\item if I am worried then I do not sleep.
		\item if I am not worried then I do sleep.
		\item If I do not sleep then I am worried.
		\item I am worried and I do sleep.
		\item If I do sleep then I am not worried.
		\item I am worried or I do not sleep
		\item I do not sleep or I am not worried.
	\end{enumerate}
}
\sol 
\begin{itemize}
	\item Original: $p\lthen \lnot q$
	\item (a): $p \lthen \lnot q$, equivalent.
	\item (b): $\lnot p \lthen q$, not equivalent.
	\item (c): $\lnot q \lthen p$, not equivalent.
	\item (d): $p\land q$, not equivalent.
	\item (e): $q \lthen\lnot p$, equivalent, the contrapositive.
	\item (f): $p \lor \lnot q$, not equivalent.
	\item (g): $\lnot q \lor \lnot p$, equivalent, logically equivalent.
\end{itemize} 

\qs{}{
	Express the operations $\lor$, $\lthen$, and $\liff$ using only $\lnot$ and $\land$.
}
\sol
\begin{align*}
	p \lor q &\equiv \lnot(\lnot (p \lor q)) \\
		&\equiv \lnot (\lnot p \land \lnot q) \\
	p \lthen q &\equiv \lnot p\lor q \\
		&\equiv \lnot (\lnot (\lnot p\lor q))\\
		&\equiv \lnot (p\land \lnot q)\\
	p \liff q &\equiv (p\lthen q) \land (q\lthen p) \\
		&\equiv (\lnot p\lor q) \land (\lnot q\lor p) \\
		&\equiv \lnot(\lnot(\lnot p\lor q)) \land \lnot(\lnot(\lnot q\lor p)) \\
		&\equiv \lnot(p\land \lnot q) \land \lnot(q\land \lnot p)
\end{align*}

\qs{Challenge}{
	Consider the NAND operation $p\lnand q \equiv \lnot(p\land q)$ can you express $\land$, $\lor$, $\lnot$, and $\lthen$ using only $\lnand$ operations? Can you express using only $\lnot$ and $\lxor$?
}
\sol Generating expressions using only NANDs:
\begin{align*}
	\lnot p &\equiv \lnot(p \land p) \\
		&\equiv p\lnand p \\
	p\land q &\equiv (p\land q)\land(p\land q) \\
		&\equiv \lnot (\lnot(p\land q)\land\lnot(p\land q)) \\
		&\equiv \lnot((p\lnand q)\land (p\lnand q)) \\
		&\equiv (p\lnand q)\lnand (p\lnand q) \\
	p\lor q &\equiv \lnot(\lnot(p\lor q)) \\
		&\equiv \lnot(\lnot p\land \lnot q) \\
		&\equiv \lnot p\lnand\lnot q \\
		&\equiv (p\lnand p)\lnand(q\lnand q) \\
	p \lthen q &\equiv \lnot p\lor q \\
		&\equiv \lnot (\lnot (\lnot p\lor q)) \\
		&\equiv \lnot (p\land \lnot q) \\
		&\equiv p\lnand\lnot q \\
		&\equiv p\lnand(q\lnand q)
\end{align*}
These sick fucks had me testing and observing truth tables for two hours. I was suspicious at times, but I assumed there must be a solution\dots\\
There is not. No matter how many XOR and NOT operations you apply, ultimately, you will always have 2 Falses and 2 Trues, or 4 Falses, or 4 Trues. \textbf{AHHHHHHHH}.

\qs{}{
	Show that $\lnot(p\lthen q) \not\equiv \lnot p\lthen \lnot q$.
}
\sol By counterexample. Suppose $p=\true\And q=\true$
\begin{align*}
\lnot(p\lthen q) &\equiv \lnot(\true\lthen\true) \\
	&\equiv \lnot(\true) \\
	&\equiv \false \\
\lnot p\lthen\lnot q &\equiv \lnot\true\lthen\lnot\true \\
	&\equiv \false\lthen\false \\
	&\equiv \true \not\equiv\lnot(p\lthen q) \tag*{\qed}
\end{align*}

\qs{}{
	Which of the following sentences have the opposite truth valuse as ``If I am worried then I did not sleep''?
	\begin{enumerate}[label=(\alph*)]
		\item if I am worried then I do not sleep.
		\item if I am not worried then I do sleep.
		\item If I do not sleep then I am worried.
		\item I am worried and I do sleep.
		\item If I do sleep then I am not worried.
		\item I am worried or I do not sleep
		\item I do not sleep or I am not worried.
	\end{enumerate}
}
\sol 
\begin{itemize}
	\item Original: $p\lthen \lnot q$
	\item (a): $p \lthen \lnot q$, No, equivalent statement.
	\item (b): $\lnot p \lthen q$, No, True 3/4 times, same as our statement. Not possible to be opposite.
	\item (c): $\lnot q \lthen p$, No, True 3/4 times.
	\item (d): $p\land q$, Yes! $p\lthen\lnot q \equiv \lnot p\lor \lnot q\equiv \lnot(p\land q)$, exactly the opposite.
	\item (e): $q \lthen\lnot p$, No, equivalent statement.
	\item (f): $p \lor \lnot q$, No, True 3/4 times.
	\item (g): $\lnot q \lor \lnot p$, No, equivalent statement.
\end{itemize} 

\qs{}{
	Show that $$p\lthen(q\lor r)\equiv (p\land\lnot q)\lthen r.$$
}
\sol
\begin{align*}
	p\lthen (q\lor r) &\equiv \lnot p \lor (q\lor r) \\
		&\equiv (\lnot p \lor q)\lor r \\
		&\equiv \lnot(\lnot(\lnot p \lor q))\lor r \\
		&\equiv \lnot(p \land\lnot q)\lor r \\
		&\equiv (p\land q)\lthen r
\end{align*}

\qs{}{
	Let $n$ be a positive integer. Find conditions that are:
	\begin{enumerate}[label=(\alph*)]
		\item necessary, but not sufficent for $n$ to be a multiple of 10.
		\item sufficient but not necessary for $n$ to be divisible by 10.
		\item necessary and sufficient for $n$ to be divisible by 10.
	\end{enumerate}
}
\sol\begin{enumerate}[label=(\alph*)]
	\item $n$ is a multiple of 10 $\lthen$ $n$ is necessarily even.
	\item $n$ is a multiple of 50 $\lthen$ sufficient to conclude that $n$ is divisible by 10.
	\item $n$'s last digit is a 0 $\liff$ $n$ is divisible by 10.
\end{enumerate}

\chapter{Week 2}
\section{Lecture 4}
\subsection*{Questions}
\qs{}{
	Write the following arguments symbolically
	\begin{itemize}
		\item If wages are raised, then buying increases.
		\item If there is a depression, then buying does not increase.
		\item Therefore, there is not a depression, or wages are not raised.
	\end{itemize}
	Decide whether this argument is valid, using three methods:
	\begin{enumerate}[label=(\alph*)]
		\item a truth table
		\item rules of inference
		\item configuring truth values
	\end{enumerate}
}
\sol 
\begin{gather*}
	\Let p = \text{``Wages are raised.''} \\
	\Let q = \text{``There is a depression.''} \\
	\Let r = \text{``Buying increases.''} \\
	\argument{
		1. & p\lthen r & \\ 
		2. & q\lthen\lnot r & \\
	}{\lnot p\lor\lnot q &}
\end{gather*}

\sol (a)
\begin{center}
	\begin{tabular}{|ccc||cc||c|}
		\hline
		\multicolumn{3}{|c||}{Variables} & \multicolumn{2}{c||}{Premises} & \multicolumn{1}{c|}{Conclusion} \\
		$p$ & $q$ & $r$ & $p\lthen r$ & $q\lthen\lnot r$ & $\lnot p\lor\lnot q$ \\ \hline
		T & T & T & T & F & F \\
		T & T & F & F & T & F \\ \cline{4-6}
		T & F & T & T & T & T \\ \cline{4-6}
		T & F & F & F & T & T \\
		F & T & T & T & F & T \\ \cline{4-6}
		F & T & F & T & T & T \\ \cline{4-6}
		F & F & T & T & T & T \\ \cline{4-6}
		F & F & F & T & T & T \\ \cline{4-6} \hline
	\end{tabular}
\end{center}
Consider rows 3, 6, 7, and 8. Whenever all the premises are true, the conclusion is true. Therefore the argument is valid. 
\begin{flushright}$\qed$\\\end{flushright}

\sol (b)
\begin{gather*}
	\argument{
		1. & p\lthen r 			 			& \text{} \\
		2. & q\lthen\lnot r 			& \text{} \\
		3. & r \lthen\lnot q 			& (\text{Contrapositive of 2.}) \\
		4. & p \lthen\lnot q 			& (\text{Transitivity of 1. and 3.}) \\ 
		5. & \lnot p \lor\lnot q	& (\text{Expansion of }\lthen) \\
	}{\lnot p\lor\lnot q & \text{}}				 
\end{gather*}
Therefore, by using laws of inference, we've proven that the conclusion follows from the premises. Therefore the argument is valid.
\begin{flushright}$\qed$\\\end{flushright}

\sol (c)
\begin{gather*}
	\longintertext{Suppose the argument is invalid\\
		Then, the premises are all true, and the conclusion is false.\\
		So, $\lnot p \lor \lnot q$ is False.}
	\argument{
		1. & p\lthen r 													& \text{} \\
		2. & q\lthen\lnot r 										& \text{} \\		
		3. & \lnot (p\land q) \text{ is False} 	& \text{(De Morgan's Law)} \\
		4. & p\land q \text{ is True} 					& \text{(Follows from 3.)} \\							
		5. & p \text{ is True} 									& \text{(Specalisation)} \\			
		6. & q \text{ is True} 									& \text{(Specalisation)} \\			
		7. & r \text{ is True} 									& \text{(Follows from 1., given 5.)} \\			
		8. & r \text{ is False}\qquad \contra 	& \text{(Follows from 2., given 6.)} \\
	}{\lnot p\lor\lnot q &}
\end{gather*}
We've identified a contradiction! If we assume invalidity, we see contradictions arise. Which means our original assumption was incorrect. Which means the argument is valid.
\begin{flushright}$\qed$\\\end{flushright}

\qs{}{
	Is the following argument valid? Again, use all three methods.
	$$
		\argument{
			1. & p\lthen q &\\
			2. & p\lor r &\\
			3. & p\lor\lnot r &\\
		}{q &}
	$$
}
\sol (a)
\begin{center}
	\begin{tabular}{|ccc||ccc||c|l}
		\cline{1-7}
		\multicolumn{3}{|c||}{Variables} & \multicolumn{3}{c||}{Premises} & \multicolumn{1}{c|}{Conclusion} & \\ 
		$p$ & $q$ & $r$ & $p\lthen q$ & $p\lor r$ & $p\lor\lnot r$ & $q$ \\ \cline{1-7}
		T & T & T & T & T & T & T & $\leftarrow$\\
		T & T & F & T & T & T & T & $\leftarrow$\\
		T & F & T & F & T & T & F & \\
		T & F & F & F & T & T & F & \\
		F & T & T & T & T & F & T & \\
		F & T & F & T & F & T & T & \\
		F & F & T & T & T & F & F & \\
		F & F & F & T & F & T & F & \\ \cline{1-7} 
	\end{tabular}
\end{center}
Observing rows 1, and 2, we can see that when all the premises are True, the conclusion is True. Therefore the argument is valid.
\begin{flushright}$\qed$\end{flushright}

\sol (b)
$$
\argument{
	1. & p\lthen q &\\
	2. & p\lor r &\\
	3. & p\lor\lnot r &\\
	4. & \lnot r\lor p & (\text{Commutativity of 3.}) \\
	5. & r \lthen p	& (\text{Logical Equivalence of 4.}) \\
	6. & r \lthen q & (\text{Transitivity of 5. and 1.}) \\
	7. & q & (\text{Division of Cases of 2. by 1., 6.}) \\
}{q &}
$$
Therefore, we've shown using rules of inference that the Argument is logically valid.
\begin{flushright}$\qed$\end{flushright}

\sol (c) \\
Suppose the argument is invalid\\
Then, the premises are all true, and the conclusion is false.\\
So, $q$ is False.
$$
	\argument{
		1. & p\lthen q &\\
		2. & p\lor r &\\
		3. & p\lor\lnot r &\\
		4. & p \text{ is False } & (\text{We know from 1. given } q) \\
		5. & r \text{ is True } & (\text{We know from 2. given } p) \\
		6. & r \text{ is False }\qquad\contra & (\text{We know from 3. given } p) \\
	}{q &}
$$
We've identified a contradiction! If we assume invalidity, we see contradictions arise. Which means our original assumption was incorrect. Which means the argument is valid.

\qs{}{
	Is the following argument valid?
	$$
		\argument{
			1. & p\lthen q & \\
			2. & q\lthen r & \\
			3. & \lnot p\lor\lnot q & \\
		}{r &}
	$$
}	
\sol I'll use rules of inference, since it's my weakest solution method.
\begin{gather*}
	\argument{
		1. & p\lthen q & \\
		2. & q\lthen r & \\
		3. & \lnot p\lor\lnot q & \\
		4. & p\lthen \lnot q\qquad\contra & (\text{Collection of 1.}\lor\text{ to }\lthen) \\
	}{r &}
\end{gather*}
This is a contradiction, because $p$ cannot similateously imply $q$ and $\lnot q$, no matter what truth value it takes. Therefore the argument is invalid.
\begin{flushright}$\qed$\end{flushright}

\qs{}{
	Determine whether or not the following argument is valid, using all three methods.
	\begin{itemize}
		\item If new messages are queued, then the filesystem is locked.
		\item The filesystem is not locked if and only if the system is functioning normally.
		\item New messages will not be sent to the message buffer only if they are queued.
		\item New messages will not be sent to the message buffer.
		\item Therefore, the system is functioning normally.
	\end{itemize}
}
\sol \begin{gather*}
	\longintertext{Let $p$ be the statement ``New messages are queued.''\\
	Let $q$ be the statement ``The filesystem is locked.''\\
	Let $r$ be the statement ``New messages are sent to the message buffer.''\\
	Let $s$ be the statement ``The system is functioning normally.''\\
	Then the argument can be written symbolically}
	\argument{
		1. & p\lthen q &\\
		2. & \lnot q\liff s &\\
		3. & p \lthen \lnot r &\\
		4. & \lnot r &\\
	}{s&\\}
	\longintertext{Let's check this arguments validity with a truth table.}
	\begin{array}{|cccc||cccc||c|l}
		\cline{1-9}
		\multicolumn{4}{|c||}{\text{Variables}} & \multicolumn{4}{c||}{\text{Premises}} & \multicolumn{1}{c|}{\text{Conclusion}} \\
		p & q & r & s & p\lthen q & \lnot q\liff s & p \lthen \lnot r & \lnot r & s \\ \cline{1-9}
		T & T & T & T & T & F & F & F & T & \\
		T & T & T & F & T & T & F & F & F & \\
		T & T & F & T & T & F & T & T & T & \\
		T & T & F & F & T & T & T & T & F & \leftarrow \\
		T & F & T & T & F & T & F & F & T & \\
		T & F & T & F & F & F & F & F & F & \\
		T & F & F & T & F & T & T & T & T & \\
		T & F & F & F & F & F & T & T & F & \\
		F & T & T & T & T & F & T & F & T & \\
		F & T & T & F & T & T & T & F & F & \\
		F & T & F & T & T & F & T & T & T & \\
		F & T & F & F & T & T & T & T & F & \leftarrow \\
		F & F & T & T & T & T & T & F & T & \\
		F & F & T & F & T & F & T & F & F & \\
		F & F & F & T & T & T & T & T & T & \\
		F & F & F & F & T & F & T & T & F & \\ \cline{1-9}
	\end{array}
	\longintertext{Observe rows 4 and 12, where in all the premises are true, but the conclusion is false. Therefore, the argument is invalid.\\
	Now let's prove this invalidity using rules of inference.}
	\argument{
		1. & p\lthen q &\\
		2. & \lnot q\liff s &\\
		3. & p\lthen \lnot r &\\
		4. & \lnot r &\\
		5. & s\lthen\lnot q & (\text{Falls from biconditional 2.}) \\
		6. & q\lthen\lnot s & (\text{Contrapositve of 5.}) \\
		7. & p\lthen\lnot s & (\text{Transitivity of 1. to 6.})\\
		8. & \lnot p\lor q & (\text{Expansion of 1.}\lthen)\\ 
		9. & \lnot p\lor\lnot r & (\text{Expansion of 3.}\lthen)\\
	}{s&\\}
	\longintertext{Therefore, we can see that the argument is invalid.\\
	Finally, lets check in validity by searching for values for the variables.\\
	Let's assume the argument is valid. All the premises are true, and the conclusion is false.\\
	$s$ is false.}
	\argument{
		1. & p\lthen q &\\
		2. & \lnot q\liff s &\\
		3. & p \lthen \lnot r &\\
		4. & \lnot r &\\
		5. & s\lthen \lnot q & (\text{Falls from biconditional 2.})\\
		6. & \lnot q \text{ is False} & (\text{Follows from 5.})\\
		7. & q \text{ is True} & (\text{Follows from 6.})\\
		8. & p \text{ is True} & (\text{Follows from 1. given }q)\\
		9. & r \text{ is False} & (\text{Negation of 4.})\\
	}{s&\\}
\end{gather*}

\newpage
\section{Lecture 5}
\subsection*{Questions}
\qs{}{example}

\newpage
\section{Lecture 6}
\subsection*{Questions}
\qs{}{example}

\chapter{Week 3}
\section{Lecture 7}
\subsection*{Questions}
\qs{}{example}

\newpage
\section{Lecture 8}
\subsection*{Questions}
\qs{}{}

\newpage
\section{Lecture 9}
\subsection*{Questions}
\qs{}{}

\chapter{Week 4}
\section{Lecture 10}
\subsection*{Questions}
\qs{}{}

\newpage
\section{Lecture 11}
\subsection*{Questions}
\qs{}{}

\newpage
\section{Lecture 12}
\subsection*{Questions}
\qs{}{}

\chapter{Week 5}
\section{Lecture 13}
\subsection*{Questions}
\qs{}{}

\newpage
\section{Lecture 14}
\subsection*{Questions}
\qs{}{}

\newpage
\section{Lecture 15}
\subsection*{Questions}
\qs{}{}

\chapter{Week 6}
\section{Lecture 16}
\subsection*{Questions}
\qs{}{}

\newpage
\section{Lecture 17}
\section*{Questions}
\qs{}{}

\newpage
\section{Lecture 18}
\subsection*{Questions}
\qs{}{}

\chapter{Week 7}
\section{Lecture 19}
\subsection*{Questions}
\qs{}{}

\newpage
\section{Lecture 20}
\subsection*{Questions}
\qs{}{}

\newpage
\section{Lecture 21}
\subsection*{Questions}
\qs{}{}


\end{document}
