\documentclass[a4paper, 11pt]{report}

%%%%%%%%%%%%%%%%%%%%%%%%%%%%%%%%%
% PACKAGE IMPORTS
%%%%%%%%%%%%%%%%%%%%%%%%%%%%%%%%%
\usepackage[tmargin=2cm,rmargin=1in,lmargin=1in,margin=0.85in,bmargin=2cm,footskip=.2in]{geometry}
\usepackage[none]{hyphenat}
\usepackage{amsmath,amsfonts,amsthm,amssymb,mathtools}
\allowdisplaybreaks
\usepackage{undertilde}
\usepackage{xfrac}
\usepackage[makeroom]{cancel}
\usepackage{mathtools}
\usepackage{bookmark}
\usepackage{enumitem}
\usepackage{kbordermatrix}
\renewcommand{\kbldelim}{(} % Change left delimiter to (
\renewcommand{\kbrdelim}{)} % Change right delimiter to )
\usepackage{hyperref,theoremref}
\hypersetup{
	pdftitle={Assignment},
	colorlinks=true, linkcolor=doc!90,
	bookmarksnumbered=true,
	bookmarksopen=true
}
\usepackage[most,many,breakable]{tcolorbox}
\usepackage{xcolor}
\usepackage{varwidth}
\usepackage{varwidth}
\usepackage{etoolbox}
%\usepackage{authblk}
\usepackage{nameref}
\usepackage{multicol,array}
\usepackage{tikz-cd}
\usepackage[ruled,vlined,linesnumbered]{algorithm2e}
\usepackage{comment} % enables the use of multi-line comments (\ifx \fi) 
\usepackage{import}
\usepackage{xifthen}
\usepackage{pdfpages}
\usepackage{svg}
\usepackage{transparent}
\usepackage{pgfplots}
\pgfplotsset{compat=1.18}
\usetikzlibrary{calc}
\usetikzlibrary{graphs}
\usetikzlibrary{graphs.standard}
% \usetikzlibrary{graphdrawing}

\newcommand\mycommfont[1]{\footnotesize\ttfamily\textcolor{blue}{#1}}
\SetCommentSty{mycommfont}
\newcommand{\incfig}[1]{%
    \def\svgwidth{\columnwidth}
    \import{./figures/}{#1.pdf_tex}
}


\usepackage{tikzsymbols}
% \renewcommand\qedsymbol{$\Laughey$}

\definecolor{commentgreen}{RGB}{2,112,10}
%%
%% Julia definition (c) 2014 Jubobs
%%
\lstdefinelanguage{Julia}%
  {morekeywords={abstract,break,case,catch,const,continue,do,else,elseif,%
      end,export,false,for,function,immutable,import,importall,if,in,%
      macro,module,otherwise,quote,return,switch,true,try,type,typealias,%
      using,while},%
   sensitive=true,%
   alsoother={$},%
   morecomment=[l]\#,%
   morecomment=[n]{\#=}{=\#},%
   morestring=[s]{"}{"},%
   morestring=[m]{'}{'},%
}[keywords,comments,strings]%

\lstset{%
    language        	= Julia,
    basicstyle      	= \ttfamily,
    keywordstyle    	= \bfseries\color{blue},
    stringstyle     	= \color{magenta},
    commentstyle    	= \color{commentgreen},
    showstringspaces	= false,
		numbers						= left,
		tabsize						= 4,
}

\definecolor{stringyellow}{RGB}{227, 78, 48}
%% 
%% Shamelessly stolen from Vivi on Stackoverflow
%% https://tex.stackexchange.com/questions/75116/what-can-i-use-to-typeset-matlab-code-in-my-document
%%
\lstset{language=Matlab,%
    %basicstyle=\color{red},
    breaklines=true,%
    morekeywords={matlab2tikz},
		morekeywords={subtitle}
    keywordstyle=\color{blue},%
    morekeywords=[2]{1}, keywordstyle=[2]{\color{black}},
    identifierstyle=\color{black},%
    stringstyle=\color{stringyellow},
    commentstyle=\color{commentgreen},%
    showstringspaces=false,%without this there will be a symbol in the places where there is a space
    numbers=left,%
		firstnumber=1,
    % numberstyle={\tiny \color{black}},% size of the numbers
    % numbersep=9pt, % this defines how far the numbers are from the text
    emph=[1]{for,end,break},emphstyle=[1]\color{red}, %some words to emphasise
    %emph=[2]{word1,word2}, emphstyle=[2]{style},    
}

%% 
%% Shamelessly stolen from egreg on Stackoverflow
%% https://tex.stackexchange.com/questions/280681/how-to-have-multiple-lines-of-intertext-within-align-environment
%%
\newlength{\normalparindent}
\AtBeginDocument{\setlength{\normalparindent}{\parindent}}
\newcommand{\longintertext}[1]{%
  \intertext{%
    \parbox{\linewidth}{%
      \setlength{\parindent}{\normalparindent}
      \noindent#1%
    }%
  }%
}

%\usepackage{import}
%\usepackage{xifthen}
%\usepackage{pdfpages}
%\usepackage{transparent}

%%%%%%%%%%%%%%%%%%%%%%%%%%%%%%
% SELF MADE COLORS
%%%%%%%%%%%%%%%%%%%%%%%%%%%%%%
\definecolor{myg}{RGB}{56, 140, 70}
\definecolor{myb}{RGB}{45, 111, 177}
\definecolor{myr}{RGB}{199, 68, 64}
\definecolor{mytheorembg}{HTML}{F2F2F9}
\definecolor{mytheoremfr}{HTML}{00007B}
\definecolor{mylenmabg}{HTML}{FFFAF8}
\definecolor{mylenmafr}{HTML}{983b0f}
\definecolor{mypropbg}{HTML}{f2fbfc}
\definecolor{mypropfr}{HTML}{191971}
\definecolor{myexamplebg}{HTML}{F2FBF8}
\definecolor{myexamplefr}{HTML}{88D6D1}
\definecolor{myexampleti}{HTML}{2A7F7F}
\definecolor{mydefinitbg}{HTML}{E5E5FF}
\definecolor{mydefinitfr}{HTML}{3F3FA3}
\definecolor{notesgreen}{RGB}{0,162,0}
\definecolor{myp}{RGB}{197, 92, 212}
\definecolor{mygr}{HTML}{2C3338}
\definecolor{myred}{RGB}{127,0,0}
\definecolor{myyellow}{RGB}{169,121,69}
\definecolor{myexercisebg}{HTML}{F2FBF8}
\definecolor{myexercisefg}{HTML}{88D6D1}

%%%%%%%%%%%%%%%%%%%%%%%%%%%%
% TCOLORBOX SETUPS
%%%%%%%%%%%%%%%%%%%%%%%%%%%%
\setlength{\parindent}{0pt}

%================================
% THEOREM BOX
%================================
\tcbuselibrary{theorems,skins,hooks}
\newtcbtheorem[number within=section]{Theorem}{Theorem}
{%
	enhanced,
	breakable,
	colback = mytheorembg,
	frame hidden,
	boxrule = 0sp,
	borderline west = {2pt}{0pt}{mytheoremfr},
	sharp corners,
	detach title,
	before upper = \tcbtitle\par\smallskip,
	coltitle = mytheoremfr,
	fonttitle = \bfseries\sffamily,
	description font = \mdseries,
	separator sign none,
	segmentation style={solid, mytheoremfr},
}
{th}

\tcbuselibrary{theorems,skins,hooks}
\newtcbtheorem[number within=chapter]{theorem}{Theorem}
{%
	enhanced,
	breakable,
	colback = mytheorembg,
	frame hidden,
	boxrule = 0sp,
	borderline west = {2pt}{0pt}{mytheoremfr},
	sharp corners,
	detach title,
	before upper = \tcbtitle\par\smallskip,
	coltitle = mytheoremfr,
	fonttitle = \bfseries\sffamily,
	description font = \mdseries,
	separator sign none,
	segmentation style={solid, mytheoremfr},
}
{th}


\tcbuselibrary{theorems,skins,hooks}
\newtcolorbox{Theoremcon}
{%
	enhanced
	,breakable
	,colback = mytheorembg
	,frame hidden
	,boxrule = 0sp
	,borderline west = {2pt}{0pt}{mytheoremfr}
	,sharp corners
	,description font = \mdseries
	,separator sign none
}

%================================
% Corollery
%================================
\tcbuselibrary{theorems,skins,hooks}
\newtcbtheorem[number within=section]{Corollary}{Corollary}
{%
	enhanced
	,breakable
	,colback = myp!10
	,frame hidden
	,boxrule = 0sp
	,borderline west = {2pt}{0pt}{myp!85!black}
	,sharp corners
	,detach title
	,before upper = \tcbtitle\par\smallskip
	,coltitle = myp!85!black
	,fonttitle = \bfseries\sffamily
	,description font = \mdseries
	,separator sign none
	,segmentation style={solid, myp!85!black}
}
{th}
\tcbuselibrary{theorems,skins,hooks}
\newtcbtheorem[number within=chapter]{corollary}{Corollary}
{%
	enhanced
	,breakable
	,colback = myp!10
	,frame hidden
	,boxrule = 0sp
	,borderline west = {2pt}{0pt}{myp!85!black}
	,sharp corners
	,detach title
	,before upper = \tcbtitle\par\smallskip
	,coltitle = myp!85!black
	,fonttitle = \bfseries\sffamily
	,description font = \mdseries
	,separator sign none
	,segmentation style={solid, myp!85!black}
}
{th}

%================================
% LENMA
%================================
\tcbuselibrary{theorems,skins,hooks}
\newtcbtheorem[number within=section]{Lenma}{Lenma}
{%
	enhanced,
	breakable,
	colback = mylenmabg,
	frame hidden,
	boxrule = 0sp,
	borderline west = {2pt}{0pt}{mylenmafr},
	sharp corners,
	detach title,
	before upper = \tcbtitle\par\smallskip,
	coltitle = mylenmafr,
	fonttitle = \bfseries\sffamily,
	description font = \mdseries,
	separator sign none,
	segmentation style={solid, mylenmafr},
}
{th}

\tcbuselibrary{theorems,skins,hooks}
\newtcbtheorem[number within=chapter]{lenma}{Lenma}
{%
	enhanced,
	breakable,
	colback = mylenmabg,
	frame hidden,
	boxrule = 0sp,
	borderline west = {2pt}{0pt}{mylenmafr},
	sharp corners,
	detach title,
	before upper = \tcbtitle\par\smallskip,
	coltitle = mylenmafr,
	fonttitle = \bfseries\sffamily,
	description font = \mdseries,
	separator sign none,
	segmentation style={solid, mylenmafr},
}
{th}

%================================
% PROPOSITION
%================================
\tcbuselibrary{theorems,skins,hooks}
\newtcbtheorem[number within=section]{Prop}{Proposition}
{%
	enhanced,
	breakable,
	colback = mypropbg,
	frame hidden,
	boxrule = 0sp,
	borderline west = {2pt}{0pt}{mypropfr},
	sharp corners,
	detach title,
	before upper = \tcbtitle\par\smallskip,
	coltitle = mypropfr,
	fonttitle = \bfseries\sffamily,
	description font = \mdseries,
	separator sign none,
	segmentation style={solid, mypropfr},
}
{th}

\tcbuselibrary{theorems,skins,hooks}
\newtcbtheorem[number within=chapter]{prop}{Proposition}
{%
	enhanced,
	breakable,
	colback = mypropbg,
	frame hidden,
	boxrule = 0sp,
	borderline west = {2pt}{0pt}{mypropfr},
	sharp corners,
	detach title,
	before upper = \tcbtitle\par\smallskip,
	coltitle = mypropfr,
	fonttitle = \bfseries\sffamily,
	description font = \mdseries,
	separator sign none,
	segmentation style={solid, mypropfr},
}
{th}

%================================
% CLAIM
%================================
\tcbuselibrary{theorems,skins,hooks}
\newtcbtheorem[number within=section]{claim}{Claim}
{%
	enhanced
	,breakable
	,colback = myg!10
	,frame hidden
	,boxrule = 0sp
	,borderline west = {2pt}{0pt}{myg}
	,sharp corners
	,detach title
	,before upper = \tcbtitle\par\smallskip
	,coltitle = myg!85!black
	,fonttitle = \bfseries\sffamily
	,description font = \mdseries
	,separator sign none
	,segmentation style={solid, myg!85!black}
}
{th}

%================================
% Exercise
%================================
\tcbuselibrary{theorems,skins,hooks}
\newtcbtheorem[number within=section]{Exercise}{Exercise}
{%
	enhanced,
	breakable,
	colback = myexercisebg,
	frame hidden,
	boxrule = 0sp,
	borderline west = {2pt}{0pt}{myexercisefg},
	sharp corners,
	detach title,
	before upper = \tcbtitle\par\smallskip,
	coltitle = myexercisefg,
	fonttitle = \bfseries\sffamily,
	description font = \mdseries,
	separator sign none,
	segmentation style={solid, myexercisefg},
}
{th}

\tcbuselibrary{theorems,skins,hooks}
\newtcbtheorem[number within=chapter]{exercise}{Exercise}
{%
	enhanced,
	breakable,
	colback = myexercisebg,
	frame hidden,
	boxrule = 0sp,
	borderline west = {2pt}{0pt}{myexercisefg},
	sharp corners,
	detach title,
	before upper = \tcbtitle\par\smallskip,
	coltitle = myexercisefg,
	fonttitle = \bfseries\sffamily,
	description font = \mdseries,
	separator sign none,
	segmentation style={solid, myexercisefg},
}
{th}

%================================
% EXAMPLE BOX
%================================
\newtcbtheorem[number within=section]{Example}{Example}
{%
	colback = myexamplebg
	,breakable
	,colframe = myexamplefr
	,coltitle = myexampleti
	,boxrule = 1pt
	,sharp corners
	,detach title
	,before upper=\tcbtitle\par\smallskip
	,fonttitle = \bfseries
	,description font = \mdseries
	,separator sign none
	,description delimiters parenthesis
}
{ex}

\newtcbtheorem[number within=chapter]{example}{Example}
{%
	colback = myexamplebg
	,breakable
	,colframe = myexamplefr
	,coltitle = myexampleti
	,boxrule = 1pt
	,sharp corners
	,detach title
	,before upper=\tcbtitle\par\smallskip
	,fonttitle = \bfseries
	,description font = \mdseries
	,separator sign none
	,description delimiters parenthesis
}
{ex}

%================================
% DEFINITION BOX
%================================
\newtcbtheorem[number within=section]{Definition}{Definition}{enhanced,
	before skip=2mm,after skip=2mm, colback=red!5,colframe=red!80!black,boxrule=0.5mm,
	attach boxed title to top left={xshift=1cm,yshift*=1mm-\tcboxedtitleheight}, varwidth boxed title*=-3cm,
	boxed title style={frame code={
					\path[fill=tcbcolback]
					([yshift=-1mm,xshift=-1mm]frame.north west)
					arc[start angle=0,end angle=180,radius=1mm]
					([yshift=-1mm,xshift=1mm]frame.north east)
					arc[start angle=180,end angle=0,radius=1mm];
					\path[left color=tcbcolback!60!black,right color=tcbcolback!60!black,
						middle color=tcbcolback!80!black]
					([xshift=-2mm]frame.north west) -- ([xshift=2mm]frame.north east)
					[rounded corners=1mm]-- ([xshift=1mm,yshift=-1mm]frame.north east)
					-- (frame.south east) -- (frame.south west)
					-- ([xshift=-1mm,yshift=-1mm]frame.north west)
					[sharp corners]-- cycle;
				},interior engine=empty,
		},
	fonttitle=\bfseries,
	title={#2},#1}{def}
\newtcbtheorem[number within=chapter]{definition}{Definition}{enhanced,
	before skip=2mm,after skip=2mm, colback=red!5,colframe=red!80!black,boxrule=0.5mm,
	attach boxed title to top left={xshift=1cm,yshift*=1mm-\tcboxedtitleheight}, varwidth boxed title*=-3cm,
	boxed title style={frame code={
					\path[fill=tcbcolback]
					([yshift=-1mm,xshift=-1mm]frame.north west)
					arc[start angle=0,end angle=180,radius=1mm]
					([yshift=-1mm,xshift=1mm]frame.north east)
					arc[start angle=180,end angle=0,radius=1mm];
					\path[left color=tcbcolback!60!black,right color=tcbcolback!60!black,
						middle color=tcbcolback!80!black]
					([xshift=-2mm]frame.north west) -- ([xshift=2mm]frame.north east)
					[rounded corners=1mm]-- ([xshift=1mm,yshift=-1mm]frame.north east)
					-- (frame.south east) -- (frame.south west)
					-- ([xshift=-1mm,yshift=-1mm]frame.north west)
					[sharp corners]-- cycle;
				},interior engine=empty,
		},
	fonttitle=\bfseries,
	title={#2},#1}{def}

%================================
% Solution BOX
%================================
\makeatletter
\newtcbtheorem{question}{Question}{enhanced,
	breakable,
	colback=white,
	colframe=myb!80!black,
	attach boxed title to top left={yshift*=-\tcboxedtitleheight},
	fonttitle=\bfseries,
	title={#2},
	boxed title size=title,
	boxed title style={%
			sharp corners,
			rounded corners=northwest,
			colback=tcbcolframe,
			boxrule=0pt,
		},
	underlay boxed title={%
			\path[fill=tcbcolframe] (title.south west)--(title.south east)
			to[out=0, in=180] ([xshift=5mm]title.east)--
			(title.center-|frame.east)
			[rounded corners=\kvtcb@arc] |-
			(frame.north) -| cycle;
		},
	#1
}{def}
\makeatother

%================================
% SOLUTION BOX
%================================
\makeatletter
\newtcolorbox{solution}{enhanced,
	breakable,
	colback=white,
	colframe=myg!80!black,
	attach boxed title to top left={yshift*=-\tcboxedtitleheight},
	title=Solution,
	boxed title size=title,
	boxed title style={%
			sharp corners,
			rounded corners=northwest,
			colback=tcbcolframe,
			boxrule=0pt,
		},
	underlay boxed title={%
			\path[fill=tcbcolframe] (title.south west)--(title.south east)
			to[out=0, in=180] ([xshift=5mm]title.east)--
			(title.center-|frame.east)
			[rounded corners=\kvtcb@arc] |-
			(frame.north) -| cycle;
		},
}
\makeatother

%================================
% Question BOX
%================================
\makeatletter
\newtcbtheorem{qstion}{Question}{enhanced,
	breakable,
	colback=white,
	colframe=mygr,
	attach boxed title to top left={yshift*=-\tcboxedtitleheight},
	fonttitle=\bfseries,
	title={#2},
	boxed title size=title,
	boxed title style={%
			sharp corners,
			rounded corners=northwest,
			colback=tcbcolframe,
			boxrule=0pt,
		},
	underlay boxed title={%
			\path[fill=tcbcolframe] (title.south west)--(title.south east)
			to[out=0, in=180] ([xshift=5mm]title.east)--
			(title.center-|frame.east)
			[rounded corners=\kvtcb@arc] |-
			(frame.north) -| cycle;
		},
	#1
}{def}
\makeatother

\newtcbtheorem[number within=chapter]{wconc}{Wrong Concept}{
	breakable,
	enhanced,
	colback=white,
	colframe=myr,
	arc=0pt,
	outer arc=0pt,
	fonttitle=\bfseries\sffamily\large,
	colbacktitle=myr,
	attach boxed title to top left={},
	boxed title style={
			enhanced,
			skin=enhancedfirst jigsaw,
			arc=3pt,
			bottom=0pt,
			interior style={fill=myr}
		},
	#1
}{def}

%================================
% NOTE BOX
%================================
\usetikzlibrary{arrows,calc,shadows.blur}
\tcbuselibrary{skins}
\newtcolorbox{note}[1][]{%
	enhanced jigsaw,
	colback=gray!20!white,%
	colframe=gray!80!black,
	size=small,
	boxrule=1pt,
	title=\textbf{Note:-},
	halign title=flush center,
	coltitle=black,
	breakable,
	drop shadow=black!50!white,
	attach boxed title to top left={xshift=1cm,yshift=-\tcboxedtitleheight/2,yshifttext=-\tcboxedtitleheight/2},
	minipage boxed title=1.5cm,
	boxed title style={%
			colback=white,
			size=fbox,
			boxrule=1pt,
			boxsep=2pt,
			underlay={%
					\coordinate (dotA) at ($(interior.west) + (-0.5pt,0)$);
					\coordinate (dotB) at ($(interior.east) + (0.5pt,0)$);
					\begin{scope}
						\clip (interior.north west) rectangle ([xshift=3ex]interior.east);
						\filldraw [white, blur shadow={shadow opacity=60, shadow yshift=-.75ex}, rounded corners=2pt] (interior.north west) rectangle (interior.south east);
					\end{scope}
					\begin{scope}[gray!80!black]
						\fill (dotA) circle (2pt);
						\fill (dotB) circle (2pt);
					\end{scope}
				},
		},
	#1,
}

%%%%%%%%%%%%%%%%%%%%%%%%%%%%%%
% SELF MADE COMMANDS
%%%%%%%%%%%%%%%%%%%%%%%%%%%%%%
\newcommand{\thm}[2]{\begin{Theorem}{#1}{}#2\end{Theorem}}
\newcommand{\cor}[2]{\begin{Corollary}{#1}{}#2\end{Corollary}}
\newcommand{\mlenma}[2]{\begin{Lenma}{#1}{}#2\end{Lenma}}
\newcommand{\mprop}[2]{\begin{Prop}{#1}{}#2\end{Prop}}
\newcommand{\clm}[3]{\begin{claim}{#1}{#2}#3\end{claim}}
\newcommand{\wc}[2]{\begin{wconc}{#1}{}\setlength{\parindent}{1cm}#2\end{wconc}}
\newcommand{\thmcon}[1]{\begin{Theoremcon}{#1}\end{Theoremcon}}
\newcommand{\ex}[2]{\begin{Example}{#1}{}#2\end{Example}}
\newcommand{\dfn}[2]{\begin{Definition}[colbacktitle=red!75!black]{#1}{}#2\end{Definition}}
\newcommand{\dfnc}[2]{\begin{definition}[colbacktitle=red!75!black]{#1}{}#2\end{definition}}
\newcommand{\qs}[2]{\begin{question}{#1}{}#2\end{question}}
\newcommand{\pf}[2]{\begin{myproof}[#1]#2\end{myproof}}
\newcommand{\nt}[1]{\begin{note}#1\end{note}}

\newcommand*\circled[1]{\tikz[baseline=(char.base)]{
		\node[shape=circle,draw,inner sep=1pt] (char) {#1};}}
\newcommand\getcurrentref[1]{%
	\ifnumequal{\value{#1}}{0}
	{??}
	{\the\value{#1}}%
}
\newcommand{\getCurrentSectionNumber}{\getcurrentref{section}}
\newenvironment{myproof}[1][\proofname]{%
	\proof[\bfseries #1: ]%
}{\endproof}

\newcommand{\mclm}[2]{\begin{myclaim}[#1]#2\end{myclaim}}
\newenvironment{myclaim}[1][\claimname]{\proof[\bfseries #1: ]}{}

\newcounter{mylabelcounter}

\makeatletter
\newcommand{\setword}[2]{%
	\phantomsection
	#1\def\@currentlabel{\unexpanded{#1}}\label{#2}%
}
\makeatother

\tikzset{
	symbol/.style={
			draw=none,
			every to/.append style={
					edge node={node [sloped, allow upside down, auto=false]{$#1$}}}
		}
}

% deliminators
\DeclarePairedDelimiter{\abs}{\lvert}{\rvert}
\DeclarePairedDelimiter{\norm}{\lVert}{\rVert}

\DeclarePairedDelimiter{\ceil}{\lceil}{\rceil}
\DeclarePairedDelimiter{\floor}{\lfloor}{\rfloor}
\DeclarePairedDelimiter{\round}{\lfloor}{\rceil}

\newsavebox\diffdbox
\newcommand{\slantedromand}{{\mathpalette\makesl{d}}}
\newcommand{\makesl}[2]{%
\begingroup
\sbox{\diffdbox}{$\mathsurround=0pt#1\mathrm{#2}$}%
\pdfsave
\pdfsetmatrix{1 0 0.2 1}%
\rlap{\usebox{\diffdbox}}%
\pdfrestore
\hskip\wd\diffdbox
\endgroup
}
\newcommand{\dd}[1][]{\ensuremath{\mathop{}\!\ifstrempty{#1}{%
\slantedromand\@ifnextchar^{\hspace{0.2ex}}{\hspace{0.1ex}}}%
{\slantedromand\hspace{0.2ex}^{#1}}}}
\ProvideDocumentCommand\dv{o m g}{%
  \ensuremath{%
    \IfValueTF{#3}{%
      \IfNoValueTF{#1}{%
        \frac{\dd #2}{\dd #3}%
      }{%
        \frac{\dd^{#1} #2}{\dd #3^{#1}}%
      }%
    }{%
      \IfNoValueTF{#1}{%
        \frac{\dd}{\dd #2}%
      }{%
        \frac{\dd^{#1}}{\dd #2^{#1}}%
      }%
    }%
  }%
}
\providecommand*{\pdv}[3][]{\frac{\partial^{#1}#2}{\partial#3^{#1}}}
%  - others
\DeclareMathOperator{\Lap}{\mathcal{L}}
\DeclareMathOperator{\Var}{Var} % varience
\DeclareMathOperator{\Cov}{Cov} % covarience
\DeclareMathOperator{\E}{E} % expected

% Since the amsthm package isn't loaded

% I dot not prefer the slanted \leq ;P
% % I prefer the slanted \leq
% \let\oldleq\leq % save them in case they're every wanted
% \let\oldgeq\geq
% \renewcommand{\leq}{\leqslant}
% \renewcommand{\geq}{\geqslant}

% % redefine matrix env to allow for alignment, use r as default
% \renewcommand*\env@matrix[1][r]{\hskip -\arraycolsep
%     \let\@ifnextchar\new@ifnextchar
%     \array{*\c@MaxMatrixCols #1}}

%\usepackage{framed}
%\usepackage{titletoc}
%\usepackage{etoolbox}
%\usepackage{lmodern}

%\patchcmd{\tableofcontents}{\contentsname}{\sffamily\contentsname}{}{}

%\renewenvironment{leftbar}
%{\def\FrameCommand{\hspace{6em}%
%		{\color{myyellow}\vrule width 2pt depth 6pt}\hspace{1em}}%
%	\MakeFramed{\parshape 1 0cm \dimexpr\textwidth-6em\relax\FrameRestore}\vskip2pt%
%}
%{\endMakeFramed}

%\titlecontents{chapter}
%[0em]{\vspace*{2\baselineskip}}
%{\parbox{4.5em}{%
%		\hfill\Huge\sffamily\bfseries\color{myred}\thecontentspage}%
%	\vspace*{-2.3\baselineskip}\leftbar\textsc{\small\chaptername~\thecontentslabel}\\\sffamily}
%{}{\endleftbar}
%\titlecontents{section}
%[8.4em]
%{\sffamily\contentslabel{3em}}{}{}
%{\hspace{0.5em}\nobreak\itshape\color{myred}\contentspage}
%\titlecontents{subsection}
%[8.4em]
%{\sffamily\contentslabel{3em}}{}{}  
%{\hspace{0.5em}\nobreak\itshape\color{myred}\contentspage}

%%%%%%%%%%%%%%%%%%%%%%%%%%%%%%%%%%%%%%%%%%%
% TABLE OF CONTENTS
%%%%%%%%%%%%%%%%%%%%%%%%%%%%%%%%%%%%%%%%%%%
\usepackage{tikz}
\definecolor{doc}{RGB}{0,60,110}
\usepackage{titletoc}
\contentsmargin{0cm}
\titlecontents{chapter}[3.7pc]
{\addvspace{30pt}%
	\begin{tikzpicture}[remember picture, overlay]%
		\draw[fill=doc!60,draw=doc!60] (-7,-.1) rectangle (-0.9,.5);%
		\pgftext[left,x=-3.5cm,y=0.2cm]{\color{white}\Large\sc\bfseries Chapter\ \thecontentslabel};%
	\end{tikzpicture}\color{doc!60}\large\sc\bfseries}%
{}
{}
{\;\titlerule\;\large\sc\bfseries Page \thecontentspage
	\begin{tikzpicture}[remember picture, overlay]
		\draw[fill=doc!60,draw=doc!60] (2pt,0) rectangle (4,0.1pt);
	\end{tikzpicture}}%
\titlecontents{section}[3.7pc]
{\addvspace{2pt}}
{\contentslabel[\thecontentslabel]{2pc}}
{}
{\hfill\small \thecontentspage}
[]
\titlecontents*{subsection}[3.7pc]
{\addvspace{-1pt}\small}
{}
{}
{\ --- \small\thecontentspage}
[ \textbullet\ ][]

\makeatletter
\renewcommand{\tableofcontents}{%
	\chapter*{%
	  \vspace*{-20\p@}%
	  \begin{tikzpicture}[remember picture, overlay]%
		  \pgftext[right,x=15cm,y=0.2cm]{\color{doc!60}\Huge\sc\bfseries \contentsname};%
		  \draw[fill=doc!60,draw=doc!60] (13,-.75) rectangle (20,1);%
		  \clip (13,-.75) rectangle (20,1);
		  \pgftext[right,x=15cm,y=0.2cm]{\color{white}\Huge\sc\bfseries \contentsname};%
	  \end{tikzpicture}}%
	\@starttoc{toc}}
\makeatother

\newcommand{\inv}{^{-1}}
\newcommand{\opname}{\operatorname}
\newcommand{\surjto}{\twoheadrightarrow}
% \newcommand{\injto}{\hookrightarrow}
\newcommand{\injto}{\rightarrowtail}
\newcommand{\bijto}{\leftrightarrow}

\newcommand{\liff}{\leftrightarrow}
\newcommand{\notliff}{\mathrel{\ooalign{$\leftrightarrow$\cr\hidewidth$/$\hidewidth}}}
\newcommand{\lthen}{\rightarrow}
\let\varlnot\lnot
\newcommand{\ordsim}{\mathord{\sim}}
\renewcommand{\lnot}{\ordsim}
\newcommand{\lxor}{\oplus}
\newcommand{\lnand}{\barwedge}
\newcommand{\divs}{\mathrel{\mid}}
\newcommand{\ndivs}{\mathrel{\nmid}}
\def\contra{\tikz[baseline, x=0.22em, y=0.22em, line width=0.032em]\draw (0,2.83)--(2.83,0) (0.71,3.54)--(3.54,0.71) (0,0.71)--(2.83,3.54) (0.71,0)--(3.54,2.83);}

\newcommand{\On}{\mathrm{On}} % ordinals
\DeclareMathOperator{\img}{im} % Image
\DeclareMathOperator{\Img}{Im} % Image
\DeclareMathOperator{\coker}{coker} % Cokernel
\DeclareMathOperator{\Coker}{Coker} % Cokernel
\DeclareMathOperator{\Ker}{Ker} % Kernel
\DeclareMathOperator{\rank}{rank}
\DeclareMathOperator{\Spec}{Spec} % spectrum
\DeclareMathOperator{\Tr}{Tr} % trace
\DeclareMathOperator{\pr}{pr} % projection
\DeclareMathOperator{\ext}{ext} % extension
\DeclareMathOperator{\pred}{pred} % predecessor
\DeclareMathOperator{\dom}{dom} % domain
\DeclareMathOperator{\ran}{ran} % range
\DeclareMathOperator{\Hom}{Hom} % homomorphism
\DeclareMathOperator{\Mor}{Mor} % morphisms
\DeclareMathOperator{\End}{End} % endomorphism
\DeclareMathOperator{\Span}{span}
\newcommand{\Mod}{\mathbin{\mathrm{mod}}}

\newcommand{\eps}{\epsilon}
\newcommand{\veps}{\varepsilon}
\newcommand{\ol}{\overline}
\newcommand{\ul}{\underline}
\newcommand{\wt}{\widetilde}
\newcommand{\wh}{\widehat}
\newcommand{\ut}{\utilde}
\newcommand{\unit}[1]{\ut{\hat{#1}}}
\newcommand{\emp}{\varnothing}

\newcommand{\vocab}[1]{\textbf{\color{blue} #1}}
\providecommand{\half}{\frac{1}{2}}
\newcommand{\dang}{\measuredangle} %% Directed angle
\newcommand{\ray}[1]{\overrightarrow{#1}}
\newcommand{\seg}[1]{\overline{#1}}
\newcommand{\arc}[1]{\wideparen{#1}}
\DeclareMathOperator{\cis}{cis}
\DeclareMathOperator*{\lcm}{lcm}
\DeclareMathOperator*{\argmin}{arg min}
\DeclareMathOperator*{\argmax}{arg max}
\newcommand{\cycsum}{\sum_{\mathrm{cyc}}}
\newcommand{\symsum}{\sum_{\mathrm{sym}}}
\newcommand{\cycprod}{\prod_{\mathrm{cyc}}}
\newcommand{\symprod}{\prod_{\mathrm{sym}}}
\newcommand{\parinn}{\setlength{\parindent}{1cm}}
\newcommand{\parinf}{\setlength{\parindent}{0cm}}
% \newcommand{\norm}{\|\cdot\|}
\newcommand{\inorm}{\norm_{\infty}}
\newcommand{\opensets}{\{V_{\alpha}\}_{\alpha\in I}}
\newcommand{\oset}{V_{\alpha}}
\newcommand{\opset}[1]{V_{\alpha_{#1}}}
\newcommand{\lub}{\text{lub}}
\newcommand{\lm}{\lambda}
\newcommand{\uin}{\mathbin{\rotatebox[origin=c]{90}{$\in$}}}
\newcommand{\usubset}{\mathbin{\rotatebox[origin=c]{90}{$\subset$}}}
\newcommand{\lt}{\left}
\newcommand{\rt}{\right}
\newcommand{\bs}[1]{\boldsymbol{#1}}
\newcommand{\exs}{\exists}
\newcommand{\st}{\strut}
\newcommand{\dps}[1]{\displaystyle{#1}}

\newcommand{\sol}{\textbf{\textit{Solution:}} }
\newcommand{\solve}[1]{\textbf{\textit{Solution: }} #1 \qed}
% \newcommand{\proof}{\underline{\textit{proof:}}\\}

\DeclareMathOperator{\sech}{sech}
\DeclareMathOperator{\csch}{csch}
\DeclareMathOperator{\arcsec}{arcsec}
\DeclareMathOperator{\arccsc}{arccsc}
\DeclareMathOperator{\arccot}{arccot}
\DeclareMathOperator{\arsinh}{arsinh}
\DeclareMathOperator{\arcosh}{arcosh}
\DeclareMathOperator{\artanh}{artanh}
\DeclareMathOperator{\arcsch}{arcsch}
\DeclareMathOperator{\arsech}{arsech}
\DeclareMathOperator{\arcoth}{arcoth}

\newcommand{\sinx}{\sin x}          \newcommand{\arcsinx}{\arcsin x}    
\newcommand{\cosx}{\cos x}          \newcommand{\arccosx}{\arccosx}
\newcommand{\tanx}{\tan x}          \newcommand{\arctanx}{\arctan x}
\newcommand{\cscx}{\csc x}          \newcommand{\arccscx}{\arccsc x}
\newcommand{\secx}{\sec x}          \newcommand{\arcsecx}{\arcsec x}
\newcommand{\cotx}{\cot x}          \newcommand{\arccotx}{\arccot x}
\newcommand{\sinhx}{\sinh x}          \newcommand{\arsinhx}{\arsinh x}
\newcommand{\coshx}{\cosh x}          \newcommand{\arcoshx}{\arcosh x}
\newcommand{\tanhx}{\tanh x}          \newcommand{\artanhx}{\artanh x}
\newcommand{\cschx}{\csch x}          \newcommand{\arcschx}{\arcsch x}
\newcommand{\sechx}{\sech x}          \newcommand{\arsechx}{\arsech x}
\newcommand{\cothx}{\coth x}          \newcommand{\arcothx}{\arcoth x}
\newcommand{\lnx}{\ln x}
\newcommand{\expx}{\exp x}

\newcommand{\Theom}{\textbf{Theorem. }}
\newcommand{\Lemma}{\textbf{Lemma. }}
\newcommand{\Corol}{\textbf{Corollary. }}
\newcommand{\Remar}{\textit{Remark. }}
\newcommand{\Defin}[1]{\textbf{Definition} (#1).}
\newcommand{\Claim}{\textbf{Claim. }}
\newcommand{\Propo}{\textbf{Proposition. }}

\newcommand{\lb}{\left(}
\newcommand{\rb}{\right)}
\newcommand{\lbr}{\left\lbrace}
\newcommand{\rbr}{\right\rbrace}
\newcommand{\lsb}{\left[}
\newcommand{\rsb}{\right]}
\newcommand{\bracks}[1]{\lb #1 \rb}
\newcommand{\braces}[1]{\lbr #1 \rbr}
\newcommand{\suchthat}{\medspace\middle|\medspace}
\newcommand{\sqbracks}[1]{\lsb #1 \rsb}
\renewcommand{\abs}[1]{\left| #1 \right|}
\newcommand{\Mag}[1]{\left|\left| #1 \right|\right|}
\renewcommand{\floor}[1]{\left\lfloor #1 \right\rfloor}
\renewcommand{\ceil}[1]{\left\lceil #1 \right\rceil}

\newcommand{\cd}{\cdot}
\newcommand{\tf}{\therefore}
\newcommand{\Let}{\text{Let }}
\newcommand{\Given}{\text{Given }}
% \newcommand{\and}{\text{and }}
\newcommand{\Substitute}{\text{Substitute }}
\newcommand{\Suppose}{\text{Suppose }}
\newcommand{\WeSee}{\text{We see }}
\newcommand{\So}{\text{So }}
\newcommand{\Then}{\text{Then }}
\newcommand{\Choose}{\text{Choose }}
\newcommand{\Take}{\text{Take }}
\newcommand{\false}{\text{False}}
\newcommand{\true}{\text{True}}

\newcommand{\QED}{\hfill \qed}
\newcommand{\CONTRA}{\hfill \contra}

\newcommand{\ihat}{\hat{\imath}}
\newcommand{\jhat}{\hat{\jmath}}
\newcommand{\khat}{\hat{k}}

\newcommand{\grad}{\nabla}
\newcommand{\D}{\Delta}
\renewcommand{\d}{\mathrm{d}}

\renewcommand{\dd}[1]{\frac{\d}{\d #1}}
\newcommand{\dyd}[2][y]{\frac{\d #1}{\d #2}}

\newcommand{\ddx}{\dd{x}}       \newcommand{\ddxsq}{\dyd[^2]{x^2}}
\newcommand{\ddy}{\dd{y}}       \newcommand{\ddysq}{\dyd[^2]{y^2}}
\newcommand{\ddu}{\dd{u}}       \newcommand{\ddusq}{\dyd[^2]{u^2}}
\newcommand{\ddv}{\dd{v}}       \newcommand{\ddvsq}{\dyd[^2]{v^2}}

\newcommand{\dydx}{\dyd{x}}     \newcommand{\dydxsq}{\dyd[^2y]{x^2}}
\newcommand{\dfdx}{\dyd[f]{x}}  \newcommand{\dfdxsq}{\dyd[^2f]{x^2}}
\newcommand{\dudx}{\dyd[u]{x}}  \newcommand{\dudxsq}{\dyd[^2u]{x^2}}
\newcommand{\dvdx}{\dyd[v]{x}}  \newcommand{\dvdxsq}{\dyd[^2v]{x^2}}

\newcommand{\del}[2]{\frac{\partial #1}{\partial #2}}
\newcommand{\Del}[3]{\frac{\partial^{#1} #2}{\partial #3^{#1}}}
\newcommand{\deld}[2]{\dfrac{\partial #1}{\partial #2}}
\newcommand{\Deld}[3]{\dfrac{\partial^{#1} #2}{\partial #3^{#1}}}

\newcommand{\argument}[2]{
  \begin{array}{rll}
    #1
    \cline{2-2}
    \therefore & #2 
  \end{array}
}
% Mathfrak primes
\newcommand{\km}{\mathfrak m}
\newcommand{\kp}{\mathfrak p}
\newcommand{\kq}{\mathfrak q}

%---------------------------------------
% Blackboard Math Fonts :-
%---------------------------------------
\newcommand{\bba}{\mathbb{A}}   \newcommand{\bbn}{\mathbb{N}}
\newcommand{\bbb}{\mathbb{B}}   \newcommand{\bbo}{\mathbb{O}}
\newcommand{\bbc}{\mathbb{C}}   \newcommand{\bbp}{\mathbb{P}}
\newcommand{\bbd}{\mathbb{D}}   \newcommand{\bbq}{\mathbb{Q}}
\newcommand{\bbe}{\mathbb{E}}   \newcommand{\bbr}{\mathbb{R}}
\newcommand{\bbf}{\mathbb{F}}   \newcommand{\bbs}{\mathbb{S}}
\newcommand{\bbg}{\mathbb{G}}   \newcommand{\bbt}{\mathbb{T}}
\newcommand{\bbh}{\mathbb{H}}   \newcommand{\bbu}{\mathbb{U}}
\newcommand{\bbi}{\mathbb{I}}   \newcommand{\bbv}{\mathbb{V}}
\newcommand{\bbj}{\mathbb{J}}   \newcommand{\bbw}{\mathbb{W}}
\newcommand{\bbk}{\mathbb{K}}   \newcommand{\bbx}{\mathbb{X}}
\newcommand{\bbl}{\mathbb{L}}   \newcommand{\bby}{\mathbb{Y}}
\newcommand{\bbm}{\mathbb{M}}   \newcommand{\bbz}{\mathbb{Z}}

%---------------------------------------
% Roman Math Fonts :-
%---------------------------------------
\newcommand{\rma}{\mathrm{A}}   \newcommand{\rmn}{\mathrm{N}}
\newcommand{\rmb}{\mathrm{B}}   \newcommand{\rmo}{\mathrm{O}}
\newcommand{\rmc}{\mathrm{C}}   \newcommand{\rmp}{\mathrm{P}}
\newcommand{\rmd}{\mathrm{D}}   \newcommand{\rmq}{\mathrm{Q}}
\newcommand{\rme}{\mathrm{E}}   \newcommand{\rmr}{\mathrm{R}}
\newcommand{\rmf}{\mathrm{F}}   \newcommand{\rms}{\mathrm{S}}
\newcommand{\rmg}{\mathrm{G}}   \newcommand{\rmt}{\mathrm{T}}
\newcommand{\rmh}{\mathrm{H}}   \newcommand{\rmu}{\mathrm{U}}
\newcommand{\rmi}{\mathrm{I}}   \newcommand{\rmv}{\mathrm{V}}
\newcommand{\rmj}{\mathrm{J}}   \newcommand{\rmw}{\mathrm{W}}
\newcommand{\rmk}{\mathrm{K}}   \newcommand{\rmx}{\mathrm{X}}
\newcommand{\rml}{\mathrm{L}}   \newcommand{\rmy}{\mathrm{Y}}
\newcommand{\rmm}{\mathrm{M}}   \newcommand{\rmz}{\mathrm{Z}}

%---------------------------------------
% Calligraphic Math Fonts :-
%---------------------------------------
\newcommand{\cla}{\mathcal{A}}   \newcommand{\cln}{\mathcal{N}}
\newcommand{\clb}{\mathcal{B}}   \newcommand{\clo}{\mathcal{O}}
\newcommand{\clc}{\mathcal{C}}   \newcommand{\clp}{\mathcal{P}}
\newcommand{\cld}{\mathcal{D}}   \newcommand{\clq}{\mathcal{Q}}
\newcommand{\cle}{\mathcal{E}}   \newcommand{\clr}{\mathcal{R}}
\newcommand{\clf}{\mathcal{F}}   \newcommand{\cls}{\mathcal{S}}
\newcommand{\clg}{\mathcal{G}}   \newcommand{\clt}{\mathcal{T}}
\newcommand{\clh}{\mathcal{H}}   \newcommand{\clu}{\mathcal{U}}
\newcommand{\cli}{\mathcal{I}}   \newcommand{\clv}{\mathcal{V}}
\newcommand{\clj}{\mathcal{J}}   \newcommand{\clw}{\mathcal{W}}
\newcommand{\clk}{\mathcal{K}}   \newcommand{\clx}{\mathcal{X}}
\newcommand{\cll}{\mathcal{L}}   \newcommand{\cly}{\mathcal{Y}}
\newcommand{\calm}{\mathcal{M}}  \newcommand{\clz}{\mathcal{Z}}

%---------------------------------------
% Fraktur  Math Fonts :-
%---------------------------------------
\newcommand{\fka}{\mathfrak{A}}   \newcommand{\fkn}{\mathfrak{N}}
\newcommand{\fkb}{\mathfrak{B}}   \newcommand{\fko}{\mathfrak{O}}
\newcommand{\fkc}{\mathfrak{C}}   \newcommand{\fkp}{\mathfrak{P}}
\newcommand{\fkd}{\mathfrak{D}}   \newcommand{\fkq}{\mathfrak{Q}}
\newcommand{\fke}{\mathfrak{E}}   \newcommand{\fkr}{\mathfrak{R}}
\newcommand{\fkf}{\mathfrak{F}}   \newcommand{\fks}{\mathfrak{S}}
\newcommand{\fkg}{\mathfrak{G}}   \newcommand{\fkt}{\mathfrak{T}}
\newcommand{\fkh}{\mathfrak{H}}   \newcommand{\fku}{\mathfrak{U}}
\newcommand{\fki}{\mathfrak{I}}   \newcommand{\fkv}{\mathfrak{V}}
\newcommand{\fkj}{\mathfrak{J}}   \newcommand{\fkw}{\mathfrak{W}}
\newcommand{\fkk}{\mathfrak{K}}   \newcommand{\fkx}{\mathfrak{X}}
\newcommand{\fkl}{\mathfrak{L}}   \newcommand{\fky}{\mathfrak{Y}}
\newcommand{\fkm}{\mathfrak{M}}   \newcommand{\fkz}{\mathfrak{Z}}


\title{\Huge{MATH1061}\\Discrete Mathematics I}
\author{\huge{Problem Set 1}\\\huge{Michael Kasumagic, sID\#: 44302669}}
\date{\huge{Due: 5pm, $16^\text{th}$ of Augsut, 2024}}

\begin{document}

\maketitle

\qs{(\it10 marks)}{
  Use a truth table to determine whether the following statement is a contradiction, a tautology or neither. If it is a contradiction or a tautology, verify your answer using logical equivalences.
  $$
    ((p\liff q)\land(p\lxor r)) \lthen (q\lor r)
  $$
}
\sol Given the statement, we will construct a truth table.
\begin{gather*}
  \begin{array}{|ccc||cc||cc||c|}
    \hline
    $p$ & $q$ & $r$ & p\liff q & p\lxor r & (p\liff q) \land (p\lxor r) & q\lor r & ((p\liff q) \land (p\lxor r)) \lthen q\lor r \\ \hline 
    $T$ & $T$ & $T$ & $T$ & $F$ & $F$ & $T$ & $T$ \\
    $T$ & $T$ & $F$ & $T$ & $T$ & $T$ & $T$ & $T$ \\
    $T$ & $F$ & $T$ & $F$ & $F$ & $F$ & $T$ & $T$ \\
    $T$ & $F$ & $F$ & $F$ & $T$ & $F$ & $F$ & $T$ \\
    $F$ & $T$ & $T$ & $F$ & $T$ & $F$ & $T$ & $T$ \\
    $F$ & $T$ & $F$ & $F$ & $F$ & $F$ & $T$ & $T$ \\
    $F$ & $F$ & $T$ & $T$ & $T$ & $T$ & $T$ & $T$ \\
    $F$ & $F$ & $F$ & $T$ & $F$ & $F$ & $F$ & $T$ \\ \hline
  \end{array}
  \longintertext{Therefore, the given statement is a tautology.\\ Let's now verify this with a proof using logical equivalences. For the sake of brevity, we'll label the original statement $\rma$.}
  \rma \equiv ((p\liff q)\land(p\lxor r)) \lthen (q\lor r) 
  \intertext{First, we'll expand and simplify $p \liff q$.}
  \begin{align*}
    p\liff q &\equiv (p\lthen q) \land (q\lthen p) \tag*{(Logical Equivalence of $\liff$)} \\
      &\equiv (\lnot p\lor q) \land (\lnot q\lor p) \tag*{(Logical Equivalence of $\lthen$)} \\
      &\equiv (\lnot p\land\lnot q) \lor (\lnot p\land p) \lor (q\land\lnot q) \lor (q\land p) \tag*{(Distributivity)}\\
      &\equiv (\lnot p\land\lnot q) \lor \bot \lor \bot \lor (q\land p) \tag*{(Negation Law)} \\
      &\equiv (\lnot p\land\lnot q) \lor (q\land p) \tag*{(Identity Law)}\\
    \tf \rma &\equiv ((\lnot p\land\lnot q) \lor (q\land p) \land (p\lxor r)) \lthen (q\lor r) \tag*{(Subsittute back into $\rma$)}
  \end{align*}
  \intertext{Next, we'll expand and simplify $p \lxor r$.}
  \begin{align*}
    p\lxor r &\equiv (p\lor r) \land \lnot (p \land r) \tag*{(Logical Equivalence of $\lxor$)} \\
      &\equiv (p\lor r) \land (\lnot p \lor \lnot r) \tag*{(Apply De Morgan's Law)} \\
      &\equiv (p\land\lnot p) \lor (p\land\lnot r) \lor (r\land\lnot p) \lor (r\land\lnot r) \tag*{(Distributivity)} \\
      &\equiv \bot \lor (p\land\lnot r) \lor (r\land\lnot p) \lor \bot \tag*{(Negation Law)} \\
      &\equiv (p\land\lnot r) \lor (r\land\lnot p) \tag*{(Identity Law)}\\
    \tf\rma &\equiv (((\lnot p\land\lnot q) \lor (q\land p)) \land ((p\land\lnot r) \lor (r\land\lnot p))) \lthen (q\lor r) \tag*{(Subsittute back into $\rma$)}
  \end{align*}
  \intertext{Now, we'll attempt to distribute $p\liff q$ over $p\lxor r$.}
  \begin{align*}
    &\mathcolor{white}{\equiv} (p\liff q) \land (p\lxor r) \\
      &\equiv ((\lnot p\land\lnot q) \lor (q\land p)) \land ((p\land\lnot r) \lor (r\land\lnot p)) \tag*{(Subsittute previous results)} \\
      &\equiv ((\lnot p\land\lnot q) \land (p\land\lnot r)) \lor ((\lnot p\land\lnot q) \land (r\land\lnot p)) \\ &\mathcolor{white}{\equiv}\lor ((q\land p) \land (p\land\lnot r)) \lor ((q\land p) \land (r\land\lnot p)) \tag*{(Distributivity)} \\
      &\equiv ((\lnot p\land p) \land (\lnot q\land\lnot r)) \lor ((\lnot p\land\lnot p) \land (r\land\lnot q)) \\ &\mathcolor{white}{\equiv}\lor ((p\land p) \land (q\land\lnot r)) \lor ((\lnot p\land p) \land (r\land q)) \tag*{(Associativity)} \\
      &\equiv (\bot \land (\lnot q\land\lnot r)) \lor (\lnot p \land (r\land\lnot q)) \lor (p \land (q\land\lnot r)) \lor (\bot \land (r\land q)) \tag*{(Negation Law)} \\
      &\equiv (\bot) \lor (\lnot p \land \lnot q \land r) \lor (p \land q\land\lnot r) \lor (\bot) \tag*{(Domination Law)}\\
      &\equiv (\lnot p \land \lnot q \land r) \lor (p \land q\land\lnot r) \tag*{(Identity Law)} \\
    \tf\rma &\equiv ((\lnot p \land \lnot q \land r) \lor (p \land q\land\lnot r)) \lthen (q\lor r) \tag*{(Subsittute back into $\rma$)}
  \end{align*}
  \longintertext{Holy hell.\\ Next, we'll expand this resultant $\alpha\lthen\beta$ into $\lnot\alpha\lor\beta$.}
  \begin{align*}
    \rma &= ((\lnot p \land \lnot q \land r) \lor (p \land q\land\lnot r)) \lthen (q\lor r) \tag*{(Previous Result)} \\
      &= \lnot((\lnot p \land \lnot q \land r) \lor (p \land q\land\lnot r)) \lor (q\lor r) \tag*{(Logical Equivalence of $\lthen$)} \\
      &= (\lnot(\lnot p \land \lnot q \land r) \land \lnot(p \land q\land\lnot r)) \lor (q\lor r) \tag*{(Apply De Morgan's Law)} \\
      &= ((p \lor q \lor \lnot r) \land (\lnot p \lor \lnot q\lor r)) \lor (q\lor r) \tag*{(Apply De Morgan's Law)} \\
      &= (q\lor r) \lor ((p \lor q \lor \lnot r) \land (\lnot p \lor \lnot q\lor r)) \tag*{(Commutativty)} 
  \end{align*}
  \intertext{Finally, we'll distribute $(q\lor r)$ over the rest of the statement.}
  \begin{align*}
    \rma &= (q\lor r) \lor ((p \lor q \lor \lnot r) \land (\lnot p \lor \lnot q\lor r)) \tag*{(Previous Result)} \\
      &= ((q\lor r) \lor (p \lor q \lor \lnot r)) \land ((q\lor r)\lor (\lnot p \lor \lnot q\lor r)) \tag*{(Distributivity)} \\
      &= (q\lor r \lor p \lor q \lor \lnot r) \land (q\lor r\lor\lnot p \lor \lnot q\lor r) \tag*{(Associativity)} 
  \end{align*}
  \longintertext{We're so close now!\\ To finish it off, we can use associativity to rearrange our statement, find the tautology, then expand it outward, until we show the whole statement evaluates to a tautology.}
  \begin{align*}
    \rma &= (q\lor r \lor p \lor q \lor \lnot r) \land (q\lor r\lor\lnot p \lor \lnot q\lor r) \\
      &= (r \lor \lnot r \lor p\lor q \lor q) \land (q\lor \lnot q \lor p\lor r\lor r) \tag*{(Associativity)} \\
      &= (\top \lor p\lor q) \land (\top \lor p\lor r) \tag*{(Nagation Law and Idempotent Law)} \\
      &= (\top\lor q) \land (\top\lor r) \tag*{(Domination Law)} \\
      &= \top\land\top \tag*{(Idempotent Law)} \\
    \tf\rma&\equiv \top \tag*{(Conclusion)}
  \end{align*}\\
  \intertext{And we're done! Therefore, we've proven, by logical equivalence, that the statement}
  ((p\liff q)\land(p\lxor r))\lthen (q\lor r)
  \intertext{is logically equivalent to a tautology.$\QED$}
\end{gather*}

\pagebreak

\qs{(\it10 marks)}{
  Using the laws of logical equivalence, show that the following statement is a tautology:
  $$
    ((a\lthen r)\land(b\lthen r)\land(\lnot a\lthen b))\lthen r
  $$
}
\sol Again, we'll label the statement $\rmb$ for the sake of brevity.
\begin{gather*}
  \rmb \equiv ((a\lthen r)\land(b\lthen r)\land(\lnot a\lthen b))\lthen r
  \intertext{We'll start by expanding each $\lthen$ expression into its corresponding $\lnot\lor$ expression.}
  \begin{align*}
    (a\lthen r) &\equiv \lnot a\lor r \tag*{(Logical Equivalence of $\lthen$)} \\
    (b\lthen r) &\equiv \lnot b\lor r \tag*{(Logical Equivalence of $\lthen$)} \\
    (\lnot a\lthen b) &\equiv a\lor b \tag*{(Logical Equivalence of $\lthen$)} \\
    \tf\rmb &\equiv ((\lnot a\lor r)\land(\lnot b\lor r)\land(a\lor b))\lthen r \tag*{(Subsittute back into $\rmb$)}  
  \end{align*}
  \intertext{We will now apply the distributivity law to undistribute $r$ from $(\lnot a\lor r)\land(\lnot b\lor r)$.}
  \begin{align*}
    (\lnot a\lor r)\land(\lnot b\lor r) &\equiv r \lor (\lnot a \land \lnot b) \tag*{(Distributivity)} \\
    \tf\rmb &\equiv ((r \lor (\lnot a \land \lnot b))\land(a\lor b))\lthen r \tag*{(Subsittute back into $\rmb$)} \\
      &\equiv ((a\lor b)\land(r \lor (\lnot a \land \lnot b)))\lthen r \tag*{(Associativity)}
  \end{align*}
  \longintertext{Next, we'll now distribute $(a\lor b)$ over $r\lor(\lnot a\land\lnot b)$.}
  \begin{align*}
    &\mathcolor{white}{\equiv} (a\lor b) \land (r\lor(\lnot a\land\lnot b)) \\
      &\equiv ((a\lor b) \land r ) \lor ((a\lor b)\land(\lnot a\land\lnot b)) \tag*{(Distributivity)} \\
      &\equiv ((a\land r) \lor (b\land r)) \lor (((a\lor b)\land\lnot a)\land((a\lor b)\land\lnot b)) \tag*{(Distributivity)} \\
      &\equiv ((a\land r) \lor (b\land r)) \lor (((a\land\lnot a)\lor (b\land\lnot a))\land((a\land\lnot b)\lor(b\land\lnot b))) \tag*{(Distributivity)} \\
      &\equiv ((a\land r) \lor (b\land r)) \lor ((\bot\lor (b\land\lnot a))\land((a\land\lnot b)\lor\bot)) \tag*{(Negation Law)}\\
      &\equiv ((a\land r) \lor (b\land r)) \lor ((b\land\lnot a)\land(a\land\lnot b)) \tag*{(Identity)} \\
      &\equiv ((a\land r) \lor (b\land r)) \lor ((b\land\lnot b)\land(a\land\lnot a)) \tag*{(\text{Associativity})}\\
      &\equiv ((a\land r) \lor (b\land r)) \lor (\bot\land\bot) \tag*{(Negation Law)} \\
      &\equiv ((a\land r) \lor (b\land r)) \lor \bot \tag*{(Identity Law)}\\
      &\equiv ((a\land r) \lor (b\land r)) \tag*{(Identity Law)} \\
      &\equiv r\land(a\lor b) \tag*{(Distributivity)} \\
    \tf\rmb &\equiv (r\land(a\lor b))\lthen r \tag*{(Subsittute back into $\rmb$)}
  \end{align*}
  \intertext{Finally, let's expand the $\lthen$ into its logically equivalent $\lnot\lor$ form.}
  \begin{align*}
    \rmb &\equiv (r\land(a\lor b))\lthen r \\
      &\equiv \lnot(r\land(a\lor b))\lor r \tag*{(Logical Equivalence of $\lthen$)} \\
      &\equiv (\lnot r\lor\lnot(a\lor b))\lor r \tag*{(Applying De Morgan's Law)} \\
      &\equiv (\lnot r\lor r)\lor\lnot(a\lor b) \tag*{(\text{Associativity})} \\
      &\equiv \top\lor\lnot(a\lor b) \\
    \tf\rmb &\equiv \top \\
  \end{align*}
  \intertext{Therefore, we've shown, by use of laws of logical equivalence, that the statement}
  ((a\lthen r)\land(b\lthen r)\land(\lnot a\lthen b))\lthen r
  \intertext{is a tautology.$\QED$}
\end{gather*}

\pagebreak

\qs{(\it5 marks)}{
  Show that the following argument is valid, by adding steps using the rules of inference and/or logical equivalences. Clearly label which rule you used in each step.
  $$
    \argument{
      1. & a\land w \lthen p & \\
      2. & \lnot a \lthen l & \\
      3. & \lnot w \lthen m & \\
      4. & \lnot p & \\
      5. & e \lthen \lnot (l \lor m) & \\
    }{\lnot e & }
  $$
}
\sol 
\begin{gather*}
  \argument{
    1.  & a\land w \lthen p & \\
    2.  & \lnot a \lthen l & \\
    3.  & \lnot w \lthen m & \\
    4.  & \lnot p & \\
    5.  & e \lthen \lnot (l \lor m) & \\
    6.  & (l\lor m) \lthen \lnot e & (\text{Contrapositive of 5.}) \\
    7.  & \lnot (a\land w) & (\text{Modus Tollens of 1. given 4.}) \\
    8.  & \lnot a \lor \lnot w & (\text{De Morgan's Law on 7.}) \\
    9.  & l \lor m & (\text{Constructive Dilemma, of 2., 3., given 8.}) \\
    10. & \lnot e & (\text{Modus Ponens of 6. given 9.}) \\
  }{\lnot e & }
  \longintertext{This shows that the argument is valid. Applying rules of inference and logical equivalence to the same given premises, we've drawn the same conclusion.Therefore the argument is valid. \QED}
\end{gather*}

\pagebreak

\qs{(\it10 marks)}{
  Let $P(x), Q(x), R(x)$ and $S(x)$ denote the following predicates with domain $\bbz$:
  $$
    \begin{array}{rl}
      P(x): & x^2 = 9, \\
      Q(x): & x^2 = 6, \\
      R(x): & x\geq 0, \\
      S(x): & x \text{ is odd.}
    \end{array}
  $$
  Determine whether each of the following statements is true or false, and give brief reasons.
  \begin{enumerate}[label=(\alph*)]
    \item $\forall x\in\bbz, P(x)\lthen S(x)$
    \item $\forall x\in\bbz, P(x)\lthen R(x)$
    \item $\exists x\in\bbz: P(x)\land R(x)$
    \item $\forall x\in\bbz, Q(x)\lthen R(x)$
    \item $\forall x\in\bbz, S(x)\lthen\lnot Q(x)$
  \end{enumerate}
}
\sol (a)
$$
  \forall x\in\bbz,P(x)\lthen S(x) \equiv \forall x\in\bbz, x^2=9\lthen x\text{ is odd}
$$
is true, because if $x^2 = 9$ then $x$ is either $-\sqrt{9}=-3$ or $\sqrt{9}=3$, both of which are odd. \\
An alternative view: the statement is true because if $x^2 = 9$, we can conclude that $|x||x| = 9$.\\ 9's prime factorsation is $3\cd3$, therefore $|x|=3$. If $|x|=3$, there are two cases which satisfy this: $x=3$ and $x=-3$, both of which are odd. \\

\sol (b)
$$
  \forall x\in\bbz, P(x) \lthen R(x) \equiv \forall x\in\bbz, x^2=9\lthen x\geq 0
$$
is false, because I can provide a counter example, namely, $x=-3$. If $x=-3$, then $x^2=9$, but $x\not\geq0$. In fact, $x=-3 < 0$. \\

\sol (c)
$$
  \exists x\in\bbz: P(x)\land R(x) \equiv \exists x\in\bbz: x^2 = 9 \land x\geq 0 
$$
is true, because I can provide an example $x\in\bbz$, namely, $x=3$. If $x=3$, then $x^2=3^2=9$, and $x=3\geq0$. \\

\sol (d)
$$
  \forall x\in\bbz, Q(x)\lthen R(x) \equiv \forall x\in\bbz, x^2 = 6\lthen x\geq 0
$$
is true. $Q(x)$ is false $\forall x\in\bbz$, because $\nexists x\in\bbz:x^2 = 6$. No integer can satisfy this equation. You'd need to expand the domain of $x$ into the reals to solve this, namely $r=\sqrt{6}\in\bbr$. Since $Q(x)$ is false $\forall x\in\bbz$, it doesn't matter if $R(x)$ is true or false, because the statement, $Q(x)\lthen R(x)\equiv\bot\lthen R(x)\equiv\top\lor R(x)\equiv \top$, will evaluate to true always. The given statement is an example of a vacuous truth. \\

\sol (e) 
$$
  \forall x\in\bbz, S(x)\lthen\lnot Q(x) \equiv \forall x\in\bbz, x \text{ is odd}\lthen x^2 \neq 6
$$
is true, because if $x$ is odd, or any element in the set of integers for that matter, then it will hold that $x^2\neq 6$. In fact, no integer can satisfy the equation $x^2=6$, you'd need to expand the domain of $x$ into the real numbers to find a solution, namely $r=\sqrt{6}\in\bbr$. So, Q(x) is false $\forall x\in\bbz$, which means $\forall x\in\bbz,\lnot Q(x)$ is true. So the whole statement, $S(x)\lthen\lnot Q(x)\equiv S(x)\lthen\top\equiv \lnot S(x)\lor \top\equiv\top$ is always true. Another example of a vacuous truth.

\pagebreak

\qs{(\it10 marks)}{
  Find values of a, b and r that show that the following statement is not a tautology, and show your working.
  $$
    r \lthen ((a\lthen r)\land(b\lthen r)\land(\lnot a\lthen b))
  $$
}
\sol We can do this with a truth table. \\ For the sake of brevity, let $\rmc = r \lthen ((a\lthen r)\land(b\lthen r)\land(\lnot a\lthen b))$
\begin{gather*}
  \begin{array}{|ccc||c||ccc||cc||c|}
    \hline
    a & b & r & \lnot a & a\lthen r & b\lthen r & \lnot a\lthen b & r & (a\lthen r)\land(b\lthen r)\land(\lnot a\lthen b) & C \\  \hline
    T & T & T & F & T & T & T & T & T & T \\
    T & T & F & F & F & F & T & F & F & T \\
    T & F & T & F & T & T & T & T & T & T \\
    T & F & F & F & F & T & T & F & F & T \\
    F & T & T & T & T & T & T & T & T & T \\
    F & T & F & T & T & F & T & F & F & T \\ \hline
    F & F & T & T & T & T & F & T & F & F \\ \hline
    F & F & F & T & T & T & F & F & F & T \\ \hline
  \end{array}\\
  \longintertext{$\tf (a,b,r)=(\false,\false,\true)$ is an example configuration of variable truth values which causes the statement to evaluate to false. This, in turn, shows that the statement $\rmc$ is not a tautology.}
  \longintertext{I'm paranoid that proving this by truth table is insufficent, so let's also work through this, just in case, and consider the statement}
  r \lthen ((a\lthen r)\land(b\lthen r)\land(\lnot a\lthen b)) \equiv \lnot r \lor ((a\lthen r)\land(b\lthen r)\land(\lnot a\lthen b)) \\ \text{is true when $r$ is false, or the ands are true.}\\
  \begin{array}{rrl}
    1. & a\lthen r\equiv \lnot a\lor r: & \text{Is true if a is false or r is true.} \\
    2. & b\lthen r\equiv \lnot b\lor r: & \text{Is true if b is false or r is true.} \\
    3. & \lnot a\lthen b\equiv a\lor b: & \text{Is true if a is true or b is true.} \\
  \end{array}
  \intertext{By analysing this statement, we can see a weakness. 3. and (2. and 1.) will clash. They can't  both be true at the same time, if we select the right configuration of truth values. We will set $r$ to true, because this moves the ``responsibility'' of making the statement false, from $r$ to the ands, where we've identified a weakness.}
  \text{Let } r \text{ be true.} \\
  \text{Then } 1. \land 2. \land 3. \text{ must be true.} \\
  \tf 1. \equiv\true, 2.\equiv\true, 3.\equiv\true \\
  \true \equiv a\lthen r \equiv a\lthen\true \equiv \lnot a \lor\true \equiv \lnot a \\
  \text{So let } a \text{ be false.} \\
  \true \equiv b\lthen r \equiv b\lthen\true\equiv \lnot b\lor \true \equiv \lnot b \\
  \text{So let } b \text{ be false.} \\
  \true\equiv \lnot a \lthen b \equiv a\lor b \equiv \false\lor\false\equiv\false \tag*{\CONTRA} \\
  \intertext{This is a contradiction! So our assumption that $1.\land2.\land3.$ is true must be wrong, and the statement ``$r\lthen((a\lthen r)\land(b\lthen r)\land(\lnot a\lthen b))$ is a tautology'' causes a contradiction. Therefore, we can conclude that $r\lthen((a\lthen r)\land(b\lthen r)\land(\lnot a\lthen b))$ is not a tautology, and can provide a counterexample, namely $(a,b,r)=(\false,\false,\true)$.\QED}
\end{gather*}

\end{document}
