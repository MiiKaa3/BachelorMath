\documentclass[a4paper, 11pt]{report}

%%%%%%%%%%%%%%%%%%%%%%%%%%%%%%%%%
% PACKAGE IMPORTS
%%%%%%%%%%%%%%%%%%%%%%%%%%%%%%%%%
\usepackage[tmargin=2cm,rmargin=1in,lmargin=1in,margin=0.85in,bmargin=2cm,footskip=.2in]{geometry}
\usepackage[none]{hyphenat}
\usepackage{amsmath,amsfonts,amsthm,amssymb,mathtools}
\allowdisplaybreaks
\usepackage{undertilde}
\usepackage{xfrac}
\usepackage[makeroom]{cancel}
\usepackage{mathtools}
\usepackage{bookmark}
\usepackage{enumitem}
\usepackage{kbordermatrix}
\renewcommand{\kbldelim}{(} % Change left delimiter to (
\renewcommand{\kbrdelim}{)} % Change right delimiter to )
\usepackage{hyperref,theoremref}
\hypersetup{
	pdftitle={Assignment},
	colorlinks=true, linkcolor=doc!90,
	bookmarksnumbered=true,
	bookmarksopen=true
}
\usepackage[most,many,breakable]{tcolorbox}
\usepackage{xcolor}
\usepackage{varwidth}
\usepackage{varwidth}
\usepackage{etoolbox}
%\usepackage{authblk}
\usepackage{nameref}
\usepackage{multicol,array}
\usepackage{tikz-cd}
\usepackage[ruled,vlined,linesnumbered]{algorithm2e}
\usepackage{comment} % enables the use of multi-line comments (\ifx \fi) 
\usepackage{import}
\usepackage{xifthen}
\usepackage{pdfpages}
\usepackage{svg}
\usepackage{transparent}
\usepackage{pgfplots}
\pgfplotsset{compat=1.18}
\usetikzlibrary{calc}
\usetikzlibrary{graphs}
\usetikzlibrary{graphs.standard}
% \usetikzlibrary{graphdrawing}

\newcommand\mycommfont[1]{\footnotesize\ttfamily\textcolor{blue}{#1}}
\SetCommentSty{mycommfont}
\newcommand{\incfig}[1]{%
    \def\svgwidth{\columnwidth}
    \import{./figures/}{#1.pdf_tex}
}


\usepackage{tikzsymbols}
% \renewcommand\qedsymbol{$\Laughey$}

\definecolor{commentgreen}{RGB}{2,112,10}
%%
%% Julia definition (c) 2014 Jubobs
%%
\lstdefinelanguage{Julia}%
  {morekeywords={abstract,break,case,catch,const,continue,do,else,elseif,%
      end,export,false,for,function,immutable,import,importall,if,in,%
      macro,module,otherwise,quote,return,switch,true,try,type,typealias,%
      using,while},%
   sensitive=true,%
   alsoother={$},%
   morecomment=[l]\#,%
   morecomment=[n]{\#=}{=\#},%
   morestring=[s]{"}{"},%
   morestring=[m]{'}{'},%
}[keywords,comments,strings]%

\lstset{%
    language        	= Julia,
    basicstyle      	= \ttfamily,
    keywordstyle    	= \bfseries\color{blue},
    stringstyle     	= \color{magenta},
    commentstyle    	= \color{commentgreen},
    showstringspaces	= false,
		numbers						= left,
		tabsize						= 4,
}

\definecolor{stringyellow}{RGB}{227, 78, 48}
%% 
%% Shamelessly stolen from Vivi on Stackoverflow
%% https://tex.stackexchange.com/questions/75116/what-can-i-use-to-typeset-matlab-code-in-my-document
%%
\lstset{language=Matlab,%
    %basicstyle=\color{red},
    breaklines=true,%
    morekeywords={matlab2tikz},
		morekeywords={subtitle}
    keywordstyle=\color{blue},%
    morekeywords=[2]{1}, keywordstyle=[2]{\color{black}},
    identifierstyle=\color{black},%
    stringstyle=\color{stringyellow},
    commentstyle=\color{commentgreen},%
    showstringspaces=false,%without this there will be a symbol in the places where there is a space
    numbers=left,%
		firstnumber=1,
    % numberstyle={\tiny \color{black}},% size of the numbers
    % numbersep=9pt, % this defines how far the numbers are from the text
    emph=[1]{for,end,break},emphstyle=[1]\color{red}, %some words to emphasise
    %emph=[2]{word1,word2}, emphstyle=[2]{style},    
}

%% 
%% Shamelessly stolen from egreg on Stackoverflow
%% https://tex.stackexchange.com/questions/280681/how-to-have-multiple-lines-of-intertext-within-align-environment
%%
\newlength{\normalparindent}
\AtBeginDocument{\setlength{\normalparindent}{\parindent}}
\newcommand{\longintertext}[1]{%
  \intertext{%
    \parbox{\linewidth}{%
      \setlength{\parindent}{\normalparindent}
      \noindent#1%
    }%
  }%
}

%\usepackage{import}
%\usepackage{xifthen}
%\usepackage{pdfpages}
%\usepackage{transparent}

%%%%%%%%%%%%%%%%%%%%%%%%%%%%%%
% SELF MADE COLORS
%%%%%%%%%%%%%%%%%%%%%%%%%%%%%%
\definecolor{myg}{RGB}{56, 140, 70}
\definecolor{myb}{RGB}{45, 111, 177}
\definecolor{myr}{RGB}{199, 68, 64}
\definecolor{mytheorembg}{HTML}{F2F2F9}
\definecolor{mytheoremfr}{HTML}{00007B}
\definecolor{mylenmabg}{HTML}{FFFAF8}
\definecolor{mylenmafr}{HTML}{983b0f}
\definecolor{mypropbg}{HTML}{f2fbfc}
\definecolor{mypropfr}{HTML}{191971}
\definecolor{myexamplebg}{HTML}{F2FBF8}
\definecolor{myexamplefr}{HTML}{88D6D1}
\definecolor{myexampleti}{HTML}{2A7F7F}
\definecolor{mydefinitbg}{HTML}{E5E5FF}
\definecolor{mydefinitfr}{HTML}{3F3FA3}
\definecolor{notesgreen}{RGB}{0,162,0}
\definecolor{myp}{RGB}{197, 92, 212}
\definecolor{mygr}{HTML}{2C3338}
\definecolor{myred}{RGB}{127,0,0}
\definecolor{myyellow}{RGB}{169,121,69}
\definecolor{myexercisebg}{HTML}{F2FBF8}
\definecolor{myexercisefg}{HTML}{88D6D1}

%%%%%%%%%%%%%%%%%%%%%%%%%%%%
% TCOLORBOX SETUPS
%%%%%%%%%%%%%%%%%%%%%%%%%%%%
\setlength{\parindent}{0pt}

%================================
% THEOREM BOX
%================================
\tcbuselibrary{theorems,skins,hooks}
\newtcbtheorem[number within=section]{Theorem}{Theorem}
{%
	enhanced,
	breakable,
	colback = mytheorembg,
	frame hidden,
	boxrule = 0sp,
	borderline west = {2pt}{0pt}{mytheoremfr},
	sharp corners,
	detach title,
	before upper = \tcbtitle\par\smallskip,
	coltitle = mytheoremfr,
	fonttitle = \bfseries\sffamily,
	description font = \mdseries,
	separator sign none,
	segmentation style={solid, mytheoremfr},
}
{th}

\tcbuselibrary{theorems,skins,hooks}
\newtcbtheorem[number within=chapter]{theorem}{Theorem}
{%
	enhanced,
	breakable,
	colback = mytheorembg,
	frame hidden,
	boxrule = 0sp,
	borderline west = {2pt}{0pt}{mytheoremfr},
	sharp corners,
	detach title,
	before upper = \tcbtitle\par\smallskip,
	coltitle = mytheoremfr,
	fonttitle = \bfseries\sffamily,
	description font = \mdseries,
	separator sign none,
	segmentation style={solid, mytheoremfr},
}
{th}


\tcbuselibrary{theorems,skins,hooks}
\newtcolorbox{Theoremcon}
{%
	enhanced
	,breakable
	,colback = mytheorembg
	,frame hidden
	,boxrule = 0sp
	,borderline west = {2pt}{0pt}{mytheoremfr}
	,sharp corners
	,description font = \mdseries
	,separator sign none
}

%================================
% Corollery
%================================
\tcbuselibrary{theorems,skins,hooks}
\newtcbtheorem[number within=section]{Corollary}{Corollary}
{%
	enhanced
	,breakable
	,colback = myp!10
	,frame hidden
	,boxrule = 0sp
	,borderline west = {2pt}{0pt}{myp!85!black}
	,sharp corners
	,detach title
	,before upper = \tcbtitle\par\smallskip
	,coltitle = myp!85!black
	,fonttitle = \bfseries\sffamily
	,description font = \mdseries
	,separator sign none
	,segmentation style={solid, myp!85!black}
}
{th}
\tcbuselibrary{theorems,skins,hooks}
\newtcbtheorem[number within=chapter]{corollary}{Corollary}
{%
	enhanced
	,breakable
	,colback = myp!10
	,frame hidden
	,boxrule = 0sp
	,borderline west = {2pt}{0pt}{myp!85!black}
	,sharp corners
	,detach title
	,before upper = \tcbtitle\par\smallskip
	,coltitle = myp!85!black
	,fonttitle = \bfseries\sffamily
	,description font = \mdseries
	,separator sign none
	,segmentation style={solid, myp!85!black}
}
{th}

%================================
% LENMA
%================================
\tcbuselibrary{theorems,skins,hooks}
\newtcbtheorem[number within=section]{Lenma}{Lenma}
{%
	enhanced,
	breakable,
	colback = mylenmabg,
	frame hidden,
	boxrule = 0sp,
	borderline west = {2pt}{0pt}{mylenmafr},
	sharp corners,
	detach title,
	before upper = \tcbtitle\par\smallskip,
	coltitle = mylenmafr,
	fonttitle = \bfseries\sffamily,
	description font = \mdseries,
	separator sign none,
	segmentation style={solid, mylenmafr},
}
{th}

\tcbuselibrary{theorems,skins,hooks}
\newtcbtheorem[number within=chapter]{lenma}{Lenma}
{%
	enhanced,
	breakable,
	colback = mylenmabg,
	frame hidden,
	boxrule = 0sp,
	borderline west = {2pt}{0pt}{mylenmafr},
	sharp corners,
	detach title,
	before upper = \tcbtitle\par\smallskip,
	coltitle = mylenmafr,
	fonttitle = \bfseries\sffamily,
	description font = \mdseries,
	separator sign none,
	segmentation style={solid, mylenmafr},
}
{th}

%================================
% PROPOSITION
%================================
\tcbuselibrary{theorems,skins,hooks}
\newtcbtheorem[number within=section]{Prop}{Proposition}
{%
	enhanced,
	breakable,
	colback = mypropbg,
	frame hidden,
	boxrule = 0sp,
	borderline west = {2pt}{0pt}{mypropfr},
	sharp corners,
	detach title,
	before upper = \tcbtitle\par\smallskip,
	coltitle = mypropfr,
	fonttitle = \bfseries\sffamily,
	description font = \mdseries,
	separator sign none,
	segmentation style={solid, mypropfr},
}
{th}

\tcbuselibrary{theorems,skins,hooks}
\newtcbtheorem[number within=chapter]{prop}{Proposition}
{%
	enhanced,
	breakable,
	colback = mypropbg,
	frame hidden,
	boxrule = 0sp,
	borderline west = {2pt}{0pt}{mypropfr},
	sharp corners,
	detach title,
	before upper = \tcbtitle\par\smallskip,
	coltitle = mypropfr,
	fonttitle = \bfseries\sffamily,
	description font = \mdseries,
	separator sign none,
	segmentation style={solid, mypropfr},
}
{th}

%================================
% CLAIM
%================================
\tcbuselibrary{theorems,skins,hooks}
\newtcbtheorem[number within=section]{claim}{Claim}
{%
	enhanced
	,breakable
	,colback = myg!10
	,frame hidden
	,boxrule = 0sp
	,borderline west = {2pt}{0pt}{myg}
	,sharp corners
	,detach title
	,before upper = \tcbtitle\par\smallskip
	,coltitle = myg!85!black
	,fonttitle = \bfseries\sffamily
	,description font = \mdseries
	,separator sign none
	,segmentation style={solid, myg!85!black}
}
{th}

%================================
% Exercise
%================================
\tcbuselibrary{theorems,skins,hooks}
\newtcbtheorem[number within=section]{Exercise}{Exercise}
{%
	enhanced,
	breakable,
	colback = myexercisebg,
	frame hidden,
	boxrule = 0sp,
	borderline west = {2pt}{0pt}{myexercisefg},
	sharp corners,
	detach title,
	before upper = \tcbtitle\par\smallskip,
	coltitle = myexercisefg,
	fonttitle = \bfseries\sffamily,
	description font = \mdseries,
	separator sign none,
	segmentation style={solid, myexercisefg},
}
{th}

\tcbuselibrary{theorems,skins,hooks}
\newtcbtheorem[number within=chapter]{exercise}{Exercise}
{%
	enhanced,
	breakable,
	colback = myexercisebg,
	frame hidden,
	boxrule = 0sp,
	borderline west = {2pt}{0pt}{myexercisefg},
	sharp corners,
	detach title,
	before upper = \tcbtitle\par\smallskip,
	coltitle = myexercisefg,
	fonttitle = \bfseries\sffamily,
	description font = \mdseries,
	separator sign none,
	segmentation style={solid, myexercisefg},
}
{th}

%================================
% EXAMPLE BOX
%================================
\newtcbtheorem[number within=section]{Example}{Example}
{%
	colback = myexamplebg
	,breakable
	,colframe = myexamplefr
	,coltitle = myexampleti
	,boxrule = 1pt
	,sharp corners
	,detach title
	,before upper=\tcbtitle\par\smallskip
	,fonttitle = \bfseries
	,description font = \mdseries
	,separator sign none
	,description delimiters parenthesis
}
{ex}

\newtcbtheorem[number within=chapter]{example}{Example}
{%
	colback = myexamplebg
	,breakable
	,colframe = myexamplefr
	,coltitle = myexampleti
	,boxrule = 1pt
	,sharp corners
	,detach title
	,before upper=\tcbtitle\par\smallskip
	,fonttitle = \bfseries
	,description font = \mdseries
	,separator sign none
	,description delimiters parenthesis
}
{ex}

%================================
% DEFINITION BOX
%================================
\newtcbtheorem[number within=section]{Definition}{Definition}{enhanced,
	before skip=2mm,after skip=2mm, colback=red!5,colframe=red!80!black,boxrule=0.5mm,
	attach boxed title to top left={xshift=1cm,yshift*=1mm-\tcboxedtitleheight}, varwidth boxed title*=-3cm,
	boxed title style={frame code={
					\path[fill=tcbcolback]
					([yshift=-1mm,xshift=-1mm]frame.north west)
					arc[start angle=0,end angle=180,radius=1mm]
					([yshift=-1mm,xshift=1mm]frame.north east)
					arc[start angle=180,end angle=0,radius=1mm];
					\path[left color=tcbcolback!60!black,right color=tcbcolback!60!black,
						middle color=tcbcolback!80!black]
					([xshift=-2mm]frame.north west) -- ([xshift=2mm]frame.north east)
					[rounded corners=1mm]-- ([xshift=1mm,yshift=-1mm]frame.north east)
					-- (frame.south east) -- (frame.south west)
					-- ([xshift=-1mm,yshift=-1mm]frame.north west)
					[sharp corners]-- cycle;
				},interior engine=empty,
		},
	fonttitle=\bfseries,
	title={#2},#1}{def}
\newtcbtheorem[number within=chapter]{definition}{Definition}{enhanced,
	before skip=2mm,after skip=2mm, colback=red!5,colframe=red!80!black,boxrule=0.5mm,
	attach boxed title to top left={xshift=1cm,yshift*=1mm-\tcboxedtitleheight}, varwidth boxed title*=-3cm,
	boxed title style={frame code={
					\path[fill=tcbcolback]
					([yshift=-1mm,xshift=-1mm]frame.north west)
					arc[start angle=0,end angle=180,radius=1mm]
					([yshift=-1mm,xshift=1mm]frame.north east)
					arc[start angle=180,end angle=0,radius=1mm];
					\path[left color=tcbcolback!60!black,right color=tcbcolback!60!black,
						middle color=tcbcolback!80!black]
					([xshift=-2mm]frame.north west) -- ([xshift=2mm]frame.north east)
					[rounded corners=1mm]-- ([xshift=1mm,yshift=-1mm]frame.north east)
					-- (frame.south east) -- (frame.south west)
					-- ([xshift=-1mm,yshift=-1mm]frame.north west)
					[sharp corners]-- cycle;
				},interior engine=empty,
		},
	fonttitle=\bfseries,
	title={#2},#1}{def}

%================================
% Solution BOX
%================================
\makeatletter
\newtcbtheorem{question}{Question}{enhanced,
	breakable,
	colback=white,
	colframe=myb!80!black,
	attach boxed title to top left={yshift*=-\tcboxedtitleheight},
	fonttitle=\bfseries,
	title={#2},
	boxed title size=title,
	boxed title style={%
			sharp corners,
			rounded corners=northwest,
			colback=tcbcolframe,
			boxrule=0pt,
		},
	underlay boxed title={%
			\path[fill=tcbcolframe] (title.south west)--(title.south east)
			to[out=0, in=180] ([xshift=5mm]title.east)--
			(title.center-|frame.east)
			[rounded corners=\kvtcb@arc] |-
			(frame.north) -| cycle;
		},
	#1
}{def}
\makeatother

%================================
% SOLUTION BOX
%================================
\makeatletter
\newtcolorbox{solution}{enhanced,
	breakable,
	colback=white,
	colframe=myg!80!black,
	attach boxed title to top left={yshift*=-\tcboxedtitleheight},
	title=Solution,
	boxed title size=title,
	boxed title style={%
			sharp corners,
			rounded corners=northwest,
			colback=tcbcolframe,
			boxrule=0pt,
		},
	underlay boxed title={%
			\path[fill=tcbcolframe] (title.south west)--(title.south east)
			to[out=0, in=180] ([xshift=5mm]title.east)--
			(title.center-|frame.east)
			[rounded corners=\kvtcb@arc] |-
			(frame.north) -| cycle;
		},
}
\makeatother

%================================
% Question BOX
%================================
\makeatletter
\newtcbtheorem{qstion}{Question}{enhanced,
	breakable,
	colback=white,
	colframe=mygr,
	attach boxed title to top left={yshift*=-\tcboxedtitleheight},
	fonttitle=\bfseries,
	title={#2},
	boxed title size=title,
	boxed title style={%
			sharp corners,
			rounded corners=northwest,
			colback=tcbcolframe,
			boxrule=0pt,
		},
	underlay boxed title={%
			\path[fill=tcbcolframe] (title.south west)--(title.south east)
			to[out=0, in=180] ([xshift=5mm]title.east)--
			(title.center-|frame.east)
			[rounded corners=\kvtcb@arc] |-
			(frame.north) -| cycle;
		},
	#1
}{def}
\makeatother

\newtcbtheorem[number within=chapter]{wconc}{Wrong Concept}{
	breakable,
	enhanced,
	colback=white,
	colframe=myr,
	arc=0pt,
	outer arc=0pt,
	fonttitle=\bfseries\sffamily\large,
	colbacktitle=myr,
	attach boxed title to top left={},
	boxed title style={
			enhanced,
			skin=enhancedfirst jigsaw,
			arc=3pt,
			bottom=0pt,
			interior style={fill=myr}
		},
	#1
}{def}

%================================
% NOTE BOX
%================================
\usetikzlibrary{arrows,calc,shadows.blur}
\tcbuselibrary{skins}
\newtcolorbox{note}[1][]{%
	enhanced jigsaw,
	colback=gray!20!white,%
	colframe=gray!80!black,
	size=small,
	boxrule=1pt,
	title=\textbf{Note:-},
	halign title=flush center,
	coltitle=black,
	breakable,
	drop shadow=black!50!white,
	attach boxed title to top left={xshift=1cm,yshift=-\tcboxedtitleheight/2,yshifttext=-\tcboxedtitleheight/2},
	minipage boxed title=1.5cm,
	boxed title style={%
			colback=white,
			size=fbox,
			boxrule=1pt,
			boxsep=2pt,
			underlay={%
					\coordinate (dotA) at ($(interior.west) + (-0.5pt,0)$);
					\coordinate (dotB) at ($(interior.east) + (0.5pt,0)$);
					\begin{scope}
						\clip (interior.north west) rectangle ([xshift=3ex]interior.east);
						\filldraw [white, blur shadow={shadow opacity=60, shadow yshift=-.75ex}, rounded corners=2pt] (interior.north west) rectangle (interior.south east);
					\end{scope}
					\begin{scope}[gray!80!black]
						\fill (dotA) circle (2pt);
						\fill (dotB) circle (2pt);
					\end{scope}
				},
		},
	#1,
}

%%%%%%%%%%%%%%%%%%%%%%%%%%%%%%
% SELF MADE COMMANDS
%%%%%%%%%%%%%%%%%%%%%%%%%%%%%%
\newcommand{\thm}[2]{\begin{Theorem}{#1}{}#2\end{Theorem}}
\newcommand{\cor}[2]{\begin{Corollary}{#1}{}#2\end{Corollary}}
\newcommand{\mlenma}[2]{\begin{Lenma}{#1}{}#2\end{Lenma}}
\newcommand{\mprop}[2]{\begin{Prop}{#1}{}#2\end{Prop}}
\newcommand{\clm}[3]{\begin{claim}{#1}{#2}#3\end{claim}}
\newcommand{\wc}[2]{\begin{wconc}{#1}{}\setlength{\parindent}{1cm}#2\end{wconc}}
\newcommand{\thmcon}[1]{\begin{Theoremcon}{#1}\end{Theoremcon}}
\newcommand{\ex}[2]{\begin{Example}{#1}{}#2\end{Example}}
\newcommand{\dfn}[2]{\begin{Definition}[colbacktitle=red!75!black]{#1}{}#2\end{Definition}}
\newcommand{\dfnc}[2]{\begin{definition}[colbacktitle=red!75!black]{#1}{}#2\end{definition}}
\newcommand{\qs}[2]{\begin{question}{#1}{}#2\end{question}}
\newcommand{\pf}[2]{\begin{myproof}[#1]#2\end{myproof}}
\newcommand{\nt}[1]{\begin{note}#1\end{note}}

\newcommand*\circled[1]{\tikz[baseline=(char.base)]{
		\node[shape=circle,draw,inner sep=1pt] (char) {#1};}}
\newcommand\getcurrentref[1]{%
	\ifnumequal{\value{#1}}{0}
	{??}
	{\the\value{#1}}%
}
\newcommand{\getCurrentSectionNumber}{\getcurrentref{section}}
\newenvironment{myproof}[1][\proofname]{%
	\proof[\bfseries #1: ]%
}{\endproof}

\newcommand{\mclm}[2]{\begin{myclaim}[#1]#2\end{myclaim}}
\newenvironment{myclaim}[1][\claimname]{\proof[\bfseries #1: ]}{}

\newcounter{mylabelcounter}

\makeatletter
\newcommand{\setword}[2]{%
	\phantomsection
	#1\def\@currentlabel{\unexpanded{#1}}\label{#2}%
}
\makeatother

\tikzset{
	symbol/.style={
			draw=none,
			every to/.append style={
					edge node={node [sloped, allow upside down, auto=false]{$#1$}}}
		}
}

% deliminators
\DeclarePairedDelimiter{\abs}{\lvert}{\rvert}
\DeclarePairedDelimiter{\norm}{\lVert}{\rVert}

\DeclarePairedDelimiter{\ceil}{\lceil}{\rceil}
\DeclarePairedDelimiter{\floor}{\lfloor}{\rfloor}
\DeclarePairedDelimiter{\round}{\lfloor}{\rceil}

\newsavebox\diffdbox
\newcommand{\slantedromand}{{\mathpalette\makesl{d}}}
\newcommand{\makesl}[2]{%
\begingroup
\sbox{\diffdbox}{$\mathsurround=0pt#1\mathrm{#2}$}%
\pdfsave
\pdfsetmatrix{1 0 0.2 1}%
\rlap{\usebox{\diffdbox}}%
\pdfrestore
\hskip\wd\diffdbox
\endgroup
}
\newcommand{\dd}[1][]{\ensuremath{\mathop{}\!\ifstrempty{#1}{%
\slantedromand\@ifnextchar^{\hspace{0.2ex}}{\hspace{0.1ex}}}%
{\slantedromand\hspace{0.2ex}^{#1}}}}
\ProvideDocumentCommand\dv{o m g}{%
  \ensuremath{%
    \IfValueTF{#3}{%
      \IfNoValueTF{#1}{%
        \frac{\dd #2}{\dd #3}%
      }{%
        \frac{\dd^{#1} #2}{\dd #3^{#1}}%
      }%
    }{%
      \IfNoValueTF{#1}{%
        \frac{\dd}{\dd #2}%
      }{%
        \frac{\dd^{#1}}{\dd #2^{#1}}%
      }%
    }%
  }%
}
\providecommand*{\pdv}[3][]{\frac{\partial^{#1}#2}{\partial#3^{#1}}}
%  - others
\DeclareMathOperator{\Lap}{\mathcal{L}}
\DeclareMathOperator{\Var}{Var} % varience
\DeclareMathOperator{\Cov}{Cov} % covarience
\DeclareMathOperator{\E}{E} % expected

% Since the amsthm package isn't loaded

% I dot not prefer the slanted \leq ;P
% % I prefer the slanted \leq
% \let\oldleq\leq % save them in case they're every wanted
% \let\oldgeq\geq
% \renewcommand{\leq}{\leqslant}
% \renewcommand{\geq}{\geqslant}

% % redefine matrix env to allow for alignment, use r as default
% \renewcommand*\env@matrix[1][r]{\hskip -\arraycolsep
%     \let\@ifnextchar\new@ifnextchar
%     \array{*\c@MaxMatrixCols #1}}

%\usepackage{framed}
%\usepackage{titletoc}
%\usepackage{etoolbox}
%\usepackage{lmodern}

%\patchcmd{\tableofcontents}{\contentsname}{\sffamily\contentsname}{}{}

%\renewenvironment{leftbar}
%{\def\FrameCommand{\hspace{6em}%
%		{\color{myyellow}\vrule width 2pt depth 6pt}\hspace{1em}}%
%	\MakeFramed{\parshape 1 0cm \dimexpr\textwidth-6em\relax\FrameRestore}\vskip2pt%
%}
%{\endMakeFramed}

%\titlecontents{chapter}
%[0em]{\vspace*{2\baselineskip}}
%{\parbox{4.5em}{%
%		\hfill\Huge\sffamily\bfseries\color{myred}\thecontentspage}%
%	\vspace*{-2.3\baselineskip}\leftbar\textsc{\small\chaptername~\thecontentslabel}\\\sffamily}
%{}{\endleftbar}
%\titlecontents{section}
%[8.4em]
%{\sffamily\contentslabel{3em}}{}{}
%{\hspace{0.5em}\nobreak\itshape\color{myred}\contentspage}
%\titlecontents{subsection}
%[8.4em]
%{\sffamily\contentslabel{3em}}{}{}  
%{\hspace{0.5em}\nobreak\itshape\color{myred}\contentspage}

%%%%%%%%%%%%%%%%%%%%%%%%%%%%%%%%%%%%%%%%%%%
% TABLE OF CONTENTS
%%%%%%%%%%%%%%%%%%%%%%%%%%%%%%%%%%%%%%%%%%%
\usepackage{tikz}
\definecolor{doc}{RGB}{0,60,110}
\usepackage{titletoc}
\contentsmargin{0cm}
\titlecontents{chapter}[3.7pc]
{\addvspace{30pt}%
	\begin{tikzpicture}[remember picture, overlay]%
		\draw[fill=doc!60,draw=doc!60] (-7,-.1) rectangle (-0.9,.5);%
		\pgftext[left,x=-3.5cm,y=0.2cm]{\color{white}\Large\sc\bfseries Chapter\ \thecontentslabel};%
	\end{tikzpicture}\color{doc!60}\large\sc\bfseries}%
{}
{}
{\;\titlerule\;\large\sc\bfseries Page \thecontentspage
	\begin{tikzpicture}[remember picture, overlay]
		\draw[fill=doc!60,draw=doc!60] (2pt,0) rectangle (4,0.1pt);
	\end{tikzpicture}}%
\titlecontents{section}[3.7pc]
{\addvspace{2pt}}
{\contentslabel[\thecontentslabel]{2pc}}
{}
{\hfill\small \thecontentspage}
[]
\titlecontents*{subsection}[3.7pc]
{\addvspace{-1pt}\small}
{}
{}
{\ --- \small\thecontentspage}
[ \textbullet\ ][]

\makeatletter
\renewcommand{\tableofcontents}{%
	\chapter*{%
	  \vspace*{-20\p@}%
	  \begin{tikzpicture}[remember picture, overlay]%
		  \pgftext[right,x=15cm,y=0.2cm]{\color{doc!60}\Huge\sc\bfseries \contentsname};%
		  \draw[fill=doc!60,draw=doc!60] (13,-.75) rectangle (20,1);%
		  \clip (13,-.75) rectangle (20,1);
		  \pgftext[right,x=15cm,y=0.2cm]{\color{white}\Huge\sc\bfseries \contentsname};%
	  \end{tikzpicture}}%
	\@starttoc{toc}}
\makeatother

\newcommand{\inv}{^{-1}}
\newcommand{\opname}{\operatorname}
\newcommand{\surjto}{\twoheadrightarrow}
% \newcommand{\injto}{\hookrightarrow}
\newcommand{\injto}{\rightarrowtail}
\newcommand{\bijto}{\leftrightarrow}

\newcommand{\liff}{\leftrightarrow}
\newcommand{\notliff}{\mathrel{\ooalign{$\leftrightarrow$\cr\hidewidth$/$\hidewidth}}}
\newcommand{\lthen}{\rightarrow}
\let\varlnot\lnot
\newcommand{\ordsim}{\mathord{\sim}}
\renewcommand{\lnot}{\ordsim}
\newcommand{\lxor}{\oplus}
\newcommand{\lnand}{\barwedge}
\newcommand{\divs}{\mathrel{\mid}}
\newcommand{\ndivs}{\mathrel{\nmid}}
\def\contra{\tikz[baseline, x=0.22em, y=0.22em, line width=0.032em]\draw (0,2.83)--(2.83,0) (0.71,3.54)--(3.54,0.71) (0,0.71)--(2.83,3.54) (0.71,0)--(3.54,2.83);}

\newcommand{\On}{\mathrm{On}} % ordinals
\DeclareMathOperator{\img}{im} % Image
\DeclareMathOperator{\Img}{Im} % Image
\DeclareMathOperator{\coker}{coker} % Cokernel
\DeclareMathOperator{\Coker}{Coker} % Cokernel
\DeclareMathOperator{\Ker}{Ker} % Kernel
\DeclareMathOperator{\rank}{rank}
\DeclareMathOperator{\Spec}{Spec} % spectrum
\DeclareMathOperator{\Tr}{Tr} % trace
\DeclareMathOperator{\pr}{pr} % projection
\DeclareMathOperator{\ext}{ext} % extension
\DeclareMathOperator{\pred}{pred} % predecessor
\DeclareMathOperator{\dom}{dom} % domain
\DeclareMathOperator{\ran}{ran} % range
\DeclareMathOperator{\Hom}{Hom} % homomorphism
\DeclareMathOperator{\Mor}{Mor} % morphisms
\DeclareMathOperator{\End}{End} % endomorphism
\DeclareMathOperator{\Span}{span}
\newcommand{\Mod}{\mathbin{\mathrm{mod}}}

\newcommand{\eps}{\epsilon}
\newcommand{\veps}{\varepsilon}
\newcommand{\ol}{\overline}
\newcommand{\ul}{\underline}
\newcommand{\wt}{\widetilde}
\newcommand{\wh}{\widehat}
\newcommand{\ut}{\utilde}
\newcommand{\unit}[1]{\ut{\hat{#1}}}
\newcommand{\emp}{\varnothing}

\newcommand{\vocab}[1]{\textbf{\color{blue} #1}}
\providecommand{\half}{\frac{1}{2}}
\newcommand{\dang}{\measuredangle} %% Directed angle
\newcommand{\ray}[1]{\overrightarrow{#1}}
\newcommand{\seg}[1]{\overline{#1}}
\newcommand{\arc}[1]{\wideparen{#1}}
\DeclareMathOperator{\cis}{cis}
\DeclareMathOperator*{\lcm}{lcm}
\DeclareMathOperator*{\argmin}{arg min}
\DeclareMathOperator*{\argmax}{arg max}
\newcommand{\cycsum}{\sum_{\mathrm{cyc}}}
\newcommand{\symsum}{\sum_{\mathrm{sym}}}
\newcommand{\cycprod}{\prod_{\mathrm{cyc}}}
\newcommand{\symprod}{\prod_{\mathrm{sym}}}
\newcommand{\parinn}{\setlength{\parindent}{1cm}}
\newcommand{\parinf}{\setlength{\parindent}{0cm}}
% \newcommand{\norm}{\|\cdot\|}
\newcommand{\inorm}{\norm_{\infty}}
\newcommand{\opensets}{\{V_{\alpha}\}_{\alpha\in I}}
\newcommand{\oset}{V_{\alpha}}
\newcommand{\opset}[1]{V_{\alpha_{#1}}}
\newcommand{\lub}{\text{lub}}
\newcommand{\lm}{\lambda}
\newcommand{\uin}{\mathbin{\rotatebox[origin=c]{90}{$\in$}}}
\newcommand{\usubset}{\mathbin{\rotatebox[origin=c]{90}{$\subset$}}}
\newcommand{\lt}{\left}
\newcommand{\rt}{\right}
\newcommand{\bs}[1]{\boldsymbol{#1}}
\newcommand{\exs}{\exists}
\newcommand{\st}{\strut}
\newcommand{\dps}[1]{\displaystyle{#1}}

\newcommand{\sol}{\textbf{\textit{Solution:}} }
\newcommand{\solve}[1]{\textbf{\textit{Solution: }} #1 \qed}
% \newcommand{\proof}{\underline{\textit{proof:}}\\}

\DeclareMathOperator{\sech}{sech}
\DeclareMathOperator{\csch}{csch}
\DeclareMathOperator{\arcsec}{arcsec}
\DeclareMathOperator{\arccsc}{arccsc}
\DeclareMathOperator{\arccot}{arccot}
\DeclareMathOperator{\arsinh}{arsinh}
\DeclareMathOperator{\arcosh}{arcosh}
\DeclareMathOperator{\artanh}{artanh}
\DeclareMathOperator{\arcsch}{arcsch}
\DeclareMathOperator{\arsech}{arsech}
\DeclareMathOperator{\arcoth}{arcoth}

\newcommand{\sinx}{\sin x}          \newcommand{\arcsinx}{\arcsin x}    
\newcommand{\cosx}{\cos x}          \newcommand{\arccosx}{\arccosx}
\newcommand{\tanx}{\tan x}          \newcommand{\arctanx}{\arctan x}
\newcommand{\cscx}{\csc x}          \newcommand{\arccscx}{\arccsc x}
\newcommand{\secx}{\sec x}          \newcommand{\arcsecx}{\arcsec x}
\newcommand{\cotx}{\cot x}          \newcommand{\arccotx}{\arccot x}
\newcommand{\sinhx}{\sinh x}          \newcommand{\arsinhx}{\arsinh x}
\newcommand{\coshx}{\cosh x}          \newcommand{\arcoshx}{\arcosh x}
\newcommand{\tanhx}{\tanh x}          \newcommand{\artanhx}{\artanh x}
\newcommand{\cschx}{\csch x}          \newcommand{\arcschx}{\arcsch x}
\newcommand{\sechx}{\sech x}          \newcommand{\arsechx}{\arsech x}
\newcommand{\cothx}{\coth x}          \newcommand{\arcothx}{\arcoth x}
\newcommand{\lnx}{\ln x}
\newcommand{\expx}{\exp x}

\newcommand{\Theom}{\textbf{Theorem. }}
\newcommand{\Lemma}{\textbf{Lemma. }}
\newcommand{\Corol}{\textbf{Corollary. }}
\newcommand{\Remar}{\textit{Remark. }}
\newcommand{\Defin}[1]{\textbf{Definition} (#1).}
\newcommand{\Claim}{\textbf{Claim. }}
\newcommand{\Propo}{\textbf{Proposition. }}

\newcommand{\lb}{\left(}
\newcommand{\rb}{\right)}
\newcommand{\lbr}{\left\lbrace}
\newcommand{\rbr}{\right\rbrace}
\newcommand{\lsb}{\left[}
\newcommand{\rsb}{\right]}
\newcommand{\bracks}[1]{\lb #1 \rb}
\newcommand{\braces}[1]{\lbr #1 \rbr}
\newcommand{\suchthat}{\medspace\middle|\medspace}
\newcommand{\sqbracks}[1]{\lsb #1 \rsb}
\renewcommand{\abs}[1]{\left| #1 \right|}
\newcommand{\Mag}[1]{\left|\left| #1 \right|\right|}
\renewcommand{\floor}[1]{\left\lfloor #1 \right\rfloor}
\renewcommand{\ceil}[1]{\left\lceil #1 \right\rceil}

\newcommand{\cd}{\cdot}
\newcommand{\tf}{\therefore}
\newcommand{\Let}{\text{Let }}
\newcommand{\Given}{\text{Given }}
% \newcommand{\and}{\text{and }}
\newcommand{\Substitute}{\text{Substitute }}
\newcommand{\Suppose}{\text{Suppose }}
\newcommand{\WeSee}{\text{We see }}
\newcommand{\So}{\text{So }}
\newcommand{\Then}{\text{Then }}
\newcommand{\Choose}{\text{Choose }}
\newcommand{\Take}{\text{Take }}
\newcommand{\false}{\text{False}}
\newcommand{\true}{\text{True}}

\newcommand{\QED}{\hfill \qed}
\newcommand{\CONTRA}{\hfill \contra}

\newcommand{\ihat}{\hat{\imath}}
\newcommand{\jhat}{\hat{\jmath}}
\newcommand{\khat}{\hat{k}}

\newcommand{\grad}{\nabla}
\newcommand{\D}{\Delta}
\renewcommand{\d}{\mathrm{d}}

\renewcommand{\dd}[1]{\frac{\d}{\d #1}}
\newcommand{\dyd}[2][y]{\frac{\d #1}{\d #2}}

\newcommand{\ddx}{\dd{x}}       \newcommand{\ddxsq}{\dyd[^2]{x^2}}
\newcommand{\ddy}{\dd{y}}       \newcommand{\ddysq}{\dyd[^2]{y^2}}
\newcommand{\ddu}{\dd{u}}       \newcommand{\ddusq}{\dyd[^2]{u^2}}
\newcommand{\ddv}{\dd{v}}       \newcommand{\ddvsq}{\dyd[^2]{v^2}}

\newcommand{\dydx}{\dyd{x}}     \newcommand{\dydxsq}{\dyd[^2y]{x^2}}
\newcommand{\dfdx}{\dyd[f]{x}}  \newcommand{\dfdxsq}{\dyd[^2f]{x^2}}
\newcommand{\dudx}{\dyd[u]{x}}  \newcommand{\dudxsq}{\dyd[^2u]{x^2}}
\newcommand{\dvdx}{\dyd[v]{x}}  \newcommand{\dvdxsq}{\dyd[^2v]{x^2}}

\newcommand{\del}[2]{\frac{\partial #1}{\partial #2}}
\newcommand{\Del}[3]{\frac{\partial^{#1} #2}{\partial #3^{#1}}}
\newcommand{\deld}[2]{\dfrac{\partial #1}{\partial #2}}
\newcommand{\Deld}[3]{\dfrac{\partial^{#1} #2}{\partial #3^{#1}}}

\newcommand{\argument}[2]{
  \begin{array}{rll}
    #1
    \cline{2-2}
    \therefore & #2 
  \end{array}
}
% Mathfrak primes
\newcommand{\km}{\mathfrak m}
\newcommand{\kp}{\mathfrak p}
\newcommand{\kq}{\mathfrak q}

%---------------------------------------
% Blackboard Math Fonts :-
%---------------------------------------
\newcommand{\bba}{\mathbb{A}}   \newcommand{\bbn}{\mathbb{N}}
\newcommand{\bbb}{\mathbb{B}}   \newcommand{\bbo}{\mathbb{O}}
\newcommand{\bbc}{\mathbb{C}}   \newcommand{\bbp}{\mathbb{P}}
\newcommand{\bbd}{\mathbb{D}}   \newcommand{\bbq}{\mathbb{Q}}
\newcommand{\bbe}{\mathbb{E}}   \newcommand{\bbr}{\mathbb{R}}
\newcommand{\bbf}{\mathbb{F}}   \newcommand{\bbs}{\mathbb{S}}
\newcommand{\bbg}{\mathbb{G}}   \newcommand{\bbt}{\mathbb{T}}
\newcommand{\bbh}{\mathbb{H}}   \newcommand{\bbu}{\mathbb{U}}
\newcommand{\bbi}{\mathbb{I}}   \newcommand{\bbv}{\mathbb{V}}
\newcommand{\bbj}{\mathbb{J}}   \newcommand{\bbw}{\mathbb{W}}
\newcommand{\bbk}{\mathbb{K}}   \newcommand{\bbx}{\mathbb{X}}
\newcommand{\bbl}{\mathbb{L}}   \newcommand{\bby}{\mathbb{Y}}
\newcommand{\bbm}{\mathbb{M}}   \newcommand{\bbz}{\mathbb{Z}}

%---------------------------------------
% Roman Math Fonts :-
%---------------------------------------
\newcommand{\rma}{\mathrm{A}}   \newcommand{\rmn}{\mathrm{N}}
\newcommand{\rmb}{\mathrm{B}}   \newcommand{\rmo}{\mathrm{O}}
\newcommand{\rmc}{\mathrm{C}}   \newcommand{\rmp}{\mathrm{P}}
\newcommand{\rmd}{\mathrm{D}}   \newcommand{\rmq}{\mathrm{Q}}
\newcommand{\rme}{\mathrm{E}}   \newcommand{\rmr}{\mathrm{R}}
\newcommand{\rmf}{\mathrm{F}}   \newcommand{\rms}{\mathrm{S}}
\newcommand{\rmg}{\mathrm{G}}   \newcommand{\rmt}{\mathrm{T}}
\newcommand{\rmh}{\mathrm{H}}   \newcommand{\rmu}{\mathrm{U}}
\newcommand{\rmi}{\mathrm{I}}   \newcommand{\rmv}{\mathrm{V}}
\newcommand{\rmj}{\mathrm{J}}   \newcommand{\rmw}{\mathrm{W}}
\newcommand{\rmk}{\mathrm{K}}   \newcommand{\rmx}{\mathrm{X}}
\newcommand{\rml}{\mathrm{L}}   \newcommand{\rmy}{\mathrm{Y}}
\newcommand{\rmm}{\mathrm{M}}   \newcommand{\rmz}{\mathrm{Z}}

%---------------------------------------
% Calligraphic Math Fonts :-
%---------------------------------------
\newcommand{\cla}{\mathcal{A}}   \newcommand{\cln}{\mathcal{N}}
\newcommand{\clb}{\mathcal{B}}   \newcommand{\clo}{\mathcal{O}}
\newcommand{\clc}{\mathcal{C}}   \newcommand{\clp}{\mathcal{P}}
\newcommand{\cld}{\mathcal{D}}   \newcommand{\clq}{\mathcal{Q}}
\newcommand{\cle}{\mathcal{E}}   \newcommand{\clr}{\mathcal{R}}
\newcommand{\clf}{\mathcal{F}}   \newcommand{\cls}{\mathcal{S}}
\newcommand{\clg}{\mathcal{G}}   \newcommand{\clt}{\mathcal{T}}
\newcommand{\clh}{\mathcal{H}}   \newcommand{\clu}{\mathcal{U}}
\newcommand{\cli}{\mathcal{I}}   \newcommand{\clv}{\mathcal{V}}
\newcommand{\clj}{\mathcal{J}}   \newcommand{\clw}{\mathcal{W}}
\newcommand{\clk}{\mathcal{K}}   \newcommand{\clx}{\mathcal{X}}
\newcommand{\cll}{\mathcal{L}}   \newcommand{\cly}{\mathcal{Y}}
\newcommand{\calm}{\mathcal{M}}  \newcommand{\clz}{\mathcal{Z}}

%---------------------------------------
% Fraktur  Math Fonts :-
%---------------------------------------
\newcommand{\fka}{\mathfrak{A}}   \newcommand{\fkn}{\mathfrak{N}}
\newcommand{\fkb}{\mathfrak{B}}   \newcommand{\fko}{\mathfrak{O}}
\newcommand{\fkc}{\mathfrak{C}}   \newcommand{\fkp}{\mathfrak{P}}
\newcommand{\fkd}{\mathfrak{D}}   \newcommand{\fkq}{\mathfrak{Q}}
\newcommand{\fke}{\mathfrak{E}}   \newcommand{\fkr}{\mathfrak{R}}
\newcommand{\fkf}{\mathfrak{F}}   \newcommand{\fks}{\mathfrak{S}}
\newcommand{\fkg}{\mathfrak{G}}   \newcommand{\fkt}{\mathfrak{T}}
\newcommand{\fkh}{\mathfrak{H}}   \newcommand{\fku}{\mathfrak{U}}
\newcommand{\fki}{\mathfrak{I}}   \newcommand{\fkv}{\mathfrak{V}}
\newcommand{\fkj}{\mathfrak{J}}   \newcommand{\fkw}{\mathfrak{W}}
\newcommand{\fkk}{\mathfrak{K}}   \newcommand{\fkx}{\mathfrak{X}}
\newcommand{\fkl}{\mathfrak{L}}   \newcommand{\fky}{\mathfrak{Y}}
\newcommand{\fkm}{\mathfrak{M}}   \newcommand{\fkz}{\mathfrak{Z}}


\title{\Huge{MATH1061}\\Discrete Mathematics I}
\author{\huge{Problem Set 2}\\\huge{Michael Kasumagic, sID\#: 44302669}}
\date{\huge{Due: 5pm, $6^\text{th}$ of September, 2024}}

\begin{document}

\maketitle

\qs{(\it10 marks)}{
  Prove the following statements:
  \begin{enumerate}[label=(\alph*)]
    \item The sum of every five consecutive integers is always divisible by 5.
    \item Suppose n is an odd integer. The sum of every n consecutive integers is always divisible by n.
  \end{enumerate}
}
\sol (a) \proof We can express five consecutive integers as follows:
\begin{gather*}
  i_0, i_1, i_2, i_3, i_4,
  \longintertext{where $i_0, a\in\bbz$ and $i_a = i_0 + a$.\\ Then, the sum of five consecutive integers can be expressed}
  \begin{aligned}
    i_0 + i_1 + i_2 + i_3 + i_4 &= i_0 + i_0 + 1 + i_0 + 2 + i_0 + 3 + i_0 + 4 \\
      &= 5i_0 + 10 \\
      &= 5(i_0 + 2) \\
    \tf i_0 + i_1 + i_2 + i_3 + i_4 &= 5k,\quad k\in\bbz
  \end{aligned}\\
  \implies 5 \divs i_0 + i_1 + i_2 + i_3 + i_4
\end{gather*}
So, the sum of any 5 consecutive integers is always divisible by 5. $\QED$\\

\sol (b) \proof We can express $n$ consecutive integers, where $n$ is an odd integer, as follows:
\begin{gather*}
  i_0, i_1, i_2, \dots, i_{n-2}, i_{n-1}, i_{n},
  \longintertext{where $i_0, a, j\in\bbz,\ i_a = i_0 + a$, and $n=2j+1$}
  \begin{aligned}
    i_0 + i_1 + i_2 + \dots + i_{n-2} + i_{n-1} + i_{n} &= i_0 + i_0 + 1 + i_0 + 2 + \dots + i_0 + n-2 + i_0 + n-1 + i_0 + n \\
      &= ni_0 + (1 + 2 + 3 \dots + n-2 + n-1 + n) \\
      &= ni_0 + \sum_{a = 1}^{n} a
    \intertext{We can apply Gau\ss's formula for the sum of consecutive natural numbers,}
      &= ni_0 + \frac{n(n+1)}{2} \\
      &= n\bracks{i_0 + \frac{n+1}{2}} \\
    \tf i_0 + i_1 + i_2 + \dots + i_{n-2} + i_{n-1} + i_{n} &= nk,\quad k\in\bbz
  \end{aligned} \\
  \implies n \divs i_0 + i_1 + i_2 + \dots + i_{n-2} + i_{n-1} + i_{n}
\end{gather*}
So, the sum of any $n$ consecutive integers, where $n$ is odd, is always divisible by 5. $\QED$\\

\newpage
\qs{(\it5 marks)}{
  \begin{enumerate}[label=(\alph*)]
    \item Compute the following quantities 
      $$ \floor{3.6},\qquad \ceil{\pi},\qquad \ceil{e},\qquad \ceil{e+0.5} $$
    \item Prove or disprove the following statements: for all numbers $x$: 
      $$ \ceil{x + 0.5} = \ceil{x} + 1$$
  \end{enumerate}
}
\sol (a)
\begin{gather*}
  \floor{3.6} = 3 \\
  \ceil{\pi} = \ceil{3.1415\dots} = 4 \\
  \ceil{e} = \ceil{2.7182\dots} = 3 \\
  \ceil{e+0.5} = \ceil{2.7182\dots+0.5} = \ceil{3.2182\dots} = 4 \\
\end{gather*}

\sol (b) 
\proof Take $x=0.1$.
\begin{gather*}
  \ceil{x + 0.5 } = \ceil{0.1 + 0.5} = \ceil{0.6} = 1 \neq 2 = 1 + 1 = \ceil{0.1} + 1 = \ceil{x} + 1
\end{gather*}
Therefore, for all $x\in\bbr,\ \ceil{x+0.5} = \ceil{x} + 1$ is not true. $\QED$ \\

For the sake of interest, let's examine the expressions and see under what circumstances they are equal or not.
\proof In general, let's think of the number $x$ as being equal to $n + r$, where $n$ is an integer component, and $r$ is the real, decimal component. For example, $\pi = n + r$, where $n=3$ and $r=0.1415\dots$. By definition, $0 \leq r < 1$. With all this in mind, we can can consider
\begin{gather*}
  \ceil{x + 0.5} = \ceil{n + r + 0.5}
  \longintertext{This allows us to consider two cases, \\ Case 1: $0 \leq r \leq 0.5$}
  n < n + r + 0.5  \leq n + 1 \\
  \tf \ceil{n + r + 0.5} = n + 1 \\
  n \leq n + r \leq n + 1 \\
  \tf\ceil{n + r} + 1 = n + 1 + 1 \tag*{(!!)} 
  \intertext{Case 2: $0.5 < r < 1$}
  n + 1 < n + r + 0.5 < n+1.5 \\
  \tf \ceil{n + r + 0.5} = n + 2 \\
  n < n + r < n + 1 \\
  \tf \ceil{n + r} + 1 = n + 1 + 1 \\
  \longintertext{Therefore, for $x\in\bbr: x = n + r,\ n\in\bbz, r\in\bbr$ if $0.5<r<1$ then $\ceil{x+0.5} = \ceil{x} + 1$. However, if $0 \leq r \leq 0.5$ then $\ceil{x+0.5} \neq \ceil{x} + 1$ (see the (!!) tag). \\ Which means the given statement, $\forall x\in\bbr, \ceil{x+0.5} = \ceil{x} + 1$ is false. $\QED$}
\end{gather*}

\newpage
\qs{(\it10 marks)}{
  \begin{enumerate}[label=(\alph*)]
    \item Use the definition, prove or disprove
      $$3 \equiv -4 \Mod 7$$
    \item Use the definition, prove or disprove: for all integers x,
      $$2x \equiv -14x \Mod 8$$
    \item Prove or disprove the following statement: 
      Suppose $a, b, c, d$ are positive integers, $ac \equiv bc \Mod d$, then
      $$a \equiv b \Mod d$$
    \item Prove or disprove the following statement: 
      Suppose $a, b, c, d$ are positive integers, $ac \equiv bc \Mod d$ and $\gcd(c, d) = 1$, then
      $$a \equiv b \Mod d$$
      (Hint: you may use a fact we mentioned in Lecture 12.)
  \end{enumerate}
}
\sol (a) \proof We'll use the definition,
\begin{gather*}
  \begin{aligned}
    3\equiv -4\Mod 7 
      &\iff 7\divs(3-(-4)) \\
      &\iff 7\divs7 \equiv \true \\
  \end{aligned}
  \intertext{Therefore, by biconditional logical, $3\equiv -4\Mod 7$ is also true. $\QED$}
\end{gather*}

\sol (b) \proof Using the definition, suppose $x\in\bbz$, 
\begin{gather*}
  \begin{aligned}
    2x \equiv -14x \Mod 8
      &\iff 8 \divs (2x - (-14x)) \\
      &\iff 8 \divs 16x \\
      &\iff 8 \divs 8(2x) \equiv\true.
  \end{aligned} \\
  \intertext{Therefore, by biconditional logical, $2x \equiv -14x \Mod 8, \forall x\in\bbz \QED$}
\end{gather*}

\sol (c) I will disprove this by giving a counter example.
\proof Suppose $a=1,b=2,c=3,d=3$,
\begin{align*}
  ac \equiv bc \Mod d &\iff 3 \equiv 6 \Mod 3 \\
    &\iff 3 \divs (3 - 6) \\
    &\iff 3 \divs (-3) \\
    &\iff 3 \divs 3(-1) \equiv \true.
  \longintertext{Therefore, by biconditional logic, $ac \equiv bc \Mod d$. \\ We expect $a,b,c,d\in\bbn, ac\equiv bc\Mod d$ to imply that $a\equiv b\Mod d$, however,}
  a\equiv b\Mod d &\iff 1 \equiv 2 \Mod 3 \\
    &\iff 3 \divs (1-2) \\
    &\iff 3 \divs (-1) \\
    &\iff 3 \divs -1 \equiv \false \tag*{\contra}
  \longintertext{This counter example gives rise to a contradiction, therefore the statement that, given $a,b,c,d\in\bbn, ac\equiv bc\Mod d$ then $a \equiv b \Mod d$ is false. $\QED$}
\end{align*}

\sol (d) \proof Suppose $a,b,c,d\in\bbn$, $ac\equiv bc \Mod d$ and $\gcd(c,d) = 1 \implies c$ and $d$ are co-prime, sharing no common factors.
\begin{gather*}
  \begin{aligned}
    ac\equiv bc \Mod d 
      &\iff d \divs (ac - bc) \\
      &\iff d \divs c(a - b) \\    
  \end{aligned}
  \longintertext{Since $d$ and $c$ are co-prime, $d\ndivs c$. \\ Therefore, $d \divs (a-b)$} 
  a \equiv b \Mod d \iff d \divs (a - b) \equiv \true.
  \longintertext{Therefore, by biconditional logic, $a\equiv b\Mod d$. \\ Thus, we can conclude that, given $a,b,c,d\in\bbn, ac\equiv bc\Mod d, \gcd(c,d) = 1$ then $a\equiv b \Mod d. \QED$}
\end{gather*}

\newpage
\qs{(\it5 marks)}{
  Use the Euclidean algorithm to compute
  $$
    \gcd(101, 24)
  $$
}
\sol
\begin{gather*}
  \begin{array}{rlcrcr}
      & \gcd(101,24)  & \implies & 101  & = & 24\cd4 + 5  \\
    = & \gcd(24, 5)   & \implies & 24   & = & 5\cd4 + 4   \\
    = & \gcd(5, 4)    & \implies & 5    & = & 4\cd1 + 1   \\
    = & \gcd(4, 1)    & \implies & 4    & = & 1\cd4 + 0   \\
    = & \multicolumn{5}{l}{\gcd(1, 0) = 1}                \\
      & \multicolumn{5}{c}{\tf \gcd(101,24) = 1}
  \end{array}
\end{gather*}

\newpage
\qs{(\it10 marks)}{
  The least common multiple of the integers a, b, denoted as lcm(a, b), is defined as the smallest positive integer which is divisible by both a and b.\\
  Let $a = 2^7\cd3^2\cd5^1$ and $b = 2^3\cd3^3\cd7^1$.
  \begin{enumerate}[label=(\alph*)]
    \item Compute $\gcd(a,b)$.
    \item Compute $\lcm(a, b)$.
    \item Verify that $\gcd(a, b)\cd\lcm(a, b) = ab$.
    \item Can you prove the statement $\gcd(a, b)\cd\lcm(a, b) = ab$ for arbitrary positive integers $a$ and $b$? (Hint: use the prime factorisation.)
  \end{enumerate}
}
\sol (a) \\
The greatest common divisor of $a$ and $b$ is the largest $n\in\bbn$ such that $n\divs a$ and $n\divs b$. In other words, there exists $k,l\in\bbz$ such that
$$
  a = kn,\qquad b = ln
$$
Rearranging we can see that $n = a/k = b/l$. If we apply the Fundamental Theorem of Arthimentic, $a = 2^{x_1}\cd3^{x_2}\cd5^{x_3}\dots,\ b = 2^{y_1}\cd3^{y_2}\cd5^{y_3}\dots$, and consider that $k$ and $l$ must cancel these factors of $a$ and $b$, such that the results of those divisions is equal, we can see that 
$$
  n = 2^{\min\braces{x_1,y_1}}\cd3^{\min\braces{x_2,y_2}}\cd5^{\min\braces{x_3,y_3}}\cd\dots.
$$
We've seen this simply by considering the definition of $\gcd(a,b)$. In this specific example,
\begin{gather*}
  \gcd\bracks{2^7\cd3^2\cd5^1, b = 2^3\cd3^3\cd7^1} = n \\
  n = \frac{2^7\cd3^2\cd5^1}{2^{k_1}\cd3^{k_2}\cd5^{k_3}\cd7^{k_4}} = \frac{2^3\cd3^3\cd7^1}{2^{l_1}\cd3^{l_2}\cd5^{l_3}\cd7^{l_4}} \\
  \begin{array}{ll}
    k_1 = 7 - 3 & l_1 = 0 \\
    k_2 = 0     & l_2 = 3 - 2 \\
    k_3 = 1     & l_3 = 0 \\
    k_4 = 0     & l_4 = 1 \\
  \end{array} \\
  n = \frac{2^7\cd3^2\cd5^1}{2^{4}\cd3^{0}\cd5^{1}\cd7^{0}} = \frac{2^3\cd3^3\cd7^1}{2^{0}\cd3^{1}\cd5^{0}\cd7^{1}} \\
  \tf n = 2^3\cd3^2\cd5^0 = 2^3\cd3^2\cd7^0 \\
  \intertext{Therefore, given the prime factorisation of $a$ and $b$, $\gcd\bracks{a,b}=2^3\cd3^2$.}
\end{gather*}

\sol (b) \\
The least common multiple of $a$ and $b$ is, effectively, the smallest number we can construct using all the prime factors of $a$ and $b$. In other words, $\lcm\bracks{a,b}=n$, where $n$ is the smallest natural number such that $n\divs a$ and $n\divs b$. From this we can can conclude that there exist $k,l\in\bbz$ such that
$$
  n = ka,\qquad n=lb,
$$
and $n$ is as minimised. We can see that $ka = lb$. If we apply the Fundamental Theorem of Arthimentic, $a = 2^{x_1}\cd3^{x_2}\cd5^{x_3}\dots,\ b = 2^{y_1}\cd3^{y_2}\cd5^{y_3}\dots$, and consider that $k$ and $l$ must eqaulise the equation, we can see that
$$
  n = 2^{\max\braces{x_1,y_1}}\cd3^{\max\braces{x_2,y_2}}\cd5^{\max\braces{x_3,y_3}}\cd\dots
$$
In this specific example,
\begin{gather*}
  \lcm\bracks{2^7\cd3^2\cd5^1, b = 2^3\cd3^3\cd7^1} = n \\
  n = 2^7\cd3^2\cd5^1 \bracks{2^{k_1}\cd3^{k_2}\cd5^{k_3}\cd7^{k_4}} = 2^3\cd3^3\cd7^1\bracks{2^{l_1}\cd3^{l_2}\cd5^{l_3}\cd7^{l_4}} \\
  \begin{array}{ll}
    k_1 = 0     & l_1 = 7-3 \\
    k_2 = 3-2   & l_2 = 0 \\
    k_3 = 0     & l_3 = 1-0 \\
    k_4 = 1-0   & l_4 = 0 \\
  \end{array} \\
  n = 2^7\cd3^2\cd5^1 \bracks{2^{0}\cd3^{1}\cd5^{0}\cd7^{1}} = 2^3\cd3^3\cd7^1\bracks{2^{4}\cd3^{0}\cd5^{1}\cd7^{0}} \\
  \tf n = 2^7\cd3^3\cd5^1\cd7^1 = 2^7\cd3^3\cd5^1\cd7^1 
  \intertext{Therefore, given the prime factorisation of $a$ and $b$, $\lcm\bracks{a,b} = 2^7\cd3^3\cd5^1\cd7^1$.}
\end{gather*}

\sol (c) \\
We've established that
\begin{gather*}
  a = 2^7\cd3^2\cd5^1,\qquad b = 2^3\cd3^3\cd7^1 \\
  \gcd\bracks{a, b} = 2^3\cd3^2 \\
  \lcm\bracks{a, b} = 2^7\cd3^3\cd5^1\cd7^1 \\
  \longintertext{We seek to show that $\gcd\bracks{a,b}\cd\lcm\bracks{a,b}=a\cd b$.\\ Let's compute the LHS}
  \begin{aligned}
    \gcd\bracks{a,b}\cd\lcm\bracks{a,b}
    &=
    2^3\cd3^2 \cd 2^7\cd3^3\cd5^1\cd7^1 \\
    &=
    2^{3+7}\cd3^{2+3}\cd5^{0+1}\cd7^{0+1} \\
    &=
    2^{10}\cd3^{5}\cd5^{1}\cd7^{1} 
  \end{aligned}
  \intertext{Now let's compute the RHS.}
  \begin{aligned}
    2^7\cd3^2\cd5^1 \cd 2^3\cd3^3\cd7^1 
    &= 
    2^{7+3}\cd3^{2+3}\cd5^{1+0}\cd7^{0+1} \\
    &=
    2^{10}\cd3^{5}\cd5^{1}\cd7^{1}
  \end{aligned}
  \intertext{Therefore, LHS = RHS. Therefore, for $a=2^7\cd3^2\cd5^1$ and $b=2^3\cd3^3\cd7^1$, $\gcd\bracks{a,b}\cd\lcm\bracks{a,b} = a\cd b$.}
\end{gather*}

\sol (d) \proof As previously discussed, we can apply the Fundamental Theorem of Arthimentic to two natural numbers $a,b\in\bbn, a=p_{1}^{x_1}\cd p_{2}^{x_2}\cd p_{3}^{x_3}\dots,\ b=p_{1}^{y_1}\cd p_{2}^{y_2}\cd p_{3}^{y_3}\dots$, where $p_i$ is the $i$-th prime number. The greatest common divisor of $a$ and $b$ is
$$
  \gcd\bracks{a,b} 
  =
    p_{1}^{\min\braces{x_1,y_1}}\cd p_{2}^{\min\braces{x_2,y_2}}\cd p_{3}^{\min\braces{x_3,y_3}}\dots
  =
    \prod_{i=1}^{\infty} p_{i}^{\min\braces{x_i,y_i}}
$$
The least common multiple of $a$ and $b$ is
$$
  \lcm\bracks{a,b} 
  =
    p_{1}^{\max\braces{x_1,y_1}}\cd p_{2}^{\max\braces{x_2,y_2}}\cd p_{3}^{\max\braces{x_3,y_3}}\dots
  =
    \prod_{i=1}^{\infty} p_{i}^{\max\braces{x_i,y_i}}
$$
We seek to prove that $\gcd(a,b)\cd\lcm(a,b)=ab$.\\
Let's compute the LHS, $\gcd\bracks{a,b}\cd\lcm\bracks{a,b}$,
$$
  \prod_{i=1}^{\infty} p_{i}^{\min\braces{x_i,y_i}} \cd \prod_{i=1}^{\infty} p_{i}^{\max\braces{x_i,y_i}}
  =
    \prod_{i=1}^{\infty} \bracks{p_{i}^{\min\braces{x_i,y_i}} \cd p_{i}^{\max\braces{x_i,y_i}}}
  = 
    \prod_{i=1}^{\infty} p_{i}^{\min\braces{x_i,y_i} + \max\braces{x_i,y_i}}
$$
\begin{gather*}
  \longintertext{Let's consider $\min\braces{x_i,y_i} + \max\braces{x_i,y_i}$ \\
  $\forall i$, there are 3 cases: $x_i < y_i,\ x_i = y_i$ and $x_i > y_i$. \\
  Case 1, $x_i < y_i$:}
  \min\braces{x_i, y_i} + \max\braces{x_i, y_i}  = x_i + y_i \\
  \intertext{Case 2, $x_i > y_i$:}
  \min\braces{x_i, y_i} + \max\braces{x_i, y_i}  = y_i + x_i = x_i + y_i \\
  \intertext{Case 3, $x_i = y_i$:}
  \min\braces{x_i, y_i} + \max\braces{x_i, y_i}  = y_i + x_i = x_i + x_i = y_i + y_i = x_i + y_i \\
  \intertext{All three cases are equal, so we can conclude that $\forall i, \min\braces{x_i,y_i} + \max\braces{x_i, y_i} = x_i + y_i$.}
  \tf LHS = \gcd\bracks{a,b}\cd\lcm\bracks{a,b} = \prod_{i=1}^{\infty} p_{i}^{x_i + y_i}
  \intertext{Let's now compute the RHS, $ab$,}
  p_{1}^{x_1}\cd p_{2}^{x_2}\cd p_{3}^{x_3}\dots \cd p_{1}^{y_1}\cd p_{2}^{y_2}\cd p_{3}^{y_3}\dots
  =
    \prod_{i=1}^\infty p_i^{x_i} \cd \prod_{i=1}^\infty p_i^{y_i}
  =
    \prod_{i=1}^\infty \bracks{p_i^{x_i} \cd p_i^{y_i}}
  =
    \prod_{i=1}^\infty p_i^{x_i + y_i}
  \intertext{Therefore, LHS = RHS}
  \intertext{Given two arbitrary integers, $a$ and $b$, we can apply the Fundamental Theorem of Arthimentic to them, and use their prime factorisations and the definitions of $\gcd(a,b)$ and $\lcm(a,b)$ to conclude that $\gcd(a,b)\cd\lcm(a,b) = ab \QED$}
\end{gather*}

\newpage
\qs{(\it10 marks)}{
  A sequence is defined recursively as:
  \begin{gather*}
    a_0 = 1,\qquad a_1 = 2, \\
    a_n = 4a_{n-1} - 3a_{n-2},\qquad n\geq 2.
  \end{gather*}
  Prove the formula
  $$
    a_n = \frac{3^n + 1}{2}
  $$
}
\sol
\proof We will use the principle of strong mathematical induction to prove the formula. Let $P(n)$ be the predicate ``$a_n = \bracks{3^n + 1} / 2$.'' Let's consider the first 2 terms of the sequence
$$
  \begin{array}{|c|r|r|}
    n                   & 0                   & 1                     \\ \hline
    a_n                 & 1                   & 2                     \\
    \dfrac{3^n + 1}{2}  & \dfrac{3^0 + 1}{2}=1& \dfrac{3^1 + 1}{2}=2  \\
  \end{array}
$$
Basis Step: P(0) and P(1) are True. We've proved this in the above table. \\
Inductive Hypothesis: Suppose that, for some integer $k: 1 \leq k \leq n$, $P(0),\ P(1),\ P(2),\ \dots\ ,\ P(k)$ are true. \\
Inductive Step: We seek to prove $P(n+1)$.
\begin{gather*}
  a_{n+1} = 4a_{n} - 3a_{n-1} 
  \intertext{From the inductive step, $P(n)$ and $P(n-1)$ are true.}
  \begin{aligned}
    \tf a_{n+1} &= 4\frac{3^n + 1}{2} - 3\frac{3^{n-1} + 1}{2} \\
      &= 4\frac{3^n + 1}{2} - \frac{3^{n} + 1}{2} \\
      &= \frac{3^n + 1}{2}\bracks{4 - 1} \\
      &= \frac{3^n + 1}{2}\bracks{3} \\
      &= \frac{3^{n+1} + 1}{2} \\
  \end{aligned}
  \longintertext{$P(n+1)$ is true \\ Therefore, by the principle of strong mathematical induction, it follows that $P(n)$ is true for all integers, $n\geq 0$. \QED}
\end{gather*}

\end{document}
