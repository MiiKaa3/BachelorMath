\documentclass[a4paper]{gtart}

\usepackage[inner=34mm,outer=34mm,textheight=235mm]{geometry}

\usepackage{fancyheadings, amsmath, amssymb, latexsym, enumerate}

\hoffset=-1.2cm
%\voffset=.4cm
\textwidth=16.4cm
\textheight=23.5cm

\pagestyle{fancy}
\lhead[]{{\sc MATH1061\,/\,7861}}
\chead[]{{\sc Assignment 3}}
\rhead[]{{\sc Semester 2/2024}}
\lfoot[]{}
\cfoot[]{}
\rfoot[]{}

\newcommand{\Z}{\mathbb{Z}}
\newcommand{\R}{\mathbb{R}}
\newcommand{\true}{\mathrm{T}}
\newcommand{\false}{\mathrm{F}}
\newcommand{\nto}{\not \to}

\renewcommand\refname{}%{\normalsize{\bf{PUBLICATIONS}}}
\pagenumbering{arabic}

\begin{document}

Due \textbf{Friday 4 October at 5pm} on blackboard.

\medskip%

Marks will be deducted for sloppy working. Clearly state your assumptions and conclusions, and justify all steps in your work. \textbf{You may use any fact we have proved in class, but please be sure to specify the lecture number and problem number, to indicate which result(s) you have cited.}

\begin{enumerate}

\item[Q1]

Prove the following set identities

(1)
\begin{align*}
(A \cup B ) \times C = (A\times C) \cup (B\times C)
\end{align*} 
(2)
\begin{align*}
(A^{\operatorname{c}} \cap B)^{\operatorname{c}} \cap D = (D-A^{\operatorname{c}}) \cup (D-B)
\end{align*}

\hfill  \emph{(10 marks)}

\medskip 

\item[Q2]
 Suppose $f:A\to B$ and $g:B\to A$ are functions, and $\operatorname{id}_A$ is the identity function on $A$, $\operatorname{id}_B$ is the identity function on $B$. In particular,  $\operatorname{id}_A(x)=x$ for any $x\in A$ and similarly $\operatorname{id}_B(x)=x$ for any $x\in B$.

(1) Suppose $f\circ g= \operatorname{id}_B$ and  $g\circ f= \operatorname{id}_A$. Prove that both $f$ and $g$ are bijections.


(2) Suppose $g$ is onto and  $f\circ g= \operatorname{id}_B$. Prove that $g\circ f= \operatorname{id}_A$.

\hfill \emph{(15 marks)}

\medskip 

\item[Q3]
(1) Show that $\mathbb{Z}^+\times \mathbb{Z}^+$ is in bijection with $\mathbb{Z}^+\times \mathbb{Z}^+\times \mathbb{Z}^{+}$. Deduce that $\mathbb{Z}^+\times \mathbb{Z}^+\times \mathbb{Z}^{+}$ is countable. 


(2) Show that $\mathbb{Z}^+\times \mathbb{Z}^+\times \mathbb{Z}^+$ is in bijection with $\mathbb{Z}^+\times \mathbb{Z}^+\times \mathbb{Z}^{+}\times \mathbb{Z}^+$. Deduce that $\mathbb{Z}^+\times \mathbb{Z}^+\times \mathbb{Z}^{+}\times \mathbb{Z}^+$ is countable. 


(3) Is the set $\mathbb{Z}^
+\times\dots\times \mathbb{Z}^+$ (Cartesian product $n$ times) countable? A yes or no would suffice.

\hfill  (15 marks)

\medskip

\item[Q4] (MATH 1061 ONLY)
 
Let $A$ be the set of all logical statements. Define a relation on $A$: for $p,q\in A$, $p$ is related to $q$ if and only if $p\wedge q$ and $p\vee q$ have the same truth value. 

Determine if the above relation is (1) reflexive, (2) symmetric, (3) transitive. If your answer is yes for any of the three properties, please prove your answer;  if your answer is no, please find a counterexample.
 




\hfill  (10 marks)

\item[Q4] (MATH 7861 ONLY)


Define a relation on $\mathcal{P}(\mathbb{Z}^+)$: for $A,B\subseteq \mathbb{Z}^+$, $A$ is related to $B$ if and only if there exists $C\subseteq \mathbb{Z}^{+}$ such that $\{A,B,C\}$ forms a partition of $\mathbb{Z}^+$.

Determine if the above relation is (1) reflexive, (2) symmetric, (3) transitive. If your answer is yes for any of the three properties, please prove your answer;  if your answer is no, please find a counterexample.
 
\hfill  (10 marks)


\end{enumerate}

\end{document}
