\documentclass[a4paper, 11pt]{report}

\input{../../../latex_template/preamble}
%From M275 "Topology" at SJSU
\newcommand{\id}{\mathrm{id}}
\newcommand{\taking}[1]{\xrightarrow{#1}}
\newcommand{\inv}{^{-1}}

%From M170 "Introduction to Graph Theory" at SJSU
\DeclareMathOperator{\diam}{diam}
\DeclareMathOperator{\ord}{ord}
\newcommand{\defeq}{\overset{\mathrm{def}}{=}}

%From the USAMO .tex files
\newcommand{\ts}{\textsuperscript}
\newcommand{\dg}{^\circ}
\newcommand{\ii}{\item}

% % From Math 55 and Math 145 at Harvard
% \newenvironment{subproof}[1][Proof]{%
% \begin{proof}[#1] \renewcommand{\qedsymbol}{$\blacksquare$}}%
% {\end{proof}}

\newcommand{\liff}{\leftrightarrow}
\newcommand{\lthen}{\rightarrow}
\newcommand{\opname}{\operatorname}
\newcommand{\surjto}{\twoheadrightarrow}
\newcommand{\injto}{\hookrightarrow}
\newcommand{\On}{\mathrm{On}} % ordinals
\DeclareMathOperator{\img}{im} % Image
\DeclareMathOperator{\Img}{Im} % Image
\DeclareMathOperator{\coker}{coker} % Cokernel
\DeclareMathOperator{\Coker}{Coker} % Cokernel
\DeclareMathOperator{\Ker}{Ker} % Kernel
\DeclareMathOperator{\rank}{rank}
\DeclareMathOperator{\Spec}{Spec} % spectrum
\DeclareMathOperator{\Tr}{Tr} % trace
\DeclareMathOperator{\pr}{pr} % projection
\DeclareMathOperator{\ext}{ext} % extension
\DeclareMathOperator{\pred}{pred} % predecessor
\DeclareMathOperator{\dom}{dom} % domain
\DeclareMathOperator{\ran}{ran} % range
\DeclareMathOperator{\Hom}{Hom} % homomorphism
\DeclareMathOperator{\Mor}{Mor} % morphisms
\DeclareMathOperator{\End}{End} % endomorphism

\newcommand{\eps}{\epsilon}
\newcommand{\veps}{\varepsilon}
\newcommand{\ol}{\overline}
\newcommand{\ul}{\underline}
\newcommand{\wt}{\widetilde}
\newcommand{\wh}{\widehat}
\newcommand{\vocab}[1]{\textbf{\color{blue} #1}}
\providecommand{\half}{\frac{1}{2}}
\newcommand{\dang}{\measuredangle} %% Directed angle
\newcommand{\ray}[1]{\overrightarrow{#1}}
\newcommand{\seg}[1]{\overline{#1}}
\newcommand{\arc}[1]{\wideparen{#1}}
\DeclareMathOperator{\cis}{cis}
\DeclareMathOperator*{\lcm}{lcm}
\DeclareMathOperator*{\argmin}{arg min}
\DeclareMathOperator*{\argmax}{arg max}
\newcommand{\cycsum}{\sum_{\mathrm{cyc}}}
\newcommand{\symsum}{\sum_{\mathrm{sym}}}
\newcommand{\cycprod}{\prod_{\mathrm{cyc}}}
\newcommand{\symprod}{\prod_{\mathrm{sym}}}
\newcommand{\Qed}{\begin{flushright}\qed\end{flushright}}
\newcommand{\parinn}{\setlength{\parindent}{1cm}}
\newcommand{\parinf}{\setlength{\parindent}{0cm}}
% \newcommand{\norm}{\|\cdot\|}
\newcommand{\inorm}{\norm_{\infty}}
\newcommand{\opensets}{\{V_{\alpha}\}_{\alpha\in I}}
\newcommand{\oset}{V_{\alpha}}
\newcommand{\opset}[1]{V_{\alpha_{#1}}}
\newcommand{\lub}{\text{lub}}
\newcommand{\del}[2]{\frac{\partial #1}{\partial #2}}
\newcommand{\Del}[3]{\frac{\partial^{#1} #2}{\partial^{#1} #3}}
\newcommand{\deld}[2]{\dfrac{\partial #1}{\partial #2}}
\newcommand{\Deld}[3]{\dfrac{\partial^{#1} #2}{\partial^{#1} #3}}
\newcommand{\lm}{\lambda}
\newcommand{\uin}{\mathbin{\rotatebox[origin=c]{90}{$\in$}}}
\newcommand{\usubset}{\mathbin{\rotatebox[origin=c]{90}{$\subset$}}}
\newcommand{\lt}{\left}
\newcommand{\rt}{\right}
\newcommand{\bs}[1]{\boldsymbol{#1}}
\newcommand{\exs}{\exists}
\newcommand{\st}{\strut}
\newcommand{\dps}[1]{\displaystyle{#1}}

\newcommand{\sol}{\setlength{\parindent}{0cm}\textbf{\textit{Solution:}}\setlength{\parindent}{1cm} }
\newcommand{\solve}[1]{\setlength{\parindent}{0cm}\textbf{\textit{Solution: }}\setlength{\parindent}{1cm}#1 \Qed}

\DeclareMathOperator{\sech}{sech}
\DeclareMathOperator{\csch}{csch}
\DeclareMathOperator{\arcsec}{arcsec}
\DeclareMathOperator{\arccsc}{arccsc}
\DeclareMathOperator{\arccot}{arccot}
\DeclareMathOperator{\arsinh}{arsinh}
\DeclareMathOperator{\arcosh}{arcosh}
\DeclareMathOperator{\artanh}{artanh}
\DeclareMathOperator{\arcsch}{arcsch}
\DeclareMathOperator{\arsech}{arsech}
\DeclareMathOperator{\arcoth}{arcoth}

\newcommand{\sinx}{\sin x}          \newcommand{\arcsinx}{\arcsin x}    
\newcommand{\cosx}{\cos x}          \newcommand{\arccosx}{\arccosx}
\newcommand{\tanx}{\tan x}          \newcommand{\arctanx}{\arctan x}
\newcommand{\cscx}{\csc x}          \newcommand{\arccscx}{\arccsc x}
\newcommand{\secx}{\sec x}          \newcommand{\arcsecx}{\arcsec x}
\newcommand{\cotx}{\cot x}          \newcommand{\arccotx}{\arccot x}
\newcommand{\sinhx}{\sinh x}          \newcommand{\arsinhx}{\arsinh x}
\newcommand{\coshx}{\cosh x}          \newcommand{\arcoshx}{\arcosh x}
\newcommand{\tanhx}{\tanh x}          \newcommand{\artanhx}{\artanh x}
\newcommand{\cschx}{\csch x}          \newcommand{\arcschx}{\arcsch x}
\newcommand{\sechx}{\sech x}          \newcommand{\arsechx}{\arsech x}
\newcommand{\cothx}{\coth x}          \newcommand{\arcothx}{\arcoth x}
\newcommand{\lnx}{\ln x}
\newcommand{\expx}{\exp x}

\newcommand{\bba}{\mathbb{A}}   \newcommand{\bbn}{\mathbb{N}}
\newcommand{\bbb}{\mathbb{B}}   \newcommand{\bbo}{\mathbb{O}}
\newcommand{\bbc}{\mathbb{C}}   \newcommand{\bbp}{\mathbb{P}}
\newcommand{\bbd}{\mathbb{D}}   \newcommand{\bbq}{\mathbb{Q}}
\newcommand{\bbe}{\mathbb{E}}   \newcommand{\bbr}{\mathbb{R}}
\newcommand{\bbf}{\mathbb{F}}   \newcommand{\bbs}{\mathbb{S}}
\newcommand{\bbg}{\mathbb{G}}   \newcommand{\bbt}{\mathbb{T}}
\newcommand{\bbh}{\mathbb{H}}   \newcommand{\bbu}{\mathbb{U}}
\newcommand{\bbi}{\mathbb{I}}    \newcommand{\bbv}{\mathbb{V}}
\newcommand{\bbj}{\mathbb{J}}   \newcommand{\bbw}{\mathbb{W}}
\newcommand{\bbk}{\mathbb{K}}   \newcommand{\bbx}{\mathbb{X}}
\newcommand{\bbl}{\mathbb{L}}    \newcommand{\bby}{\mathbb{Y}}
\newcommand{\bbm}{\mathbb{M}}   \newcommand{\bbz}{\mathbb{Z}}

\newcommand{\lb}{\left(}
\newcommand{\rb}{\right)}
\newcommand{\lbr}{\left\lbrace}
\newcommand{\rbr}{\right\rbrace}
\newcommand{\lsb}{\left[}
\newcommand{\rsb}{\right]}
\newcommand{\suchthat}{\medspace\middle|\medspace}
\newcommand{\bracks}[1]{\lb #1 \rb}
\newcommand{\braces}[1]{\lbr #1 \rbr}
\newcommand{\sqbracks}[1]{\lsb #1 \rsb}

\renewcommand{\floor}[1]{\lfloor #1 \rfloor}
\renewcommand{\ceil}[1]{\lceil #1 \rceil}

\newcommand{\cd}{\cdot}
\newcommand{\tf}{\therefore}
\newcommand{\Let}{\text{Let }}
\newcommand{\Given}{\text{Given }}
\newcommand{\Suppose}{\text{Suppose }}
\newcommand{\WeSee}{\text{We see }}
\newcommand{\So}{\text{So }}

\newcommand{\QED}{\hfill \qed}

\renewcommand{\dd}[1]{\frac{d}{d#1}}
\newcommand{\dyd}[2][y]{\frac{d#1}{d#2}}

\newcommand{\ddx}{\dd{x}}       \newcommand{\ddxsq}{\dyd[^2]{x^2}}
\newcommand{\ddy}{\dd{y}}       \newcommand{\ddysq}{\dyd[^2]{y^2}}
\newcommand{\ddu}{\dd{u}}       \newcommand{\ddusq}{\dyd[^2]{u^2}}
\newcommand{\ddv}{\dd{v}}       \newcommand{\ddvsq}{\dyd[^2]{v^2}}

\newcommand{\dydx}{\dyd{x}}     \newcommand{\dydxsq}{\dyd[^2y]{x^2}}
\newcommand{\dfdx}{\dyd[f]{x}}  \newcommand{\dfdxsq}{\dyd[^2f]{x^2}}
\newcommand{\dudx}{\dyd[u]{x}}  \newcommand{\dudxsq}{\dyd[^2u]{x^2}}
\newcommand{\dvdx}{\dyd[v]{x}}  \newcommand{\dvdxsq}{\dyd[^2v]{x^2}}

% Mathfrak primes
\newcommand{\km}{\mathfrak{m}}
\newcommand{\kp}{\mathfrak{p}}
\newcommand{\kq}{\mathfrak{q}}

%---------------------------------------
% Blackboard Math Fonts :-
%---------------------------------------
\newcommand{\bba}{\mathbb{A}}   \newcommand{\bbn}{\mathbb{N}}
\newcommand{\bbb}{\mathbb{B}}   \newcommand{\bbo}{\mathbb{O}}
\newcommand{\bbc}{\mathbb{C}}   \newcommand{\bbp}{\mathbb{P}}
\newcommand{\bbd}{\mathbb{D}}   \newcommand{\bbq}{\mathbb{Q}}
\newcommand{\bbe}{\mathbb{E}}   \newcommand{\bbr}{\mathbb{R}}
\newcommand{\bbf}{\mathbb{F}}   \newcommand{\bbs}{\mathbb{S}}
\newcommand{\bbg}{\mathbb{G}}   \newcommand{\bbt}{\mathbb{T}}
\newcommand{\bbh}{\mathbb{H}}   \newcommand{\bbu}{\mathbb{U}}
\newcommand{\bbi}{\mathbb{I}}   \newcommand{\bbv}{\mathbb{V}}
\newcommand{\bbj}{\mathbb{J}}   \newcommand{\bbw}{\mathbb{W}}
\newcommand{\bbk}{\mathbb{K}}   \newcommand{\bbx}{\mathbb{X}}
\newcommand{\bbl}{\mathbb{L}}   \newcommand{\bby}{\mathbb{Y}}
\newcommand{\bbm}{\mathbb{M}}   \newcommand{\bbz}{\mathbb{Z}}

%---------------------------------------
% Roman Math Fonts :-
%---------------------------------------
\newcommand{\rma}{\mathrm{A}}   \newcommand{\rmn}{\mathrm{N}}
\newcommand{\rmb}{\mathrm{B}}   \newcommand{\rmo}{\mathrm{O}}
\newcommand{\rmc}{\mathrm{C}}   \newcommand{\rmp}{\mathrm{P}}
\newcommand{\rmd}{\mathrm{D}}   \newcommand{\rmq}{\mathrm{Q}}
\newcommand{\rme}{\mathrm{E}}   \newcommand{\rmr}{\mathrm{R}}
\newcommand{\rmf}{\mathrm{F}}   \newcommand{\rms}{\mathrm{S}}
\newcommand{\rmg}{\mathrm{G}}   \newcommand{\rmt}{\mathrm{T}}
\newcommand{\rmh}{\mathrm{H}}   \newcommand{\rmu}{\mathrm{U}}
\newcommand{\rmi}{\mathrm{I}}   \newcommand{\rmv}{\mathrm{V}}
\newcommand{\rmj}{\mathrm{J}}   \newcommand{\rmw}{\mathrm{W}}
\newcommand{\rmk}{\mathrm{K}}   \newcommand{\rmx}{\mathrm{X}}
\newcommand{\rml}{\mathrm{L}}   \newcommand{\rmy}{\mathrm{Y}}
\newcommand{\rmm}{\mathrm{M}}   \newcommand{\rmz}{\mathrm{Z}}

%---------------------------------------
% Calligraphic Math Fonts :-
%---------------------------------------
\newcommand{\cla}{\mathcal{A}}   \newcommand{\cln}{\mathcal{N}}
\newcommand{\clb}{\mathcal{B}}   \newcommand{\clo}{\mathcal{O}}
\newcommand{\clc}{\mathcal{C}}   \newcommand{\clp}{\mathcal{P}}
\newcommand{\cld}{\mathcal{D}}   \newcommand{\clq}{\mathcal{Q}}
\newcommand{\cle}{\mathcal{E}}   \newcommand{\clr}{\mathcal{R}}
\newcommand{\clf}{\mathcal{F}}   \newcommand{\cls}{\mathcal{S}}
\newcommand{\clg}{\mathcal{G}}   \newcommand{\clt}{\mathcal{T}}
\newcommand{\clh}{\mathcal{H}}   \newcommand{\clu}{\mathcal{U}}
\newcommand{\cli}{\mathcal{I}}   \newcommand{\clv}{\mathcal{V}}
\newcommand{\clj}{\mathcal{J}}   \newcommand{\clw}{\mathcal{W}}
\newcommand{\clk}{\mathcal{K}}   \newcommand{\clx}{\mathcal{X}}
\newcommand{\cll}{\mathcal{L}}   \newcommand{\cly}{\mathcal{Y}}
\newcommand{\calm}{\mathcal{M}}  \newcommand{\clz}{\mathcal{Z}}

%---------------------------------------
% Fraktur  Math Fonts :-
%---------------------------------------
\newcommand{\fka}{\mathfrak{A}}   \newcommand{\fkn}{\mathfrak{N}}
\newcommand{\fkb}{\mathfrak{B}}   \newcommand{\fko}{\mathfrak{O}}
\newcommand{\fkc}{\mathfrak{C}}   \newcommand{\fkp}{\mathfrak{P}}
\newcommand{\fkd}{\mathfrak{D}}   \newcommand{\fkq}{\mathfrak{Q}}
\newcommand{\fke}{\mathfrak{E}}   \newcommand{\fkr}{\mathfrak{R}}
\newcommand{\fkf}{\mathfrak{F}}   \newcommand{\fks}{\mathfrak{S}}
\newcommand{\fkg}{\mathfrak{G}}   \newcommand{\fkt}{\mathfrak{T}}
\newcommand{\fkh}{\mathfrak{H}}   \newcommand{\fku}{\mathfrak{U}}
\newcommand{\fki}{\mathfrak{I}}   \newcommand{\fkv}{\mathfrak{V}}
\newcommand{\fkj}{\mathfrak{J}}   \newcommand{\fkw}{\mathfrak{W}}
\newcommand{\fkk}{\mathfrak{K}}   \newcommand{\fkx}{\mathfrak{X}}
\newcommand{\fkl}{\mathfrak{L}}   \newcommand{\fky}{\mathfrak{Y}}
\newcommand{\fkm}{\mathfrak{M}}   \newcommand{\fkz}{\mathfrak{Z}}


\title{\Huge{MATH1072}\\Advanced Multivariate Calculus and Ordinary Differential Equations}
\author{\huge{Problem Set 3}\\\huge{Michael Kasumagic, s4430266}}
\date{\huge{Due: 1pm, $30^\text{th}$ of September, 2024}}

\begin{document}
\begin{center}
{\bf SCHOOL OF MATHEMATICS AND PHYSICS}
\end{center}
\centerline{\large\bf MATH1072}
\vspace{.1cm}   
\centerline{\large\bf Assignment 3}
\vspace{.1cm}
\centerline{\large\bf Semester Two 2024}

\vspace{3mm}
%%%%%%%%%%%%%%%%%%%%%%%%%%%%%%%%%
\hrule
\vspace{3mm}

{\it Submit your answers - along with this sheet - by 1pm on the 30th of September, using the blackboard assignment submission system. Assignments must consist of a single PDF.
}

You may find some of these problems challenging. Attendance at weekly tutorials is assumed.

\vspace{1cm}

\begin{tabular}{rl}
	Family name: & $\clk$asumagic \\
	& \\
	& \\
	Given names: & $\calm$ichael $\cla$llan \\
	& \\
	& \\
	Student number: & 44302669 \\
	& \\
	& \\
\end{tabular}

\hrule ${}^{}$\\
Marker's use only
\\
\hrule ${}^{}$\\
Each question marked out of 3.
\begin{itemize}
\item Mark of 0: You have not submitted a relevant answer, or you have no strategy present in your submission.\\
\item Mark of 1: Your submission has some relevance, but does not demonstrate deep understanding or sound mathematical technique. \\ %This topic needs more attention!\\
\item Mark of 2: You have the right approach, but need to fine-tune some aspects of your calculations.\\
\item Mark of 3: You have demonstrated a good understanding of the topic and techniques involved, with well-executed calculations. \\ %A mathematician in the making?\\
\end{itemize}
\begin{tabular}{rrrrrrrr}
Q1a & \hspace{2cm} & Q1b: & \hspace{2cm} & Q1c: & \hspace{2cm} & Q1d:  & \\[.5cm]
Q2a:& \hspace{2cm} & Q2b: & \hspace{2cm} & Q2c: & \hspace{2cm} & Q3a:  & \\[.5cm]
Q3b:& \hspace{2cm} & Q3c: & \hspace{2cm} & Q3d: & \hspace{2cm} & Q4:  & \\[.5cm]
& \hspace{2cm} &  & \hspace{2cm} & & \hspace{2cm} &  & \\[.5cm]
&&&&&&& \\
Total (out of 36): &&&&&&& \\
\end{tabular}

\vfill
\newpage
\qs{Inital Value Problem}{
	A population of Christmas beetles in Brisbane grows at a rate proportional to their current population. In the absence of external factors, the population will double in one week's time. On any given day, there is a net migration into the area of 10 beetles, 11 beetles are eaten by a local Magpie population, and 2 die of natural causes.
	
	\begin{enumerate}[label=(\alph*)]
	\item Write a initial value problem to describe the change in population at time $t$, given that the initial population of Christmas beetles is $P(0) = 100$. You \textbf{must} write all the parameters in your model \textbf{explicitly}.
	\item Solve the initial value problem from part (a) to find the population $P(t)$, at any time $t$.
	\item Use MATLAB to plot your solution from part (b) from time $t = 0$ to $t = 100$. Will the beetle population survive?
	\item Use MATLAB to plot the solution to part (a) from time $t = 0$ to $t = 100$ for 30 different initial populations of Christmas beetles from $P(0) = 20$ to $P(0) = 50$. Recall that initial population sizes must be integer valued. What can you say about the stability of the population of Christmas beetles from this plot?
	\end{enumerate}
}

\newpage
\qs{Fluid Flow}{
	In this question, we take the cartesian variables as $x\equiv x_1,$ $y\equiv x_2,$ and $z\equiv x_3.$ We also take the basis vectors as $\ihat\equiv\ut{e}_1$, $\jhat\equiv\ut{e}_2,$ and $\khat\equiv\ut{e}_3$. \\
	
	In fluid mechanics, fluids which are \textbf{incompressible} and \textbf{inviscid} are referred to as \textbf{ideal}.
	
	Let $\ut{u}=\ut{u}(x,y,z,t)=u_1\ut{e}_1+u_2\ut{e}_2+u_3\ut{e}_3$ be the velocity of an ideal fluid at an arbitrary point in space and time. Its motion is governed by the Euler equation,
	$$
		\del{\ut{u}}{t} + (\ut{u}\cd\nabla)\ut{u} =\ut{F} - \frac{\nabla P}{\rho}.
	$$
	Here $P$ is the fluid's pressure, $\rho$ its constant density, $\ut{F}$ is the external force on the fluid at any given point, and
	$$
		(\ut{u}\cd\nabla)\ut{u} \equiv \sum_{i=1}^3 u_i\del{\ut{u}}{x_i}.
	$$
	Suppose the system is simplified in three ways:
	\begin{itemize}
		\item The flow is \textbf{steady}. $\ut{u}=\ut{u}(x,y,z)$ is not changing with time.
		\item $\ut{u}$ is a conservative vector field. This occurs when the fluid is \textbf{irrotational}, though we won't elaborate on that here.
		\item The external force is also conservative, with $\ut{f}=-\nabla V$, for some scalar potential $V = V(x_1,x_2,x_3)$.  
	\end{itemize}
	
	\begin{enumerate}[label=(\alph*)]
		\item Show that in this case, 
		$$
			(\ut{u}\cdot\nabla)\ut{u}=\frac{1}{2}\nabla ||\ut{u}||^2.
		$$ (Hint, you can assume that Clairaut's theorem applies)
		\item Hence show that the Euler equation simplifies to
		$$
			\frac{1}{2}||\ut{u}||^2+V+\frac{P}{\rho}=\text{constant}
		$$
		\item If $V$ is a constant, what can you say about the relationship between a fluid's speed and its pressure?
	\end{enumerate}
}

\newpage
\qs{First Order Differential Equation}{
	Consider Newton's second law of motion which states that,
	\[ 
		F = ma \tag{1}\label{eq:newtonsecond}
	\]
	or: net force, $F$, is equal to mass $m$, times acceleration $a$.
	\begin{enumerate}[label=(\alph*)]
		\item Use equation \eqref{eq:newtonsecond} to write out the equation of motion of a particle of mass $m$, subject to a frictional force proportional to the square of the velocity $v(t)$, completely in terms of the particle's velocity $v(t)$.
		\item Solve the first-order differential equation from part (a) to find the particle's velocity $v(t)$ at time $t$, with initial velocity $v_0$.
		\item Use your solution from part (b) to solve for the position of the particle $x(t)$ at time $t$, with initial position $x_0$.
		\item Use Euler's method with step size 0.1 to estimate the particle's position $x(0.5)$ at time $t = 0.5$ of your solution in part (b). Take $m = 1$, $\beta = 2$, $x_0 = 0$, and $v_0 = 1$. Calculate the error in using Euler's method, rounded to the fourth decimal.
	\end{enumerate}
}

\newpage
\qs{Line Integral}{
	Evaluate the line integral
	$$
		\int_C x e^y\ \d{s},
	$$
	where $C$ is the line segment from $(2,0)$ to $(5,4)$.
}

\end{document}
