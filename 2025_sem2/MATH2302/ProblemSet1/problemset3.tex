\documentclass[a4paper,11pt]{report}

\input{../../../latex_template/preamble}
%From M275 "Topology" at SJSU
\newcommand{\id}{\mathrm{id}}
\newcommand{\taking}[1]{\xrightarrow{#1}}
\newcommand{\inv}{^{-1}}

%From M170 "Introduction to Graph Theory" at SJSU
\DeclareMathOperator{\diam}{diam}
\DeclareMathOperator{\ord}{ord}
\newcommand{\defeq}{\overset{\mathrm{def}}{=}}

%From the USAMO .tex files
\newcommand{\ts}{\textsuperscript}
\newcommand{\dg}{^\circ}
\newcommand{\ii}{\item}

% % From Math 55 and Math 145 at Harvard
% \newenvironment{subproof}[1][Proof]{%
% \begin{proof}[#1] \renewcommand{\qedsymbol}{$\blacksquare$}}%
% {\end{proof}}

\newcommand{\liff}{\leftrightarrow}
\newcommand{\lthen}{\rightarrow}
\newcommand{\opname}{\operatorname}
\newcommand{\surjto}{\twoheadrightarrow}
\newcommand{\injto}{\hookrightarrow}
\newcommand{\On}{\mathrm{On}} % ordinals
\DeclareMathOperator{\img}{im} % Image
\DeclareMathOperator{\Img}{Im} % Image
\DeclareMathOperator{\coker}{coker} % Cokernel
\DeclareMathOperator{\Coker}{Coker} % Cokernel
\DeclareMathOperator{\Ker}{Ker} % Kernel
\DeclareMathOperator{\rank}{rank}
\DeclareMathOperator{\Spec}{Spec} % spectrum
\DeclareMathOperator{\Tr}{Tr} % trace
\DeclareMathOperator{\pr}{pr} % projection
\DeclareMathOperator{\ext}{ext} % extension
\DeclareMathOperator{\pred}{pred} % predecessor
\DeclareMathOperator{\dom}{dom} % domain
\DeclareMathOperator{\ran}{ran} % range
\DeclareMathOperator{\Hom}{Hom} % homomorphism
\DeclareMathOperator{\Mor}{Mor} % morphisms
\DeclareMathOperator{\End}{End} % endomorphism

\newcommand{\eps}{\epsilon}
\newcommand{\veps}{\varepsilon}
\newcommand{\ol}{\overline}
\newcommand{\ul}{\underline}
\newcommand{\wt}{\widetilde}
\newcommand{\wh}{\widehat}
\newcommand{\vocab}[1]{\textbf{\color{blue} #1}}
\providecommand{\half}{\frac{1}{2}}
\newcommand{\dang}{\measuredangle} %% Directed angle
\newcommand{\ray}[1]{\overrightarrow{#1}}
\newcommand{\seg}[1]{\overline{#1}}
\newcommand{\arc}[1]{\wideparen{#1}}
\DeclareMathOperator{\cis}{cis}
\DeclareMathOperator*{\lcm}{lcm}
\DeclareMathOperator*{\argmin}{arg min}
\DeclareMathOperator*{\argmax}{arg max}
\newcommand{\cycsum}{\sum_{\mathrm{cyc}}}
\newcommand{\symsum}{\sum_{\mathrm{sym}}}
\newcommand{\cycprod}{\prod_{\mathrm{cyc}}}
\newcommand{\symprod}{\prod_{\mathrm{sym}}}
\newcommand{\Qed}{\begin{flushright}\qed\end{flushright}}
\newcommand{\parinn}{\setlength{\parindent}{1cm}}
\newcommand{\parinf}{\setlength{\parindent}{0cm}}
% \newcommand{\norm}{\|\cdot\|}
\newcommand{\inorm}{\norm_{\infty}}
\newcommand{\opensets}{\{V_{\alpha}\}_{\alpha\in I}}
\newcommand{\oset}{V_{\alpha}}
\newcommand{\opset}[1]{V_{\alpha_{#1}}}
\newcommand{\lub}{\text{lub}}
\newcommand{\del}[2]{\frac{\partial #1}{\partial #2}}
\newcommand{\Del}[3]{\frac{\partial^{#1} #2}{\partial^{#1} #3}}
\newcommand{\deld}[2]{\dfrac{\partial #1}{\partial #2}}
\newcommand{\Deld}[3]{\dfrac{\partial^{#1} #2}{\partial^{#1} #3}}
\newcommand{\lm}{\lambda}
\newcommand{\uin}{\mathbin{\rotatebox[origin=c]{90}{$\in$}}}
\newcommand{\usubset}{\mathbin{\rotatebox[origin=c]{90}{$\subset$}}}
\newcommand{\lt}{\left}
\newcommand{\rt}{\right}
\newcommand{\bs}[1]{\boldsymbol{#1}}
\newcommand{\exs}{\exists}
\newcommand{\st}{\strut}
\newcommand{\dps}[1]{\displaystyle{#1}}

\newcommand{\sol}{\setlength{\parindent}{0cm}\textbf{\textit{Solution:}}\setlength{\parindent}{1cm} }
\newcommand{\solve}[1]{\setlength{\parindent}{0cm}\textbf{\textit{Solution: }}\setlength{\parindent}{1cm}#1 \Qed}

\DeclareMathOperator{\sech}{sech}
\DeclareMathOperator{\csch}{csch}
\DeclareMathOperator{\arcsec}{arcsec}
\DeclareMathOperator{\arccsc}{arccsc}
\DeclareMathOperator{\arccot}{arccot}
\DeclareMathOperator{\arsinh}{arsinh}
\DeclareMathOperator{\arcosh}{arcosh}
\DeclareMathOperator{\artanh}{artanh}
\DeclareMathOperator{\arcsch}{arcsch}
\DeclareMathOperator{\arsech}{arsech}
\DeclareMathOperator{\arcoth}{arcoth}

\newcommand{\sinx}{\sin x}          \newcommand{\arcsinx}{\arcsin x}    
\newcommand{\cosx}{\cos x}          \newcommand{\arccosx}{\arccosx}
\newcommand{\tanx}{\tan x}          \newcommand{\arctanx}{\arctan x}
\newcommand{\cscx}{\csc x}          \newcommand{\arccscx}{\arccsc x}
\newcommand{\secx}{\sec x}          \newcommand{\arcsecx}{\arcsec x}
\newcommand{\cotx}{\cot x}          \newcommand{\arccotx}{\arccot x}
\newcommand{\sinhx}{\sinh x}          \newcommand{\arsinhx}{\arsinh x}
\newcommand{\coshx}{\cosh x}          \newcommand{\arcoshx}{\arcosh x}
\newcommand{\tanhx}{\tanh x}          \newcommand{\artanhx}{\artanh x}
\newcommand{\cschx}{\csch x}          \newcommand{\arcschx}{\arcsch x}
\newcommand{\sechx}{\sech x}          \newcommand{\arsechx}{\arsech x}
\newcommand{\cothx}{\coth x}          \newcommand{\arcothx}{\arcoth x}
\newcommand{\lnx}{\ln x}
\newcommand{\expx}{\exp x}

\newcommand{\bba}{\mathbb{A}}   \newcommand{\bbn}{\mathbb{N}}
\newcommand{\bbb}{\mathbb{B}}   \newcommand{\bbo}{\mathbb{O}}
\newcommand{\bbc}{\mathbb{C}}   \newcommand{\bbp}{\mathbb{P}}
\newcommand{\bbd}{\mathbb{D}}   \newcommand{\bbq}{\mathbb{Q}}
\newcommand{\bbe}{\mathbb{E}}   \newcommand{\bbr}{\mathbb{R}}
\newcommand{\bbf}{\mathbb{F}}   \newcommand{\bbs}{\mathbb{S}}
\newcommand{\bbg}{\mathbb{G}}   \newcommand{\bbt}{\mathbb{T}}
\newcommand{\bbh}{\mathbb{H}}   \newcommand{\bbu}{\mathbb{U}}
\newcommand{\bbi}{\mathbb{I}}    \newcommand{\bbv}{\mathbb{V}}
\newcommand{\bbj}{\mathbb{J}}   \newcommand{\bbw}{\mathbb{W}}
\newcommand{\bbk}{\mathbb{K}}   \newcommand{\bbx}{\mathbb{X}}
\newcommand{\bbl}{\mathbb{L}}    \newcommand{\bby}{\mathbb{Y}}
\newcommand{\bbm}{\mathbb{M}}   \newcommand{\bbz}{\mathbb{Z}}

\newcommand{\lb}{\left(}
\newcommand{\rb}{\right)}
\newcommand{\lbr}{\left\lbrace}
\newcommand{\rbr}{\right\rbrace}
\newcommand{\lsb}{\left[}
\newcommand{\rsb}{\right]}
\newcommand{\suchthat}{\medspace\middle|\medspace}
\newcommand{\bracks}[1]{\lb #1 \rb}
\newcommand{\braces}[1]{\lbr #1 \rbr}
\newcommand{\sqbracks}[1]{\lsb #1 \rsb}

\renewcommand{\floor}[1]{\lfloor #1 \rfloor}
\renewcommand{\ceil}[1]{\lceil #1 \rceil}

\newcommand{\cd}{\cdot}
\newcommand{\tf}{\therefore}
\newcommand{\Let}{\text{Let }}
\newcommand{\Given}{\text{Given }}
\newcommand{\Suppose}{\text{Suppose }}
\newcommand{\WeSee}{\text{We see }}
\newcommand{\So}{\text{So }}

\newcommand{\QED}{\hfill \qed}

\renewcommand{\dd}[1]{\frac{d}{d#1}}
\newcommand{\dyd}[2][y]{\frac{d#1}{d#2}}

\newcommand{\ddx}{\dd{x}}       \newcommand{\ddxsq}{\dyd[^2]{x^2}}
\newcommand{\ddy}{\dd{y}}       \newcommand{\ddysq}{\dyd[^2]{y^2}}
\newcommand{\ddu}{\dd{u}}       \newcommand{\ddusq}{\dyd[^2]{u^2}}
\newcommand{\ddv}{\dd{v}}       \newcommand{\ddvsq}{\dyd[^2]{v^2}}

\newcommand{\dydx}{\dyd{x}}     \newcommand{\dydxsq}{\dyd[^2y]{x^2}}
\newcommand{\dfdx}{\dyd[f]{x}}  \newcommand{\dfdxsq}{\dyd[^2f]{x^2}}
\newcommand{\dudx}{\dyd[u]{x}}  \newcommand{\dudxsq}{\dyd[^2u]{x^2}}
\newcommand{\dvdx}{\dyd[v]{x}}  \newcommand{\dvdxsq}{\dyd[^2v]{x^2}}

% Mathfrak primes
\newcommand{\km}{\mathfrak{m}}
\newcommand{\kp}{\mathfrak{p}}
\newcommand{\kq}{\mathfrak{q}}

%---------------------------------------
% Blackboard Math Fonts :-
%---------------------------------------
\newcommand{\bba}{\mathbb{A}}   \newcommand{\bbn}{\mathbb{N}}
\newcommand{\bbb}{\mathbb{B}}   \newcommand{\bbo}{\mathbb{O}}
\newcommand{\bbc}{\mathbb{C}}   \newcommand{\bbp}{\mathbb{P}}
\newcommand{\bbd}{\mathbb{D}}   \newcommand{\bbq}{\mathbb{Q}}
\newcommand{\bbe}{\mathbb{E}}   \newcommand{\bbr}{\mathbb{R}}
\newcommand{\bbf}{\mathbb{F}}   \newcommand{\bbs}{\mathbb{S}}
\newcommand{\bbg}{\mathbb{G}}   \newcommand{\bbt}{\mathbb{T}}
\newcommand{\bbh}{\mathbb{H}}   \newcommand{\bbu}{\mathbb{U}}
\newcommand{\bbi}{\mathbb{I}}   \newcommand{\bbv}{\mathbb{V}}
\newcommand{\bbj}{\mathbb{J}}   \newcommand{\bbw}{\mathbb{W}}
\newcommand{\bbk}{\mathbb{K}}   \newcommand{\bbx}{\mathbb{X}}
\newcommand{\bbl}{\mathbb{L}}   \newcommand{\bby}{\mathbb{Y}}
\newcommand{\bbm}{\mathbb{M}}   \newcommand{\bbz}{\mathbb{Z}}

%---------------------------------------
% Roman Math Fonts :-
%---------------------------------------
\newcommand{\rma}{\mathrm{A}}   \newcommand{\rmn}{\mathrm{N}}
\newcommand{\rmb}{\mathrm{B}}   \newcommand{\rmo}{\mathrm{O}}
\newcommand{\rmc}{\mathrm{C}}   \newcommand{\rmp}{\mathrm{P}}
\newcommand{\rmd}{\mathrm{D}}   \newcommand{\rmq}{\mathrm{Q}}
\newcommand{\rme}{\mathrm{E}}   \newcommand{\rmr}{\mathrm{R}}
\newcommand{\rmf}{\mathrm{F}}   \newcommand{\rms}{\mathrm{S}}
\newcommand{\rmg}{\mathrm{G}}   \newcommand{\rmt}{\mathrm{T}}
\newcommand{\rmh}{\mathrm{H}}   \newcommand{\rmu}{\mathrm{U}}
\newcommand{\rmi}{\mathrm{I}}   \newcommand{\rmv}{\mathrm{V}}
\newcommand{\rmj}{\mathrm{J}}   \newcommand{\rmw}{\mathrm{W}}
\newcommand{\rmk}{\mathrm{K}}   \newcommand{\rmx}{\mathrm{X}}
\newcommand{\rml}{\mathrm{L}}   \newcommand{\rmy}{\mathrm{Y}}
\newcommand{\rmm}{\mathrm{M}}   \newcommand{\rmz}{\mathrm{Z}}

%---------------------------------------
% Calligraphic Math Fonts :-
%---------------------------------------
\newcommand{\cla}{\mathcal{A}}   \newcommand{\cln}{\mathcal{N}}
\newcommand{\clb}{\mathcal{B}}   \newcommand{\clo}{\mathcal{O}}
\newcommand{\clc}{\mathcal{C}}   \newcommand{\clp}{\mathcal{P}}
\newcommand{\cld}{\mathcal{D}}   \newcommand{\clq}{\mathcal{Q}}
\newcommand{\cle}{\mathcal{E}}   \newcommand{\clr}{\mathcal{R}}
\newcommand{\clf}{\mathcal{F}}   \newcommand{\cls}{\mathcal{S}}
\newcommand{\clg}{\mathcal{G}}   \newcommand{\clt}{\mathcal{T}}
\newcommand{\clh}{\mathcal{H}}   \newcommand{\clu}{\mathcal{U}}
\newcommand{\cli}{\mathcal{I}}   \newcommand{\clv}{\mathcal{V}}
\newcommand{\clj}{\mathcal{J}}   \newcommand{\clw}{\mathcal{W}}
\newcommand{\clk}{\mathcal{K}}   \newcommand{\clx}{\mathcal{X}}
\newcommand{\cll}{\mathcal{L}}   \newcommand{\cly}{\mathcal{Y}}
\newcommand{\calm}{\mathcal{M}}  \newcommand{\clz}{\mathcal{Z}}

%---------------------------------------
% Fraktur  Math Fonts :-
%---------------------------------------
\newcommand{\fka}{\mathfrak{A}}   \newcommand{\fkn}{\mathfrak{N}}
\newcommand{\fkb}{\mathfrak{B}}   \newcommand{\fko}{\mathfrak{O}}
\newcommand{\fkc}{\mathfrak{C}}   \newcommand{\fkp}{\mathfrak{P}}
\newcommand{\fkd}{\mathfrak{D}}   \newcommand{\fkq}{\mathfrak{Q}}
\newcommand{\fke}{\mathfrak{E}}   \newcommand{\fkr}{\mathfrak{R}}
\newcommand{\fkf}{\mathfrak{F}}   \newcommand{\fks}{\mathfrak{S}}
\newcommand{\fkg}{\mathfrak{G}}   \newcommand{\fkt}{\mathfrak{T}}
\newcommand{\fkh}{\mathfrak{H}}   \newcommand{\fku}{\mathfrak{U}}
\newcommand{\fki}{\mathfrak{I}}   \newcommand{\fkv}{\mathfrak{V}}
\newcommand{\fkj}{\mathfrak{J}}   \newcommand{\fkw}{\mathfrak{W}}
\newcommand{\fkk}{\mathfrak{K}}   \newcommand{\fkx}{\mathfrak{X}}
\newcommand{\fkl}{\mathfrak{L}}   \newcommand{\fky}{\mathfrak{Y}}
\newcommand{\fkm}{\mathfrak{M}}   \newcommand{\fkz}{\mathfrak{Z}}

\usepackage{float}

\begin{document}
\begin{center}
  {\bf\small Universtiy of Queensland \\ School of Mathematics and Physics}
\end{center}
\begin{center}
	{\Large\bf MATH2302 Discrete Mathematics II \\ Semester 2 2025 \\ Problem Set 3} \\ \vspace{1em}
	Michael Kasumagic, 44302669 \\
  Applied Class \#1 \\
	Due 3pm Friday 17 October 2025
\end{center}

\qs{10 marks}{
  For each of the following words: (i) draw the polygon represented by the word; (ii) determine whether the resulting surface is orientable or non-orientable; (iii) compute the Euler characteristic; (iv) use  ii. and iii. to identify the surface  in the form that it is given in the classification theorem. 

  \begin{enumerate}[label=(\alph*), itemsep=0pt]
    \item $a\,b\,c\,a^{-1}\,b\,c^{-1}$
    \item $a\,b\,c\,a^{-1}\,d\,b^{-1}\,c^{-1}\,d^{-1}$
  \end{enumerate}
}
\sol (a)
\begin{figure}[H]\begin{center}
  \begin{tikzpicture}[scale=2, line cap=round, line join=round]
    \coordinate (P0) at ({cos(120)},  {sin(120)});
    \coordinate (P1) at ({cos(60)},   {sin(60)});
    \coordinate (P2) at ({cos(0)},    {sin(0)});
    \coordinate (P3) at ({cos(-60)},  {sin(-60)});
    \coordinate (P4) at ({cos(-120)}, {sin(-120)});
    \coordinate (P5) at ({cos(180)},  {sin(180)});

    \node[above left]  at (P0) {$\alpha$};
    \node[above right] at (P1) {$\beta$};
    \node[right]       at (P2) {$\gamma$};
    \node[below right] at (P3) {$\delta$};
    \node[below left]  at (P4) {$\veps$};
    \node[left]        at (P5) {$\zeta$};
  
    \draw[thick] (P0)--(P1)--(P2)--(P3)--(P4)--(P5)--cycle;

    \tikzset{arr/.style={->, line width=1pt, >=Stealth}}

    \path (P5)--(P0) node[midway, left] {$a$};
    \draw[arr] (P5) -- ($(P5)!0.57!(P0)$);
    
    \path (P0)--(P1) node[midway, above] {$b$};
    \draw[arr] (P0) -- ($(P0)!0.57!(P1)$);

    \path (P1)--(P2) node[midway, above right] {$c$};
    \draw[arr] (P1) -- ($(P1)!0.57!(P2)$);

    \path (P2)--(P3) node[midway, right] {$a\inv$};
    \draw[arr] (P3) -- ($(P3)!0.57!(P2)$);

    \path (P3)--(P4) node[midway, below] {$b$};
    \draw[arr] (P3) -- ($(P3)!0.57!(P4)$);

    \path (P4)--(P5) node[midway, below left] {$c\inv$};
    \draw[arr] (P5) -- ($(P5)!0.57!(P4)$);
  \end{tikzpicture}
  \caption{A drawing of the polygon represented by the word $abca\inv bc\inv$}
\end{center}\end{figure}\vspace{-1em}

A word describes an orientable surface if and only if every symbol appears with both signs. \\
${}^{}$\hspace{1em}$a$ appears with both signs. $c$ appears with both signs. \\
${}^{}$\hspace{1em}However, $b$ only appears with one sign. \\
Therefore, this word describes a non-orientable surface. \vspace{0.5em}

$F = 1$ face, trivially. \\
The word has length $6=2\cd3\iff n = 3$. \\
$E = n = 3$ edges. \\
Calculating the vertices now,\\
${}^{}$\hspace{1em}Let's start with $a$ and $a\inv$. We note that $\alpha\liff\gamma$ and $\zeta\liff\delta$. \\
${}^{}$\hspace{1em}Next we'll consider $b$ and $b$. We note that $\alpha\liff\delta$ and $\beta\liff\veps$. \\
${}^{}$\hspace{1em}Finally, let's consider $c$ and $c\inv$, and note $\beta\liff\zeta$ and $\gamma\liff\veps$. \\
All together, we have
$$
  \alpha\liff\gamma\liff\veps\liff\beta\liff\zeta\liff\delta
$$
Therefore, the surface has 1 vertex. $V=1$ vertex. \\
Therefore, the surface has Euler characteristic, $\chi = V - E + F = 1 - 3 + 1 = -1$. \vspace{0.5em}

Therefore, a non-orientable surface with $\chi=-1 = 2-g \iff g=3$, is the connected sum of 3 projective planes, by the classification theorem.

\newpage
\sol (b) \vspace{0.5em}
\begin{figure}[H]\begin{center}
  \begin{tikzpicture}[scale=2, line cap=round, line join=round]
    \coordinate (P0) at ({cos(112.5)},  {sin(112.5)});  
    \coordinate (P1) at ({cos(67.5)},   {sin(67.5)});   
    \coordinate (P2) at ({cos(22.5)},   {sin(22.5)});   
    \coordinate (P3) at ({cos(-22.5)},  {sin(-22.5)}); 
    \coordinate (P4) at ({cos(-67.5)},  {sin(-67.5)});  
    \coordinate (P5) at ({cos(-112.5)}, {sin(-112.5)}); 
    \coordinate (P6) at ({cos(-157.5)}, {sin(-157.5)}); 
    \coordinate (P7) at ({cos(-202.5)}, {sin(-202.5)});

    \node[above left]  at (P0) {$\alpha$};
    \node[above right] at (P1) {$\beta$};
    \node[right]       at (P2) {$\gamma$};
    \node[right] at (P3) {$\delta$};
    \node[below right]  at (P4) {$\veps$};
    \node[below left]        at (P5) {$\zeta$};
    \node[left]        at (P6) {$\eta$};
    \node[left]        at (P7) {$\theta$};
  
    \draw[thick] (P0)--(P1)--(P2)--(P3)--(P4)--(P5)--(P6)--(P7)--cycle;

    \tikzset{arr/.style={->, line width=1pt, >=Stealth}}

    \path (P7)--(P0) node[midway, above left] {$a$};
    \draw[arr] (P7) -- ($(P7)!0.57!(P0)$);
    
    \path (P0)--(P1) node[midway, above] {$b$};
    \draw[arr] (P0) -- ($(P0)!0.57!(P1)$);

    \path (P1)--(P2) node[midway, above right] {$c$};
    \draw[arr] (P1) -- ($(P1)!0.57!(P2)$);

    \path (P2)--(P3) node[midway, right] {$a\inv$};
    \draw[arr] (P3) -- ($(P3)!0.57!(P2)$);

    \path (P3)--(P4) node[midway, below right] {$d$};
    \draw[arr] (P3) -- ($(P3)!0.57!(P4)$);

    \path (P4)--(P5) node[midway, below] {$b\inv$};
    \draw[arr] (P5) -- ($(P5)!0.57!(P4)$);

    \path (P5)--(P6) node[midway, below left] {$c\inv$};
    \draw[arr] (P6) -- ($(P6)!0.57!(P5)$);

    \path (P6)--(P7) node[midway, left] {$d\inv$};
    \draw[arr] (P7) -- ($(P7)!0.57!(P6)$);
  \end{tikzpicture}
  \caption{A drawing of the polygon represented by the word $abca\inv db\inv c\inv d\inv$}
\end{center}\end{figure}\vspace{-1em}

A word describes an orientable surface if and only if every symbol appears with both signs. \\
${}^{}$\hspace{1em}$a$ appears with both signs. \\
${}^{}$\hspace{1em}$b$ appears with both signs. \\
${}^{}$\hspace{1em}$c$ appears with both signs. \\
${}^{}$\hspace{1em}$d$ appears with both signs. \\
Therefore, this word describes a orientable surface. \vspace{0.5em}

$F = 1$ face, trivially. \\
The word has length $8=2\cd4\iff n = 4$. \\
$E = n = 4$ edges. \\
Calculating the vertices now,\\
${}^{}$\hspace{1em}Let's start with $a$ and $a\inv$. We note that $\alpha\liff\gamma$ and $\theta\liff\delta$. \\
${}^{}$\hspace{1em}Next we'll consider $b$ and $b\inv$. We note that $\beta\liff\veps$ and $\alpha\liff\zeta$. \\
${}^{}$\hspace{1em}Next we'll consider $c$ and $c\inv$. We note that $\gamma\liff\zeta$ and $\beta\liff\eta$. \\
${}^{}$\hspace{1em}Finally, let's consider $d$ and $d\inv$, and note $\veps\liff\eta$ and $\delta\liff\theta$. \\
All together, we have
$$
  \alpha\liff\gamma\liff\zeta,\qquad\theta\liff\delta,\qquad\beta\liff\veps\liff\eta
$$
Therefore, the polygon has 3 vertices. \\
$V=3$ vertices. \\
Therefore, the polygon has Euler characteristic, $\chi = V - E + F = 3 - 4 + 1 = 0$. \vspace{0.5em}

Therefore, a orientable surface with $\chi=0 = 2-2g \iff g=1$, is the connected sum of 1 torus, by the classification theorem, ie, a torus.

\newpage
\qs{5 marks}{
  Consider the closed surface represented by the following word: 
  \[
    a\,b\,c\,a^{-1}\,d\,b\,c^{-1}\,d^{-1}.
  \]
  Use the word rules from  lectures to  identify the surface in the form stated in the classification theorem.
  }
\sol\vspace{0.5em}

Firstly, we'll note that $b$ appears with only one sign, hence the surface the word represents is non-orientable. \vspace{-1em}
\begin{align*}
  \intertext{We'll start transforming the word by applying rule 6, ``moving outside $xx$ pairs,'' to the two $b$'s.}
  a\,b\,c\,a\inv\,d\,b\,c\inv\,d\inv &\stackrel{r.6}{=} a\,b\,b\,d\inv\,a\,c\inv\,c\inv\,d\inv
  \intertext{We'll then apply rule 6 again to the two $d\inv$'s.}
  a\,b\,b\,d\inv\,a\,c\inv\,c\inv\,d\inv &\stackrel{r.6}{=} a\,b\,b\,d\inv\,d\inv\,c\,c\,a\inv
  \intertext{Next, we'll apply rule 2, ``cycling edges,'' to move the $a\inv$ to the front of the word.}
  a\,b\,b\,d\inv\,d\inv\,c\,c\,a\inv &\stackrel{r.2}{=} a\inv\,a\,b\,b\,d\inv\,d\inv\,c\,c
  \intertext{For the sake of sematics (not-shortcutting anything), we'll apply rule 1, ``relabelling edges,'' to swap around $a$ and $a\inv$.}
  a\inv\,a\,b\,b\,d\inv\,d\inv\,c\,c &\stackrel{r.1}{=} a\,a\inv\,b\,b\,d\inv\,d\inv\,c\,c
  \intertext{Now, we can apply rule 5, ``cancelling $xx\inv$ pairs'' to cancel out the $aa\inv$ at the front of the word.}
  a\,a\inv\,b\,b\,d\inv\,d\inv\,c\,c &\stackrel{r.5}{=} b\,b\,d\inv\,d\inv\,c\,c
  \intertext{Finally, we'll clean this up by applying rule 1 twice: to relabel $b$ as $a$; and to relabel $d\inv$ as $b$.}
  b\,b\,d\inv\,d\inv\,c\,c &\stackrel{r.1}{=} a\,a\,b\,b\,c\,c
\end{align*}
In accordance with the classification theorem, the word $abca^{-1}dbc^{-1}d^{-1}$, therefore, represents a surface which is non-orientable and the sum of 3 projective planes with Euler characteristic $-1$.

\newpage
\qs{8 marks}{
  Consider the sequence $S=(5,3,3,3,1,1)$.
  \begin{enumerate}[label=(\alph*), itemsep=0pt]
    \item Show that $S$ is graphical, and draw a graph with the degree sequence $S$.
    \item Prove that the graph with the degree sequence $S$ is unique up to graph isomorphism.
  \end{enumerate}
}
\sol (a) \vspace{0.5em}

To show $S$ is graphical, we'll use Theorem 25.76. First, we'll note that $S$ is monotone decreasing ($d_1=5\geq d_2=3\geq ... \geq d_p = d_6 = 1$), with $p=6\geq2$ and $\Delta = 5\geq1$. Therefore, we can apply the theorem; the first reduction:
\[
  (5,3,3,3,1,1) \xrightarrow{Thm. 25.76} (3-1, 3-1, 3-1, 1-1, 1-1) = (2,2,2,0,0)
\]
The sequence is monotone decreasing, with $p=5\geq2$ and $\Delta=2\geq1$. The second reduction:
\[
  (2,2,2,0,0) \xrightarrow{Thm. 25.76} (2-1, 2-1, 0, 0) = (1,1,0,0)
\]
The sequence is monotone decreasing, with $p=4\geq2$ and $\Delta=1\geq1$. The third reduction:
\[
  (1,1,0,0) \xrightarrow{Thm. 25.76} (1-1, 0, 0) = (0,0,0)
\]
Because the $\Delta$ for this sequence is $0<1$, the recurrsion ceases, but this reduced graph is trivially graphical because it represents the degree sequence of some graph (namely, three vertices with no edges). Therefore, $S$ is graphical.

\begin{figure}[H]\begin{center}
  \begin{tikzpicture}[
    scale=2,
    v/.style={circle, fill=black, inner sep=1.6pt},
    every label/.style={inner sep=1pt}
  ]
    \node[v, label=right:$v_0$] (a) at (0,0) {};
    \node[v, label=above left:$v_1$] (b) at (-1,1) {};
    \node[v, label=above:$v_2$] (c) at (0,2) {};
    \node[v, label=above right:$v_3$] (d) at (1,1) {};
    \node[v, label=below:$v_4$] (e) at (-1,-1) {};
    \node[v, label=below:$v_5$] (f) at (1,-1) {};
    
    \draw (a)--(b) (a)--(c) (a)--(d) (a)--(e) (a)--(f);
    \draw (b)--(c) (d)--(c) (b)--(d);
  \end{tikzpicture}
  \caption{A drawing of a simple graph with degree sequence $S = (5,3,3,3,1,1)$.}
\end{center}\end{figure}\vspace{-1em}

\newpage
\sol (b) \vspace{0.5em}

\Claim\ The graph with degree sequence $S=(5,3,3,3,1,1)$ is unique up to graph isomorphism. \\
\Proof\ By construction.
\begin{list}{}{\setlength{\leftmargin}{0.6in}\setlength{\topsep}{0pt}}\item 
  The graph, $G$, with degree sequence $S$ is graphical.

  $\abs{S}=6$, hence $\abs{V(G)}=6$. \\
  $\Delta(S)=5 = \abs{V(G)}-1$, so the vertex with the greatest degree, call it $v_0$ has $d(v_0)=5$, and must be connected to all other vertices in the graph. \\
  Therefore, $\braces{\braces{v_0,v_1}, \braces{v_0,v_2}, \braces{v_0,v_3}, \braces{v_0,v_4}\braces{v_0,v_5}}= M \subseteq E(G)$.

  There are two vertices, call them $v_4$ and $v_5$ with $d(v_4)=d(v_5)=1$. \\
  The only choice for the other endpoint of these edges is $v_0$. \\
  These edges, namely $\braces{v_4,v_0}$ and $\braces{v_5,v_0}$, are already in the constructed set.

  The remaining three vertices, $v_1,\ v_2,\ v_3$ have $d(v_1)=d(v_2)=d(v_3)=3$. \\
  Each of these has one edge joining it to $v_0$, namely $\braces{v_1,v_0},\ \braces{v_2,v_0},\ \braces{v_3,v_0}$, all of which are already included in our constructed set. \\
  $v_1$ has 2 incident edges left to account for and only 2 choices: $v_2$ and $v_3$. \\
  $v_2$ has 2 incident edges left to account for and only 2 choices: $v_1$ and $v_3$. \\
  $v_3$ has 2 incident edges left to account for and only 2 choices: $v_1$ and $v_2$. \\
  Therefore, $\braces{\braces{v_1,v_2},\braces{v_1,v_3}}\cup\braces{\braces{v_2,v_1},\braces{v_2,v_3}}\cup\braces{\braces{v_3,v_1},\braces{v_3,v_1}}$ \\ $= \braces{\braces{v_1,v_2},\braces{v_1,v_3},\braces{v_2,v_3}} = T \subseteq E(G)$

  $M\cup T = \braces{\braces{v_0,v_1}, \braces{v_0,v_2}, \braces{v_0,v_3}, \braces{v_0,v_4}, \braces{v_0,v_5}, \braces{v_1,v_2}, \braces{v_1,v_3}, \braces{v_2,v_3}} \subseteq E(G)$

  By the Handshake Theorem, $\dps{\sum_{i=0}^{p-1}d(v_i) = d(v_0) + d(v_1) + d(v_2) + d(v_3) + d(v_4) + d(v_5)}$ \\ $= 5 + 3 + 3 + 3 + 1 + 1 = 16 = 2q\iff q=8$, therefore, $\abs{E(G)}=8$.

  $\abs{M \cup T} = 8$. 

  Therefore, $M\cup T = E(G)$.

  By our construction, this is the only graph with degree sequence $S$ (a consequence of the various forced decisions made throughout the process). \\
\end{list}
Therefore, the graph with degree sequence $S$ is unique up to graph isomorphism. $\QED$

\newpage
\qs{7 marks}{
  Suppose that $G$ is a graph on $4k+3$ vertices, where each vertex has degree at least $2k+1$, for some integer $k \geq 1$. 
  \begin{enumerate}[label=(\alph*), itemsep=0pt]
    \item  Can $G$ be $(2k+1)$-regular? Justify your answer. 
    \item  Prove that $G$ is connected. 
  \end{enumerate}
}
\sol (a) \vspace{0.3em}

Suppose $G$ were $(2k+1)$-regular. \\
Then it would be a graph with $4k+3$ vertices, where each vertex has degree $2k+1$. \vspace{0.3em}

Then, we could calculate this graph's degree total: 
\[
  (4k+3)(2k+1) = 8k^2+10k+3 = 2(4k^2+5k+1)+1 = 2z+1,\ z\in\bbz,
\]
which has odd parity. \vspace{0.3em}

We can also calculate this graphs degree total with the Handshake Theorem, namely:
\[
  \sum_{\forall v\in V(G)} d(v) = 2q,\ q\in\bbz,
\]
which has even parity.\vspace{0.3em}

Therefore, no $(2k+1)$-regular graph with $4k+3$ vertices exists, because of this clear contradiction.\\
Therefore, $G$ cannot be $(2k+1)$-regular. \\

\sol (b) \vspace{0.5em}

Suppose $G$ is a graph on $4k+3$ vertices such that each vertex has degree at least $2k+1$, for some integer $k\geq1$. \\
\Claim\ G is connected.
\proof\ By contradiction.
\begin{list}{}{\setlength{\leftmargin}{0.6in}\setlength{\topsep}{0pt}}\item 
  For the sake of contradiction, suppose that $G$ is disconected. \\
  Then $G$ can be decomposed into at least 2 connected subgraphs. 

  Let one of these components have $t$ vertices. \\
  Then each vertex in this component has degree at most $t-1$.

  Since, $\delta(G)\geq 2k+1,$ we know $t-1\geq 2k+1$,\ $t\geq 2k+2$.

  Now, since $G$ can be decomposed into \textit{at least} 2 components,
  \begin{equation*}
    \abs{V(G)}\geq 2(2k+2) = 4k+4 > 4k+3 = \abs{V(G)} \tag*{\CONTRA}
  \end{equation*}
  This is a clear contradiction because $V(G)$ must simultaneously be equal to $4k+3$ and greater than or equal to $4k+4$. Absurd. \\
  Therefore, $G$ is not disconected.  
\end{list}
Therefore, $G$ is connected. $\QED$


\end{document}
