\documentclass[a4paper,11pt]{report}

\input{../../../latex_template/preamble}
%From M275 "Topology" at SJSU
\newcommand{\id}{\mathrm{id}}
\newcommand{\taking}[1]{\xrightarrow{#1}}
\newcommand{\inv}{^{-1}}

%From M170 "Introduction to Graph Theory" at SJSU
\DeclareMathOperator{\diam}{diam}
\DeclareMathOperator{\ord}{ord}
\newcommand{\defeq}{\overset{\mathrm{def}}{=}}

%From the USAMO .tex files
\newcommand{\ts}{\textsuperscript}
\newcommand{\dg}{^\circ}
\newcommand{\ii}{\item}

% % From Math 55 and Math 145 at Harvard
% \newenvironment{subproof}[1][Proof]{%
% \begin{proof}[#1] \renewcommand{\qedsymbol}{$\blacksquare$}}%
% {\end{proof}}

\newcommand{\liff}{\leftrightarrow}
\newcommand{\lthen}{\rightarrow}
\newcommand{\opname}{\operatorname}
\newcommand{\surjto}{\twoheadrightarrow}
\newcommand{\injto}{\hookrightarrow}
\newcommand{\On}{\mathrm{On}} % ordinals
\DeclareMathOperator{\img}{im} % Image
\DeclareMathOperator{\Img}{Im} % Image
\DeclareMathOperator{\coker}{coker} % Cokernel
\DeclareMathOperator{\Coker}{Coker} % Cokernel
\DeclareMathOperator{\Ker}{Ker} % Kernel
\DeclareMathOperator{\rank}{rank}
\DeclareMathOperator{\Spec}{Spec} % spectrum
\DeclareMathOperator{\Tr}{Tr} % trace
\DeclareMathOperator{\pr}{pr} % projection
\DeclareMathOperator{\ext}{ext} % extension
\DeclareMathOperator{\pred}{pred} % predecessor
\DeclareMathOperator{\dom}{dom} % domain
\DeclareMathOperator{\ran}{ran} % range
\DeclareMathOperator{\Hom}{Hom} % homomorphism
\DeclareMathOperator{\Mor}{Mor} % morphisms
\DeclareMathOperator{\End}{End} % endomorphism

\newcommand{\eps}{\epsilon}
\newcommand{\veps}{\varepsilon}
\newcommand{\ol}{\overline}
\newcommand{\ul}{\underline}
\newcommand{\wt}{\widetilde}
\newcommand{\wh}{\widehat}
\newcommand{\vocab}[1]{\textbf{\color{blue} #1}}
\providecommand{\half}{\frac{1}{2}}
\newcommand{\dang}{\measuredangle} %% Directed angle
\newcommand{\ray}[1]{\overrightarrow{#1}}
\newcommand{\seg}[1]{\overline{#1}}
\newcommand{\arc}[1]{\wideparen{#1}}
\DeclareMathOperator{\cis}{cis}
\DeclareMathOperator*{\lcm}{lcm}
\DeclareMathOperator*{\argmin}{arg min}
\DeclareMathOperator*{\argmax}{arg max}
\newcommand{\cycsum}{\sum_{\mathrm{cyc}}}
\newcommand{\symsum}{\sum_{\mathrm{sym}}}
\newcommand{\cycprod}{\prod_{\mathrm{cyc}}}
\newcommand{\symprod}{\prod_{\mathrm{sym}}}
\newcommand{\Qed}{\begin{flushright}\qed\end{flushright}}
\newcommand{\parinn}{\setlength{\parindent}{1cm}}
\newcommand{\parinf}{\setlength{\parindent}{0cm}}
% \newcommand{\norm}{\|\cdot\|}
\newcommand{\inorm}{\norm_{\infty}}
\newcommand{\opensets}{\{V_{\alpha}\}_{\alpha\in I}}
\newcommand{\oset}{V_{\alpha}}
\newcommand{\opset}[1]{V_{\alpha_{#1}}}
\newcommand{\lub}{\text{lub}}
\newcommand{\del}[2]{\frac{\partial #1}{\partial #2}}
\newcommand{\Del}[3]{\frac{\partial^{#1} #2}{\partial^{#1} #3}}
\newcommand{\deld}[2]{\dfrac{\partial #1}{\partial #2}}
\newcommand{\Deld}[3]{\dfrac{\partial^{#1} #2}{\partial^{#1} #3}}
\newcommand{\lm}{\lambda}
\newcommand{\uin}{\mathbin{\rotatebox[origin=c]{90}{$\in$}}}
\newcommand{\usubset}{\mathbin{\rotatebox[origin=c]{90}{$\subset$}}}
\newcommand{\lt}{\left}
\newcommand{\rt}{\right}
\newcommand{\bs}[1]{\boldsymbol{#1}}
\newcommand{\exs}{\exists}
\newcommand{\st}{\strut}
\newcommand{\dps}[1]{\displaystyle{#1}}

\newcommand{\sol}{\setlength{\parindent}{0cm}\textbf{\textit{Solution:}}\setlength{\parindent}{1cm} }
\newcommand{\solve}[1]{\setlength{\parindent}{0cm}\textbf{\textit{Solution: }}\setlength{\parindent}{1cm}#1 \Qed}

\DeclareMathOperator{\sech}{sech}
\DeclareMathOperator{\csch}{csch}
\DeclareMathOperator{\arcsec}{arcsec}
\DeclareMathOperator{\arccsc}{arccsc}
\DeclareMathOperator{\arccot}{arccot}
\DeclareMathOperator{\arsinh}{arsinh}
\DeclareMathOperator{\arcosh}{arcosh}
\DeclareMathOperator{\artanh}{artanh}
\DeclareMathOperator{\arcsch}{arcsch}
\DeclareMathOperator{\arsech}{arsech}
\DeclareMathOperator{\arcoth}{arcoth}

\newcommand{\sinx}{\sin x}          \newcommand{\arcsinx}{\arcsin x}    
\newcommand{\cosx}{\cos x}          \newcommand{\arccosx}{\arccosx}
\newcommand{\tanx}{\tan x}          \newcommand{\arctanx}{\arctan x}
\newcommand{\cscx}{\csc x}          \newcommand{\arccscx}{\arccsc x}
\newcommand{\secx}{\sec x}          \newcommand{\arcsecx}{\arcsec x}
\newcommand{\cotx}{\cot x}          \newcommand{\arccotx}{\arccot x}
\newcommand{\sinhx}{\sinh x}          \newcommand{\arsinhx}{\arsinh x}
\newcommand{\coshx}{\cosh x}          \newcommand{\arcoshx}{\arcosh x}
\newcommand{\tanhx}{\tanh x}          \newcommand{\artanhx}{\artanh x}
\newcommand{\cschx}{\csch x}          \newcommand{\arcschx}{\arcsch x}
\newcommand{\sechx}{\sech x}          \newcommand{\arsechx}{\arsech x}
\newcommand{\cothx}{\coth x}          \newcommand{\arcothx}{\arcoth x}
\newcommand{\lnx}{\ln x}
\newcommand{\expx}{\exp x}

\newcommand{\bba}{\mathbb{A}}   \newcommand{\bbn}{\mathbb{N}}
\newcommand{\bbb}{\mathbb{B}}   \newcommand{\bbo}{\mathbb{O}}
\newcommand{\bbc}{\mathbb{C}}   \newcommand{\bbp}{\mathbb{P}}
\newcommand{\bbd}{\mathbb{D}}   \newcommand{\bbq}{\mathbb{Q}}
\newcommand{\bbe}{\mathbb{E}}   \newcommand{\bbr}{\mathbb{R}}
\newcommand{\bbf}{\mathbb{F}}   \newcommand{\bbs}{\mathbb{S}}
\newcommand{\bbg}{\mathbb{G}}   \newcommand{\bbt}{\mathbb{T}}
\newcommand{\bbh}{\mathbb{H}}   \newcommand{\bbu}{\mathbb{U}}
\newcommand{\bbi}{\mathbb{I}}    \newcommand{\bbv}{\mathbb{V}}
\newcommand{\bbj}{\mathbb{J}}   \newcommand{\bbw}{\mathbb{W}}
\newcommand{\bbk}{\mathbb{K}}   \newcommand{\bbx}{\mathbb{X}}
\newcommand{\bbl}{\mathbb{L}}    \newcommand{\bby}{\mathbb{Y}}
\newcommand{\bbm}{\mathbb{M}}   \newcommand{\bbz}{\mathbb{Z}}

\newcommand{\lb}{\left(}
\newcommand{\rb}{\right)}
\newcommand{\lbr}{\left\lbrace}
\newcommand{\rbr}{\right\rbrace}
\newcommand{\lsb}{\left[}
\newcommand{\rsb}{\right]}
\newcommand{\suchthat}{\medspace\middle|\medspace}
\newcommand{\bracks}[1]{\lb #1 \rb}
\newcommand{\braces}[1]{\lbr #1 \rbr}
\newcommand{\sqbracks}[1]{\lsb #1 \rsb}

\renewcommand{\floor}[1]{\lfloor #1 \rfloor}
\renewcommand{\ceil}[1]{\lceil #1 \rceil}

\newcommand{\cd}{\cdot}
\newcommand{\tf}{\therefore}
\newcommand{\Let}{\text{Let }}
\newcommand{\Given}{\text{Given }}
\newcommand{\Suppose}{\text{Suppose }}
\newcommand{\WeSee}{\text{We see }}
\newcommand{\So}{\text{So }}

\newcommand{\QED}{\hfill \qed}

\renewcommand{\dd}[1]{\frac{d}{d#1}}
\newcommand{\dyd}[2][y]{\frac{d#1}{d#2}}

\newcommand{\ddx}{\dd{x}}       \newcommand{\ddxsq}{\dyd[^2]{x^2}}
\newcommand{\ddy}{\dd{y}}       \newcommand{\ddysq}{\dyd[^2]{y^2}}
\newcommand{\ddu}{\dd{u}}       \newcommand{\ddusq}{\dyd[^2]{u^2}}
\newcommand{\ddv}{\dd{v}}       \newcommand{\ddvsq}{\dyd[^2]{v^2}}

\newcommand{\dydx}{\dyd{x}}     \newcommand{\dydxsq}{\dyd[^2y]{x^2}}
\newcommand{\dfdx}{\dyd[f]{x}}  \newcommand{\dfdxsq}{\dyd[^2f]{x^2}}
\newcommand{\dudx}{\dyd[u]{x}}  \newcommand{\dudxsq}{\dyd[^2u]{x^2}}
\newcommand{\dvdx}{\dyd[v]{x}}  \newcommand{\dvdxsq}{\dyd[^2v]{x^2}}

% Mathfrak primes
\newcommand{\km}{\mathfrak{m}}
\newcommand{\kp}{\mathfrak{p}}
\newcommand{\kq}{\mathfrak{q}}

%---------------------------------------
% Blackboard Math Fonts :-
%---------------------------------------
\newcommand{\bba}{\mathbb{A}}   \newcommand{\bbn}{\mathbb{N}}
\newcommand{\bbb}{\mathbb{B}}   \newcommand{\bbo}{\mathbb{O}}
\newcommand{\bbc}{\mathbb{C}}   \newcommand{\bbp}{\mathbb{P}}
\newcommand{\bbd}{\mathbb{D}}   \newcommand{\bbq}{\mathbb{Q}}
\newcommand{\bbe}{\mathbb{E}}   \newcommand{\bbr}{\mathbb{R}}
\newcommand{\bbf}{\mathbb{F}}   \newcommand{\bbs}{\mathbb{S}}
\newcommand{\bbg}{\mathbb{G}}   \newcommand{\bbt}{\mathbb{T}}
\newcommand{\bbh}{\mathbb{H}}   \newcommand{\bbu}{\mathbb{U}}
\newcommand{\bbi}{\mathbb{I}}   \newcommand{\bbv}{\mathbb{V}}
\newcommand{\bbj}{\mathbb{J}}   \newcommand{\bbw}{\mathbb{W}}
\newcommand{\bbk}{\mathbb{K}}   \newcommand{\bbx}{\mathbb{X}}
\newcommand{\bbl}{\mathbb{L}}   \newcommand{\bby}{\mathbb{Y}}
\newcommand{\bbm}{\mathbb{M}}   \newcommand{\bbz}{\mathbb{Z}}

%---------------------------------------
% Roman Math Fonts :-
%---------------------------------------
\newcommand{\rma}{\mathrm{A}}   \newcommand{\rmn}{\mathrm{N}}
\newcommand{\rmb}{\mathrm{B}}   \newcommand{\rmo}{\mathrm{O}}
\newcommand{\rmc}{\mathrm{C}}   \newcommand{\rmp}{\mathrm{P}}
\newcommand{\rmd}{\mathrm{D}}   \newcommand{\rmq}{\mathrm{Q}}
\newcommand{\rme}{\mathrm{E}}   \newcommand{\rmr}{\mathrm{R}}
\newcommand{\rmf}{\mathrm{F}}   \newcommand{\rms}{\mathrm{S}}
\newcommand{\rmg}{\mathrm{G}}   \newcommand{\rmt}{\mathrm{T}}
\newcommand{\rmh}{\mathrm{H}}   \newcommand{\rmu}{\mathrm{U}}
\newcommand{\rmi}{\mathrm{I}}   \newcommand{\rmv}{\mathrm{V}}
\newcommand{\rmj}{\mathrm{J}}   \newcommand{\rmw}{\mathrm{W}}
\newcommand{\rmk}{\mathrm{K}}   \newcommand{\rmx}{\mathrm{X}}
\newcommand{\rml}{\mathrm{L}}   \newcommand{\rmy}{\mathrm{Y}}
\newcommand{\rmm}{\mathrm{M}}   \newcommand{\rmz}{\mathrm{Z}}

%---------------------------------------
% Calligraphic Math Fonts :-
%---------------------------------------
\newcommand{\cla}{\mathcal{A}}   \newcommand{\cln}{\mathcal{N}}
\newcommand{\clb}{\mathcal{B}}   \newcommand{\clo}{\mathcal{O}}
\newcommand{\clc}{\mathcal{C}}   \newcommand{\clp}{\mathcal{P}}
\newcommand{\cld}{\mathcal{D}}   \newcommand{\clq}{\mathcal{Q}}
\newcommand{\cle}{\mathcal{E}}   \newcommand{\clr}{\mathcal{R}}
\newcommand{\clf}{\mathcal{F}}   \newcommand{\cls}{\mathcal{S}}
\newcommand{\clg}{\mathcal{G}}   \newcommand{\clt}{\mathcal{T}}
\newcommand{\clh}{\mathcal{H}}   \newcommand{\clu}{\mathcal{U}}
\newcommand{\cli}{\mathcal{I}}   \newcommand{\clv}{\mathcal{V}}
\newcommand{\clj}{\mathcal{J}}   \newcommand{\clw}{\mathcal{W}}
\newcommand{\clk}{\mathcal{K}}   \newcommand{\clx}{\mathcal{X}}
\newcommand{\cll}{\mathcal{L}}   \newcommand{\cly}{\mathcal{Y}}
\newcommand{\calm}{\mathcal{M}}  \newcommand{\clz}{\mathcal{Z}}

%---------------------------------------
% Fraktur  Math Fonts :-
%---------------------------------------
\newcommand{\fka}{\mathfrak{A}}   \newcommand{\fkn}{\mathfrak{N}}
\newcommand{\fkb}{\mathfrak{B}}   \newcommand{\fko}{\mathfrak{O}}
\newcommand{\fkc}{\mathfrak{C}}   \newcommand{\fkp}{\mathfrak{P}}
\newcommand{\fkd}{\mathfrak{D}}   \newcommand{\fkq}{\mathfrak{Q}}
\newcommand{\fke}{\mathfrak{E}}   \newcommand{\fkr}{\mathfrak{R}}
\newcommand{\fkf}{\mathfrak{F}}   \newcommand{\fks}{\mathfrak{S}}
\newcommand{\fkg}{\mathfrak{G}}   \newcommand{\fkt}{\mathfrak{T}}
\newcommand{\fkh}{\mathfrak{H}}   \newcommand{\fku}{\mathfrak{U}}
\newcommand{\fki}{\mathfrak{I}}   \newcommand{\fkv}{\mathfrak{V}}
\newcommand{\fkj}{\mathfrak{J}}   \newcommand{\fkw}{\mathfrak{W}}
\newcommand{\fkk}{\mathfrak{K}}   \newcommand{\fkx}{\mathfrak{X}}
\newcommand{\fkl}{\mathfrak{L}}   \newcommand{\fky}{\mathfrak{Y}}
\newcommand{\fkm}{\mathfrak{M}}   \newcommand{\fkz}{\mathfrak{Z}}

\usepackage{float}
\usetikzlibrary{calc,arrows,decorations.markings}

\begin{document}
\begin{center}
  {\bf\small Universtiy of Queensland \\ School of Mathematics and Physics}
\end{center}
\begin{center}
	{\Large\bf MATH2302 Discrete Mathematics II \\ Semester 2 2025 \\ Problem Set 4} \\ \vspace{1em}
	Michael Kasumagic, 44302669 \\
  Applied Class \#1 \\
	Due 3pm Friday 28 October 2025
\end{center}

\qs{6 marks}{
  For each of the graphs shown below, determine whether it is Hamiltonian. Justify your answer. 

  \vspace{0.5cm}
  \begin{minipage}{0.48\textwidth}\centering
  % ---------- FIRST TIKZ ----------
  \begin{tikzpicture}[
      scale=1,
      dot/.style={circle,fill,inner sep=1.5pt},
      edge/.style={line width=0.9pt}
  ]

  % radii
  \pgfmathsetmacro{\Rout}{1.0}
  \pgfmathsetmacro{\Rin}{0.45}

  % angles (top vertex at 90°, 72° apart)
  \foreach \k in {0,...,4}{
    \pgfmathsetmacro{\ang}{90 - 72*\k}
    \coordinate (O\k) at ({\Rout*cos(\ang)},{\Rout*sin(\ang)});
    \coordinate (I\k) at ({\Rin*cos(\ang)},{\Rin*sin(\ang)});
  }

  \coordinate (O) at (0,0);
  % --- Draw outer and inner pentagons ---
  \draw[edge] (O0)--(O1)--(O2)--(O3)--(O4)--cycle;
  %\draw[edge] (I0)--(I1)--(I2)--(I3)--(I4)--cycle;

  % --- Connect corresponding vertices ---
  \foreach \k in {0,...,4}{
    \draw[edge] (O)--(I\k);
  }

  % --- Dots at vertices ---
  \foreach \P in {O0,O1,O2,O3,O4,I0,I1,I2,I3,I4}
    \node[dot] at (\P) {};

  \node [dot] at (O) {};

  \node at (-1.5,0) {(a)};

  \draw[edge] (O0)--(I4) (O0)--(I1) (O2)--(I1) (O3)--(I4) (O2)--(I3) (O3)--(I2) (O1)--(I0) (O1)--(I2) (O4)--(I0) (O4)--(I3);


  \end{tikzpicture}
  \end{minipage}\hfill
  \begin{minipage}{0.48\textwidth}\centering
  % ---------- SECOND TIKZ ----------
  \begin{tikzpicture}[
      scale=1,
      dot/.style={circle,fill,inner sep=1.5pt},
      edge/.style={line width=0.9pt}
  ]
  \pgfmathsetmacro{\s}{0.866025403784}
  \pgfmathsetmacro{\m}{0.65}
  \pgfmathsetmacro{\r}{0.30}

  \coordinate (O1) at (1,0);
  \coordinate (O2) at (0.5,\s);
  \coordinate (O3) at (-0.5,\s);
  \coordinate (O4) at (-1,0);
  \coordinate (O5) at (-0.5,-\s);
  \coordinate (O6) at (0.5,-\s);

  \coordinate (M1) at (\m*1,\m*0);
  \coordinate (M2) at (\m*0.5,\m*\s);
  \coordinate (M3) at (\m*-0.5,\m*\s);
  \coordinate (M4) at (\m*-1,\m*0);
  \coordinate (M5) at (\m*-0.5,\m*-\s);
  \coordinate (M6) at (\m*0.5,\m*-\s);

  \coordinate (I1) at (\r*1,\r*0);
  \coordinate (I2) at (\r*0.5,\r*\s);
  \coordinate (I3) at (\r*-0.5,\r*\s);
  \coordinate (I4) at (\r*-1,\r*0);
  \coordinate (I5) at (\r*-0.5,\r*-\s);
  \coordinate (I6) at (\r*0.5,\r*-\s);

  \draw[edge] (O1)--(O2)--(O3)--(O4)--(O5)--(O6)--cycle;
  \draw[edge] (M1)--(M2)--(M3)--(M4)--(M5)--(M6)--cycle;
  \draw[edge] (I1)--(I2)--(I3)--(I4)--(I5)--(I6)--cycle;

  \draw[edge] (O1)--(I1) (O2)--(I2) (O3)--(I3)
              (O4)--(I4) (O5)--(I5) (O6)--(I6);

  \foreach \P in {O1,O2,O3,O4,O5,O6,M1,M2,M3,M4,M5,M6,I1,I2,I3,I4,I5,I6}
    \node[dot] at (\P) {};

  % midpoints on middle edges 
  \coordinate (Mm12) at ($(M1)!0.5!(M2)$);
  \coordinate (Mm23) at ($(M2)!0.5!(M3)$);
  \coordinate (Mm34) at ($(M3)!0.5!(M4)$);
  \coordinate (Mm45) at ($(M4)!0.5!(M5)$);
  \coordinate (Mm56) at ($(M5)!0.5!(M6)$);
  \coordinate (Mm61) at ($(M6)!0.5!(M1)$);
  \foreach \Q in {Mm12,Mm23,Mm34,Mm45,Mm56,Mm61} \node[dot] at (\Q) {};

  \node at (-1.5,0) {(b)};
  \end{tikzpicture}
  \end{minipage}
}
\sol \vspace{0.5em}

Graph (a) is Hamiltonian. I present one such spanning cycle:

\begin{center}\begin{tikzpicture}[
      scale=2,
      dot/.style={circle,fill,inner sep=1.5pt},
      edge/.style={line width=0.9pt}
  ]

  % radii
  \pgfmathsetmacro{\Rout}{1.0}
  \pgfmathsetmacro{\Rin}{0.45}

  % angles (top vertex at 90°, 72° apart)
  \foreach \k in {0,...,4}{
    \pgfmathsetmacro{\ang}{90 - 72*\k}
    \coordinate (O\k) at ({\Rout*cos(\ang)},{\Rout*sin(\ang)});
    \coordinate (I\k) at ({\Rin*cos(\ang)},{\Rin*sin(\ang)});
  }

  \coordinate (O) at (0,0);
  % --- Draw outer and inner pentagons ---
  \draw[edge] (O0)--(O1)--(O2)--(O3)--(O4)--cycle;
  %\draw[edge] (I0)--(I1)--(I2)--(I3)--(I4)--cycle;

  % --- Connect corresponding vertices ---
  \foreach \k in {0,...,4}{
    \draw[edge] (O)--(I\k);
  }

  % --- Dots at vertices ---
  \foreach \P in {O0,O1,O2,O3,O4,I0,I1,I2,I3,I4}
    \node[dot] at (\P) {};

  \node [dot] at (O) {};

  \draw[edge] (O0)--(I4) (O0)--(I1) (O2)--(I1) (O3)--(I4) (O2)--(I3) (O3)--(I2) (O1)--(I0) (O1)--(I2) (O4)--(I0) (O4)--(I3);

  \draw[->, red, ultra thick] (O) -- (I0);
  \draw[->, red, ultra thick] (I0) -- (O1);
  \draw[->, red, ultra thick] (O1) -- (I2);
  \draw[->, red, ultra thick] (I2) -- (O3);
  \draw[->, red, ultra thick] (O3) -- (I4);
  \draw[->, red, ultra thick] (I4) -- (O0);
  \draw[->, red, ultra thick] (O0) -- (O4);
  \draw[->, red, ultra thick] (O4) -- (I3);
  \draw[->, red, ultra thick] (I3) -- (O2);
  \draw[->, red, ultra thick] (O2) -- (I1);
  \draw[->, red, ultra thick] (I1) -- (O);

\end{tikzpicture}\end{center}

Graph (b) is not Hamiltonian. We show this by considering Theorem 26.85. Consider the set $S$, which contains only the highlighted vertices. Then consider the graph $G-S$, in particular, we count the number of components the graph has.

\begin{center}\begin{tikzpicture}[
      scale=2,
      dot/.style={circle,fill,inner sep=1.5pt},
      highlightDot/.style={circle,fill,inner sep=2.5pt},
      edge/.style={line width=0.9pt}
  ]
  \pgfmathsetmacro{\s}{0.866025403784}
  \pgfmathsetmacro{\m}{0.65}
  \pgfmathsetmacro{\r}{0.30}

  \coordinate (O1) at (1,0);
  \coordinate (O2) at (0.5,\s);
  \coordinate (O3) at (-0.5,\s);
  \coordinate (O4) at (-1,0);
  \coordinate (O5) at (-0.5,-\s);
  \coordinate (O6) at (0.5,-\s);

  \coordinate (M1) at (\m*1,\m*0);
  \coordinate (M2) at (\m*0.5,\m*\s);
  \coordinate (M3) at (\m*-0.5,\m*\s);
  \coordinate (M4) at (\m*-1,\m*0);
  \coordinate (M5) at (\m*-0.5,\m*-\s);
  \coordinate (M6) at (\m*0.5,\m*-\s);

  \coordinate (I1) at (\r*1,\r*0);
  \coordinate (I2) at (\r*0.5,\r*\s);
  \coordinate (I3) at (\r*-0.5,\r*\s);
  \coordinate (I4) at (\r*-1,\r*0);
  \coordinate (I5) at (\r*-0.5,\r*-\s);
  \coordinate (I6) at (\r*0.5,\r*-\s);

  \draw[edge] (O1)--(O2)--(O3)--(O4)--(O5)--(O6)--cycle;
  \draw[edge] (M1)--(M2)--(M3)--(M4)--(M5)--(M6)--cycle;
  \draw[edge] (I1)--(I2)--(I3)--(I4)--(I5)--(I6)--cycle;

  \draw[edge] (O1)--(I1) (O2)--(I2) (O3)--(I3)
              (O4)--(I4) (O5)--(I5) (O6)--(I6);


  \foreach \P in {O1,O2,O3,O4,O5,O6} \node[dot] at (\P) {};
  \foreach \P in {M1,M2,M3,M4,M5,M6} \node[highlightDot, red] at (\P) {};
  \foreach \P in {I1,I2,I3,I4,I5,I6} \node[dot] at (\P) {};

  % midpoints on middle edges 
  \coordinate (Mm12) at ($(M1)!0.5!(M2)$);
  \coordinate (Mm23) at ($(M2)!0.5!(M3)$);
  \coordinate (Mm34) at ($(M3)!0.5!(M4)$);
  \coordinate (Mm45) at ($(M4)!0.5!(M5)$);
  \coordinate (Mm56) at ($(M5)!0.5!(M6)$);
  \coordinate (Mm61) at ($(M6)!0.5!(M1)$);
  \foreach \Q in {Mm12,Mm23,Mm34,Mm45,Mm56,Mm61} \node[dot] at (\Q) {};

\end{tikzpicture}
\hspace{8em}
\begin{tikzpicture}[
      scale=2,
      dot/.style={circle,fill,inner sep=1.5pt},
      edge/.style={line width=0.9pt}
  ]
  \pgfmathsetmacro{\s}{0.866025403784}
  \pgfmathsetmacro{\m}{0.65}
  \pgfmathsetmacro{\r}{0.30}

  \coordinate (O1) at (1,0);
  \coordinate (O2) at (0.5,\s);
  \coordinate (O3) at (-0.5,\s);
  \coordinate (O4) at (-1,0);
  \coordinate (O5) at (-0.5,-\s);
  \coordinate (O6) at (0.5,-\s);

  \coordinate (M1) at (\m*1,\m*0);
  \coordinate (M2) at (\m*0.5,\m*\s);
  \coordinate (M3) at (\m*-0.5,\m*\s);
  \coordinate (M4) at (\m*-1,\m*0);
  \coordinate (M5) at (\m*-0.5,\m*-\s);
  \coordinate (M6) at (\m*0.5,\m*-\s);

  \coordinate (I1) at (\r*1,\r*0);
  \coordinate (I2) at (\r*0.5,\r*\s);
  \coordinate (I3) at (\r*-0.5,\r*\s);
  \coordinate (I4) at (\r*-1,\r*0);
  \coordinate (I5) at (\r*-0.5,\r*-\s);
  \coordinate (I6) at (\r*0.5,\r*-\s);

  \draw[edge] (O1)--(O2)--(O3)--(O4)--(O5)--(O6)--cycle;
  \draw[edge] (I1)--(I2)--(I3)--(I4)--(I5)--(I6)--cycle;

  \foreach \P in {O1,O2,O3,O4,O5,O6} \node[dot] at (\P) {};
  \foreach \P in {I1,I2,I3,I4,I5,I6} \node[dot] at (\P) {};

  % midpoints on middle edges 
  \coordinate (Mm12) at ($(M1)!0.5!(M2)$);
  \coordinate (Mm23) at ($(M2)!0.5!(M3)$);
  \coordinate (Mm34) at ($(M3)!0.5!(M4)$);
  \coordinate (Mm45) at ($(M4)!0.5!(M5)$);
  \coordinate (Mm56) at ($(M5)!0.5!(M6)$);
  \coordinate (Mm61) at ($(M6)!0.5!(M1)$);
  \foreach \Q in {Mm12,Mm23,Mm34,Mm45,Mm56,Mm61} \node[dot] at (\Q) {};

  \node[left] at (O1) {(1)};
  \node[left] at (I1) {(2)};
  \node[above] at (Mm23) {(3)};
  \node[above] at (Mm12) {(4)};
  \node[above] at (Mm61) {(5)};
  \node[above] at (Mm56) {(6)};
  \node[above] at (Mm45) {(7)};
  \node[above] at (Mm34) {(8)};

\end{tikzpicture}\end{center}
We highlighted 6 vertices for $S\subseteq G$, hence $\abs{S}=6$. The graph $G-S$, has componets: the outer ring, the inner ring, and the 6 isolated vertices. Hence, $\abs{\operatorname{Comp}(G-S)}=8$. 
\[
  \abs{S}=6 < 8 = \abs{\operatorname{Comp}(G-S)}.
\]
Therefore, by Theorem 26.85, the graph is not Hamiltonian.

\newpage
\qs{6 marks}{
  For any integer $n\geq 4$, let $K_n$ be the complete graph on vertices  $1,2,\ldots,n$. Let   $G_n$ be the graph obtained from  $K_n$ by deleting two edges $\{1,2\}$ and $\{3,4\}$. What is the vertex chromatic number of $G_n$? Justify your answer. 
}
\sol 


\newpage
\qs{8 marks}{
  Use Dijkstra's algorithm  to determine the distance from $s$ to each other vertex in the weighted graph shown below, and state a shortest path from $s$ to $e$.
  \begin{center}
   \begin{tikzpicture}[
      scale=1,
      v/.style={circle,fill,inner sep=1.6pt},
      e/.style={line width=0.9pt}
  ]
  % vertices
  \node (s) at (-2,0)   [v,label=left:$s$] {};
  \node (a) at (0,1.6)  [v,label=above:$a$] {};
  \node (b) at (0,-1.6) [v,label=below:$b$] {};
  \node (c) at (2,1.6)  [v,label=above:$c$] {};
  \node (d) at (2,-1.6) [v,label=below:$d$] {};
  \node (e) at (4,0)    [v,label=right:$e$] {};
  \node (f) at (3,2.4)  [v,label=above:$f$] {};
  \node (g) at (3,-2.4) [v,label=below:$g$] {};

  % edges with weights
  \draw[e] (s)--(a) node[midway,above left]{2};
  \draw[e] (s)--(b) node[midway,below left]{5};

  \draw[e] (a)--(b) node[midway,left]{2};

  \draw[e] (a)--(c) node[midway,below]{6};
  \draw[e] (a)--(d) node[pos=0.3, below]{3};

  \draw[e] (b)--(c) node[pos=0.3,above]{4};
  \draw[e] (b)--(d) node[midway,above]{6};

  \draw[e] (c)--(d) node[midway,right]{1}; 

  \draw[e] (c)--(e) node[midway,above]{7};
  \draw[e] (d)--(e) node[midway,above]{5};

  \draw[e] (c)--(f) node[midway,below]{3};
  \draw[e] (f)--(e) node[midway,above right]{4};

  \draw[e] (d)--(g) node[pos=0.6,above]{2};
  \draw[e] (g)--(e) node[midway,below]{4};

  \draw[e] (a)--(f) node[midway,above]{5};
  \draw[e] (b)--(g) node[midway,below]{4};
  \end{tikzpicture}
  \end{center}
}
\sol
\begin{center}
  \begin{tabular}{c|c|c|c|c|c|c|c|c}
    $s$ & $a$ & $b$ & $c$ & $d$ & $f$ & $g$ & $e$ & Vertex Added to $S$ \\ \hline
    $\boxed{0}$ & $(2,s)$ & $(5,s)$ & $\infty$ & $\infty$ & $\infty$ & $\infty$ & $\infty$ & $s$ \\
    \phantom{\boxed{(0,a)}} & $\boxed{(2,s)}$ & $(4,a)$ & $(8,a)$ & $(5,a)$ & $(7,a)$ & $\infty$ & $\infty$ & $a$ \\
      & & $\boxed{(4,a)}$ & $(8,a)$ & $(5,a)$ & $(7,a)$ & $(8,b)$ & $\infty$ & $b$ \\
      & & & $(6,d)$ & \boxed{(5,a)} & $(7,a)$ & $(7,d)$ & $(10,d)$ & $d$ \\
      & & & $\boxed{(6,d)}$ & & $(7,a)$ & $(7,d)$ & $(10,d)$ & $c$ \\
      & & & & & $\boxed{(7,a)}$ & $(7,d)$ & $(10,d)$ & $f$ \\
      & & & & & & $\boxed{(7,d)}$ & $(10,d)$ & $g$ \\
      & & & & & & & $\boxed{(10,d)}$ & $e$ \\
  \end{tabular}
\end{center}

From the final entry in the table, we can see that the shortest path from $s$ to $e$ has distance 10. Starting at the end, $e$, we can work our way back to $d$. Back to $a$. Then back to the starting point $s$. Therefore, the shortest path is
$$
  s \to a \to d \to e\quad\text{with}\quad d(s,e)=10.
$$

\begin{figure}[H]\begin{center}\begin{tikzpicture}[
  scale=1,
  v/.style={circle,fill,inner sep=1.6pt},
  e/.style={line width=0.9pt}
  ]
  % vertices
  \node (s) at (-2,0)   [v,label=left:$s$] {};
  \node (a) at (0,1.6)  [v,label=above:$a$] {};
  \node (b) at (0,-1.6) [v,label=below:$b$] {};
  \node (c) at (2,1.6)  [v,label=above:$c$] {};
  \node (d) at (2,-1.6) [v,label=below:$d$] {};
  \node (e) at (4,0)    [v,label=right:$e$] {};
  \node (f) at (3,2.4)  [v,label=above:$f$] {};
  \node (g) at (3,-2.4) [v,label=below:$g$] {};

  % edges with weights
  \draw[e] (s)--(a) node[midway,above left]{2};
  \draw[e] (s)--(b) node[midway,below left]{5};

  \draw[e] (a)--(b) node[midway,left]{2};

  \draw[e] (a)--(c) node[midway,below]{6};
  \draw[e] (a)--(d) node[pos=0.3, below]{3};

  \draw[e] (b)--(c) node[pos=0.3,above]{4};
  \draw[e] (b)--(d) node[midway,above]{6};

  \draw[e] (c)--(d) node[midway,right]{1}; 

  \draw[e] (c)--(e) node[midway,above]{7};
  \draw[e] (d)--(e) node[midway,above]{5};

  \draw[e] (c)--(f) node[midway,below]{3};
  \draw[e] (f)--(e) node[midway,above right]{4};

  \draw[e] (d)--(g) node[pos=0.6,above]{2};
  \draw[e] (g)--(e) node[midway,below]{4};

  \draw[e] (a)--(f) node[midway,above]{5};
  \draw[e] (b)--(g) node[midway,below]{4};

  \draw[red, ultra thick] (s) -- (a) -- (d) -- (e);
\end{tikzpicture}
\caption{An illustraion of the shortest path, $s\to a\to d\to e$ which has $\text{distance}=10$}
\end{center}\end{figure}



\newpage
\qs{10 marks}{
  Use Algorithm 33.113 from the notes to find a maximum weight perfect matching
  in the weighted complete bipartite graph $G$ with parts $V_1 = \{v_1,v_2,...,v_6\}$ and $V_2 =\{u_1,u_2,...,u_6\}$. The weight $w(v_iu_j)$ of the edge $v_iu_j$ is given by $w(v_iu_j)= m_{ij}$,  where $M =[m_{ij}]$
  is given by
  \[
    \begin{bmatrix}
    7 & 5 & 9 & 4 & 7 & 6\\
    6 & 9 & 9 & 7 & 5 & 5\\
    5 & 6 & 4 & 8 & 9 & 7\\
    8 & 7 & 5 & 6 & 8 & 4\\
    6 & 5 & 7 & 5 & 9 & 8\\
    7 & 6 & 6 & 9 & 5 & 9
    \end{bmatrix}.
  \]
}
\sol


\end{document}
