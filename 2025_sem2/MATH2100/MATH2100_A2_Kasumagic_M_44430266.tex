\documentclass[a4paper,11pt]{report}

\input{../../latex_template/preamble}
%From M275 "Topology" at SJSU
\newcommand{\id}{\mathrm{id}}
\newcommand{\taking}[1]{\xrightarrow{#1}}
\newcommand{\inv}{^{-1}}

%From M170 "Introduction to Graph Theory" at SJSU
\DeclareMathOperator{\diam}{diam}
\DeclareMathOperator{\ord}{ord}
\newcommand{\defeq}{\overset{\mathrm{def}}{=}}

%From the USAMO .tex files
\newcommand{\ts}{\textsuperscript}
\newcommand{\dg}{^\circ}
\newcommand{\ii}{\item}

% % From Math 55 and Math 145 at Harvard
% \newenvironment{subproof}[1][Proof]{%
% \begin{proof}[#1] \renewcommand{\qedsymbol}{$\blacksquare$}}%
% {\end{proof}}

\newcommand{\liff}{\leftrightarrow}
\newcommand{\lthen}{\rightarrow}
\newcommand{\opname}{\operatorname}
\newcommand{\surjto}{\twoheadrightarrow}
\newcommand{\injto}{\hookrightarrow}
\newcommand{\On}{\mathrm{On}} % ordinals
\DeclareMathOperator{\img}{im} % Image
\DeclareMathOperator{\Img}{Im} % Image
\DeclareMathOperator{\coker}{coker} % Cokernel
\DeclareMathOperator{\Coker}{Coker} % Cokernel
\DeclareMathOperator{\Ker}{Ker} % Kernel
\DeclareMathOperator{\rank}{rank}
\DeclareMathOperator{\Spec}{Spec} % spectrum
\DeclareMathOperator{\Tr}{Tr} % trace
\DeclareMathOperator{\pr}{pr} % projection
\DeclareMathOperator{\ext}{ext} % extension
\DeclareMathOperator{\pred}{pred} % predecessor
\DeclareMathOperator{\dom}{dom} % domain
\DeclareMathOperator{\ran}{ran} % range
\DeclareMathOperator{\Hom}{Hom} % homomorphism
\DeclareMathOperator{\Mor}{Mor} % morphisms
\DeclareMathOperator{\End}{End} % endomorphism

\newcommand{\eps}{\epsilon}
\newcommand{\veps}{\varepsilon}
\newcommand{\ol}{\overline}
\newcommand{\ul}{\underline}
\newcommand{\wt}{\widetilde}
\newcommand{\wh}{\widehat}
\newcommand{\vocab}[1]{\textbf{\color{blue} #1}}
\providecommand{\half}{\frac{1}{2}}
\newcommand{\dang}{\measuredangle} %% Directed angle
\newcommand{\ray}[1]{\overrightarrow{#1}}
\newcommand{\seg}[1]{\overline{#1}}
\newcommand{\arc}[1]{\wideparen{#1}}
\DeclareMathOperator{\cis}{cis}
\DeclareMathOperator*{\lcm}{lcm}
\DeclareMathOperator*{\argmin}{arg min}
\DeclareMathOperator*{\argmax}{arg max}
\newcommand{\cycsum}{\sum_{\mathrm{cyc}}}
\newcommand{\symsum}{\sum_{\mathrm{sym}}}
\newcommand{\cycprod}{\prod_{\mathrm{cyc}}}
\newcommand{\symprod}{\prod_{\mathrm{sym}}}
\newcommand{\Qed}{\begin{flushright}\qed\end{flushright}}
\newcommand{\parinn}{\setlength{\parindent}{1cm}}
\newcommand{\parinf}{\setlength{\parindent}{0cm}}
% \newcommand{\norm}{\|\cdot\|}
\newcommand{\inorm}{\norm_{\infty}}
\newcommand{\opensets}{\{V_{\alpha}\}_{\alpha\in I}}
\newcommand{\oset}{V_{\alpha}}
\newcommand{\opset}[1]{V_{\alpha_{#1}}}
\newcommand{\lub}{\text{lub}}
\newcommand{\del}[2]{\frac{\partial #1}{\partial #2}}
\newcommand{\Del}[3]{\frac{\partial^{#1} #2}{\partial^{#1} #3}}
\newcommand{\deld}[2]{\dfrac{\partial #1}{\partial #2}}
\newcommand{\Deld}[3]{\dfrac{\partial^{#1} #2}{\partial^{#1} #3}}
\newcommand{\lm}{\lambda}
\newcommand{\uin}{\mathbin{\rotatebox[origin=c]{90}{$\in$}}}
\newcommand{\usubset}{\mathbin{\rotatebox[origin=c]{90}{$\subset$}}}
\newcommand{\lt}{\left}
\newcommand{\rt}{\right}
\newcommand{\bs}[1]{\boldsymbol{#1}}
\newcommand{\exs}{\exists}
\newcommand{\st}{\strut}
\newcommand{\dps}[1]{\displaystyle{#1}}

\newcommand{\sol}{\setlength{\parindent}{0cm}\textbf{\textit{Solution:}}\setlength{\parindent}{1cm} }
\newcommand{\solve}[1]{\setlength{\parindent}{0cm}\textbf{\textit{Solution: }}\setlength{\parindent}{1cm}#1 \Qed}

\DeclareMathOperator{\sech}{sech}
\DeclareMathOperator{\csch}{csch}
\DeclareMathOperator{\arcsec}{arcsec}
\DeclareMathOperator{\arccsc}{arccsc}
\DeclareMathOperator{\arccot}{arccot}
\DeclareMathOperator{\arsinh}{arsinh}
\DeclareMathOperator{\arcosh}{arcosh}
\DeclareMathOperator{\artanh}{artanh}
\DeclareMathOperator{\arcsch}{arcsch}
\DeclareMathOperator{\arsech}{arsech}
\DeclareMathOperator{\arcoth}{arcoth}

\newcommand{\sinx}{\sin x}          \newcommand{\arcsinx}{\arcsin x}    
\newcommand{\cosx}{\cos x}          \newcommand{\arccosx}{\arccosx}
\newcommand{\tanx}{\tan x}          \newcommand{\arctanx}{\arctan x}
\newcommand{\cscx}{\csc x}          \newcommand{\arccscx}{\arccsc x}
\newcommand{\secx}{\sec x}          \newcommand{\arcsecx}{\arcsec x}
\newcommand{\cotx}{\cot x}          \newcommand{\arccotx}{\arccot x}
\newcommand{\sinhx}{\sinh x}          \newcommand{\arsinhx}{\arsinh x}
\newcommand{\coshx}{\cosh x}          \newcommand{\arcoshx}{\arcosh x}
\newcommand{\tanhx}{\tanh x}          \newcommand{\artanhx}{\artanh x}
\newcommand{\cschx}{\csch x}          \newcommand{\arcschx}{\arcsch x}
\newcommand{\sechx}{\sech x}          \newcommand{\arsechx}{\arsech x}
\newcommand{\cothx}{\coth x}          \newcommand{\arcothx}{\arcoth x}
\newcommand{\lnx}{\ln x}
\newcommand{\expx}{\exp x}

\newcommand{\bba}{\mathbb{A}}   \newcommand{\bbn}{\mathbb{N}}
\newcommand{\bbb}{\mathbb{B}}   \newcommand{\bbo}{\mathbb{O}}
\newcommand{\bbc}{\mathbb{C}}   \newcommand{\bbp}{\mathbb{P}}
\newcommand{\bbd}{\mathbb{D}}   \newcommand{\bbq}{\mathbb{Q}}
\newcommand{\bbe}{\mathbb{E}}   \newcommand{\bbr}{\mathbb{R}}
\newcommand{\bbf}{\mathbb{F}}   \newcommand{\bbs}{\mathbb{S}}
\newcommand{\bbg}{\mathbb{G}}   \newcommand{\bbt}{\mathbb{T}}
\newcommand{\bbh}{\mathbb{H}}   \newcommand{\bbu}{\mathbb{U}}
\newcommand{\bbi}{\mathbb{I}}    \newcommand{\bbv}{\mathbb{V}}
\newcommand{\bbj}{\mathbb{J}}   \newcommand{\bbw}{\mathbb{W}}
\newcommand{\bbk}{\mathbb{K}}   \newcommand{\bbx}{\mathbb{X}}
\newcommand{\bbl}{\mathbb{L}}    \newcommand{\bby}{\mathbb{Y}}
\newcommand{\bbm}{\mathbb{M}}   \newcommand{\bbz}{\mathbb{Z}}

\newcommand{\lb}{\left(}
\newcommand{\rb}{\right)}
\newcommand{\lbr}{\left\lbrace}
\newcommand{\rbr}{\right\rbrace}
\newcommand{\lsb}{\left[}
\newcommand{\rsb}{\right]}
\newcommand{\suchthat}{\medspace\middle|\medspace}
\newcommand{\bracks}[1]{\lb #1 \rb}
\newcommand{\braces}[1]{\lbr #1 \rbr}
\newcommand{\sqbracks}[1]{\lsb #1 \rsb}

\renewcommand{\floor}[1]{\lfloor #1 \rfloor}
\renewcommand{\ceil}[1]{\lceil #1 \rceil}

\newcommand{\cd}{\cdot}
\newcommand{\tf}{\therefore}
\newcommand{\Let}{\text{Let }}
\newcommand{\Given}{\text{Given }}
\newcommand{\Suppose}{\text{Suppose }}
\newcommand{\WeSee}{\text{We see }}
\newcommand{\So}{\text{So }}

\newcommand{\QED}{\hfill \qed}

\renewcommand{\dd}[1]{\frac{d}{d#1}}
\newcommand{\dyd}[2][y]{\frac{d#1}{d#2}}

\newcommand{\ddx}{\dd{x}}       \newcommand{\ddxsq}{\dyd[^2]{x^2}}
\newcommand{\ddy}{\dd{y}}       \newcommand{\ddysq}{\dyd[^2]{y^2}}
\newcommand{\ddu}{\dd{u}}       \newcommand{\ddusq}{\dyd[^2]{u^2}}
\newcommand{\ddv}{\dd{v}}       \newcommand{\ddvsq}{\dyd[^2]{v^2}}

\newcommand{\dydx}{\dyd{x}}     \newcommand{\dydxsq}{\dyd[^2y]{x^2}}
\newcommand{\dfdx}{\dyd[f]{x}}  \newcommand{\dfdxsq}{\dyd[^2f]{x^2}}
\newcommand{\dudx}{\dyd[u]{x}}  \newcommand{\dudxsq}{\dyd[^2u]{x^2}}
\newcommand{\dvdx}{\dyd[v]{x}}  \newcommand{\dvdxsq}{\dyd[^2v]{x^2}}

% Mathfrak primes
\newcommand{\km}{\mathfrak{m}}
\newcommand{\kp}{\mathfrak{p}}
\newcommand{\kq}{\mathfrak{q}}

%---------------------------------------
% Blackboard Math Fonts :-
%---------------------------------------
\newcommand{\bba}{\mathbb{A}}   \newcommand{\bbn}{\mathbb{N}}
\newcommand{\bbb}{\mathbb{B}}   \newcommand{\bbo}{\mathbb{O}}
\newcommand{\bbc}{\mathbb{C}}   \newcommand{\bbp}{\mathbb{P}}
\newcommand{\bbd}{\mathbb{D}}   \newcommand{\bbq}{\mathbb{Q}}
\newcommand{\bbe}{\mathbb{E}}   \newcommand{\bbr}{\mathbb{R}}
\newcommand{\bbf}{\mathbb{F}}   \newcommand{\bbs}{\mathbb{S}}
\newcommand{\bbg}{\mathbb{G}}   \newcommand{\bbt}{\mathbb{T}}
\newcommand{\bbh}{\mathbb{H}}   \newcommand{\bbu}{\mathbb{U}}
\newcommand{\bbi}{\mathbb{I}}   \newcommand{\bbv}{\mathbb{V}}
\newcommand{\bbj}{\mathbb{J}}   \newcommand{\bbw}{\mathbb{W}}
\newcommand{\bbk}{\mathbb{K}}   \newcommand{\bbx}{\mathbb{X}}
\newcommand{\bbl}{\mathbb{L}}   \newcommand{\bby}{\mathbb{Y}}
\newcommand{\bbm}{\mathbb{M}}   \newcommand{\bbz}{\mathbb{Z}}

%---------------------------------------
% Roman Math Fonts :-
%---------------------------------------
\newcommand{\rma}{\mathrm{A}}   \newcommand{\rmn}{\mathrm{N}}
\newcommand{\rmb}{\mathrm{B}}   \newcommand{\rmo}{\mathrm{O}}
\newcommand{\rmc}{\mathrm{C}}   \newcommand{\rmp}{\mathrm{P}}
\newcommand{\rmd}{\mathrm{D}}   \newcommand{\rmq}{\mathrm{Q}}
\newcommand{\rme}{\mathrm{E}}   \newcommand{\rmr}{\mathrm{R}}
\newcommand{\rmf}{\mathrm{F}}   \newcommand{\rms}{\mathrm{S}}
\newcommand{\rmg}{\mathrm{G}}   \newcommand{\rmt}{\mathrm{T}}
\newcommand{\rmh}{\mathrm{H}}   \newcommand{\rmu}{\mathrm{U}}
\newcommand{\rmi}{\mathrm{I}}   \newcommand{\rmv}{\mathrm{V}}
\newcommand{\rmj}{\mathrm{J}}   \newcommand{\rmw}{\mathrm{W}}
\newcommand{\rmk}{\mathrm{K}}   \newcommand{\rmx}{\mathrm{X}}
\newcommand{\rml}{\mathrm{L}}   \newcommand{\rmy}{\mathrm{Y}}
\newcommand{\rmm}{\mathrm{M}}   \newcommand{\rmz}{\mathrm{Z}}

%---------------------------------------
% Calligraphic Math Fonts :-
%---------------------------------------
\newcommand{\cla}{\mathcal{A}}   \newcommand{\cln}{\mathcal{N}}
\newcommand{\clb}{\mathcal{B}}   \newcommand{\clo}{\mathcal{O}}
\newcommand{\clc}{\mathcal{C}}   \newcommand{\clp}{\mathcal{P}}
\newcommand{\cld}{\mathcal{D}}   \newcommand{\clq}{\mathcal{Q}}
\newcommand{\cle}{\mathcal{E}}   \newcommand{\clr}{\mathcal{R}}
\newcommand{\clf}{\mathcal{F}}   \newcommand{\cls}{\mathcal{S}}
\newcommand{\clg}{\mathcal{G}}   \newcommand{\clt}{\mathcal{T}}
\newcommand{\clh}{\mathcal{H}}   \newcommand{\clu}{\mathcal{U}}
\newcommand{\cli}{\mathcal{I}}   \newcommand{\clv}{\mathcal{V}}
\newcommand{\clj}{\mathcal{J}}   \newcommand{\clw}{\mathcal{W}}
\newcommand{\clk}{\mathcal{K}}   \newcommand{\clx}{\mathcal{X}}
\newcommand{\cll}{\mathcal{L}}   \newcommand{\cly}{\mathcal{Y}}
\newcommand{\calm}{\mathcal{M}}  \newcommand{\clz}{\mathcal{Z}}

%---------------------------------------
% Fraktur  Math Fonts :-
%---------------------------------------
\newcommand{\fka}{\mathfrak{A}}   \newcommand{\fkn}{\mathfrak{N}}
\newcommand{\fkb}{\mathfrak{B}}   \newcommand{\fko}{\mathfrak{O}}
\newcommand{\fkc}{\mathfrak{C}}   \newcommand{\fkp}{\mathfrak{P}}
\newcommand{\fkd}{\mathfrak{D}}   \newcommand{\fkq}{\mathfrak{Q}}
\newcommand{\fke}{\mathfrak{E}}   \newcommand{\fkr}{\mathfrak{R}}
\newcommand{\fkf}{\mathfrak{F}}   \newcommand{\fks}{\mathfrak{S}}
\newcommand{\fkg}{\mathfrak{G}}   \newcommand{\fkt}{\mathfrak{T}}
\newcommand{\fkh}{\mathfrak{H}}   \newcommand{\fku}{\mathfrak{U}}
\newcommand{\fki}{\mathfrak{I}}   \newcommand{\fkv}{\mathfrak{V}}
\newcommand{\fkj}{\mathfrak{J}}   \newcommand{\fkw}{\mathfrak{W}}
\newcommand{\fkk}{\mathfrak{K}}   \newcommand{\fkx}{\mathfrak{X}}
\newcommand{\fkl}{\mathfrak{L}}   \newcommand{\fky}{\mathfrak{Y}}
\newcommand{\fkm}{\mathfrak{M}}   \newcommand{\fkz}{\mathfrak{Z}}


\usepackage{listings}
\usepackage{xcolor}

\definecolor{mathematicaPurple}{rgb}{0.36,0.2,0.5}
\definecolor{mathematicaBlue}{rgb}{0.25,0.35,0.75}
\definecolor{mathematicaGreen}{rgb}{0.0,0.5,0.0}

\lstdefinelanguage{Mathematica}{
  morekeywords={Plot, Solve, Table, Do, Module, If, True, False, Integrate,
    Sum, Limit, Sin, Cos, Exp, Log, Simplify, FullSimplify, Eigenvalues,
    Eigenvectors, D, DSolve, NDSolve, Piecewise},
  sensitive=true,
  morecomment=[l](*),
  morestring=[b]"
}

\lstset{
  language=Mathematica,
  basicstyle=\ttfamily,
  keywordstyle=\color{mathematicaBlue}\bfseries,
  stringstyle=\color{mathematicaGreen},
  commentstyle=\color{mathematicaPurple}\itshape,
  showstringspaces=false,
  breaklines=true,
  columns=flexible,
}

\begin{document}
\begin{center}
{\bf School of Mathematics and Physics, UQ}
\end{center}
\begin{center}
	{\large\bf MATH2100 Applied Mathematical Analysis \\ Semester 2 2025 \\ Problem Set 2} \\ \vspace{1em}
	Michael Kasumagic, 44302669 \\
	Professor Ivana Carrizo Molina \\
	Due 1pm Monday 8 September 2025
\end{center}

\qs{27 marks}{
  The Duffling equation $\ddot{x} - c\dot{x} -4x + x^3 = 0$, with damping constant $c$, models the motin of a mechanical system in a twin-well potential field.
  \begin{enumerate}[label=(\alph*), itemsep=-3pt, topsep=0pt]
    \item Rewrite this second order ODE as a system of first order ODEs.
    \item For $c=0$
    \begin{enumerate}[label=(\roman*), itemsep=-3pt, topsep=0pt]
      \item Find all critical points of the system.
      \item Use linearisation to classify the nature and stability of the critical points
      \item Use the chain rule to solve for the phase curves $y_1 = y_2(y_1)$.
      \item Does the linearised system exhibit qualitatively similar behaviour as the nonlinear?
    \end{enumerate}
    \item (Mathematica) Verify results from part (b)(ii), (iii), and (iv) by plotting a vector field.
    \item For $c=2$
    \begin{enumerate}[label=(\roman*), itemsep=-3pt, topsep=0pt]
      \item Find all critical points of the system.
      \item Calculate the linearised system about each critical point. Classify their nature and stability. If a critical point is a saddle or a node, identify the relevant eigenvalues and eigenvector. If a critical point is a spiral or a centre, describe the direction of rotation, vertical nullclines, and direction field along the axes.
      \item Does the linearised system exhibit qualitatively similar behaviour to the nonlinear?
      \item Sketch a phase portrait for the non-linear system, by hand. Clearly identify each of the critical points and determine the heteroclinic orbit.
    \end{enumerate}
  \end{enumerate}
}
\sol (a)


\newpage
\qs{6 marks}{
  Consider the non-linear system of ODEs
  $$
    \begin{array}{l}
      y_1' = f_1'(y_1,y_2) = -2y_1(3-y_2-y_1) \\
      y_2' = f_2'(y_1,y_2) = y_2(-4+y_1+2y_2)      
    \end{array}
  $$
  \begin{enumerate}[label=(\alph*), itemsep=-3pt, topsep=0pt]
    \item (Mathematica) Find all critical points of the system.
    \item (Mathematica) Calculate the linearised system about each critical point. Classify their type and stability.
    \item (Mathematica) Find the nullcines of the system.
    \item (Mathematica) Sketch a phase portrait for the non-linear system. Clearly identify each critical point and nullcine.
  \end{enumerate}
}
\sol (a)


\newpage
\qs{7 marks}{
  \begin{enumerate}[label=(\alph*), itemsep=-3pt, topsep=0pt]
    \item (Mathematica) Solve the first order DE
    $$
      \dyd{t} + \frac{2}{t}y(t) = \frac{1}{t^2}
    $$
    \item Consider the IVP 
    $$
      ty'' + 2ty' - 2y = -2,\quad y(0) = 1,\qquad y'(0) = 2.
    $$
    \begin{enumerate}[label=(\roman*), itemsep=-3pt, topsep=0pt]
      \item Apply the Laplace transform to solve for $Y(s)$.
      \item Apply the inverse Laplace transform to $Y(s)$, yielding the solution $y(t)$.
    \end{enumerate}
  \end{enumerate}
}
\sol (a)
\begin{lstlisting}[language=Mathematica]
In[1]   :=  DSolve[a'[t] + (2/t) a[t] == 1/t^2, a[t], t]
Out[1]  :=  {{a[t] -> 1/t + C[1]/t^2}}
\end{lstlisting}
Hence,
$$
  y(t) = \frac{1}{t} + \frac{c}{t^2}
$$

\sol (b)(i) \\
We'll apply the Laplace transform, then solve for $\laplace{y(t)}(s) =: Y(s)$
\begin{align*}
  \laplace{ty'' + 2ty' - 2y}(s) &= \laplace{-2}(s)
  \intertext{First we'll apply linearity,}
  \laplace{ty''}(s) + 2\laplace{ty'}(s) - 2\laplace{y}(s) &= -2\laplace{1}(s)
  \intertext{From the table we have the identities}
  \ignorealign{\laplace{tf(t)} = -\dd{s}\laplace{f(t)}(s) = -\dyd[F]{s} =: -F'(s)} \\
  \ignorealign{\laplace{f^{(n)}(t)}(s) = s^nF(s) - \sum_{i=0}^{n-1} s^if^{(n-i)}(0)} \\
  \intertext{From left to right...}
  \laplace{ty''}(s) &= -\dyd{s} \bracks{s^2Y(s) - sy(0) - y'(0)} \\
    &= -\dyd{s} \bracks{s^2Y(s) - s - 2} \\
    &= -2sY(s) - s^2Y'(s) + 1 + 0  \\
    &= -2sY(s) - s^2Y'(s) + 1  \\
  \laplace{ty'}(s) &= -\dyd{s}\bracks{sY(s) + y(0)} \\
    &= -\dyd{s}\bracks{sY(s) + 1} \\
    &= -Y(s) - sY'(s) \\
  \laplace{y}(s) &= Y \\
  \laplace{1}(s) &= \frac{1}{s} \\
  \intertext{After substituting these values and doing some simplifying, we find that}
  Y'(s^2 + 2s) + Y(2s - 4)  &= 1 + \frac{2}{s}.
\end{align*}


\newpage
\qs{10 marks}{
  \begin{enumerate}[label=(\alph*), itemsep=-3pt, topsep=0pt]
    \item Find the Laplace transform of $f(t) = t^5e^{-4t}-7\sin(6t)$.
    \item Let $f(t)$ satisfy the following identity
    $$
      f(t) = (t-1)^2u(t-1) + \int_{0}^{t} f(\tau)\sin(t-\tau)\d\tau.
    $$
    Find the Laplace transform of $f(t)$.
    \item (Mathematica) Use the Laplace-transforms method to solve the IVP
    $$
      \begin{cases}
        y_1'(t) = 3y_1(t) - 4y_2(t),\\
        y_2'(t) = 3y_2(t) - 4y_1(t)
      \end{cases}
    $$
    with initial conditions, $y_1(0)=0$ and $y_2(0)=1$.
    \item (Mathematica) Use \texttt{DSolve} to solve the same IVP, and show the results are equal.
  \end{enumerate}
}
\sol (a) \\
We seek the Laplace transform of $f(t) = t^5e^{-4t}-7\sin(6t)$,
\begin{align*}
  \laplace{f(t)}(s) &= \laplace{t^5e^{-4t}-7\sin(6t)}(s). \\
  \intertext{The Laplace transform is a linear operator, hence}
    &= \laplace{t^5e^{-4t}}(s) + \laplace{-7\sin(6t)}(s) \\
    &= \laplace{t^5e^{-4t}}(s) - 7\laplace{\sin(6t)}(s)
  \intertext{We'll use the table to evaluate these,}
  \ignorealign{\laplace{e^{\alpha t}g(t)}(s) = \laplace{g(t)}(s-\alpha)\qquad \laplace{t^n}(s) = \frac{n!}{s^{n+1}}}
  \intertext{Here, $\alpha=-4,~ g(t)=t^5$, and $n=5$. Therefore}
  \ignorealign{\laplace{t^5e^{-4t}}(s) = \laplace{t^5}(s+4) = \frac{5!}{(s+4)^{s+1}} = \frac{120}{(s+4)^6}}
  \intertext{Now we'll evaluate the other using the table.}
  \laplace{\sin(\alpha t)}(s) &= \frac{\alpha}{s^2 + \alpha^2}
  \intertext{Here, $\alpha=6$, therefore,}
  \ignorealign{\laplace{\sin(6t)}(s) = \frac{6}{s^2 + 6^2} = \frac{6}{s^2 + 36}}
  \intertext{Hence, our transformation comes to}
  \laplace{f(t)}(s) &= \laplace{t^5e^{-4t}}(s) - 7\laplace{\sin(6t)}(s) \\
    &= \frac{120}{(s+4)^6} - 7\cd\frac{6}{s^2 + 36} \\
    &= \frac{120}{(s+4)^6} - \frac{42}{s^2 + 36}
\end{align*}

\newpage
\sol (b) \\
We seek the Laplace transform of 
\begin{align*}
  f(t) &= (t-1)^2u(t-1) + \int_{0}^{t} f(\tau)\sin(t-\tau)\d\tau, \\
  \laplace{f(t)}(s) &= \laplace{(t-1)^2u(t-1) + \int_{0}^{t} f(\tau)\sin(t-\tau)\d\tau}(s).
  \intertext{First, we'll use the linearity of the Laplace transform operator,}
  \laplace{f(t)}(s) &= \laplace{(t-1)^2u(t-1)}(s) + \laplace{\int_{0}^{t} f(\tau)\sin(t-\tau)\d\tau}(s).
  \intertext{Second shifting theorem states}
  \laplace{f(t-k)u(t-k)}(s) &= e^{-ks}\laplace{f(t)}(s).
  \intertext{In our case $k=1$, and $f(t)=t^2$, hence}
  \ignorealign{\laplace{(t-1)^2u(t-1)}(s) = e^{-1s}\laplace{t^2}(s) = e^{-s}\cd\frac{2!}{s^{2+1}} = \frac{2e^{-s}}{s^{3}}}
  \intertext{Convolution theorem states}
  \laplace{\int_{0}^{t} f(\tau)g(t-\tau)\d\tau}(s) &= \laplace{f(t)}(s)\cd\laplace{g(t)}(s).
  \intertext{We have $f(\tau)$ and $g(t)=\sin(t)$. Hence,}
  \laplace{\int_{0}^{t} f(\tau)\sin(t-\tau)\d\tau}(s) &= \laplace{f(t)}(s)\cd\laplace{\sin t}(s) \\
    &= \laplace{f(t)}(s)\cd\frac{1}{s^2+1^2}
  \intertext{Let $F(s) := \laplace{f(t)}(s)$. Hence, we have}
  \laplace{f(t){(s)}} = F(s) &= \frac{2e^{-s}}{s^{3}} + F(s)\cd\frac{1}{s^2+1} \\
  \implies F(s)\bracks{1 - \frac{1}{s^2+1}} &= \frac{2e^{-s}}{s^{3}} \\
  \tf F(s) &= \frac{2e^{-s}}{s^{3}} \div \bracks{1 - \frac{1}{s^2+1}} \\
    &= \frac{2e^{-s}}{s^{3}}\bracks{\frac{s^2+1}{s^2}} \\
  \intertext{Therefore, an arbitrary function defined by the identify}
  f(t) &= (t-1)^2u(t-1) + \int_{0}^{t} f(\tau)\sin(t-\tau)\d\tau
  \intertext{has Laplace transform}
  \laplace{f(t)}(s) &= \frac{2e^{-s}(s^2+1)}{s^{5}}.
\end{align*}

\newpage
\sol (c)(i)
\begin{lstlisting}[language=Mathematica]
In[1]   :=  A = {{3, -4}, {-4, 3}};
            y0 = {0, 1};
            LapY = Inverse[s*IdentityMatrix[2] - A] . y0;
            y = InverseLaplaceTransform[LapY, s, t]
Out[4]  := {-(1/2) Exp[-t] (-1 + Exp[8 t]), 1/2 Exp[-t] (1 + Exp[8 t])}
\end{lstlisting}

First we set the matrix \texttt{A} to the coefficents, as they were given in the question, namely
$$
  \mathbf{Y}'(t) = A\mathbf{Y}(t) \iff \begin{pmatrix} y_1'(t) \\ y_2'(t) \end{pmatrix} = \begin{pmatrix} 3 & -4 \\ -4 & 3 \end{pmatrix} \begin{pmatrix} y_1(t) \\ y_2(t) \end{pmatrix}^\rmt \implies A = \begin{pmatrix} 3 & -4 \\ -4 & 3 \end{pmatrix}.
$$
Next we set a vector to represent the intial condition, \texttt{y0}, by
$$
  \textbf{Y}(0) = \begin{pmatrix} y_1 (0) \\ y_2(0) \end{pmatrix} = \begin{pmatrix} 0 \\ 1 \end{pmatrix}
$$
Applying the Laplace transform to
$$ 
  \mathbf Y'(t)=A\mathbf Y(t)
$$
with $\mathbf Y(0)=\mathbf y_0$ yields
$$
  s\mathbf Y(s)-\mathbf y_0=A\mathbf Y(s),
$$
hence
$$
  (sI-A)\mathbf{Y}(s)=\mathbf y_0\quad \text{and}\quad \mathbf{Y}(s)=(sI-A)^{-1}\mathbf y_0.
$$
The inverse Laplace transform gets us our desired result! \\

\sol (c)(ii)
\begin{lstlisting}[language=Mathematica]
In[1]   :=  eqs = {y1'[t] == 3 y1[t] - 4 y2[t], y2'[t] == 3 y2[t] - 4 y1[t], 
              y1[0] == 0, y2[0] == 1};             
In[2]   :=  solDS = DSolve[eqs, {y1, y2}, t][[1]];
In[3]   :=  solLap = {y1[t] -> -(1/2) Exp[-t] (-1 + Exp[8 t]),
              y2[t] -> 1/2 Exp[-t] (1 + Exp[8 t])};
In[4]   :=  ({y1[t], y2[t]} /. solDS) == ({y1[t], y2[t]} /. solLap)
Out[4]  :=  True
\end{lstlisting}
We set the system of DEs, along with the initial values in the \texttt{eqs} variable. We then solve it numerically and store the first solution it found in \texttt{solDS}. Next we store the Laplace solution from the previous question in \texttt{solLap}. Finally, we compare the solutions, and find that they are equal.

\end{document}