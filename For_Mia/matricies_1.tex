\documentclass[12pt]{report}

\input{../latex_template/preamble}
%From M275 "Topology" at SJSU
\newcommand{\id}{\mathrm{id}}
\newcommand{\taking}[1]{\xrightarrow{#1}}
\newcommand{\inv}{^{-1}}

%From M170 "Introduction to Graph Theory" at SJSU
\DeclareMathOperator{\diam}{diam}
\DeclareMathOperator{\ord}{ord}
\newcommand{\defeq}{\overset{\mathrm{def}}{=}}

%From the USAMO .tex files
\newcommand{\ts}{\textsuperscript}
\newcommand{\dg}{^\circ}
\newcommand{\ii}{\item}

% % From Math 55 and Math 145 at Harvard
% \newenvironment{subproof}[1][Proof]{%
% \begin{proof}[#1] \renewcommand{\qedsymbol}{$\blacksquare$}}%
% {\end{proof}}

\newcommand{\liff}{\leftrightarrow}
\newcommand{\lthen}{\rightarrow}
\newcommand{\opname}{\operatorname}
\newcommand{\surjto}{\twoheadrightarrow}
\newcommand{\injto}{\hookrightarrow}
\newcommand{\On}{\mathrm{On}} % ordinals
\DeclareMathOperator{\img}{im} % Image
\DeclareMathOperator{\Img}{Im} % Image
\DeclareMathOperator{\coker}{coker} % Cokernel
\DeclareMathOperator{\Coker}{Coker} % Cokernel
\DeclareMathOperator{\Ker}{Ker} % Kernel
\DeclareMathOperator{\rank}{rank}
\DeclareMathOperator{\Spec}{Spec} % spectrum
\DeclareMathOperator{\Tr}{Tr} % trace
\DeclareMathOperator{\pr}{pr} % projection
\DeclareMathOperator{\ext}{ext} % extension
\DeclareMathOperator{\pred}{pred} % predecessor
\DeclareMathOperator{\dom}{dom} % domain
\DeclareMathOperator{\ran}{ran} % range
\DeclareMathOperator{\Hom}{Hom} % homomorphism
\DeclareMathOperator{\Mor}{Mor} % morphisms
\DeclareMathOperator{\End}{End} % endomorphism

\newcommand{\eps}{\epsilon}
\newcommand{\veps}{\varepsilon}
\newcommand{\ol}{\overline}
\newcommand{\ul}{\underline}
\newcommand{\wt}{\widetilde}
\newcommand{\wh}{\widehat}
\newcommand{\vocab}[1]{\textbf{\color{blue} #1}}
\providecommand{\half}{\frac{1}{2}}
\newcommand{\dang}{\measuredangle} %% Directed angle
\newcommand{\ray}[1]{\overrightarrow{#1}}
\newcommand{\seg}[1]{\overline{#1}}
\newcommand{\arc}[1]{\wideparen{#1}}
\DeclareMathOperator{\cis}{cis}
\DeclareMathOperator*{\lcm}{lcm}
\DeclareMathOperator*{\argmin}{arg min}
\DeclareMathOperator*{\argmax}{arg max}
\newcommand{\cycsum}{\sum_{\mathrm{cyc}}}
\newcommand{\symsum}{\sum_{\mathrm{sym}}}
\newcommand{\cycprod}{\prod_{\mathrm{cyc}}}
\newcommand{\symprod}{\prod_{\mathrm{sym}}}
\newcommand{\Qed}{\begin{flushright}\qed\end{flushright}}
\newcommand{\parinn}{\setlength{\parindent}{1cm}}
\newcommand{\parinf}{\setlength{\parindent}{0cm}}
% \newcommand{\norm}{\|\cdot\|}
\newcommand{\inorm}{\norm_{\infty}}
\newcommand{\opensets}{\{V_{\alpha}\}_{\alpha\in I}}
\newcommand{\oset}{V_{\alpha}}
\newcommand{\opset}[1]{V_{\alpha_{#1}}}
\newcommand{\lub}{\text{lub}}
\newcommand{\del}[2]{\frac{\partial #1}{\partial #2}}
\newcommand{\Del}[3]{\frac{\partial^{#1} #2}{\partial^{#1} #3}}
\newcommand{\deld}[2]{\dfrac{\partial #1}{\partial #2}}
\newcommand{\Deld}[3]{\dfrac{\partial^{#1} #2}{\partial^{#1} #3}}
\newcommand{\lm}{\lambda}
\newcommand{\uin}{\mathbin{\rotatebox[origin=c]{90}{$\in$}}}
\newcommand{\usubset}{\mathbin{\rotatebox[origin=c]{90}{$\subset$}}}
\newcommand{\lt}{\left}
\newcommand{\rt}{\right}
\newcommand{\bs}[1]{\boldsymbol{#1}}
\newcommand{\exs}{\exists}
\newcommand{\st}{\strut}
\newcommand{\dps}[1]{\displaystyle{#1}}

\newcommand{\sol}{\setlength{\parindent}{0cm}\textbf{\textit{Solution:}}\setlength{\parindent}{1cm} }
\newcommand{\solve}[1]{\setlength{\parindent}{0cm}\textbf{\textit{Solution: }}\setlength{\parindent}{1cm}#1 \Qed}

\DeclareMathOperator{\sech}{sech}
\DeclareMathOperator{\csch}{csch}
\DeclareMathOperator{\arcsec}{arcsec}
\DeclareMathOperator{\arccsc}{arccsc}
\DeclareMathOperator{\arccot}{arccot}
\DeclareMathOperator{\arsinh}{arsinh}
\DeclareMathOperator{\arcosh}{arcosh}
\DeclareMathOperator{\artanh}{artanh}
\DeclareMathOperator{\arcsch}{arcsch}
\DeclareMathOperator{\arsech}{arsech}
\DeclareMathOperator{\arcoth}{arcoth}

\newcommand{\sinx}{\sin x}          \newcommand{\arcsinx}{\arcsin x}    
\newcommand{\cosx}{\cos x}          \newcommand{\arccosx}{\arccosx}
\newcommand{\tanx}{\tan x}          \newcommand{\arctanx}{\arctan x}
\newcommand{\cscx}{\csc x}          \newcommand{\arccscx}{\arccsc x}
\newcommand{\secx}{\sec x}          \newcommand{\arcsecx}{\arcsec x}
\newcommand{\cotx}{\cot x}          \newcommand{\arccotx}{\arccot x}
\newcommand{\sinhx}{\sinh x}          \newcommand{\arsinhx}{\arsinh x}
\newcommand{\coshx}{\cosh x}          \newcommand{\arcoshx}{\arcosh x}
\newcommand{\tanhx}{\tanh x}          \newcommand{\artanhx}{\artanh x}
\newcommand{\cschx}{\csch x}          \newcommand{\arcschx}{\arcsch x}
\newcommand{\sechx}{\sech x}          \newcommand{\arsechx}{\arsech x}
\newcommand{\cothx}{\coth x}          \newcommand{\arcothx}{\arcoth x}
\newcommand{\lnx}{\ln x}
\newcommand{\expx}{\exp x}

\newcommand{\bba}{\mathbb{A}}   \newcommand{\bbn}{\mathbb{N}}
\newcommand{\bbb}{\mathbb{B}}   \newcommand{\bbo}{\mathbb{O}}
\newcommand{\bbc}{\mathbb{C}}   \newcommand{\bbp}{\mathbb{P}}
\newcommand{\bbd}{\mathbb{D}}   \newcommand{\bbq}{\mathbb{Q}}
\newcommand{\bbe}{\mathbb{E}}   \newcommand{\bbr}{\mathbb{R}}
\newcommand{\bbf}{\mathbb{F}}   \newcommand{\bbs}{\mathbb{S}}
\newcommand{\bbg}{\mathbb{G}}   \newcommand{\bbt}{\mathbb{T}}
\newcommand{\bbh}{\mathbb{H}}   \newcommand{\bbu}{\mathbb{U}}
\newcommand{\bbi}{\mathbb{I}}    \newcommand{\bbv}{\mathbb{V}}
\newcommand{\bbj}{\mathbb{J}}   \newcommand{\bbw}{\mathbb{W}}
\newcommand{\bbk}{\mathbb{K}}   \newcommand{\bbx}{\mathbb{X}}
\newcommand{\bbl}{\mathbb{L}}    \newcommand{\bby}{\mathbb{Y}}
\newcommand{\bbm}{\mathbb{M}}   \newcommand{\bbz}{\mathbb{Z}}

\newcommand{\lb}{\left(}
\newcommand{\rb}{\right)}
\newcommand{\lbr}{\left\lbrace}
\newcommand{\rbr}{\right\rbrace}
\newcommand{\lsb}{\left[}
\newcommand{\rsb}{\right]}
\newcommand{\suchthat}{\medspace\middle|\medspace}
\newcommand{\bracks}[1]{\lb #1 \rb}
\newcommand{\braces}[1]{\lbr #1 \rbr}
\newcommand{\sqbracks}[1]{\lsb #1 \rsb}

\renewcommand{\floor}[1]{\lfloor #1 \rfloor}
\renewcommand{\ceil}[1]{\lceil #1 \rceil}

\newcommand{\cd}{\cdot}
\newcommand{\tf}{\therefore}
\newcommand{\Let}{\text{Let }}
\newcommand{\Given}{\text{Given }}
\newcommand{\Suppose}{\text{Suppose }}
\newcommand{\WeSee}{\text{We see }}
\newcommand{\So}{\text{So }}

\newcommand{\QED}{\hfill \qed}

\renewcommand{\dd}[1]{\frac{d}{d#1}}
\newcommand{\dyd}[2][y]{\frac{d#1}{d#2}}

\newcommand{\ddx}{\dd{x}}       \newcommand{\ddxsq}{\dyd[^2]{x^2}}
\newcommand{\ddy}{\dd{y}}       \newcommand{\ddysq}{\dyd[^2]{y^2}}
\newcommand{\ddu}{\dd{u}}       \newcommand{\ddusq}{\dyd[^2]{u^2}}
\newcommand{\ddv}{\dd{v}}       \newcommand{\ddvsq}{\dyd[^2]{v^2}}

\newcommand{\dydx}{\dyd{x}}     \newcommand{\dydxsq}{\dyd[^2y]{x^2}}
\newcommand{\dfdx}{\dyd[f]{x}}  \newcommand{\dfdxsq}{\dyd[^2f]{x^2}}
\newcommand{\dudx}{\dyd[u]{x}}  \newcommand{\dudxsq}{\dyd[^2u]{x^2}}
\newcommand{\dvdx}{\dyd[v]{x}}  \newcommand{\dvdxsq}{\dyd[^2v]{x^2}}

% Mathfrak primes
\newcommand{\km}{\mathfrak{m}}
\newcommand{\kp}{\mathfrak{p}}
\newcommand{\kq}{\mathfrak{q}}

%---------------------------------------
% Blackboard Math Fonts :-
%---------------------------------------
\newcommand{\bba}{\mathbb{A}}   \newcommand{\bbn}{\mathbb{N}}
\newcommand{\bbb}{\mathbb{B}}   \newcommand{\bbo}{\mathbb{O}}
\newcommand{\bbc}{\mathbb{C}}   \newcommand{\bbp}{\mathbb{P}}
\newcommand{\bbd}{\mathbb{D}}   \newcommand{\bbq}{\mathbb{Q}}
\newcommand{\bbe}{\mathbb{E}}   \newcommand{\bbr}{\mathbb{R}}
\newcommand{\bbf}{\mathbb{F}}   \newcommand{\bbs}{\mathbb{S}}
\newcommand{\bbg}{\mathbb{G}}   \newcommand{\bbt}{\mathbb{T}}
\newcommand{\bbh}{\mathbb{H}}   \newcommand{\bbu}{\mathbb{U}}
\newcommand{\bbi}{\mathbb{I}}   \newcommand{\bbv}{\mathbb{V}}
\newcommand{\bbj}{\mathbb{J}}   \newcommand{\bbw}{\mathbb{W}}
\newcommand{\bbk}{\mathbb{K}}   \newcommand{\bbx}{\mathbb{X}}
\newcommand{\bbl}{\mathbb{L}}   \newcommand{\bby}{\mathbb{Y}}
\newcommand{\bbm}{\mathbb{M}}   \newcommand{\bbz}{\mathbb{Z}}

%---------------------------------------
% Roman Math Fonts :-
%---------------------------------------
\newcommand{\rma}{\mathrm{A}}   \newcommand{\rmn}{\mathrm{N}}
\newcommand{\rmb}{\mathrm{B}}   \newcommand{\rmo}{\mathrm{O}}
\newcommand{\rmc}{\mathrm{C}}   \newcommand{\rmp}{\mathrm{P}}
\newcommand{\rmd}{\mathrm{D}}   \newcommand{\rmq}{\mathrm{Q}}
\newcommand{\rme}{\mathrm{E}}   \newcommand{\rmr}{\mathrm{R}}
\newcommand{\rmf}{\mathrm{F}}   \newcommand{\rms}{\mathrm{S}}
\newcommand{\rmg}{\mathrm{G}}   \newcommand{\rmt}{\mathrm{T}}
\newcommand{\rmh}{\mathrm{H}}   \newcommand{\rmu}{\mathrm{U}}
\newcommand{\rmi}{\mathrm{I}}   \newcommand{\rmv}{\mathrm{V}}
\newcommand{\rmj}{\mathrm{J}}   \newcommand{\rmw}{\mathrm{W}}
\newcommand{\rmk}{\mathrm{K}}   \newcommand{\rmx}{\mathrm{X}}
\newcommand{\rml}{\mathrm{L}}   \newcommand{\rmy}{\mathrm{Y}}
\newcommand{\rmm}{\mathrm{M}}   \newcommand{\rmz}{\mathrm{Z}}

%---------------------------------------
% Calligraphic Math Fonts :-
%---------------------------------------
\newcommand{\cla}{\mathcal{A}}   \newcommand{\cln}{\mathcal{N}}
\newcommand{\clb}{\mathcal{B}}   \newcommand{\clo}{\mathcal{O}}
\newcommand{\clc}{\mathcal{C}}   \newcommand{\clp}{\mathcal{P}}
\newcommand{\cld}{\mathcal{D}}   \newcommand{\clq}{\mathcal{Q}}
\newcommand{\cle}{\mathcal{E}}   \newcommand{\clr}{\mathcal{R}}
\newcommand{\clf}{\mathcal{F}}   \newcommand{\cls}{\mathcal{S}}
\newcommand{\clg}{\mathcal{G}}   \newcommand{\clt}{\mathcal{T}}
\newcommand{\clh}{\mathcal{H}}   \newcommand{\clu}{\mathcal{U}}
\newcommand{\cli}{\mathcal{I}}   \newcommand{\clv}{\mathcal{V}}
\newcommand{\clj}{\mathcal{J}}   \newcommand{\clw}{\mathcal{W}}
\newcommand{\clk}{\mathcal{K}}   \newcommand{\clx}{\mathcal{X}}
\newcommand{\cll}{\mathcal{L}}   \newcommand{\cly}{\mathcal{Y}}
\newcommand{\calm}{\mathcal{M}}  \newcommand{\clz}{\mathcal{Z}}

%---------------------------------------
% Fraktur  Math Fonts :-
%---------------------------------------
\newcommand{\fka}{\mathfrak{A}}   \newcommand{\fkn}{\mathfrak{N}}
\newcommand{\fkb}{\mathfrak{B}}   \newcommand{\fko}{\mathfrak{O}}
\newcommand{\fkc}{\mathfrak{C}}   \newcommand{\fkp}{\mathfrak{P}}
\newcommand{\fkd}{\mathfrak{D}}   \newcommand{\fkq}{\mathfrak{Q}}
\newcommand{\fke}{\mathfrak{E}}   \newcommand{\fkr}{\mathfrak{R}}
\newcommand{\fkf}{\mathfrak{F}}   \newcommand{\fks}{\mathfrak{S}}
\newcommand{\fkg}{\mathfrak{G}}   \newcommand{\fkt}{\mathfrak{T}}
\newcommand{\fkh}{\mathfrak{H}}   \newcommand{\fku}{\mathfrak{U}}
\newcommand{\fki}{\mathfrak{I}}   \newcommand{\fkv}{\mathfrak{V}}
\newcommand{\fkj}{\mathfrak{J}}   \newcommand{\fkw}{\mathfrak{W}}
\newcommand{\fkk}{\mathfrak{K}}   \newcommand{\fkx}{\mathfrak{X}}
\newcommand{\fkl}{\mathfrak{L}}   \newcommand{\fky}{\mathfrak{Y}}
\newcommand{\fkm}{\mathfrak{M}}   \newcommand{\fkz}{\mathfrak{Z}}

\usepackage{kbordermatrix}
\renewcommand{\kbldelim}{[}% Left delimiter
\renewcommand{\kbrdelim}{]}% Right delimiter

\title{\Huge{General Mathematics 2}\\Topic 3: Matrices}
\author{\huge{Michael Kasumagic}}
\date{\huge{Term 3, 2025}}

\begin{document}

\maketitle
\newpage% or \cleardouble \pd
% \pdfbookmark[<level>]{<title>}{<dest>}
\pdfbookmark[section]{\contentsname}{toc}
\tableofcontents
\pagebreak

\chapter{Matrix Basics}
\section{Introduction}
A matrix is an array or group of numbers. We display them in columns and rows, and surround them in square brackets. We typically denote them with capital letters, $A$, $B$, etc. 

\begin{equation}
	A = \begin{bmatrix} 5 \end{bmatrix} \qquad
	B = \begin{bmatrix} 1 & 4 \\ 9 & 7 \end{bmatrix} \qquad
	C = \begin{bmatrix} 4 & 5 \\ 2 & 9 \\ 6 & 2	\end{bmatrix} \qquad
	D = \begin{bmatrix} 1 & 3 & 3 & 1 \\ 2 & 6 & 6 & 2 \\ 1 & 0 & 0 & 8 \\ 1 & 7 & 7 & 6	\end{bmatrix} \qquad
	E = \begin{bmatrix} 5 & 2 & 8 & 6 & 3 & 7 \\ 4 & 1 & 6 & 5 & 8 & 3 \\ 5 & 7 & 3 & 5 & 6 & 9	\end{bmatrix}
	\label{eq:MatrixSizeExamples}
\end{equation}

\section{Identifying Matrix Sizes}
The first property that we're going to understand is the order of a matrix, or its size. This will become increasingly important as we learn more about matrices.
\dfn{Matrix Size/Order}{
	We say an $m{\times}n$ matrix has $m$ rows and $n$ columns, or has order $m$ by $n$. \\ 
	
	Sadly, this is something we just need to memorise; there's no ``reason'' for this convetion, it's just an arbitary convention. Just memorise $m{\times}n$, rows by columns.
}
\ex{Order of a Matrix}{
	$$
		\begin{bmatrix} 7 & 8 & 9 \\ 5 & 6 & 7 \end{bmatrix}
	$$
	I can count the individual numbers, and find that this matrix has 2 rows. \\
	I can count the individual numbers, and find that this matrix has 3 columns. \\
	$$
		\kbordermatrix{
			  & 1 & 2 & 3 \\
			1 & 7 & 8 & 9 \\
			2 & 5 & 6 & 7 
	  }
	$$
	Therefore this matrix has order $2{\times}3$.
}
\qs{Name those orders!}{
	What are sizes of the matrices I've drawn in \eqref{eq:MatrixSizeExamples}?
}
\nt{
	It's good to note here that an matrix can be arbitarily large. You could imagine a company might manipulate a matrix which contains some data for each of it's 10 million customers.
}

\section{Identifying Matrix Elements}
Next we learn how to denote and refer to specific elements from a given matrix.
\dfn{Matrix Elements}{
	An element of a matrix is a specific entry in a matrix. \\

	Suppose you have a matrix $A$. We would denote the entry in row $i$, column $j$ by $a_{ij}$.
}

\chapter{Using Matrices to Model Practical Situations}
\chapter{Matrix Addition, Subtraction}
\chapter{Scalar Multiplication}
\chapter{Matrix Multiplication and Powers}
\chapter{Communication and Connections}
\chapter{Problem Solving and Modelling with Matrices}
\chapter{Solutions}
\begin{multicols}{3}
	1.1.1(a) $1{\times}1$ (b) $2{\times}2$ (c) $3{\times}2$ (d) $4{\times}4$ (e) $3{\times}6$ \\


\end{multicols}





















\pagebreak
\chapter{template.tex - Delete later!}
\section{Random Examples}
\dfn{Limit of Sequence in $\bbr$}{Let $\{s_n\}$ be a sequence in $\bbr$. We say $$\lim_{n\to\infty}s_n=s$$ where $s\in\bbr$ if $\forall$ real numbers $\eps>0$ $\exists$ natural number $N$ such that for $n>N$ $$s-\eps<s_n<s+\eps\text{ i.e. }|s-s_n|<\eps$$}
\qs{}{Is the set ${x-}$axis${\setminus\{\text{Origin}\}}$ a closed set}
\sol We have to take its complement and check whether that set is a open set i.e. if it is a union of open balls
\nt{We will do topology in Normed Linear Space  (Mainly $\bbr^n$ and occasionally $\bbc^n$)using the language of Metric Space}
\clm{Topology}{}{Topology is cool}
\ex{Open Set and Close Set}{
	\begin{tabular}{rl}
		Open Set:   & $\bullet$ $\phi$                                              \\
		            & $\bullet$ $\bigcup\limits_{x\in X}B_r(x)$ (Any $r>0$ will do) \\[3mm]
		            & $\bullet$ $B_r(x)$ is open                                    \\
		Closed Set: & $\bullet$ $X,\ \phi$                                          \\
		            & $\bullet$ $\overline{B_r(x)}$                                 \\
		            & $x-$axis $\cup$ $y-$axis
	\end{tabular}}
\thm{}{If $x\in$ open set $V$ then $\exists$ $\delta>0$ such that $B_{\delta}(x)\subset V$}
\begin{myproof}By openness of $V$, $x\in B_r(u)\subset V$
	\begin{center}
		\begin{tikzpicture}
			\draw[red] (0,0) circle [x radius=3.5cm, y radius=2cm] ;
			\draw (3,1.6) node[red]{$V$};
			\draw [blue] (1,0) circle (1.45cm) ;
			\filldraw[blue] (1,0) circle (1pt) node[anchor=north]{$u$};
			\draw (2.9,0.4) node[blue]{$B_r(u)$};
			\draw [green!40!black] (1.7,0) circle (0.5cm) node [yshift=0.7cm]{$B_{\delta}(x)$} ;
			\filldraw[green!40!black] (1.7,0) circle (1pt) node[anchor=west]{$x$};
		\end{tikzpicture}
	\end{center}

	Given $x\in B_r(u)\subset V$, we want $\delta>0$ such that $x\in B_{\delta} (x)\subset B_r(u)\subset V$. Let $d=d(u,x)$. Choose $\delta $ such that $d+\delta<r$ (e.g. $\delta<\frac{r-d}{2}$)

	If $y\in B_{\delta}(x)$ we will be done by showing that $d(u,y)<r$ but $$d(u,y)\leq d(u,x)+d(x,y)<d+\delta<r$$
\end{myproof}

\cor{}{By the result of the proof, we can then show...}
\mlenma{}{Suppose $\vec{v_1}, \dots, \vec{v_n} \in \bbr^n$ is subspace of $\bbr^n$.}
\mprop{}{$1 + 1 = 2$.}

\section{Random}
\dfn{Normed Linear Space and Norm $\boldsymbol{\|\cdot\|}$}{Let $V$ be a vector space over $\bbr$ (or $\bbc$). A norm on $V$ is function $\|\cdot\|\ V\to \bbr_{\geq 0}$ satisfying \begin{enumerate}[label=\bfseries\tiny\protect\circled{\small\arabic*}]
		\item \label{n:1}$\|x\|=0 \iff x=0$ $\forall$ $x\in V$
		\item \label{n:2}	$\|\lambda x\|=|\lambda|\|x\|$ $\forall$ $\lambda\in\bbr$(or $\bbc$), $x\in V$
		\item \label{n:3} $\|x+y\| \leq \|x\|+\|y\|$ $\forall$ $x,y\in V$ (Triangle Inequality/Subadditivity)
	\end{enumerate}And $V$ is called a normed linear space.

	$\bullet $ Same definition works with $V$ a vector space over $\bbc$ (again $\|\cdot\|\to\bbr_{\geq 0}$) where \ref{n:2} becomes $\|\lambda x\|=|\lambda|\|x\|$ $\forall$ $\lambda\in\bbc$, $x\in V$, where for $\lambda=a+ib$, $|\lambda|=\sqrt{a^2+b^2}$ }


\ex{$\bs{p-}$Norm}{\label{pnorm}$V={\bbr}^m$, $p\in\bbr_{\geq 0}$. Define for $x=(x_1,x_2,\cdots,x_m)\in\bbr^m$ $$\|x\|_p=\Big(|x_1|^p+|x_2|^p+\cdots+|x_m|^p\Big)^{\frac1p}$$(In school $p=2$)}
\textbf{Special Case $\bs{p=1}$}: $\|x\|_1=|x_1|+|x_2|+\cdots+|x_m|$ is clearly a norm by usual triangle inequality. \par
\textbf{Special Case $\bs{p\to\infty\ (\bbr^m$ with $\|\cdot\|_{\infty})}$}: $\|x\|_{\infty}=\max\{|x_1|,|x_2|,\cdots,|x_m|\}$\\
For $m=1$ these $p-$norms are nothing but $|x|$.
Now exercise
\qs{}{\label{exs1}Prove that triangle inequality is true if $p\geq 1$ for $p-$norms. (What goes wrong for $p<1$ ?)}
\sol{\textbf{For Property \ref{n:3} for norm-2}	\subsubsection*{\textbf{When field is $\bbr:$}} We have to show\begin{align*}
		         & \sum_i(x_i+y_i)^2\leq \left(\sqrt{\sum_ix_i^2} +\sqrt{\sum_iy_i^2}\right)^2                                       \\
		\implies & \sum_i (x_i^2+2x_iy_i+y_i^2)\leq \sum_ix_i^2+2\sqrt{\left[\sum_ix_i^2\right]\left[\sum_iy_i^2\right]}+\sum_iy_i^2 \\
		\implies & \left[\sum_ix_iy_i\right]^2\leq \left[\sum_ix_i^2\right]\left[\sum_iy_i^2\right]
	\end{align*}So in other words prove $\langle x,y\rangle^2 \leq \langle x,x\rangle\langle y,y\rangle$ where
	$$\langle x,y\rangle =\sum\limits_i x_iy_i$$

	\begin{note}
		\begin{itemize}
			\item $\|x\|^2=\langle x,x\rangle$
			\item $\langle x,y\rangle=\langle y,x\rangle$
			\item $\langle \cdot,\cdot\rangle$ is $\bbr-$linear in each slot i.e. \begin{align*}
				      \langle rx+x',y\rangle=r\langle x,y\rangle+\langle x',y\rangle	\text{ and similarly for second slot}
			      \end{align*}Here in $\langle x,y\rangle$ $x$ is in first slot and $y$ is in second slot.
		\end{itemize}
	\end{note}Now the statement is just the Cauchy-Schwartz Inequality. For proof $$\langle x,y\rangle^2\leq \langle x,x\rangle\langle y,y\rangle $$ expand everything of $\langle x-\lambda y,x-\lambda y\rangle$ which is going to give a quadratic equation in variable $\lambda $ \begin{align*}
		\langle x-\lambda y,x-\lambda y\rangle & =\langle x,x-\lambda y\rangle-\lambda\langle y,x-\lambda y\rangle                                       \\
		                                       & =\langle x ,x\rangle -\lambda\langle x,y\rangle -\lambda\langle y,x\rangle +\lambda^2\langle y,y\rangle \\
		                                       & =\langle x,x\rangle -2\lambda\langle x,y\rangle+\lambda^2\langle y,y\rangle
	\end{align*}Now unless $x=\lambda y$ we have $\langle x-\lambda y,x-\lambda y\rangle>0$ Hence the quadratic equation has no root therefore the discriminant is greater than zero.

	\subsubsection*{\textbf{When field is $\bbc:$}}Modify the definition by $$\langle x,y\rangle=\sum_i\overline{x_i}y_i$$Then we still have $\langle x,x\rangle\geq 0$}

\section{Algorithms}
\begin{algorithm}[H]
\KwIn{This is some input}
\KwOut{This is some output}
\SetAlgoLined
\SetNoFillComment
\tcc{This is a comment}
\vspace{3mm}
some code here\;
$x \leftarrow 0$\;
$y \leftarrow 0$\;
\uIf{$ x > 5$} {
    x is greater than 5 \tcp*{This is also a comment}
}
\Else {
    x is less than or equal to 5\;
}
\ForEach{y in 0..5} {
    $y \leftarrow y + 1$\;
}
\For{$y$ in $0..5$} {
    $y \leftarrow y - 1$\;
}
\While{$x > 5$} {
    $x \leftarrow x - 1$\;
}
\Return Return something here\;
\caption{what}
\end{algorithm}

\end{document}
