\documentclass[a4paper,12pt]{report}

%%%%%%%%%%%%%%%%%%%%%%%%%%%%%%%%%
% PACKAGE IMPORTS
%%%%%%%%%%%%%%%%%%%%%%%%%%%%%%%%%
\usepackage[tmargin=2cm,rmargin=1in,lmargin=1in,margin=0.85in,bmargin=2cm,footskip=.2in]{geometry}
\usepackage[none]{hyphenat}
\usepackage{amsmath,amsfonts,amsthm,amssymb,mathtools}
\allowdisplaybreaks
\usepackage{undertilde}
\usepackage{xfrac}
\usepackage[makeroom]{cancel}
\usepackage{mathtools}
\usepackage{bookmark}
\usepackage{enumitem}
\usepackage{kbordermatrix}
\renewcommand{\kbldelim}{(} % Change left delimiter to (
\renewcommand{\kbrdelim}{)} % Change right delimiter to )
\usepackage{hyperref,theoremref}
\hypersetup{
	pdftitle={Assignment},
	colorlinks=true, linkcolor=doc!90,
	bookmarksnumbered=true,
	bookmarksopen=true
}
\usepackage[most,many,breakable]{tcolorbox}
\usepackage{xcolor}
\usepackage{varwidth}
\usepackage{varwidth}
\usepackage{etoolbox}
%\usepackage{authblk}
\usepackage{nameref}
\usepackage{multicol,array}
\usepackage{tikz-cd}
\usepackage[ruled,vlined,linesnumbered]{algorithm2e}
\usepackage{comment} % enables the use of multi-line comments (\ifx \fi) 
\usepackage{import}
\usepackage{xifthen}
\usepackage{pdfpages}
\usepackage{svg}
\usepackage{transparent}
\usepackage{pgfplots}
\pgfplotsset{compat=1.18}
\usetikzlibrary{calc}
\usetikzlibrary{graphs}
\usetikzlibrary{graphs.standard}
% \usetikzlibrary{graphdrawing}

\newcommand\mycommfont[1]{\footnotesize\ttfamily\textcolor{blue}{#1}}
\SetCommentSty{mycommfont}
\newcommand{\incfig}[1]{%
    \def\svgwidth{\columnwidth}
    \import{./figures/}{#1.pdf_tex}
}


\usepackage{tikzsymbols}
% \renewcommand\qedsymbol{$\Laughey$}

\definecolor{commentgreen}{RGB}{2,112,10}
%%
%% Julia definition (c) 2014 Jubobs
%%
\lstdefinelanguage{Julia}%
  {morekeywords={abstract,break,case,catch,const,continue,do,else,elseif,%
      end,export,false,for,function,immutable,import,importall,if,in,%
      macro,module,otherwise,quote,return,switch,true,try,type,typealias,%
      using,while},%
   sensitive=true,%
   alsoother={$},%
   morecomment=[l]\#,%
   morecomment=[n]{\#=}{=\#},%
   morestring=[s]{"}{"},%
   morestring=[m]{'}{'},%
}[keywords,comments,strings]%

\lstset{%
    language        	= Julia,
    basicstyle      	= \ttfamily,
    keywordstyle    	= \bfseries\color{blue},
    stringstyle     	= \color{magenta},
    commentstyle    	= \color{commentgreen},
    showstringspaces	= false,
		numbers						= left,
		tabsize						= 4,
}

\definecolor{stringyellow}{RGB}{227, 78, 48}
%% 
%% Shamelessly stolen from Vivi on Stackoverflow
%% https://tex.stackexchange.com/questions/75116/what-can-i-use-to-typeset-matlab-code-in-my-document
%%
\lstset{language=Matlab,%
    %basicstyle=\color{red},
    breaklines=true,%
    morekeywords={matlab2tikz},
		morekeywords={subtitle}
    keywordstyle=\color{blue},%
    morekeywords=[2]{1}, keywordstyle=[2]{\color{black}},
    identifierstyle=\color{black},%
    stringstyle=\color{stringyellow},
    commentstyle=\color{commentgreen},%
    showstringspaces=false,%without this there will be a symbol in the places where there is a space
    numbers=left,%
		firstnumber=1,
    % numberstyle={\tiny \color{black}},% size of the numbers
    % numbersep=9pt, % this defines how far the numbers are from the text
    emph=[1]{for,end,break},emphstyle=[1]\color{red}, %some words to emphasise
    %emph=[2]{word1,word2}, emphstyle=[2]{style},    
}

%% 
%% Shamelessly stolen from egreg on Stackoverflow
%% https://tex.stackexchange.com/questions/280681/how-to-have-multiple-lines-of-intertext-within-align-environment
%%
\newlength{\normalparindent}
\AtBeginDocument{\setlength{\normalparindent}{\parindent}}
\newcommand{\longintertext}[1]{%
  \intertext{%
    \parbox{\linewidth}{%
      \setlength{\parindent}{\normalparindent}
      \noindent#1%
    }%
  }%
}

%\usepackage{import}
%\usepackage{xifthen}
%\usepackage{pdfpages}
%\usepackage{transparent}

%%%%%%%%%%%%%%%%%%%%%%%%%%%%%%
% SELF MADE COLORS
%%%%%%%%%%%%%%%%%%%%%%%%%%%%%%
\definecolor{myg}{RGB}{56, 140, 70}
\definecolor{myb}{RGB}{45, 111, 177}
\definecolor{myr}{RGB}{199, 68, 64}
\definecolor{mytheorembg}{HTML}{F2F2F9}
\definecolor{mytheoremfr}{HTML}{00007B}
\definecolor{mylenmabg}{HTML}{FFFAF8}
\definecolor{mylenmafr}{HTML}{983b0f}
\definecolor{mypropbg}{HTML}{f2fbfc}
\definecolor{mypropfr}{HTML}{191971}
\definecolor{myexamplebg}{HTML}{F2FBF8}
\definecolor{myexamplefr}{HTML}{88D6D1}
\definecolor{myexampleti}{HTML}{2A7F7F}
\definecolor{mydefinitbg}{HTML}{E5E5FF}
\definecolor{mydefinitfr}{HTML}{3F3FA3}
\definecolor{notesgreen}{RGB}{0,162,0}
\definecolor{myp}{RGB}{197, 92, 212}
\definecolor{mygr}{HTML}{2C3338}
\definecolor{myred}{RGB}{127,0,0}
\definecolor{myyellow}{RGB}{169,121,69}
\definecolor{myexercisebg}{HTML}{F2FBF8}
\definecolor{myexercisefg}{HTML}{88D6D1}

%%%%%%%%%%%%%%%%%%%%%%%%%%%%
% TCOLORBOX SETUPS
%%%%%%%%%%%%%%%%%%%%%%%%%%%%
\setlength{\parindent}{0pt}

%================================
% THEOREM BOX
%================================
\tcbuselibrary{theorems,skins,hooks}
\newtcbtheorem[number within=section]{Theorem}{Theorem}
{%
	enhanced,
	breakable,
	colback = mytheorembg,
	frame hidden,
	boxrule = 0sp,
	borderline west = {2pt}{0pt}{mytheoremfr},
	sharp corners,
	detach title,
	before upper = \tcbtitle\par\smallskip,
	coltitle = mytheoremfr,
	fonttitle = \bfseries\sffamily,
	description font = \mdseries,
	separator sign none,
	segmentation style={solid, mytheoremfr},
}
{th}

\tcbuselibrary{theorems,skins,hooks}
\newtcbtheorem[number within=chapter]{theorem}{Theorem}
{%
	enhanced,
	breakable,
	colback = mytheorembg,
	frame hidden,
	boxrule = 0sp,
	borderline west = {2pt}{0pt}{mytheoremfr},
	sharp corners,
	detach title,
	before upper = \tcbtitle\par\smallskip,
	coltitle = mytheoremfr,
	fonttitle = \bfseries\sffamily,
	description font = \mdseries,
	separator sign none,
	segmentation style={solid, mytheoremfr},
}
{th}


\tcbuselibrary{theorems,skins,hooks}
\newtcolorbox{Theoremcon}
{%
	enhanced
	,breakable
	,colback = mytheorembg
	,frame hidden
	,boxrule = 0sp
	,borderline west = {2pt}{0pt}{mytheoremfr}
	,sharp corners
	,description font = \mdseries
	,separator sign none
}

%================================
% Corollery
%================================
\tcbuselibrary{theorems,skins,hooks}
\newtcbtheorem[number within=section]{Corollary}{Corollary}
{%
	enhanced
	,breakable
	,colback = myp!10
	,frame hidden
	,boxrule = 0sp
	,borderline west = {2pt}{0pt}{myp!85!black}
	,sharp corners
	,detach title
	,before upper = \tcbtitle\par\smallskip
	,coltitle = myp!85!black
	,fonttitle = \bfseries\sffamily
	,description font = \mdseries
	,separator sign none
	,segmentation style={solid, myp!85!black}
}
{th}
\tcbuselibrary{theorems,skins,hooks}
\newtcbtheorem[number within=chapter]{corollary}{Corollary}
{%
	enhanced
	,breakable
	,colback = myp!10
	,frame hidden
	,boxrule = 0sp
	,borderline west = {2pt}{0pt}{myp!85!black}
	,sharp corners
	,detach title
	,before upper = \tcbtitle\par\smallskip
	,coltitle = myp!85!black
	,fonttitle = \bfseries\sffamily
	,description font = \mdseries
	,separator sign none
	,segmentation style={solid, myp!85!black}
}
{th}

%================================
% LENMA
%================================
\tcbuselibrary{theorems,skins,hooks}
\newtcbtheorem[number within=section]{Lenma}{Lenma}
{%
	enhanced,
	breakable,
	colback = mylenmabg,
	frame hidden,
	boxrule = 0sp,
	borderline west = {2pt}{0pt}{mylenmafr},
	sharp corners,
	detach title,
	before upper = \tcbtitle\par\smallskip,
	coltitle = mylenmafr,
	fonttitle = \bfseries\sffamily,
	description font = \mdseries,
	separator sign none,
	segmentation style={solid, mylenmafr},
}
{th}

\tcbuselibrary{theorems,skins,hooks}
\newtcbtheorem[number within=chapter]{lenma}{Lenma}
{%
	enhanced,
	breakable,
	colback = mylenmabg,
	frame hidden,
	boxrule = 0sp,
	borderline west = {2pt}{0pt}{mylenmafr},
	sharp corners,
	detach title,
	before upper = \tcbtitle\par\smallskip,
	coltitle = mylenmafr,
	fonttitle = \bfseries\sffamily,
	description font = \mdseries,
	separator sign none,
	segmentation style={solid, mylenmafr},
}
{th}

%================================
% PROPOSITION
%================================
\tcbuselibrary{theorems,skins,hooks}
\newtcbtheorem[number within=section]{Prop}{Proposition}
{%
	enhanced,
	breakable,
	colback = mypropbg,
	frame hidden,
	boxrule = 0sp,
	borderline west = {2pt}{0pt}{mypropfr},
	sharp corners,
	detach title,
	before upper = \tcbtitle\par\smallskip,
	coltitle = mypropfr,
	fonttitle = \bfseries\sffamily,
	description font = \mdseries,
	separator sign none,
	segmentation style={solid, mypropfr},
}
{th}

\tcbuselibrary{theorems,skins,hooks}
\newtcbtheorem[number within=chapter]{prop}{Proposition}
{%
	enhanced,
	breakable,
	colback = mypropbg,
	frame hidden,
	boxrule = 0sp,
	borderline west = {2pt}{0pt}{mypropfr},
	sharp corners,
	detach title,
	before upper = \tcbtitle\par\smallskip,
	coltitle = mypropfr,
	fonttitle = \bfseries\sffamily,
	description font = \mdseries,
	separator sign none,
	segmentation style={solid, mypropfr},
}
{th}

%================================
% CLAIM
%================================
\tcbuselibrary{theorems,skins,hooks}
\newtcbtheorem[number within=section]{claim}{Claim}
{%
	enhanced
	,breakable
	,colback = myg!10
	,frame hidden
	,boxrule = 0sp
	,borderline west = {2pt}{0pt}{myg}
	,sharp corners
	,detach title
	,before upper = \tcbtitle\par\smallskip
	,coltitle = myg!85!black
	,fonttitle = \bfseries\sffamily
	,description font = \mdseries
	,separator sign none
	,segmentation style={solid, myg!85!black}
}
{th}

%================================
% Exercise
%================================
\tcbuselibrary{theorems,skins,hooks}
\newtcbtheorem[number within=section]{Exercise}{Exercise}
{%
	enhanced,
	breakable,
	colback = myexercisebg,
	frame hidden,
	boxrule = 0sp,
	borderline west = {2pt}{0pt}{myexercisefg},
	sharp corners,
	detach title,
	before upper = \tcbtitle\par\smallskip,
	coltitle = myexercisefg,
	fonttitle = \bfseries\sffamily,
	description font = \mdseries,
	separator sign none,
	segmentation style={solid, myexercisefg},
}
{th}

\tcbuselibrary{theorems,skins,hooks}
\newtcbtheorem[number within=chapter]{exercise}{Exercise}
{%
	enhanced,
	breakable,
	colback = myexercisebg,
	frame hidden,
	boxrule = 0sp,
	borderline west = {2pt}{0pt}{myexercisefg},
	sharp corners,
	detach title,
	before upper = \tcbtitle\par\smallskip,
	coltitle = myexercisefg,
	fonttitle = \bfseries\sffamily,
	description font = \mdseries,
	separator sign none,
	segmentation style={solid, myexercisefg},
}
{th}

%================================
% EXAMPLE BOX
%================================
\newtcbtheorem[number within=section]{Example}{Example}
{%
	colback = myexamplebg
	,breakable
	,colframe = myexamplefr
	,coltitle = myexampleti
	,boxrule = 1pt
	,sharp corners
	,detach title
	,before upper=\tcbtitle\par\smallskip
	,fonttitle = \bfseries
	,description font = \mdseries
	,separator sign none
	,description delimiters parenthesis
}
{ex}

\newtcbtheorem[number within=chapter]{example}{Example}
{%
	colback = myexamplebg
	,breakable
	,colframe = myexamplefr
	,coltitle = myexampleti
	,boxrule = 1pt
	,sharp corners
	,detach title
	,before upper=\tcbtitle\par\smallskip
	,fonttitle = \bfseries
	,description font = \mdseries
	,separator sign none
	,description delimiters parenthesis
}
{ex}

%================================
% DEFINITION BOX
%================================
\newtcbtheorem[number within=section]{Definition}{Definition}{enhanced,
	before skip=2mm,after skip=2mm, colback=red!5,colframe=red!80!black,boxrule=0.5mm,
	attach boxed title to top left={xshift=1cm,yshift*=1mm-\tcboxedtitleheight}, varwidth boxed title*=-3cm,
	boxed title style={frame code={
					\path[fill=tcbcolback]
					([yshift=-1mm,xshift=-1mm]frame.north west)
					arc[start angle=0,end angle=180,radius=1mm]
					([yshift=-1mm,xshift=1mm]frame.north east)
					arc[start angle=180,end angle=0,radius=1mm];
					\path[left color=tcbcolback!60!black,right color=tcbcolback!60!black,
						middle color=tcbcolback!80!black]
					([xshift=-2mm]frame.north west) -- ([xshift=2mm]frame.north east)
					[rounded corners=1mm]-- ([xshift=1mm,yshift=-1mm]frame.north east)
					-- (frame.south east) -- (frame.south west)
					-- ([xshift=-1mm,yshift=-1mm]frame.north west)
					[sharp corners]-- cycle;
				},interior engine=empty,
		},
	fonttitle=\bfseries,
	title={#2},#1}{def}
\newtcbtheorem[number within=chapter]{definition}{Definition}{enhanced,
	before skip=2mm,after skip=2mm, colback=red!5,colframe=red!80!black,boxrule=0.5mm,
	attach boxed title to top left={xshift=1cm,yshift*=1mm-\tcboxedtitleheight}, varwidth boxed title*=-3cm,
	boxed title style={frame code={
					\path[fill=tcbcolback]
					([yshift=-1mm,xshift=-1mm]frame.north west)
					arc[start angle=0,end angle=180,radius=1mm]
					([yshift=-1mm,xshift=1mm]frame.north east)
					arc[start angle=180,end angle=0,radius=1mm];
					\path[left color=tcbcolback!60!black,right color=tcbcolback!60!black,
						middle color=tcbcolback!80!black]
					([xshift=-2mm]frame.north west) -- ([xshift=2mm]frame.north east)
					[rounded corners=1mm]-- ([xshift=1mm,yshift=-1mm]frame.north east)
					-- (frame.south east) -- (frame.south west)
					-- ([xshift=-1mm,yshift=-1mm]frame.north west)
					[sharp corners]-- cycle;
				},interior engine=empty,
		},
	fonttitle=\bfseries,
	title={#2},#1}{def}

%================================
% Solution BOX
%================================
\makeatletter
\newtcbtheorem{question}{Question}{enhanced,
	breakable,
	colback=white,
	colframe=myb!80!black,
	attach boxed title to top left={yshift*=-\tcboxedtitleheight},
	fonttitle=\bfseries,
	title={#2},
	boxed title size=title,
	boxed title style={%
			sharp corners,
			rounded corners=northwest,
			colback=tcbcolframe,
			boxrule=0pt,
		},
	underlay boxed title={%
			\path[fill=tcbcolframe] (title.south west)--(title.south east)
			to[out=0, in=180] ([xshift=5mm]title.east)--
			(title.center-|frame.east)
			[rounded corners=\kvtcb@arc] |-
			(frame.north) -| cycle;
		},
	#1
}{def}
\makeatother

%================================
% SOLUTION BOX
%================================
\makeatletter
\newtcolorbox{solution}{enhanced,
	breakable,
	colback=white,
	colframe=myg!80!black,
	attach boxed title to top left={yshift*=-\tcboxedtitleheight},
	title=Solution,
	boxed title size=title,
	boxed title style={%
			sharp corners,
			rounded corners=northwest,
			colback=tcbcolframe,
			boxrule=0pt,
		},
	underlay boxed title={%
			\path[fill=tcbcolframe] (title.south west)--(title.south east)
			to[out=0, in=180] ([xshift=5mm]title.east)--
			(title.center-|frame.east)
			[rounded corners=\kvtcb@arc] |-
			(frame.north) -| cycle;
		},
}
\makeatother

%================================
% Question BOX
%================================
\makeatletter
\newtcbtheorem{qstion}{Question}{enhanced,
	breakable,
	colback=white,
	colframe=mygr,
	attach boxed title to top left={yshift*=-\tcboxedtitleheight},
	fonttitle=\bfseries,
	title={#2},
	boxed title size=title,
	boxed title style={%
			sharp corners,
			rounded corners=northwest,
			colback=tcbcolframe,
			boxrule=0pt,
		},
	underlay boxed title={%
			\path[fill=tcbcolframe] (title.south west)--(title.south east)
			to[out=0, in=180] ([xshift=5mm]title.east)--
			(title.center-|frame.east)
			[rounded corners=\kvtcb@arc] |-
			(frame.north) -| cycle;
		},
	#1
}{def}
\makeatother

\newtcbtheorem[number within=chapter]{wconc}{Wrong Concept}{
	breakable,
	enhanced,
	colback=white,
	colframe=myr,
	arc=0pt,
	outer arc=0pt,
	fonttitle=\bfseries\sffamily\large,
	colbacktitle=myr,
	attach boxed title to top left={},
	boxed title style={
			enhanced,
			skin=enhancedfirst jigsaw,
			arc=3pt,
			bottom=0pt,
			interior style={fill=myr}
		},
	#1
}{def}

%================================
% NOTE BOX
%================================
\usetikzlibrary{arrows,calc,shadows.blur}
\tcbuselibrary{skins}
\newtcolorbox{note}[1][]{%
	enhanced jigsaw,
	colback=gray!20!white,%
	colframe=gray!80!black,
	size=small,
	boxrule=1pt,
	title=\textbf{Note:-},
	halign title=flush center,
	coltitle=black,
	breakable,
	drop shadow=black!50!white,
	attach boxed title to top left={xshift=1cm,yshift=-\tcboxedtitleheight/2,yshifttext=-\tcboxedtitleheight/2},
	minipage boxed title=1.5cm,
	boxed title style={%
			colback=white,
			size=fbox,
			boxrule=1pt,
			boxsep=2pt,
			underlay={%
					\coordinate (dotA) at ($(interior.west) + (-0.5pt,0)$);
					\coordinate (dotB) at ($(interior.east) + (0.5pt,0)$);
					\begin{scope}
						\clip (interior.north west) rectangle ([xshift=3ex]interior.east);
						\filldraw [white, blur shadow={shadow opacity=60, shadow yshift=-.75ex}, rounded corners=2pt] (interior.north west) rectangle (interior.south east);
					\end{scope}
					\begin{scope}[gray!80!black]
						\fill (dotA) circle (2pt);
						\fill (dotB) circle (2pt);
					\end{scope}
				},
		},
	#1,
}

%%%%%%%%%%%%%%%%%%%%%%%%%%%%%%
% SELF MADE COMMANDS
%%%%%%%%%%%%%%%%%%%%%%%%%%%%%%
\newcommand{\thm}[2]{\begin{Theorem}{#1}{}#2\end{Theorem}}
\newcommand{\cor}[2]{\begin{Corollary}{#1}{}#2\end{Corollary}}
\newcommand{\mlenma}[2]{\begin{Lenma}{#1}{}#2\end{Lenma}}
\newcommand{\mprop}[2]{\begin{Prop}{#1}{}#2\end{Prop}}
\newcommand{\clm}[3]{\begin{claim}{#1}{#2}#3\end{claim}}
\newcommand{\wc}[2]{\begin{wconc}{#1}{}\setlength{\parindent}{1cm}#2\end{wconc}}
\newcommand{\thmcon}[1]{\begin{Theoremcon}{#1}\end{Theoremcon}}
\newcommand{\ex}[2]{\begin{Example}{#1}{}#2\end{Example}}
\newcommand{\dfn}[2]{\begin{Definition}[colbacktitle=red!75!black]{#1}{}#2\end{Definition}}
\newcommand{\dfnc}[2]{\begin{definition}[colbacktitle=red!75!black]{#1}{}#2\end{definition}}
\newcommand{\qs}[2]{\begin{question}{#1}{}#2\end{question}}
\newcommand{\pf}[2]{\begin{myproof}[#1]#2\end{myproof}}
\newcommand{\nt}[1]{\begin{note}#1\end{note}}

\newcommand*\circled[1]{\tikz[baseline=(char.base)]{
		\node[shape=circle,draw,inner sep=1pt] (char) {#1};}}
\newcommand\getcurrentref[1]{%
	\ifnumequal{\value{#1}}{0}
	{??}
	{\the\value{#1}}%
}
\newcommand{\getCurrentSectionNumber}{\getcurrentref{section}}
\newenvironment{myproof}[1][\proofname]{%
	\proof[\bfseries #1: ]%
}{\endproof}

\newcommand{\mclm}[2]{\begin{myclaim}[#1]#2\end{myclaim}}
\newenvironment{myclaim}[1][\claimname]{\proof[\bfseries #1: ]}{}

\newcounter{mylabelcounter}

\makeatletter
\newcommand{\setword}[2]{%
	\phantomsection
	#1\def\@currentlabel{\unexpanded{#1}}\label{#2}%
}
\makeatother

\tikzset{
	symbol/.style={
			draw=none,
			every to/.append style={
					edge node={node [sloped, allow upside down, auto=false]{$#1$}}}
		}
}

% deliminators
\DeclarePairedDelimiter{\abs}{\lvert}{\rvert}
\DeclarePairedDelimiter{\norm}{\lVert}{\rVert}

\DeclarePairedDelimiter{\ceil}{\lceil}{\rceil}
\DeclarePairedDelimiter{\floor}{\lfloor}{\rfloor}
\DeclarePairedDelimiter{\round}{\lfloor}{\rceil}

\newsavebox\diffdbox
\newcommand{\slantedromand}{{\mathpalette\makesl{d}}}
\newcommand{\makesl}[2]{%
\begingroup
\sbox{\diffdbox}{$\mathsurround=0pt#1\mathrm{#2}$}%
\pdfsave
\pdfsetmatrix{1 0 0.2 1}%
\rlap{\usebox{\diffdbox}}%
\pdfrestore
\hskip\wd\diffdbox
\endgroup
}
\newcommand{\dd}[1][]{\ensuremath{\mathop{}\!\ifstrempty{#1}{%
\slantedromand\@ifnextchar^{\hspace{0.2ex}}{\hspace{0.1ex}}}%
{\slantedromand\hspace{0.2ex}^{#1}}}}
\ProvideDocumentCommand\dv{o m g}{%
  \ensuremath{%
    \IfValueTF{#3}{%
      \IfNoValueTF{#1}{%
        \frac{\dd #2}{\dd #3}%
      }{%
        \frac{\dd^{#1} #2}{\dd #3^{#1}}%
      }%
    }{%
      \IfNoValueTF{#1}{%
        \frac{\dd}{\dd #2}%
      }{%
        \frac{\dd^{#1}}{\dd #2^{#1}}%
      }%
    }%
  }%
}
\providecommand*{\pdv}[3][]{\frac{\partial^{#1}#2}{\partial#3^{#1}}}
%  - others
\DeclareMathOperator{\Lap}{\mathcal{L}}
\DeclareMathOperator{\Var}{Var} % varience
\DeclareMathOperator{\Cov}{Cov} % covarience
\DeclareMathOperator{\E}{E} % expected

% Since the amsthm package isn't loaded

% I dot not prefer the slanted \leq ;P
% % I prefer the slanted \leq
% \let\oldleq\leq % save them in case they're every wanted
% \let\oldgeq\geq
% \renewcommand{\leq}{\leqslant}
% \renewcommand{\geq}{\geqslant}

% % redefine matrix env to allow for alignment, use r as default
% \renewcommand*\env@matrix[1][r]{\hskip -\arraycolsep
%     \let\@ifnextchar\new@ifnextchar
%     \array{*\c@MaxMatrixCols #1}}

%\usepackage{framed}
%\usepackage{titletoc}
%\usepackage{etoolbox}
%\usepackage{lmodern}

%\patchcmd{\tableofcontents}{\contentsname}{\sffamily\contentsname}{}{}

%\renewenvironment{leftbar}
%{\def\FrameCommand{\hspace{6em}%
%		{\color{myyellow}\vrule width 2pt depth 6pt}\hspace{1em}}%
%	\MakeFramed{\parshape 1 0cm \dimexpr\textwidth-6em\relax\FrameRestore}\vskip2pt%
%}
%{\endMakeFramed}

%\titlecontents{chapter}
%[0em]{\vspace*{2\baselineskip}}
%{\parbox{4.5em}{%
%		\hfill\Huge\sffamily\bfseries\color{myred}\thecontentspage}%
%	\vspace*{-2.3\baselineskip}\leftbar\textsc{\small\chaptername~\thecontentslabel}\\\sffamily}
%{}{\endleftbar}
%\titlecontents{section}
%[8.4em]
%{\sffamily\contentslabel{3em}}{}{}
%{\hspace{0.5em}\nobreak\itshape\color{myred}\contentspage}
%\titlecontents{subsection}
%[8.4em]
%{\sffamily\contentslabel{3em}}{}{}  
%{\hspace{0.5em}\nobreak\itshape\color{myred}\contentspage}

%%%%%%%%%%%%%%%%%%%%%%%%%%%%%%%%%%%%%%%%%%%
% TABLE OF CONTENTS
%%%%%%%%%%%%%%%%%%%%%%%%%%%%%%%%%%%%%%%%%%%
\usepackage{tikz}
\definecolor{doc}{RGB}{0,60,110}
\usepackage{titletoc}
\contentsmargin{0cm}
\titlecontents{chapter}[3.7pc]
{\addvspace{30pt}%
	\begin{tikzpicture}[remember picture, overlay]%
		\draw[fill=doc!60,draw=doc!60] (-7,-.1) rectangle (-0.9,.5);%
		\pgftext[left,x=-3.5cm,y=0.2cm]{\color{white}\Large\sc\bfseries Chapter\ \thecontentslabel};%
	\end{tikzpicture}\color{doc!60}\large\sc\bfseries}%
{}
{}
{\;\titlerule\;\large\sc\bfseries Page \thecontentspage
	\begin{tikzpicture}[remember picture, overlay]
		\draw[fill=doc!60,draw=doc!60] (2pt,0) rectangle (4,0.1pt);
	\end{tikzpicture}}%
\titlecontents{section}[3.7pc]
{\addvspace{2pt}}
{\contentslabel[\thecontentslabel]{2pc}}
{}
{\hfill\small \thecontentspage}
[]
\titlecontents*{subsection}[3.7pc]
{\addvspace{-1pt}\small}
{}
{}
{\ --- \small\thecontentspage}
[ \textbullet\ ][]

\makeatletter
\renewcommand{\tableofcontents}{%
	\chapter*{%
	  \vspace*{-20\p@}%
	  \begin{tikzpicture}[remember picture, overlay]%
		  \pgftext[right,x=15cm,y=0.2cm]{\color{doc!60}\Huge\sc\bfseries \contentsname};%
		  \draw[fill=doc!60,draw=doc!60] (13,-.75) rectangle (20,1);%
		  \clip (13,-.75) rectangle (20,1);
		  \pgftext[right,x=15cm,y=0.2cm]{\color{white}\Huge\sc\bfseries \contentsname};%
	  \end{tikzpicture}}%
	\@starttoc{toc}}
\makeatother

\newcommand{\inv}{^{-1}}
\newcommand{\opname}{\operatorname}
\newcommand{\surjto}{\twoheadrightarrow}
% \newcommand{\injto}{\hookrightarrow}
\newcommand{\injto}{\rightarrowtail}
\newcommand{\bijto}{\leftrightarrow}

\newcommand{\liff}{\leftrightarrow}
\newcommand{\notliff}{\mathrel{\ooalign{$\leftrightarrow$\cr\hidewidth$/$\hidewidth}}}
\newcommand{\lthen}{\rightarrow}
\let\varlnot\lnot
\newcommand{\ordsim}{\mathord{\sim}}
\renewcommand{\lnot}{\ordsim}
\newcommand{\lxor}{\oplus}
\newcommand{\lnand}{\barwedge}
\newcommand{\divs}{\mathrel{\mid}}
\newcommand{\ndivs}{\mathrel{\nmid}}
\def\contra{\tikz[baseline, x=0.22em, y=0.22em, line width=0.032em]\draw (0,2.83)--(2.83,0) (0.71,3.54)--(3.54,0.71) (0,0.71)--(2.83,3.54) (0.71,0)--(3.54,2.83);}

\newcommand{\On}{\mathrm{On}} % ordinals
\DeclareMathOperator{\img}{im} % Image
\DeclareMathOperator{\Img}{Im} % Image
\DeclareMathOperator{\coker}{coker} % Cokernel
\DeclareMathOperator{\Coker}{Coker} % Cokernel
\DeclareMathOperator{\Ker}{Ker} % Kernel
\DeclareMathOperator{\rank}{rank}
\DeclareMathOperator{\Spec}{Spec} % spectrum
\DeclareMathOperator{\Tr}{Tr} % trace
\DeclareMathOperator{\pr}{pr} % projection
\DeclareMathOperator{\ext}{ext} % extension
\DeclareMathOperator{\pred}{pred} % predecessor
\DeclareMathOperator{\dom}{dom} % domain
\DeclareMathOperator{\ran}{ran} % range
\DeclareMathOperator{\Hom}{Hom} % homomorphism
\DeclareMathOperator{\Mor}{Mor} % morphisms
\DeclareMathOperator{\End}{End} % endomorphism
\DeclareMathOperator{\Span}{span}
\newcommand{\Mod}{\mathbin{\mathrm{mod}}}

\newcommand{\eps}{\epsilon}
\newcommand{\veps}{\varepsilon}
\newcommand{\ol}{\overline}
\newcommand{\ul}{\underline}
\newcommand{\wt}{\widetilde}
\newcommand{\wh}{\widehat}
\newcommand{\ut}{\utilde}
\newcommand{\unit}[1]{\ut{\hat{#1}}}
\newcommand{\emp}{\varnothing}

\newcommand{\vocab}[1]{\textbf{\color{blue} #1}}
\providecommand{\half}{\frac{1}{2}}
\newcommand{\dang}{\measuredangle} %% Directed angle
\newcommand{\ray}[1]{\overrightarrow{#1}}
\newcommand{\seg}[1]{\overline{#1}}
\newcommand{\arc}[1]{\wideparen{#1}}
\DeclareMathOperator{\cis}{cis}
\DeclareMathOperator*{\lcm}{lcm}
\DeclareMathOperator*{\argmin}{arg min}
\DeclareMathOperator*{\argmax}{arg max}
\newcommand{\cycsum}{\sum_{\mathrm{cyc}}}
\newcommand{\symsum}{\sum_{\mathrm{sym}}}
\newcommand{\cycprod}{\prod_{\mathrm{cyc}}}
\newcommand{\symprod}{\prod_{\mathrm{sym}}}
\newcommand{\parinn}{\setlength{\parindent}{1cm}}
\newcommand{\parinf}{\setlength{\parindent}{0cm}}
% \newcommand{\norm}{\|\cdot\|}
\newcommand{\inorm}{\norm_{\infty}}
\newcommand{\opensets}{\{V_{\alpha}\}_{\alpha\in I}}
\newcommand{\oset}{V_{\alpha}}
\newcommand{\opset}[1]{V_{\alpha_{#1}}}
\newcommand{\lub}{\text{lub}}
\newcommand{\lm}{\lambda}
\newcommand{\uin}{\mathbin{\rotatebox[origin=c]{90}{$\in$}}}
\newcommand{\usubset}{\mathbin{\rotatebox[origin=c]{90}{$\subset$}}}
\newcommand{\lt}{\left}
\newcommand{\rt}{\right}
\newcommand{\bs}[1]{\boldsymbol{#1}}
\newcommand{\exs}{\exists}
\newcommand{\st}{\strut}
\newcommand{\dps}[1]{\displaystyle{#1}}

\newcommand{\sol}{\textbf{\textit{Solution:}} }
\newcommand{\solve}[1]{\textbf{\textit{Solution: }} #1 \qed}
% \newcommand{\proof}{\underline{\textit{proof:}}\\}

\DeclareMathOperator{\sech}{sech}
\DeclareMathOperator{\csch}{csch}
\DeclareMathOperator{\arcsec}{arcsec}
\DeclareMathOperator{\arccsc}{arccsc}
\DeclareMathOperator{\arccot}{arccot}
\DeclareMathOperator{\arsinh}{arsinh}
\DeclareMathOperator{\arcosh}{arcosh}
\DeclareMathOperator{\artanh}{artanh}
\DeclareMathOperator{\arcsch}{arcsch}
\DeclareMathOperator{\arsech}{arsech}
\DeclareMathOperator{\arcoth}{arcoth}

\newcommand{\sinx}{\sin x}          \newcommand{\arcsinx}{\arcsin x}    
\newcommand{\cosx}{\cos x}          \newcommand{\arccosx}{\arccosx}
\newcommand{\tanx}{\tan x}          \newcommand{\arctanx}{\arctan x}
\newcommand{\cscx}{\csc x}          \newcommand{\arccscx}{\arccsc x}
\newcommand{\secx}{\sec x}          \newcommand{\arcsecx}{\arcsec x}
\newcommand{\cotx}{\cot x}          \newcommand{\arccotx}{\arccot x}
\newcommand{\sinhx}{\sinh x}          \newcommand{\arsinhx}{\arsinh x}
\newcommand{\coshx}{\cosh x}          \newcommand{\arcoshx}{\arcosh x}
\newcommand{\tanhx}{\tanh x}          \newcommand{\artanhx}{\artanh x}
\newcommand{\cschx}{\csch x}          \newcommand{\arcschx}{\arcsch x}
\newcommand{\sechx}{\sech x}          \newcommand{\arsechx}{\arsech x}
\newcommand{\cothx}{\coth x}          \newcommand{\arcothx}{\arcoth x}
\newcommand{\lnx}{\ln x}
\newcommand{\expx}{\exp x}

\newcommand{\Theom}{\textbf{Theorem. }}
\newcommand{\Lemma}{\textbf{Lemma. }}
\newcommand{\Corol}{\textbf{Corollary. }}
\newcommand{\Remar}{\textit{Remark. }}
\newcommand{\Defin}[1]{\textbf{Definition} (#1).}
\newcommand{\Claim}{\textbf{Claim. }}
\newcommand{\Propo}{\textbf{Proposition. }}

\newcommand{\lb}{\left(}
\newcommand{\rb}{\right)}
\newcommand{\lbr}{\left\lbrace}
\newcommand{\rbr}{\right\rbrace}
\newcommand{\lsb}{\left[}
\newcommand{\rsb}{\right]}
\newcommand{\bracks}[1]{\lb #1 \rb}
\newcommand{\braces}[1]{\lbr #1 \rbr}
\newcommand{\suchthat}{\medspace\middle|\medspace}
\newcommand{\sqbracks}[1]{\lsb #1 \rsb}
\renewcommand{\abs}[1]{\left| #1 \right|}
\newcommand{\Mag}[1]{\left|\left| #1 \right|\right|}
\renewcommand{\floor}[1]{\left\lfloor #1 \right\rfloor}
\renewcommand{\ceil}[1]{\left\lceil #1 \right\rceil}

\newcommand{\cd}{\cdot}
\newcommand{\tf}{\therefore}
\newcommand{\Let}{\text{Let }}
\newcommand{\Given}{\text{Given }}
% \newcommand{\and}{\text{and }}
\newcommand{\Substitute}{\text{Substitute }}
\newcommand{\Suppose}{\text{Suppose }}
\newcommand{\WeSee}{\text{We see }}
\newcommand{\So}{\text{So }}
\newcommand{\Then}{\text{Then }}
\newcommand{\Choose}{\text{Choose }}
\newcommand{\Take}{\text{Take }}
\newcommand{\false}{\text{False}}
\newcommand{\true}{\text{True}}

\newcommand{\QED}{\hfill \qed}
\newcommand{\CONTRA}{\hfill \contra}

\newcommand{\ihat}{\hat{\imath}}
\newcommand{\jhat}{\hat{\jmath}}
\newcommand{\khat}{\hat{k}}

\newcommand{\grad}{\nabla}
\newcommand{\D}{\Delta}
\renewcommand{\d}{\mathrm{d}}

\renewcommand{\dd}[1]{\frac{\d}{\d #1}}
\newcommand{\dyd}[2][y]{\frac{\d #1}{\d #2}}

\newcommand{\ddx}{\dd{x}}       \newcommand{\ddxsq}{\dyd[^2]{x^2}}
\newcommand{\ddy}{\dd{y}}       \newcommand{\ddysq}{\dyd[^2]{y^2}}
\newcommand{\ddu}{\dd{u}}       \newcommand{\ddusq}{\dyd[^2]{u^2}}
\newcommand{\ddv}{\dd{v}}       \newcommand{\ddvsq}{\dyd[^2]{v^2}}

\newcommand{\dydx}{\dyd{x}}     \newcommand{\dydxsq}{\dyd[^2y]{x^2}}
\newcommand{\dfdx}{\dyd[f]{x}}  \newcommand{\dfdxsq}{\dyd[^2f]{x^2}}
\newcommand{\dudx}{\dyd[u]{x}}  \newcommand{\dudxsq}{\dyd[^2u]{x^2}}
\newcommand{\dvdx}{\dyd[v]{x}}  \newcommand{\dvdxsq}{\dyd[^2v]{x^2}}

\newcommand{\del}[2]{\frac{\partial #1}{\partial #2}}
\newcommand{\Del}[3]{\frac{\partial^{#1} #2}{\partial #3^{#1}}}
\newcommand{\deld}[2]{\dfrac{\partial #1}{\partial #2}}
\newcommand{\Deld}[3]{\dfrac{\partial^{#1} #2}{\partial #3^{#1}}}

\newcommand{\argument}[2]{
  \begin{array}{rll}
    #1
    \cline{2-2}
    \therefore & #2 
  \end{array}
}
% Mathfrak primes
\newcommand{\km}{\mathfrak m}
\newcommand{\kp}{\mathfrak p}
\newcommand{\kq}{\mathfrak q}

%---------------------------------------
% Blackboard Math Fonts :-
%---------------------------------------
\newcommand{\bba}{\mathbb{A}}   \newcommand{\bbn}{\mathbb{N}}
\newcommand{\bbb}{\mathbb{B}}   \newcommand{\bbo}{\mathbb{O}}
\newcommand{\bbc}{\mathbb{C}}   \newcommand{\bbp}{\mathbb{P}}
\newcommand{\bbd}{\mathbb{D}}   \newcommand{\bbq}{\mathbb{Q}}
\newcommand{\bbe}{\mathbb{E}}   \newcommand{\bbr}{\mathbb{R}}
\newcommand{\bbf}{\mathbb{F}}   \newcommand{\bbs}{\mathbb{S}}
\newcommand{\bbg}{\mathbb{G}}   \newcommand{\bbt}{\mathbb{T}}
\newcommand{\bbh}{\mathbb{H}}   \newcommand{\bbu}{\mathbb{U}}
\newcommand{\bbi}{\mathbb{I}}   \newcommand{\bbv}{\mathbb{V}}
\newcommand{\bbj}{\mathbb{J}}   \newcommand{\bbw}{\mathbb{W}}
\newcommand{\bbk}{\mathbb{K}}   \newcommand{\bbx}{\mathbb{X}}
\newcommand{\bbl}{\mathbb{L}}   \newcommand{\bby}{\mathbb{Y}}
\newcommand{\bbm}{\mathbb{M}}   \newcommand{\bbz}{\mathbb{Z}}

%---------------------------------------
% Roman Math Fonts :-
%---------------------------------------
\newcommand{\rma}{\mathrm{A}}   \newcommand{\rmn}{\mathrm{N}}
\newcommand{\rmb}{\mathrm{B}}   \newcommand{\rmo}{\mathrm{O}}
\newcommand{\rmc}{\mathrm{C}}   \newcommand{\rmp}{\mathrm{P}}
\newcommand{\rmd}{\mathrm{D}}   \newcommand{\rmq}{\mathrm{Q}}
\newcommand{\rme}{\mathrm{E}}   \newcommand{\rmr}{\mathrm{R}}
\newcommand{\rmf}{\mathrm{F}}   \newcommand{\rms}{\mathrm{S}}
\newcommand{\rmg}{\mathrm{G}}   \newcommand{\rmt}{\mathrm{T}}
\newcommand{\rmh}{\mathrm{H}}   \newcommand{\rmu}{\mathrm{U}}
\newcommand{\rmi}{\mathrm{I}}   \newcommand{\rmv}{\mathrm{V}}
\newcommand{\rmj}{\mathrm{J}}   \newcommand{\rmw}{\mathrm{W}}
\newcommand{\rmk}{\mathrm{K}}   \newcommand{\rmx}{\mathrm{X}}
\newcommand{\rml}{\mathrm{L}}   \newcommand{\rmy}{\mathrm{Y}}
\newcommand{\rmm}{\mathrm{M}}   \newcommand{\rmz}{\mathrm{Z}}

%---------------------------------------
% Calligraphic Math Fonts :-
%---------------------------------------
\newcommand{\cla}{\mathcal{A}}   \newcommand{\cln}{\mathcal{N}}
\newcommand{\clb}{\mathcal{B}}   \newcommand{\clo}{\mathcal{O}}
\newcommand{\clc}{\mathcal{C}}   \newcommand{\clp}{\mathcal{P}}
\newcommand{\cld}{\mathcal{D}}   \newcommand{\clq}{\mathcal{Q}}
\newcommand{\cle}{\mathcal{E}}   \newcommand{\clr}{\mathcal{R}}
\newcommand{\clf}{\mathcal{F}}   \newcommand{\cls}{\mathcal{S}}
\newcommand{\clg}{\mathcal{G}}   \newcommand{\clt}{\mathcal{T}}
\newcommand{\clh}{\mathcal{H}}   \newcommand{\clu}{\mathcal{U}}
\newcommand{\cli}{\mathcal{I}}   \newcommand{\clv}{\mathcal{V}}
\newcommand{\clj}{\mathcal{J}}   \newcommand{\clw}{\mathcal{W}}
\newcommand{\clk}{\mathcal{K}}   \newcommand{\clx}{\mathcal{X}}
\newcommand{\cll}{\mathcal{L}}   \newcommand{\cly}{\mathcal{Y}}
\newcommand{\calm}{\mathcal{M}}  \newcommand{\clz}{\mathcal{Z}}

%---------------------------------------
% Fraktur  Math Fonts :-
%---------------------------------------
\newcommand{\fka}{\mathfrak{A}}   \newcommand{\fkn}{\mathfrak{N}}
\newcommand{\fkb}{\mathfrak{B}}   \newcommand{\fko}{\mathfrak{O}}
\newcommand{\fkc}{\mathfrak{C}}   \newcommand{\fkp}{\mathfrak{P}}
\newcommand{\fkd}{\mathfrak{D}}   \newcommand{\fkq}{\mathfrak{Q}}
\newcommand{\fke}{\mathfrak{E}}   \newcommand{\fkr}{\mathfrak{R}}
\newcommand{\fkf}{\mathfrak{F}}   \newcommand{\fks}{\mathfrak{S}}
\newcommand{\fkg}{\mathfrak{G}}   \newcommand{\fkt}{\mathfrak{T}}
\newcommand{\fkh}{\mathfrak{H}}   \newcommand{\fku}{\mathfrak{U}}
\newcommand{\fki}{\mathfrak{I}}   \newcommand{\fkv}{\mathfrak{V}}
\newcommand{\fkj}{\mathfrak{J}}   \newcommand{\fkw}{\mathfrak{W}}
\newcommand{\fkk}{\mathfrak{K}}   \newcommand{\fkx}{\mathfrak{X}}
\newcommand{\fkl}{\mathfrak{L}}   \newcommand{\fky}{\mathfrak{Y}}
\newcommand{\fkm}{\mathfrak{M}}   \newcommand{\fkz}{\mathfrak{Z}}


\begin{document}
\begin{center}
{\bf School of Mathematics and Physics, UQ}
\end{center}
\centerline{\large\bf MATH2001, Assignment 3, Summer Semester, 2024-2025}

\vspace{3mm}

{\bf Due on 23 January at 14:00AEST.} Each question is marked out of 10 then homogeneously rescaled up to a total marks of 13. {\bf Total marks: $\frac{13}{60}(Q1+Q2+Q3+Q4+Q5+Q6) $}. 

Submit your assignment online via the Assignment 3 submission link in Blackboard. \\
\textbf{Michael Kasumagic, s4430266}

\qs{Polarising Integral}{
    Evaluate the following integral by first converting the integral to polar coordinates
    $$
        \int_{0}^{3}\int_{-\sqrt{9-x^2}}^{0} e^{x^2 + y^2}\ \d y\d x
    $$
}
\sol
\begin{gather*}
    \intertext{From the limits of the integral, we can see that $x$ ranges from 0 to 3, and $y$ ranges from $-\sqrt{9-x^2}$, for some fixed $x$ to 0. The $y$ bound $-\sqrt{9-x^2}$ is particularly interesting, because it corresponds to a bottom semi-circle. Rearranging, we can find the equation of the circle,}
    y = -\sqrt{9 - x^2},\qquad y^2 = 9 - x^2,\qquad x^2 + y^2 = 3^2 \implies r = 3
    \intertext{Now we can make the conversion,}
    x = r\cos\theta \\
    y = r\sin\theta \\
    \longintertext{The bounds of $x$ correspond to the bounds of $r$. So, where $x$ ranges from 0 to 3, $r$ ranges from 0 to 3. \\ The bounds of $y$ correspond with the bounds of $\theta$. So, where $y$ ranges from the bottom of the semi circle, with radius 3; so $\theta=-\pi/2$, to $y=0$, which corresponds to $\theta=0$. \\ The last thing we need to convert our integral, is to note that $e^{x^2+y^2}$ can be rewritten as $e^{r^2}$. \\ Now, we can rewrite the integral.}
    \begin{aligned}
      I = \int_{0}^{3}\int_{-\sqrt{9-x^2}}^{0} e^{x^2 + y^2}\ \d y\d x &= \int_{-\pi/2}^{0}\int_{0}^{3} e^{r^2}r\ \d r\d\theta \\
      \mathclap{\Let u = r^2 \Rightarrow u' = 2r \iff \d u = 2r\d r \iff r\d r = \frac{1}{2}\d u} \\
      \int_{0}^{3} e^{r^2}r\ \d r &= \frac{1}{2}\int_{0^2=0}^{3^2=9} e^u\ \d u \\ 
        &= \frac{1}{2}\sqbracks{\vphantom{\frac{1}{2}}e^u}_0^9 \\
        &= \frac{1}{2}\bracks{e^9 - e^0} \\
        &= \frac{1}{2}\bracks{e^9 - 1} \\
      \tf I &= \frac{1}{2}\bracks{e^9 - 1}\int_{-\pi/2}^{0}\d\theta \\
        &= \frac{1}{2}\bracks{e^9 - 1}\sqbracks{\vphantom{\frac{1}{2}}\theta}_{-\pi/2}^0 \\
      \tf I &= \frac{\pi}{4}\bracks{e^9 - 1} \approx 6363.3618
    \end{aligned}
\end{gather*}

\newpage
\qs{Volume of a Bounded Region Within a Cylinder}{
    Use a triple integral to determine the volume of a the region below $z=6-x$, above $z=-\sqrt{4x^2 + 4y^2}$, inside the cylinder $x^2 + y^2 = 3$ with $x\leq 0$.
}
\sol
\begin{gather*}
  \longintertext{The upper bound of $z$ is the plane $z=6-x$ and its lower bound is the upside-down cone, $z=-\sqrt{4x^2 + 4y^2}=-\sqrt{4\bracks{x^2 + y^2}}$.\\ The cylinder is described by the circle equation $x^2 + y^2 = r^2 = 3$, which implies that the radius of the cylinder is $\sqrt{3}$.\\ These facts make it really natural to express the integral using cylindrical coordinates,}
  \begin{aligned}
    x &= r\cos\theta \\
    y &= r\sin\theta \\
    \d x\d y &= r\d r\d \theta
  \end{aligned}
  \longintertext{$r$ ranges from 0 to $\sqrt{3}$, the radius of the cylinder. $\theta$ ranges from $\pi/2$ to $3\pi/2$, since we're only dealing with the left side of the cylinder, $x\leq0$. Finally, $z$ will range from the bottom to the top of the region (the height, in a sense), namely from $-\sqrt{4r^2}=-2r$ to $6 - r\cos\theta$.\\ Therefore, the volume of the region of interest is given by the triple integral}
  V = \int_{\theta=\pi/2}^{3\pi/2}\int_{r=0}^{\sqrt{3}}\int_{z=-2r}^{6-r\cos\theta} r\ \d z\d r\d\theta.
\end{gather*}
Now, we'll evaluate the integral to find our volume.
\begin{align*}
  V &= \int_{\theta=\pi/2}^{3\pi/2}\int_{r=0}^{\sqrt{3}}r\int_{z=-2r}^{6-r\cos\theta} 1\ \d z\d r\d\theta \\
    &= \int_{\theta=\pi/2}^{3\pi/2}\int_{r=0}^{\sqrt{3}}r \sqbracks{\vphantom{\frac{1}{2}}z}_{-2r}^{6-r\cos\theta}\ \d r\d\theta \\
    &= \int_{\theta=\pi/2}^{3\pi/2}\int_{r=0}^{\sqrt{3}}r \bracks{6-r\cos\theta + 2r}\ \d r\d\theta \\
    &= \int_{\theta=\pi/2}^{3\pi/2}\int_{r=0}^{\sqrt{3}} 6r - r^2\cos\theta + 2r^2\ \d r\d\theta \\
    &= \int_{\theta=\pi/2}^{3\pi/2}\bracks{
      \int_{r=0}^{\sqrt{3}} 6r\d r - 
      \int_{r=0}^{\sqrt{3}}r^2\cos\theta\d r +
      \int_{r=0}^{\sqrt{3}}2r^2\d r
    }\d\theta \\
    &= \int_{\theta=\pi/2}^{3\pi/2}\bracks{
      6\int_{r=0}^{\sqrt{3}} r\d r - 
      \cos\theta\int_{r=0}^{\sqrt{3}}r^2\d r +
      2\int_{r=0}^{\sqrt{3}}r^2\d r
    }\d\theta \\
    &= \int_{\theta=\pi/2}^{3\pi/2}\bracks{
      6\sqbracks{\frac{1}{2}r^2}_{0}^{\sqrt{3}} - 
      \cos\theta\sqbracks{\frac{1}{3}r^3}_{r=0}^{\sqrt{3}} +
      2\sqbracks{\frac{1}{3}r^3}_{r=0}^{\sqrt{3}}
    }\d\theta \\
    &= \int_{\theta=\pi/2}^{3\pi/2}\bracks{
      6\bracks{\frac{3}{2} - \frac{0}{2}} - 
      \cos\theta\bracks{\frac{\sqrt{27}}{3} - \frac{0}{3}} +
      2\bracks{\frac{\sqrt{27}}{3} - \frac{0}{3}}
    }\d\theta \\
    &= \int_{\theta=\pi/2}^{3\pi/2}\bracks{
      6\bracks{\frac{3}{2}} - 
      \cos\theta\bracks{\frac{3\sqrt{3}}{3}} +
      2\bracks{\frac{3\sqrt{3}}{3}}
    }\d\theta \\
    &= \int_{\theta=\pi/2}^{3\pi/2}9 - \sqrt{3}\cos\theta + 2\sqrt{3}\ \d\theta \\
    &= \int_{\theta=\pi/2}^{3\pi/2}9\d\theta - \int_{\theta=\pi/2}^{3\pi/2}\sqrt{3}\cos\theta\d\theta + \int_{\theta=\pi/2}^{3\pi/2}2\sqrt{3}\d\theta \\
    &= 9\int_{\theta=\pi/2}^{3\pi/2}1\d\theta - \sqrt{3}\int_{\theta=\pi/2}^{3\pi/2}\cos\theta\d\theta + 2\sqrt{3}\int_{\theta=\pi/2}^{3\pi/2}1\d\theta \\
    &= 9\sqbracks{\vphantom{\frac{1}{2}}\theta}_{\pi/2}^{3\pi/2} - \sqrt{3}\sqbracks{\vphantom{\frac{1}{2}}\sin\theta}_{\pi/2}^{3\pi/2} + 2\sqrt{3}\sqbracks{\vphantom{\frac{1}{2}}\theta}_{\pi/2}^{3\pi/2} \\
    &= 9\bracks{\frac{3\pi}{2} - \frac{\pi}{2}} - \sqrt{3}\bracks{\sin\frac{3\pi}{2} - \sin\frac{\pi}{2}} + 2\sqrt{3}\bracks{\frac{3\pi}{2} - \frac{\pi}{2}} \\
    &= 9\pi - \sqrt{3}\bracks{-1 - 1} + 2\pi\sqrt{3} \\
    &= 9\pi + 2\sqrt{3} + 2\pi\sqrt{3} \\
  \tf V &= 2\sqrt{3} + \bracks{9 + 2\sqrt{3}}\pi \approx 42.6212\text{ units}^3 \\
\end{align*}

\newpage
\qs{Line Integral Over Vector Field}{
    Evaluate $\int_C \ut{F}\cd\d\ut{r}$ where $\ut{F} = (6x - 2y)\ihat + x^2\jhat$ for each of the following curves.
    \begin{enumerate}[label=(\roman*)]
        \item $C$ is the line segment from $(6,-3)$ to $(0,0)$ followed by the line segment from $(0,0)$ to $(6,3)$.
        \item $C$ is the line segment from $(6,-3)$ to $(6,3)$.
    \end{enumerate}
}
\sol (i) \\
We'll break up the line integral into two for each line segment
$$
    \int_C \ut{F}\cd\d\ut{r} = \int_{C_1}\ut{F}\cd\d\ut{r} + \int_{C_2}\ut{F}\cd\d\ut{r},
$$
where $C_1$ is the segment from $(6,-3)$ to $(0,0)$, and $C_2$ is the segment from $(0,0)$ to $(6,3)$. We can parameterise these line segments, \\
$C_1$:
\begin{align*}
  x &= 6-6t &\qquad y &= -3+3t,\ t\in[0,1] \\
  \dyd[x]{t} = -6 &\Rightarrow \d x = -6\d t &\qquad \dyd[y]{t} = 3 &\Rightarrow \d y = 3\d t
  \intertext{$C_2$:}
  x &= 6t &\qquad y &= 3t,\ t\in[0,1] \\
  \dyd[x]{t} = 6 &\Rightarrow \d x = 6\d t &\qquad \dyd[y]{t} = 3 &\Rightarrow \d y = 3\d t
\end{align*}
Finally, utilising the fact that $\ut{F} = P(x,y)\ihat + Q(x,y)\jhat$, and $\int_C\ut{F}\cdot\d\ut{r} = \int_CP(x,y)\d x + \int_CQ(x,y)\d y$, we can break up the line integral into 4, single variable integrals.
\begin{align*}
  I = \int_C \ut{F}\cd\d\ut{r} &= \int_{C_1}\ut{F}\cd\d\ut{r} + \int_{C_2}\ut{F}\cd\d\ut{r} \\
    &= \int_{C_1}P(x,y)\d x + \int_{C_1}Q(x,y)\d y + \int_{C_2}P(x,y)\d x + \int_{C_2}Q(x,y)\d y \\
    &= \int_{C_1}(6x - 2y)\d x + \int_{C_1}\bracks{x^2}\d y + \int_{C_2}
    (6x - 2y)\d x + \int_{C_2}\bracks{x^2}\d y \\
  \intertext{Now, we'll substitute the approriate $x$ and $y$ and the approriate bounds for $C_1$ and $C_2$, respectively.}
  I &= \int_0^1 -6(6(6-6t) - 2(-3+3t))\d t + \int_0^1 3\bracks{(6-6t)^2}\d t \\ &\phantom{=} + \int_0^1 6(6(6t) - 2(3t))\d t + \int_0^1 3\bracks{(6t)^2}\d t \\
    &= \int_0^1 (-252+252t)\d t + \int_0^1 \bracks{108-216t+108t^{2}}\d t \\ &\phantom{=} + \int_0^1 180t\d t + \int_0^1 108t^2\d t \\
    &= \int_0^1 (-252+252t) + \bracks{108-216t+108t^{2}} + 180t + 108t^2\ \d t \\
    &= \int_0^1 108t^2 + 108t^2 + 252t - 216t + 180t - 252 + 108\ \d t \\
    &= \int_0^1 216t^2 + 216t - 144\ \d t \\
    &= \sqbracks{72t^3 + 108t^2 - 144t}_0^1 \\
    &= 72(1)^3 + 108(1)^2 - 144(1) - 0 \\
    &= 72 + 108 - 144 - 0 \\
  \tf I &= 36
\end{align*}

\sol(ii) \\
We're going to follow a similar process as we did in (i). Except, this time, we only need to parameterise the one line segment, from (6,-3) to (6,3).
\begin{align*}
  x &= 6 &\qquad y &= -3+6t,\ t\in[0,1] \\
  \dyd[x]{t} = 0 &\Rightarrow \d x = 0\d t &\qquad \dyd[y]{t} = 6 &\Rightarrow \d y = 6\d t
\end{align*}
Now, we'll toss these functions into the integral, and evaluate.
\begin{align*}
  I = \int_C \ut{F}\cd\d\ut{r} &= \int_C (6x - 2y)\d x + \int_C x^2\d y \\
    &= \int_0^1 0(6(6) - 2(-3+6t))\d t + \int_0^1 6(6)^2\d t \\
    &= 0 + \int_0^1 216\d t \\
    &= \int_0^1 216\d t \\
    &= \sqbracks{\vphantom{\frac{1}{2}}216t}_0^1 \\
    &= 216(1) - 216(0) \\
  \tf I &= 216
\end{align*}

\newpage
\qs{Finding Potential Function}{
    Find the potential function $f(x,y)$ for the vector field
    $$
        \ut{F} = y^2(1 + \cos(x+y))\ihat + (2xy - 2y + y^2\cos(x+y) + 2y\sin(x+y))\jhat
    $$
    that satisfies $\grad f = \ut{F}$.
}
\sol
Given that $\grad f = \ut{F}$, we can deduce that
$$
  \del{f}{x} = y^2(1 + \cos(x+y))\qquad \del{f}{y} = 2xy - 2y + y^2\cos(x+y) + 2y\sin(x+y)
$$
To find our potential function $f(x,y)$ then, we'll start by integrating its partial derivative with respect to $x$, with respect to $x$.
\begin{align*}
  f(x,y) = \int\del{f}{x}\d x &= \int y^2(1 + \cos(x+y))\d x \\
    &= y^2 \int 1\d x + y^2\int\cos(x+y)\d x \\
  \tf f(x,y) &= xy^2 + y^2\sin(x+y) + g(y) \\
  \intertext{Next, we'll differentiate this expression with respect to $y$, and compare it to the other expression of $\deld{f}{y}$, which is an equivalent.}
  \del{f}{y} &= \del{}{y}\bracks{xy^2 + y^2\sin(x+y) + g(y)} \\
    &= 2xy + 2y\sin(x+y) + y^2\cos(x+y) + g'(y) \\
    &= 2xy + 2y\sin(x+y) + y^2\cos(x+y) - 2y \\
  \implies g'(y) &= -2y \\
  \iff g(y) &= -2\int y\d y \\
    &= -y^2 + A_i,
\end{align*}
where $A_i$ is an arbitrary constant. Therefore,
$$
  f(x,y) = 2xy - y^2 + 2y\sin(x+y) + y^2\cos(x+y) + A_i.
$$

\newpage
\qs{Flux Over Surface of Bounded Solid}{
    Evaluate $\iint_S\ut{F}\cd\d S$ where $\ut{F} = y\ihat + 2x\jhat + (z-8)\khat$ and $S$ is the surface of the solid bounded by $4x + 2y + z = 8,\ z=0,\ y=0$ and $x=0$ with the positive orientation. Note that all four surfaces of the solid are included in $S$.
}
\sol \\
Since these bounds bound a solid region, we can apply the divergence theorem to find the flux over the surface of solid.
$$
  \iint_S \ut{F}\cd\d\ut{S} = \iiint_V (\grad\cd\ut{F})\d V
$$
First, we'll calculate $\Div\ut{F}$
\begin{align*}
  \grad\cd\ut{F} &= \del{}{x}(y) + \del{}{y}(2x) + \del{}{z}(z-8) \\
    &= 0 + 0 + 1 \\
    &= 1
\end{align*}
Next, we're going to assess the vertices which bound the solid. First, we note that the three planes $x=0,\ y=0,\ z=0$ all intersect at the vertex $(0,0,0)$. If we hold $x$ and $y$ at 0, $z$ will reach its maximum of 8, so $(0,0,8)$ is a vertex. If $x$ and $z$ are held at 0, $y$ reaches its maximum at 4, so $(0,4,0)$ is a vertex. Finally, if $y$ and $z$ are held at 0, $x$ reaches its maximum at 2, so $(2,0,0)$ is the last vertex. \\
Thus, the solid, $V$, is bound by the verticies
$$
  (0,0,0),\quad (2,0,0),\quad (0,4,0),\quad (0,0,8)
$$
The volume of the solid is given by halving the scalar triple product,
$$
  \frac{1}{2}\cd\frac{1}{3}(\ut{a}\times\ut{b})\cd\ut{c} = \frac{1}{6}\begin{vmatrix} 2 & 0 & 0 \\ 0 & 4 & 0 \\ 0 & 0 & 8 \end{vmatrix} = \frac{1}{6}(2\cd4\cd8) = \frac{64}{6} = \frac{32}{3}
$$
By the divergence theorem then,
$$
  I = \iint_S \ut{F}\cd\d\ut{S} = \iiint_V (\grad\cd\ut{F})\d V = \iiint_V 1\d V = \frac{32}{3} \approx 10.6667
$$

\newpage
\qs{Divergence Theorem}{
    Use the Divergence Theorem to evaluate $\iint_S\ut{F}\cd\d S$ where $\ut{F} = 2xz\ihat + (1-4xy^2)\jhat + (2z-z^2)\khat$ and $S$ is the surface of the solid bounded by $z=6-2x^2-2y^2$ and the plane $z=0$. Note that both of the surfaces of this solid are included in $S$.
}
\sol
Divergence theorem states that
$$
  \iint_S \ut{F}\cd\d\ut{S} = \iiint_V (\grad\cd\ut{F})\d V
$$
We'll start by calculating $\Div\ut{F}$,
\begin{align*}
  \grad\cd\ut{F} &= \del{}{x}(2xz) + \del{}{y}(1-4xy^2) + \del{}{z}(2z-z^2) \\
    &= 2z - 8xy + 2 - 2z \\
    &= 2 - 8xy
\end{align*}
We'll use cylindrical coordinates,
$$
  x^2 + y^2 = r^2,\quad x = r\cos\theta,\quad y = r\sin\theta,\quad z = h,\quad \d V = r\ \d r\d\theta\d h
$$
$r$ will range between $0$ and $0=6-2r^2\iff r^2 = 3 \iff r = \sqrt{3}$. \\
$\theta$ will range between $0$ and $2\pi$, the full circular rotation. \\
$h$ will range between $0$ and the bounding surface $6 - 2x^2 - 2y^2 = 6 - 2(x^2+y^2) = 6 - 2r^2$.
$$
  \tf \iiint_V (\grad\cd\ut{F})\d V = \iiint_V (2 - 8xy)\d V = \iiint_V (2 - 8r^2\cos\theta\sin\theta)r\ \d r\d\theta\d h
$$
Let's go ahead and evaluate this integral
\begin{align*}
  I = \iint_S \ut{F}\cd\d\ut{S} &= \iiint_V (\grad\cd\ut{F})\d V \\
    &= \iiint_V (2 - 8r^2\cos\theta\sin\theta)r\ \d r\d\theta\d h \\
    &= \iiint_V 2r - 8r^3\cos\theta\sin\theta\ \d r\d\theta\d h \\
    &= \int_{0}^{2\pi}\int_{0}^{\sqrt{3}}\int_{0}^{6-2r^2} 2r - 8r^3\cos\theta\sin\theta\ \d h\d r\d\theta \\
    &= \int_{0}^{2\pi}\int_{0}^{\sqrt{3}}\sqbracks{\vphantom{\frac{1}{2}}2hr - 8hr^3\cos\theta\sin\theta}_{0}^{6-2r^2}\d r\d\theta \\
    &= \int_{0}^{2\pi}\int_{0}^{\sqrt{3}}\bracks{2(6-2r^2)r - 8(6-2r^2)r^3\cos\theta\sin\theta - \bracks{2(0)r - 8(0)r^3\cos\theta\sin\theta}}\d r\d\theta \\
    &= \int_{0}^{2\pi}\int_{0}^{\sqrt{3}}\bracks{2(6-2r^2)r - 8(6-2r^2)r^3\cos\theta\sin\theta}\d r\d\theta \\
    &= \int_{0}^{2\pi}\int_{0}^{\sqrt{3}}\bracks{12r-4r^3 + 16r^5\cos\theta\sin\theta - 48r^3\cos\theta\sin\theta}\d r\d\theta \\
    &= \int_{0}^{2\pi}\sqbracks{6r^2 - r^4 + \frac{8}{3}r^6\cos\theta\sin\theta - 12r^4\cos\theta\sin\theta}_{0}^{\sqrt{3}}\d\theta \\
    &= \int_{0}^{2\pi}\bracks{6(\sqrt{3})^2 - (\sqrt{3})^4 + \frac{8}{3}(\sqrt{3})^6\cos\theta\sin\theta - 12(\sqrt{3})^4\cos\theta\sin\theta}\d\theta \\
    &= \int_{0}^{2\pi}\bracks{18 - 9 + 72\cos\theta\sin\theta - 108\cos\theta\sin\theta}\d\theta \\
    &= \int_{0}^{2\pi} 9 - 36\cos\theta\sin\theta\ \d\theta \\
    &= \int_{0}^{2\pi} 9 - 18\cd 2\cos\theta\sin\theta\ \d\theta \\
    &= \int_{0}^{2\pi} 9 - 18\sin2\theta\ \d\theta \\
    &= \sqbracks{\vphantom{\frac{1}{2}}9\theta + 9\cos2\theta}_{0}^{2\pi} \\
    &= 9(2\pi) + 9\cos(2(2\pi)) - 9(0) - 9\cos(2(0)) \\
    &= 18\pi + 9 - 0 - 9 \\
  \tf I &= 18\pi \approx 56.5487\\
\end{align*}


\end{document}
