\documentclass{report}

\input{../../latex_template/preamble}
%From M275 "Topology" at SJSU
\newcommand{\id}{\mathrm{id}}
\newcommand{\taking}[1]{\xrightarrow{#1}}
\newcommand{\inv}{^{-1}}

%From M170 "Introduction to Graph Theory" at SJSU
\DeclareMathOperator{\diam}{diam}
\DeclareMathOperator{\ord}{ord}
\newcommand{\defeq}{\overset{\mathrm{def}}{=}}

%From the USAMO .tex files
\newcommand{\ts}{\textsuperscript}
\newcommand{\dg}{^\circ}
\newcommand{\ii}{\item}

% % From Math 55 and Math 145 at Harvard
% \newenvironment{subproof}[1][Proof]{%
% \begin{proof}[#1] \renewcommand{\qedsymbol}{$\blacksquare$}}%
% {\end{proof}}

\newcommand{\liff}{\leftrightarrow}
\newcommand{\lthen}{\rightarrow}
\newcommand{\opname}{\operatorname}
\newcommand{\surjto}{\twoheadrightarrow}
\newcommand{\injto}{\hookrightarrow}
\newcommand{\On}{\mathrm{On}} % ordinals
\DeclareMathOperator{\img}{im} % Image
\DeclareMathOperator{\Img}{Im} % Image
\DeclareMathOperator{\coker}{coker} % Cokernel
\DeclareMathOperator{\Coker}{Coker} % Cokernel
\DeclareMathOperator{\Ker}{Ker} % Kernel
\DeclareMathOperator{\rank}{rank}
\DeclareMathOperator{\Spec}{Spec} % spectrum
\DeclareMathOperator{\Tr}{Tr} % trace
\DeclareMathOperator{\pr}{pr} % projection
\DeclareMathOperator{\ext}{ext} % extension
\DeclareMathOperator{\pred}{pred} % predecessor
\DeclareMathOperator{\dom}{dom} % domain
\DeclareMathOperator{\ran}{ran} % range
\DeclareMathOperator{\Hom}{Hom} % homomorphism
\DeclareMathOperator{\Mor}{Mor} % morphisms
\DeclareMathOperator{\End}{End} % endomorphism

\newcommand{\eps}{\epsilon}
\newcommand{\veps}{\varepsilon}
\newcommand{\ol}{\overline}
\newcommand{\ul}{\underline}
\newcommand{\wt}{\widetilde}
\newcommand{\wh}{\widehat}
\newcommand{\vocab}[1]{\textbf{\color{blue} #1}}
\providecommand{\half}{\frac{1}{2}}
\newcommand{\dang}{\measuredangle} %% Directed angle
\newcommand{\ray}[1]{\overrightarrow{#1}}
\newcommand{\seg}[1]{\overline{#1}}
\newcommand{\arc}[1]{\wideparen{#1}}
\DeclareMathOperator{\cis}{cis}
\DeclareMathOperator*{\lcm}{lcm}
\DeclareMathOperator*{\argmin}{arg min}
\DeclareMathOperator*{\argmax}{arg max}
\newcommand{\cycsum}{\sum_{\mathrm{cyc}}}
\newcommand{\symsum}{\sum_{\mathrm{sym}}}
\newcommand{\cycprod}{\prod_{\mathrm{cyc}}}
\newcommand{\symprod}{\prod_{\mathrm{sym}}}
\newcommand{\Qed}{\begin{flushright}\qed\end{flushright}}
\newcommand{\parinn}{\setlength{\parindent}{1cm}}
\newcommand{\parinf}{\setlength{\parindent}{0cm}}
% \newcommand{\norm}{\|\cdot\|}
\newcommand{\inorm}{\norm_{\infty}}
\newcommand{\opensets}{\{V_{\alpha}\}_{\alpha\in I}}
\newcommand{\oset}{V_{\alpha}}
\newcommand{\opset}[1]{V_{\alpha_{#1}}}
\newcommand{\lub}{\text{lub}}
\newcommand{\del}[2]{\frac{\partial #1}{\partial #2}}
\newcommand{\Del}[3]{\frac{\partial^{#1} #2}{\partial^{#1} #3}}
\newcommand{\deld}[2]{\dfrac{\partial #1}{\partial #2}}
\newcommand{\Deld}[3]{\dfrac{\partial^{#1} #2}{\partial^{#1} #3}}
\newcommand{\lm}{\lambda}
\newcommand{\uin}{\mathbin{\rotatebox[origin=c]{90}{$\in$}}}
\newcommand{\usubset}{\mathbin{\rotatebox[origin=c]{90}{$\subset$}}}
\newcommand{\lt}{\left}
\newcommand{\rt}{\right}
\newcommand{\bs}[1]{\boldsymbol{#1}}
\newcommand{\exs}{\exists}
\newcommand{\st}{\strut}
\newcommand{\dps}[1]{\displaystyle{#1}}

\newcommand{\sol}{\setlength{\parindent}{0cm}\textbf{\textit{Solution:}}\setlength{\parindent}{1cm} }
\newcommand{\solve}[1]{\setlength{\parindent}{0cm}\textbf{\textit{Solution: }}\setlength{\parindent}{1cm}#1 \Qed}

\DeclareMathOperator{\sech}{sech}
\DeclareMathOperator{\csch}{csch}
\DeclareMathOperator{\arcsec}{arcsec}
\DeclareMathOperator{\arccsc}{arccsc}
\DeclareMathOperator{\arccot}{arccot}
\DeclareMathOperator{\arsinh}{arsinh}
\DeclareMathOperator{\arcosh}{arcosh}
\DeclareMathOperator{\artanh}{artanh}
\DeclareMathOperator{\arcsch}{arcsch}
\DeclareMathOperator{\arsech}{arsech}
\DeclareMathOperator{\arcoth}{arcoth}

\newcommand{\sinx}{\sin x}          \newcommand{\arcsinx}{\arcsin x}    
\newcommand{\cosx}{\cos x}          \newcommand{\arccosx}{\arccosx}
\newcommand{\tanx}{\tan x}          \newcommand{\arctanx}{\arctan x}
\newcommand{\cscx}{\csc x}          \newcommand{\arccscx}{\arccsc x}
\newcommand{\secx}{\sec x}          \newcommand{\arcsecx}{\arcsec x}
\newcommand{\cotx}{\cot x}          \newcommand{\arccotx}{\arccot x}
\newcommand{\sinhx}{\sinh x}          \newcommand{\arsinhx}{\arsinh x}
\newcommand{\coshx}{\cosh x}          \newcommand{\arcoshx}{\arcosh x}
\newcommand{\tanhx}{\tanh x}          \newcommand{\artanhx}{\artanh x}
\newcommand{\cschx}{\csch x}          \newcommand{\arcschx}{\arcsch x}
\newcommand{\sechx}{\sech x}          \newcommand{\arsechx}{\arsech x}
\newcommand{\cothx}{\coth x}          \newcommand{\arcothx}{\arcoth x}
\newcommand{\lnx}{\ln x}
\newcommand{\expx}{\exp x}

\newcommand{\bba}{\mathbb{A}}   \newcommand{\bbn}{\mathbb{N}}
\newcommand{\bbb}{\mathbb{B}}   \newcommand{\bbo}{\mathbb{O}}
\newcommand{\bbc}{\mathbb{C}}   \newcommand{\bbp}{\mathbb{P}}
\newcommand{\bbd}{\mathbb{D}}   \newcommand{\bbq}{\mathbb{Q}}
\newcommand{\bbe}{\mathbb{E}}   \newcommand{\bbr}{\mathbb{R}}
\newcommand{\bbf}{\mathbb{F}}   \newcommand{\bbs}{\mathbb{S}}
\newcommand{\bbg}{\mathbb{G}}   \newcommand{\bbt}{\mathbb{T}}
\newcommand{\bbh}{\mathbb{H}}   \newcommand{\bbu}{\mathbb{U}}
\newcommand{\bbi}{\mathbb{I}}    \newcommand{\bbv}{\mathbb{V}}
\newcommand{\bbj}{\mathbb{J}}   \newcommand{\bbw}{\mathbb{W}}
\newcommand{\bbk}{\mathbb{K}}   \newcommand{\bbx}{\mathbb{X}}
\newcommand{\bbl}{\mathbb{L}}    \newcommand{\bby}{\mathbb{Y}}
\newcommand{\bbm}{\mathbb{M}}   \newcommand{\bbz}{\mathbb{Z}}

\newcommand{\lb}{\left(}
\newcommand{\rb}{\right)}
\newcommand{\lbr}{\left\lbrace}
\newcommand{\rbr}{\right\rbrace}
\newcommand{\lsb}{\left[}
\newcommand{\rsb}{\right]}
\newcommand{\suchthat}{\medspace\middle|\medspace}
\newcommand{\bracks}[1]{\lb #1 \rb}
\newcommand{\braces}[1]{\lbr #1 \rbr}
\newcommand{\sqbracks}[1]{\lsb #1 \rsb}

\renewcommand{\floor}[1]{\lfloor #1 \rfloor}
\renewcommand{\ceil}[1]{\lceil #1 \rceil}

\newcommand{\cd}{\cdot}
\newcommand{\tf}{\therefore}
\newcommand{\Let}{\text{Let }}
\newcommand{\Given}{\text{Given }}
\newcommand{\Suppose}{\text{Suppose }}
\newcommand{\WeSee}{\text{We see }}
\newcommand{\So}{\text{So }}

\newcommand{\QED}{\hfill \qed}

\renewcommand{\dd}[1]{\frac{d}{d#1}}
\newcommand{\dyd}[2][y]{\frac{d#1}{d#2}}

\newcommand{\ddx}{\dd{x}}       \newcommand{\ddxsq}{\dyd[^2]{x^2}}
\newcommand{\ddy}{\dd{y}}       \newcommand{\ddysq}{\dyd[^2]{y^2}}
\newcommand{\ddu}{\dd{u}}       \newcommand{\ddusq}{\dyd[^2]{u^2}}
\newcommand{\ddv}{\dd{v}}       \newcommand{\ddvsq}{\dyd[^2]{v^2}}

\newcommand{\dydx}{\dyd{x}}     \newcommand{\dydxsq}{\dyd[^2y]{x^2}}
\newcommand{\dfdx}{\dyd[f]{x}}  \newcommand{\dfdxsq}{\dyd[^2f]{x^2}}
\newcommand{\dudx}{\dyd[u]{x}}  \newcommand{\dudxsq}{\dyd[^2u]{x^2}}
\newcommand{\dvdx}{\dyd[v]{x}}  \newcommand{\dvdxsq}{\dyd[^2v]{x^2}}

% Mathfrak primes
\newcommand{\km}{\mathfrak{m}}
\newcommand{\kp}{\mathfrak{p}}
\newcommand{\kq}{\mathfrak{q}}

%---------------------------------------
% Blackboard Math Fonts :-
%---------------------------------------
\newcommand{\bba}{\mathbb{A}}   \newcommand{\bbn}{\mathbb{N}}
\newcommand{\bbb}{\mathbb{B}}   \newcommand{\bbo}{\mathbb{O}}
\newcommand{\bbc}{\mathbb{C}}   \newcommand{\bbp}{\mathbb{P}}
\newcommand{\bbd}{\mathbb{D}}   \newcommand{\bbq}{\mathbb{Q}}
\newcommand{\bbe}{\mathbb{E}}   \newcommand{\bbr}{\mathbb{R}}
\newcommand{\bbf}{\mathbb{F}}   \newcommand{\bbs}{\mathbb{S}}
\newcommand{\bbg}{\mathbb{G}}   \newcommand{\bbt}{\mathbb{T}}
\newcommand{\bbh}{\mathbb{H}}   \newcommand{\bbu}{\mathbb{U}}
\newcommand{\bbi}{\mathbb{I}}   \newcommand{\bbv}{\mathbb{V}}
\newcommand{\bbj}{\mathbb{J}}   \newcommand{\bbw}{\mathbb{W}}
\newcommand{\bbk}{\mathbb{K}}   \newcommand{\bbx}{\mathbb{X}}
\newcommand{\bbl}{\mathbb{L}}   \newcommand{\bby}{\mathbb{Y}}
\newcommand{\bbm}{\mathbb{M}}   \newcommand{\bbz}{\mathbb{Z}}

%---------------------------------------
% Roman Math Fonts :-
%---------------------------------------
\newcommand{\rma}{\mathrm{A}}   \newcommand{\rmn}{\mathrm{N}}
\newcommand{\rmb}{\mathrm{B}}   \newcommand{\rmo}{\mathrm{O}}
\newcommand{\rmc}{\mathrm{C}}   \newcommand{\rmp}{\mathrm{P}}
\newcommand{\rmd}{\mathrm{D}}   \newcommand{\rmq}{\mathrm{Q}}
\newcommand{\rme}{\mathrm{E}}   \newcommand{\rmr}{\mathrm{R}}
\newcommand{\rmf}{\mathrm{F}}   \newcommand{\rms}{\mathrm{S}}
\newcommand{\rmg}{\mathrm{G}}   \newcommand{\rmt}{\mathrm{T}}
\newcommand{\rmh}{\mathrm{H}}   \newcommand{\rmu}{\mathrm{U}}
\newcommand{\rmi}{\mathrm{I}}   \newcommand{\rmv}{\mathrm{V}}
\newcommand{\rmj}{\mathrm{J}}   \newcommand{\rmw}{\mathrm{W}}
\newcommand{\rmk}{\mathrm{K}}   \newcommand{\rmx}{\mathrm{X}}
\newcommand{\rml}{\mathrm{L}}   \newcommand{\rmy}{\mathrm{Y}}
\newcommand{\rmm}{\mathrm{M}}   \newcommand{\rmz}{\mathrm{Z}}

%---------------------------------------
% Calligraphic Math Fonts :-
%---------------------------------------
\newcommand{\cla}{\mathcal{A}}   \newcommand{\cln}{\mathcal{N}}
\newcommand{\clb}{\mathcal{B}}   \newcommand{\clo}{\mathcal{O}}
\newcommand{\clc}{\mathcal{C}}   \newcommand{\clp}{\mathcal{P}}
\newcommand{\cld}{\mathcal{D}}   \newcommand{\clq}{\mathcal{Q}}
\newcommand{\cle}{\mathcal{E}}   \newcommand{\clr}{\mathcal{R}}
\newcommand{\clf}{\mathcal{F}}   \newcommand{\cls}{\mathcal{S}}
\newcommand{\clg}{\mathcal{G}}   \newcommand{\clt}{\mathcal{T}}
\newcommand{\clh}{\mathcal{H}}   \newcommand{\clu}{\mathcal{U}}
\newcommand{\cli}{\mathcal{I}}   \newcommand{\clv}{\mathcal{V}}
\newcommand{\clj}{\mathcal{J}}   \newcommand{\clw}{\mathcal{W}}
\newcommand{\clk}{\mathcal{K}}   \newcommand{\clx}{\mathcal{X}}
\newcommand{\cll}{\mathcal{L}}   \newcommand{\cly}{\mathcal{Y}}
\newcommand{\calm}{\mathcal{M}}  \newcommand{\clz}{\mathcal{Z}}

%---------------------------------------
% Fraktur  Math Fonts :-
%---------------------------------------
\newcommand{\fka}{\mathfrak{A}}   \newcommand{\fkn}{\mathfrak{N}}
\newcommand{\fkb}{\mathfrak{B}}   \newcommand{\fko}{\mathfrak{O}}
\newcommand{\fkc}{\mathfrak{C}}   \newcommand{\fkp}{\mathfrak{P}}
\newcommand{\fkd}{\mathfrak{D}}   \newcommand{\fkq}{\mathfrak{Q}}
\newcommand{\fke}{\mathfrak{E}}   \newcommand{\fkr}{\mathfrak{R}}
\newcommand{\fkf}{\mathfrak{F}}   \newcommand{\fks}{\mathfrak{S}}
\newcommand{\fkg}{\mathfrak{G}}   \newcommand{\fkt}{\mathfrak{T}}
\newcommand{\fkh}{\mathfrak{H}}   \newcommand{\fku}{\mathfrak{U}}
\newcommand{\fki}{\mathfrak{I}}   \newcommand{\fkv}{\mathfrak{V}}
\newcommand{\fkj}{\mathfrak{J}}   \newcommand{\fkw}{\mathfrak{W}}
\newcommand{\fkk}{\mathfrak{K}}   \newcommand{\fkx}{\mathfrak{X}}
\newcommand{\fkl}{\mathfrak{L}}   \newcommand{\fky}{\mathfrak{Y}}
\newcommand{\fkm}{\mathfrak{M}}   \newcommand{\fkz}{\mathfrak{Z}}


\title{\Huge{MATH2001}\\Calculus \& Linear Algebra II}
\author{\huge{Michael Kasumagic, s4430266}}
\date{\huge{Summer Semester, 2024-25}}

\begin{document}

\maketitle
\newpage% or \cleardoublepage
% \pdfbookmark[<level>]{<title>}{<dest>}
\pdfbookmark[section]{\contentsname}{toc}
\tableofcontents
\pagebreak

\chapter{Week 1}
\section{Lecture 1}
In this course we will cover four major topics:
\begin{itemize}
  \item Ordinary Differential Equations
  \item Linear Algebra
  \item Vector Calculus
  \item Integral Calculus
\end{itemize}

\subsection{Solutions to First Order ODEs}
We are comfortable solving three types of first order ODEs by now:
\begin{itemize}
  \item Directly integrable: $\dps{\dydx = f(x)}$
  $$
    y(x) = \int f(x) \d x = F(x) + C
  $$
  \item Seperable: $\dps{\dydx = f(x)g(y)}$
  $$
    \frac{1}{g(y)}\dydx = f(x) \iff \int \frac{1}{g(y)}\frac{\d y}{\cancel{\d x}}\cancel{\d x} = \int f(x) \d x \iff G(y(x)) = F(x) + C
  $$
  $$
    \text{If } G \text{ is invertible, then } y(x) = G\inv(F(x)+C) 
  $$
  \item Linear: $\dps{\dydx = q(x) - p(x)y}$
  $$
    \Let\mu = \exp\bracks{\int p(x)\d x}\implies \mu\dydx + \mu p(x)y = \mu q(x) \iff \ddx\bracks{\mu y} = \mu q(x) \iff y(x) = \frac{1}{\mu(x)}\int \mu q(x)\d x
  $$
\end{itemize}

In many applications, we need to solve an IVP. In general this is an equation of form,
$$
  \dydx = f(x,y),\quad y(x_0) = y_0.
$$
In other words, we seek to find solutions to the ODE which pass through the point $(x_0, y_0)$ in the $x{-}y$ plane.

\ex{}{
  $$
    \dydx = x,\quad y(0)=1\text{ has a unique solution:}
  $$
  \begin{align*}
    \dydx &= x \\
    y(x) &= \frac{1}{2}x^2 + C \\
    \text{Impose } y(0) &= 1 \\ 
    \tf 1 &= \frac{1}{2}(0)^2 + C \\
    \tf C &= 1 \\
    \tf y(x) &= \frac{1}{2}x^2 + 1 
  \end{align*}
}

\ex{}{
  $$
    \dydx = 3xy^{1/3},\quad y(0)=0\text{ has more than one solution:}
  $$
  \begin{align*}
    y^{-1/3}\dydx &= 3x \\
    \int y^{-1/3}\dydx\d x &= \int 3x\d x \\
    \int y^{-1/3}\d y &= \int 3x\d x \\
    \frac{3}{2}y^{2/3} + C_1 &= \frac{3}{2}x^2 + C_2 \\
    y^{2/3} &= x^2 + C \\
    \text{Impose } y(0) &= 0 \\
    0^{2/3} &= 0^2 + C \\ 
    \implies C &= 0 \\
    \tf y^{2/3} &= x^2 \\
    \tf y &= \pm x^3
  \end{align*}
  This is problematic. Our inital value constraint hasn't allowed us to pick one particular solution. 
}
\nt{
  The previous IVP has multiple solutions because $f(x,y)=3xy^{1/3}$ is not differentiable at $y=0$.
}

\ex{}{
  $$
    \dydx = \frac{x-y}{x},\quad y(0)=1\text{ has no solutions:}
  $$
  \begin{align*}
    \dydx &= \frac{x}{x} - \frac{1}{x}y \\
      &= q(x) - p(x)y \\
    \dydx + p(x)y &= 1 \\
    \mu &= \exp\bracks{\int p(x)\d x} \\
      &= \exp\bracks{\int \frac{1}{x}\d x} \\
      &= \exp(\ln(x)) \\
      &= x \\
    \mu\dydx + \mu p(x)y &= \mu \\
    x\dydx + y &= x \\
    \ddx\bracks{xy} &= x \\
    xy &= \int x \d x \\ 
      &= \frac{1}{2}x^2 + C \\
    \Impose y(0) = 1 \\
    \tf 0\cd 1 &= \frac{1}{2}(0)^2 + C \\
    C &= 0 \\
    \tf y(x) &= \frac{1}{2}x
  \end{align*}
  However, our general solution \textbf{does not} satisfy our inital value constraint, $y(0)=\frac{1}{2}(0) = 0 \neq 1$.
} 
\nt{
  Our IVP doesn't have a solution because $f(x,y)=\frac{x-y}{x}$ is not differentiable or continuous around $x=0$.\\

  We're kind of loosley referring to ``existence and uniqueness'' theorems, or Picard-Lindel\"of Theorem, which generally states:
  $$
    \text{The IVP } \dydx = f(x,y)\quad y(x_0)=y_0 
  $$
  has a unique solution around $x_0$ if:
  \begin{enumerate}
    \item $f(x,y)$ is continuous around $(x_0,y_0)$
    \item $f(x,y)$ is differentiable with respect to $y$ around $(x_0,y_0)$, ie $\del{f}{y}$ is continuous aorund $(x_0,y_0)$.
  \end{enumerate}
}

\subsection{Existence and Uniqueness}
\thm{}{
  Consider the IVP
  $$
    \dydx = f(x,y),\quad y(x_0)=y_0
  $$
  We are concerned with the conditions under which a solution exists and is unique.
  \begin{enumerate}
    \item (existence, Peano's Theorem) If $f(x,y)$ is continuous in some rectangle
    $$
      R = \braces{(x,y)\suchthat \abs{x-x_0}<a,\ \abs{y-y_0}<b}
    $$
    then the IVP has at least one solution.
    \item (uniqueness, Picard's Theorem) If $f_y(x,y):=\del{f}{y}$ is also continuous in $R$ then there is some interval $\abs{x-x_0}\leq h\leq a$ which contains at least one solution.
  \end{enumerate}  
}
This result only tells us that a solution exists or is unique locally. Beyond $R$, we simply don't know.

\ex{}{
  $$
    \dydx = x,\quad y(0)=1
  $$
  \begin{align*}
    f(x,y) &= x \\
    f_y(x,y) &= 0
  \end{align*}
  These functions are both continuous over $\bbr^2$. Therefore there exists a unique solution \\

  $$
    \dydx = 3xy^{1/3},\quad y(0)=0\text{ has more than one solution:}
  $$
  \begin{align*}
    f(x,y) &= 3xy^{1/3} \\
    f_y(x,y) &= xy^{-2/3}
  \end{align*}
  $f(x,y)$ is continuous over $\bbr^2$ so there exists at least one solution. However, $f_y$ has a discontinuity at $y=0$, so there may or may not be unique solutions (remember, its not an iff).

  $$
    \dydx = \frac{x-y}{x},\quad y(0)=1\text{ has no solutions:}
  $$
  \begin{align*}
    f(x,y) &= \frac{x-y}{x} \\
    f_y(x,y) &= -\frac{1}{x}
  \end{align*}
  $f(x,y)$ and $f_y$ both have discontinuities when $x=0$, so we don't know from this test if there are solutions, or if the solution is unique.
}

\nt{
  These theorems are not if and only if's. They can fail. For example, take the IVP
  $$
    \dydx = \frac{1}{3}x^{-2/3},\quad y(0)=1/
  $$
  We see
  $$
    f(x,y) = \frac{1}{3}x^{-2/3},\qquad f_y(x,y) = 0
  $$
  $f$ has a discontinuity when $x=0$, so the theorem's fail to identify if this IVP has solutions. However, this IVP \textbf{does} have a unique solution,
  $$
    y(x) = x^{1/3} + 1,
  $$
  so we need to be careful we're using these theorems correctly. \textbf{If} $f$ and $f_y$ are continuous in some reigon \textbf{then} there exists a unique solution in that reigon.
}

\ex{}{
  Solve these:
  \begin{enumerate}
    \item $y' = y^{2/3},\quad y(0)=1$
    \begin{align*}
      f(x,y) &= y^{2/3} \\
      f_y(x,y) &= \frac{2}{3}y^{-1/3}
      \intertext{Therefore there exist at least one solution to the IVP.}
      y^{-2/3}y' &= 1 \\
      \int y^{-2/3}\d y &= \int 1\d x \\
      3 y^{1/3} &= x + C \\
      y^{1/3} &= \frac{1}{3}(x+C) \\
      y &= \frac{1}{27}(x+C)^3 \\
      \Impose y(0) = 1
    \end{align*}
    Imposing the IVP and expanding the cubic expression, will reveal 3 values for C, the nicest of which is 3. The one which satisfies our IVP is
    $$
      y(x) = \frac{1}{27}(x+3)^3
    $$
    Even though those other solutions exist, only one satisfies the IVP, hence this solution is unique.
    \item $y' = (3x^2+4x+2)/(2y-2),\quad y(0)=1$
    \begin{align*}
      f(x,y) &= \frac{3x^2 + 4x + 2}{2(y-1)} \\
      \intertext{Because of the discontinuity at $y=1$, our exisitence theorem fails to identify if solutions exist.}
      y' &= \frac{3x^2 + 4x + 2}{2(y-1)} \\
      2(y-1)y' &= 3x^2 + 4x + 2 \\ 
      2\int y-1\d y &= \int 3x^2 + 4x + 2\ \d x \\
      y^2-2y &= x^3 + 2x^2 + 2x + C \\
      y^2-2y+1 &= x^3 + 2x^2 + 2x + C + 1 \\
      (y-1)^2 &= x^3 + 2x^2 + 2x + C + 1 \\
      \Impose y(0) &= 1 \\
      ((1)-1)^2 &= (0)^3 + 2(0)^2 + 2(0) + C + 1 \\
      \iff C &= -1 \\
      \tf (y-1)^2 &= x^3 + 2x^2 + 2x \\
      \tf y(x) &= 1\pm\sqrt{x^3 + 2x^2 + 2x}
    \end{align*}
    The IVP has two solutions.
  \end{enumerate}
}

\subsection{Method of Succesive Approximations}
To start, we note that it is always possible to apply a variable shift and so that the IVP is expressed:
$$
  \dydx = f(x,y),\qquad y(0)=0
$$
\ex{}{
  $$
    y' = 2(x-1)(y-1),y(1)=2
  $$
  \begin{gather*}
    \Let \bar{x} = x-1 \\
    \Let \bar{y} = y-2 \\
    \So \dydx = \dyd[\bar{y}]{\bar{x}} \\
    \implies \dyd[\bar{y}]{\bar{x}} = 2\bar{x}(\bar{y}+1),\ \bar{y}(0)=0
  \end{gather*}
}
Without loss of generality we will consider this problem where the inital point is at the origin. We can restate the previous theorem 1.1.1 as follows
\thm{}{
  If $f$ and $f_y$ are continuous in some rectangle
  $$
    R = \braces{(x,y)\suchthat \abs{x}\leq a,\ \abs{y}\leq b},
  $$
  then there is some interval $\abs{x}\leq h\leq a$ which contains a uniuqe solution $y=\phi(x)$ of the IVP
  $$
    \dydx = f(x,y),\qquad y(0)=0
  $$
}

\subsubsection*{Equivilance with integral equation}
Let $y=\phi(x)$ be the solution to the IVP 
\[
  \dydx=f(x,y),\qquad y(0)=0, \tag*{(1)}
\]
and note that the function $F(x) = f(x,\phi(x))$ is a continuous function of $x$ only. We then have
\[
  \phi(x) = \int_{0}^{x} F(t)\d t = \int_{0}^{x} f(t,\phi(t))\d t. \tag*{(2)}
\]
Note that $\phi(0)=0$. This is an example of an \textit{integral equation}. Conversely, let $\phi(x)$ satisfy the integral equation (2). By the Fundamental Theorem of Integral Calculus, $\phi'(x) = f (x, \phi(x))$, which implies that $y = \phi(x)$ is a solution of the IVP (1). In other words, the IVP (1) and the integral equation (2) are equivalent, meaning that a solution of one is a solution of the other. Herein we work with (2).

\subsubsection*{Method of succesive approximations}
The goal of the approach is to generate a sequence of functions ${\phi_0, \phi_1, \dots, \phi_n, \dots}$. Starting with the initial function $\phi_0(x) = 0$ (satisfying the initial condition of (1)), the sequence is generated iteratively by
\[
  \phi_{n+1}(x) = \int_{0}^{x} f(t, \phi_n(t))\d t. \tag*{(3)}
\]
Note that each $\phi_n$ satisfies $\phi_n(0) = 0$, but generally not the integral equation (2) itself. However, if there is a $k$, such that $\phi_{k+1}(x) = \phi_k(x)$, then $\phi_k(x)$ is a solution of the integral equation (2) and hence the IVP (1). Generally this does not occur, but we may instead consider \textit{limit functions}. \\

There are 4 key points to consider:
\begin{enumerate}
  \item Do all members of the sequence exist?
  \item Does the sequence converge to a limit function $\phi$?
  \item What are the properties of $\phi$?
  \item If $\phi$ satisfies the IVP (1), are there other solutions?
\end{enumerate}

\ex{}{
  $$
    y' = 2x(y+1),\qquad y(0) = 0
  $$
  \begin{gather*}
    \phi_0(x) = 0,\qquad f(x,y) = 2x(y+1) \\
    \phi_1(x) = \int_{0}^{x} f(t, \phi_0(t))\d t = \int_{0}^{x} f(t, 0)\d t = \int_{0}^{x} 2t(0+1) \d t = t^2\Big|_{0}^{x} = x^2 \\
    \phi_2(x) = \int_{0}^{x} f(t, \phi_1(t))\d t = \int_{0}^{x} f(t, t^2)\d t = \int_{0}^{x} 2t(t^2+1) \d t = \int_{0}^{x} 2t^3+2t\ \d t = \frac{1}{2}t^4 + t^2\Big|_{0}^{x} = \frac{1}{2}x^4 + x^2
    \intertext{Similarly,}
    \phi_3(x) = x^2 + \frac{1}{2}x^4 + \frac{1}{6}x^6 \\
    \phi_4(x) = x^2 + \frac{1}{2}x^4 + \frac{1}{6}x^6 + \frac{1}{24}x^8 \\
    \intertext{\Propo}
    \phi_n(x) = \sum_{i=1}^{n} \frac{1}{i!}x^{2i}
    \intertext{\Proof By induction: True for $n=1$. Suppose True for $n=k$.}
    \Then \phi_{k+1} (x) = \int_{0}^{x} f(t, \phi_k(t))\d t = \int_{0}^{x} 2t\bracks{1+\sum_{i=1}^{k} \frac{1}{i!}t^{2i}} \d t = \int_{0}^{x} \bracks{2t+\sum_{i=1}^{k} \frac{2}{i!}t^{2i+1}} \d t = t^2 + \sum_{i=2}^{k+1} \frac{1}{i!}t^{2i}\Bigg|_{0}^{x} \\
    \tf\phi_{k+1} = x^2 + \sum_{i=2}^{k+1} \frac{1}{i!}x^{2i} = \sum_{i=1}^{k+1} \frac{1}{i!}x^{2i} \\ 
    \intertext{So the proposition is true $\forall n\in\bbn$.  $\QED$}
    \lim_{n\to\infty}\phi_n(x) = \lim_{n\to\infty} \sum_{i=1}^{n} \frac{1}{i!}x^{2i} \text{ exists}\iff \text{the series converges}
    \intertext{Applying the ratio test between two successive terms, $j$ and $j+1$, as $j$ goes to infinty,}
    \lim_{j\to\infty} \abs{\frac{\dfrac{x^{2j+2}}{(j+1)!}}{\dfrac{x^{2j}}{j!}}} = \lim_{j\to\infty} \abs{\frac{x^{2j+2}}{(j+1)!}\cd\frac{j!}{x^{2j}}} = \lim_{j\to\infty} \abs{\frac{x^2}{j+1}} = 0
    \longintertext{Therefore, the series converges! \\ Therefore, the limit, as $n\to\infty$ of $\phi_n$  exists.}
  \end{gather*}
}

\subsection{Exact First Order ODEs}
\dfn{Exact First Order ODE}{
  Recall that if $z=f(x,y)$ is a differentiable function of $x$ and $y$, where $x=g(t)$ and $y=h(t)$ are both differentiable functions of $t$, then $z$ is a differentiable function of $t$, whose derivative is given by the chain rule:
  $$
    \dyd[z]{t} = \del{f}{x}\dyd[x]{t} + \del{f}{y}\dyd[y]{t}
  $$
  Now suppose the equation
  $$
    f(x,y) = C
  $$
  defines $y$ implicitly as a function of $x$. Then $y=y(x)$ can be show to satisfy a first order ODE obtained by using the chain rule above. In this case, $z=f(x,y(x))=C$, so,
  \begin{gather*}
    \dyd[z]{x} = \dd{x}C = 0 = \del{f}{x}\dyd[x]{x} + \del{f}{y}\dydx \\
    \implies f_x + f_yy' = 0
  \end{gather*}
  A first order ODE of form
  $$
    P(x,y) + Q(x,y)\dydx = 0
  $$
  is called exact if there is a function $f(x,y)$ such that 
  $$
    f_x(x,y) = P(x,y)\quad\text{and}\quad f_y(x,y)=Q(x,y).
  $$
  The solution is then given implicitly by the equation
  $$
    f(x,y) = C,
  $$
  where $C$ can usually be determined by some intial condition.
}

\thm{Test for Exactness}{
  Let $\dps{P,Q,\del{P}{y},\del{Q}{x}}$ be continuous over some reigon of interest. Then
  $$
    P(x,y) + Q(x,y)\dydx = 0
  $$
  is an exact ODE if and only if 
  $$
    \del{P}{y} = \del{Q}{x}
  $$
  everywhere in the reigon
}
\begin{proof}
  1. Prove: ODE is exact $\dps{\implies\del{P}{y} = \del{Q}{x}}$. \\
  Recall Clairout's Theorem,
  \begin{gather*}
    \del{^2f}{x\partial y} = \del{^2f}{y\partial x} \text{ if both $f_{xy}$ and $f_{yx}$ are continuous in the reigon.} \\
    \Suppose \text{ODE is exact}\implies \exists f(x,y): \del{f}{x} = P(x,y), \del{f}{y} = Q(x,y) \\
    \del{P}{y} = \del{^2 f}{x\partial y} = \del{^2 f}{y\partial x} = \del{Q}{x},\text{ by Clairout's Theorem.}
  \end{gather*}
  2. Prove: $\dps{\del{P}{y} = \del{Q}{x}\implies}$ ODE is exact. \\
  \begin{gather*}
    \intertext{$\dps{\Suppose \del{P}{y} = \del{Q}{x}}$. We seek a function $f$ such that $f_x=P, f_y=Q$.}
    \Take f(x,y) = \int_{x_0}^x P(x',y)\d x' + \int_{y_0}^y Q(x_0, y') + C \\
    f_x(x,y) = \del{}{x}\bracks{\int_{x_0}^x P(x',y)\d x' + \cancel{\int_{y_0}^y Q(x_0, y')\d y'}} = P(x,y) \\
    f_y(x,y) = \del{}{y}\bracks{\cancel{\int_{x_0}^x P(x',y)\d x'} + \int_{y_0}^y Q(x_0, y')\d y'} = Q(x,y) \\
  \end{gather*}
  Therefore $\dps{P(x,y) + Q(x,y)\dydx = 0}$ is an exact ODE $\dps{\iff\del{P}{y} = \del{Q}{x}}$ everywhere in the reigon.
\end{proof}

\ex{}{
  $$
    \text{Solve the ODE } 2x+e^y + xe^yy' = 0
  $$
  \begin{align*}
    P(x,y) &= 2x+e^y \\
    \del{P}{y} &= e^y \\
    Q(x,y) &= xe^y \\
    \del{Q}{x} &= e^y \\
    \del{P}{y} = \del{Q}{x} &\Rightarrow \text{ODE is exact} \\
    \tf\exists f(x,y): f_x(x,y) &= P = 2x+e^y \\
    \text{and } f_y(x,y) &= Q = xe^y \\
    \implies f = \int P \d x &= \int 2x+e^y\d x \\
      &= x^2 + xe^y + g(y) \\
    \implies f_y(x,y) = xe^y &= \del{}{y}\bracks{x^2 + xe^y + g(y)} \\
     xe^y &= xe^y + \dyd[g]{y} \\
    \implies \dyd[g]{y} &= 0 \\
    \tf f(x,y) &= x^2 + xe^y + C
    \intertext{All solutions to ODE: $f(x,y)=k$.}
    \iff x^2 + xe^y &= k' \tag*{$(k'=k-C)$} \\
    \iff y &= \ln\bracks{\frac{k' - x^2}{x}}
  \end{align*}
}


\section{Lecture 2}
\subsection{Almost Exact ODEs and Integrating Factors}
Suppose we have
$$
  P(x,y) + Q(x,y)\dydx = 0
$$
and
$$
  \del{P}{y} \neq \del{Q}{x}.
$$
This is not an exact ODE, but can we do anything with it anyway? Let's consider an ``integrating factor'' (not to be confused with integrating factors used when solving linear ODEs). \\

The general idea though, is to multiple the ODE by some function, $h(x,y)$ such that the resulting ODE
$$
  h(x,y)P(x,y) + h(x,y)Q(x,y)\dydx = 0
$$
\textit{is} exact. We know that this new ODE is exact if and only if 
$$
  \del{}{y}(hP) = \del{}{x}(hQ)
$$
Let's find a $h$ which accomplishes this:
\begin{gather*}
  \del{}{y}(hP) = \del{}{x}(hQ) \\
  h_yP + hP_y = h_xQ + hQ_x \iff h_yP - h_xQ + h(P_y - Q_x) = 0 
\end{gather*}
Solving for $h$ in general requires us to solve this first order partial differential equation, which is very nasty and also outside the scope of this course. \\

However, we can consider a simplier case, where $h$ is a function of one of the varaibles $x$ or $y$. Let's try $h = h(x)$:
\begin{gather*}
  \dyd[h]{x} = h(x)\frac{P_y-Q_x}{Q} = h\hat{f}
  \intertext{Suppose $\hat{f}$ is a function of one variable, $x$. Then we are left with a first order seperable ODE which we can solve. Once we've solved for $h$, we can find $f(x,y)$ that solves the exact ODE we wanted to solve.}
\end{gather*}
This is not a great technique, it often doesn't work and requires a lot of trial an error. For example, if $h=h(x)$ didn't yield an approriate $f(x,y)$, we could try $h=h(y)$ or $h=h(x)+h(y)$, which would also give us a seperable ODE to solve. If this technique does work, it hints to some underlying symmetry in the differential system we're solving.

\ex{}{
  $$
    \text{Solve}\ (3xy+y^2) + (x^2+xy)\dydx = 0
  $$
  \begin{gather*}
    P(x,y) = 3xy + y^2\qquad Q(x,y) = x^2 + xy \\
    \del{P}{y} = 3x + 2y \neq 2x + y = \del{Q}{x}
    \intertext{So, this ODE is not exact. Can we multiply through by some integrating factor, $h$?}
    \Take h = h(x) \neq 0 \\
    h(3xy+y^2) + h(x^2+xy)\dydx = \hat{p} + \hat{q}\dydx = 0 \\
    \intertext{is exact}
    \iff \del{\hat{p}}{y} = \del{\hat{q}}{x} \iff h(3x + 2y) = h_x(x^2+xy) + h(2x+y) \iff h(x + y) = h_xx(x+y)
    \intertext{Supposing that $x+y\neq0$, we can simplify and find}
    h = h'x
    \intertext{Supposing that $x\neq0$, we can see}
    h' = \frac{1}{x}h
    \intertext{This is a seperable first order ODE we can simply solve,}
    \int \frac{1}{h} \dyd[h]{x}\d x = \int \frac{1}{x}\d x \iff \int \frac{1}{h}\d h = \int \frac{1}{x}\d x \iff \ln\abs{h} = \ln\abs{x} + \hat{\alpha} \iff h(x) = \alpha x,\ \alpha = \exp(\hat{\alpha}).
    \intertext{We're free to choose $\alpha>0$, so we'll take $\alpha=1$ for simplicity, and then multiple our original ODE by our integrating factor $h=h(x)=x$. Check:}
    h(3xy+y^2) + h(x^2+xy)\dydx = 0 \iff x(3xy+y^2) + x(x^2+xy)\dydx = 0 \\ 
    \iff (3x^2y+xy^2) + (x^3+x^2y)\dydx = 0,\ P(x,y) = 3x^2y+xy^2,\ Q(x,y) = x^3+x^2y \\
    \del{P}{y} = 3x^2+2xy = 3x^2+2xy = \del{Q}{x}
    \intertext{So this ODE is exact. Therefore, there exists some functoin, $f(x,y)$ such that $f_x=P$ and $f_y=Q$}
    \Take f(x,y) = x^3y + \frac{1}{2}x^2y^2 \implies f_x =3x^2y + xy^2 = P,\ f_y = x^3 + x^2y
    \intertext{Therefore, the solution to our ODE is}
    f(x,y) = K \iff x^3y + \frac{1}{2}x^2y^2 = K \iff \frac{1}{2}x^2y^2 + x^3y - K = 0 \iff y = \frac{-x^2 \pm \sqrt{x^3 + 2K}}{x}
    \intertext{Purely for fun, we're going to apply an inital condition, $y(1)=0$}
    \Then 0 = \frac{-1 \pm \sqrt{1 + 2K}}{1} \iff 0 = K \text{ and we choose the positive branch}
    \intertext{So our final solution is} 
    y(x) = \frac{x^2 + \sqrt{x^3}}{x} = \sqrt{x}-x
  \end{gather*}
}

\subsection{Hyperbolic Functions}
\dfn{Hyperbolic Functions}{
  \begin{align*}
    \cosh(x) &= \frac{e^x + e^{-x}}{2}, \\
    \sinh(x) &= \frac{e^x - e^{-x}}{2}, \\
    \tanh(x) &= \frac{\sinh (x)}{\cosh(x)} = \frac{1 - e^{-2x}}{1 + e^{-2x}}, \\
  \end{align*}
}
\cor{Hyperbolic-Pythagorean Identity}{
  $$
    \cosh^2(x) - \sinh^2(x) = 1
  $$
}
This follows from direct calculation. \\

Note that the Pythagorean identity $\cos^2x + \sin^2x = 1$ allows us to paramaterise the unit circle, namely by setting $x(t)=\cos t,\ y(t)=\sin t$, which gives us the equation of a unit circle, $\cos^2t+\sin^2t = x^2 + y^2 = 1$. \\

If instead, we set $x(t)=\cosh t,\ y(t)=\sinh t$, we can see
$$
  \cosh^2t - \sinh^2t = x^2 - y^2 = 1
$$
which is the equation for a hyperbola. \\
Also following from direct calculation, similar to their trigonometric counterparts, the hyperbolic functions satisfy
\begin{gather*}
  \ddx\cosh x = \frac{e^x - e^{-x}}{2} = \sinh x,\\ 
  \ddx\sinh x = \frac{e^x + e^{-x}}{2} = \cosh x
\end{gather*}
Note that $\cosh(0)=1,\ \cosh(x)\geq1$ and $\cosh(x)$ is an even function ($\cosh(-x) = \cosh(x)$); $\sinh(0)=0,\ \sinh(x)$ is an odd function $\sinh(-x) = -\sinh(x)$.

% TODO:
\ex{}{
  Prove that:
  \begin{enumerate}
    \item $\cosh^2x = \frac{1}{2}(\cosh(2x)+1)$ 
    \item $\sinh^2x = \frac{1}{2}(\cosh(2x)-1)$
    \item $\sinh(2x) = 2\sinh x\cosh x$
  \end{enumerate}
}

\begin{center}
  \begin{tikzpicture}
    \begin{axis}[
        axis lines=middle,
        samples=20,
        domain=-2.5:2.5,
        % enlargelimits=true,
        grid=both,
        grid style={dotted,gray},
        minor tick num=1,
        legend pos=south east,
        xmin=-2.5, xmax=2.5,
    ]
        \addplot[blue, thick] {cosh(x)};
        \addplot[red, thick] {sinh(x)};
        \addplot[darkgreen, thick] {tanh(x)};
        \addlegendentry{$\cosh(x)$};
        \addlegendentry{$\sinh(x)$};
        \addlegendentry{$\tanh(x)$};
    \end{axis}
  \end{tikzpicture}
\end{center}

Looking at the plots of the functions, we can deduce that
\begin{align*}
  \dom\cosh x &= \bbr       & \dom\sinh x &= \bbr & \dom\tanh x &= \bbr \\
  \ran\cosh x &= [1,\infty) & \ran\sinh x &= \bbr & \ran\tanh x &= (-1,1)
\end{align*}

\dfn{Reciprocal Hyperbolic Functions}{
  \begin{align*}
    \coth(x) &= \frac{1}{\tanh(x)} = \frac{\cosh (x)}{\sinh(x)} = \frac{1 + e^{-2x}}{1 - e^{-2x}} \\
    \sech(x) &= \frac{1}{\cosh(x)} = \frac{2}{e^{x} - e^{-x}} \\
    \csch(x) &= \frac{1}{\sech(x)} = \frac{2}{e^{x} + e^{-x}} \\
  \end{align*}
}

\subsection{Inverse Hyperbolic Functions}
\dfn{Inverse Hyperbolic Functions}{
  If $f$ is \dots $f\inv$ is denoted \dots:
  \begin{align*}
    \cosh(x) && \arcosh(x) \\
    \sinh(x) && \arsinh(x) \\
    \tanh(x) && \artanh(x)
  \end{align*}
}
\begin{center}
  \begin{tikzpicture}
    \begin{axis}[
        axis lines=middle,
        samples=20,
        domain=-2.5:2.5,
        % enlargelimits=true,
        grid=both,
        grid style={dotted,gray},
        minor tick num=1,
        legend pos=south east,
        xmin=-2.5, xmax=2.5,
    ]
        \addplot[blue, thick, domain=1:2.5] {ln(x + sqrt(x^2 - 1))};
        \addplot[red, thick] {ln(x + sqrt(x^2 + 1))};
        \addplot[darkgreen, thick, domain=-0.99:0.99] {0.5*ln((1+x)/(1-x))};
        \addlegendentry{$\arcosh(x)$};
        \addlegendentry{$\arsinh(x)$};
        \addlegendentry{$\artanh(x)$};
    \end{axis}
  \end{tikzpicture}
\end{center}
\begin{align*}
  \dom\arcosh x &= [1,\infty)       & \dom\arsinh x &= \bbr & \dom\artanh x &= (-1,1) \\
  \ran\arcosh x &= [0,\infty) & \ran\arsinh x &= \bbr & \ran\artanh x &= \bbr
\end{align*}

We have the following:
\begin{align*}
  \int\frac{\d x}{\sqrt{1 + x^2}} &= \arsinh x + C \\
  \int\frac{\d x}{\sqrt{1 - x^2}} &= \arcosh x + C,\ x > 1 
\end{align*}

\ex{}{
  Show $\dps{\ddx(\arsinhx)=\frac{1}{\sqrt{1+x^2}}}$.
  \begin{align*}
    \arsinhx &= y(x) \\
    x &= \sinh y \\
    \iff\ddx(x) &= \ddx(\sinh y) \\
    \iff 1 &= \dydx \cd \cosh y \\
    \iff \dydx &= \frac{1}{\cosh y} \\
      &= \frac{1}{\cosh(\arsinhx)} \\
      &= \frac{1}{\sqrt{\cosh^2(\arsinhx)}} \\
      &= \frac{1}{\sqrt{1 + \sinh^2(\arsinhx)}} \\
      &= \frac{1}{\sqrt{1 + \sinh(\arsinhx)\sinh(\arsinhx)}} \\
      &= \frac{1}{\sqrt{1 + x\cd x}} \\
      &= \frac{1}{\sqrt{1 + x^2}}
  \end{align*}

  Show $\dps{\ddx(\arcoshx)=\frac{1}{\sqrt{x^2 - 1}}}$.
  \begin{align*}
    \arcoshx &= y(x) \\
    x &= \cosh y \\
    \iff\ddx(x) &= \ddx(\cosh y) \\
    \iff 1 &= \dydx \cd \sinh y \\
    \iff \dydx &= \frac{1}{\sinh y} \\
      &= \frac{1}{\sinh(\arcoshx)} \\
      &= \frac{1}{\sqrt{\sinh^2(\arcoshx)}} \\
      &= \frac{1}{\sqrt{\cosh^2(\arcoshx) - 1}} \\
      &= \frac{1}{\sqrt{\cosh(\arcoshx)\cosh(\arcoshx) - 1}} \\
      &= \frac{1}{\sqrt{x\cd x - 1}} \\
      &= \frac{1}{\sqrt{x^2 - 1}}
  \end{align*}
}

\ex{}{
  Evaluate $\dps{\int\frac{\d x}{\sqrt{1+x^2}}}$
  \begin{align*}
    1 + \sinh^2 t &= \cosh^2 t \\
    \Let x &= \sinh t \\
    \implies\dyd[x]{t} = \cosh t &\Rightarrow \d x = \cosh t\ \d t \\
    \tf\int\frac{\d x}{\sqrt{1+x^2}} &= \int\frac{\cosh t}{\sqrt{1+\sinh^2t}}\d t \\
      &= \int\frac{\cosh t}{\sqrt{\cosh^2 t}}\d t \\
      &= \int\frac{\cosh t}{\cosh t}\d t \\
      &= \int \d t \\
      &= t + C\\
      &= \arsinhx + C
  \end{align*}

  Evaluate $\dps{\int\frac{\d x}{\sqrt{x^2-1}}}$
  \begin{align*}
    \cosh^2 t - 1 &= \sinh^2 t \\
    \Let x &= \cosh t \\
    \implies\dyd[x]{t} = \sinh t &\Rightarrow \d x = \sinh t\ \d t \\
    \tf\int\frac{\d x}{\sqrt{x^2-1}} &= \int\frac{\sinh t}{\sqrt{\cosh^2t - 1}}\d t \\
      &= \int\frac{\sinh t}{\sqrt{\sinh^2 t}}\d t \\
      &= \int\frac{\sinh t}{\sinh t}\d t \\
      &= \int \d t \\
      &= t + C\\
      &= \arcoshx + C,\ x\geq 1
  \end{align*}
}

% TODO:
\ex{}{
  Show that $\dps{\ddx(\artanhx) = \frac{1}{1 - x^2}}$
}

Using partial fractions, we also find that
$$
  \int\frac{\d x}{1 - x^2} = \frac{1}{2}\ln\bracks{\frac{1 + x}{1 - x}} + C
$$
In fact, we have the following identities
\begin{gather*}
  \artanhx = \frac{1}{2}\ln\bracks{\frac{1+x}{1-x}} \\
  \arsinhx = \ln\bracks{x+\sqrt{x^2+1}} \\
  \arcoshx = \ln\bracks{x+\sqrt{x^2-1}} 
\end{gather*}

% TODO:
\ex{}{
  Show that $\dps{\arsinhx = \ln\bracks{x + \sqrt{x^2 + 1}}}$
}

% TODO:
\subsection{The Cateary Problem}

\subsection{Linear Second-Order Non-Homogenous ODEs and the Wronskian}

\section{Lecture 3}
\subsection{Variation of Parameters}
We've seen that for a linear second-order, non-homogenous IVP,
$$
  y'' + p(x)y' + q(x)y = r(x),\quad y(x_0)=y_0
$$
if $p,q,r$ are continous on an open interval $I$, and the inital condition, $x_0\in I$, then there exisits a solution to the IVP. The solution will be a linear combination of the solution in the homogenous case and the particular case, $y(x) = y_H(x) + y_P(x)$. Assuming the homogenous case is a linear combination of linearly independent $y$s, ie $W(y_1, y_2) \neq 0$, we can apply variation of parameters. The process is as follows:
\begin{enumerate}
  \item Solve $y'' + p(x)y' + q(x)y = 0$, and obtain a fundamental set of solutions, $y_1, y_2$. Calculate the Wronskian, $W(y_1,y_2)(x) = W$.
  \item Set $y_P = u(x)y_1(x) + v(x)y_2(x)$ and substitute into the ODE. We also impose the condition, $u'y_1 + v'y_2 = 0$. We can freely impose this condition because we have two functions, $u,v$, and only one equation they must satisfy, the ODE.
  \item We obtain
  $$
    u(x) = -\int \frac{y_2r}{W}\d x,\qquad v(x) = \int \frac{y_1r}{W}\d x.
  $$
\end{enumerate}
This approach is a variation of the reduction of order, which prescribes taking some solution, $y$, of the associated ODE, and using it to find a particular solution.

\ex{}{
  Derivation of $u(x)$ and $v(x)$ of the variation of parameters.
  \begin{gather*}
    y'' + p(x)y' + q(x)y = r(x) \tag{1}\label{eq:LinSecOrdODE}
    \intertext{Suppose we solved the homogenous case, $y''+py'+qy=0$.}
    \implies\exists y_1(x), y_2(x): W(y_1,y_2)(x) \neq 0,\ y_H(x) = Ay_1(x) + By_2(x) \\
    y_P(x) = u(x)y_1(x) + v(x)y_2(x) \tag{2}\label{eq:VarParams} \\
    \tf y_P' = u'y_1 + uy_1' + v'y_2 + vy_2 \\
    \intertext{Impose that $u'y_1 + v'y_2 = 0$, then}
    y_P' = uy_1' + vy_2' \\
    \tf y_P'' = u'y_1' + uy_1'' + v'y_2' + vy_2'' 
    \intertext{We'll now substitute (\ref{eq:VarParams})'s derivatives back into (\ref{eq:LinSecOrdODE}), and find}
    (u'y_1' + uy_1'' + v'y_2' + vy_2'') + p(uy_1' + vy_2') + q(uy_1 + vy_2) = r
    \intertext{Consider $uy_1'' + puy_1' + quy_1$ and $vy_2'' + pvy_2' + qvy_2$, and note that they are solutions to the homogenous case, and are therefore equal to 0. So we can simply cancel them out, and are left with:}
    u'y_1' + v'y_2' = r
    \intertext{In fact, the entire system has been reduced to the system of equations}
    \left\lbrace \begin{array}{l} u'y_1 + v'y_2 = 0 \\ u'y_1' + v'y_2' = r \end{array} \right. \\
    \iff \begin{pmatrix} y_1(x) & y_2(x) \\ y_1'(x) & y_2'(x) \end{pmatrix} \begin{pmatrix} u' \\ v' \end{pmatrix} = \begin{pmatrix} 0 \\ r(x) \end{pmatrix} \\
    \begin{pmatrix} y_1(x) & y_2(x) \\ y_1'(x) & y_2'(x) \end{pmatrix} = \hat{W},\ \det\hat{W} = \det W(y_1,y_2)(x) = W \neq 0 \implies \hat{W}\text{ is invertible.} \\
    \hat{W}\inv = \frac{1}{\det\hat{W}}\begin{pmatrix} y_2'(x) & -y_2(x) \\ -y_1' & y_1(x) \end{pmatrix} \\
    \tf \begin{pmatrix} u ' \\ v' \end{pmatrix} = \hat{W}\inv\begin{pmatrix} 0 \\ r(x) \end{pmatrix} = \frac{1}{\det\hat{W}}\begin{pmatrix} y_2'(x) & -y_2(x) \\ -y_1' & y_1(x) \end{pmatrix}\begin{pmatrix} 0 \\ r(x) \end{pmatrix} \\
    \begin{pmatrix} u ' \\ v' \end{pmatrix} = \frac{1}{W}\begin{pmatrix} -y_2r \\ y_1r \end{pmatrix} \\
    \iff \left\lbrace \begin{array}{l} \dps{u' = \frac{-y_2r}{W}} \\ \dps{v' = \frac{y_1r}{W}}\end{array} \right.
    \iff \left\lbrace \begin{array}{l} \dps{u = -\int\frac{y_2r}{W}\d x} \\ \dps{v = \int\dfrac{y_1r}{W}\d x} \end{array} \right.
  \end{gather*}
}

\ex{}{
  Solve
  $$
    y'' - 4y' + 5y = \frac{2e^{2x}}{\sin x}
  $$
  using variation of parameters.
  \begin{gather*}
    y = y_H + y_P
    \intertext{Let's ansatz that $y_H = e^{\lm x}$}
    \iff \lm^2 - 4 + 5 = 0 \iff \lm_{1,2} = 2\pm i \iff y_H = Ae^{2x}\cos x + Be^{2x}\sin x \\ 
    W = \det W(y_1, y_2)(x) = \det\begin{pmatrix} e^{2x}\cos x & e^{2x}\sin x \\ 2e^{2x}\cosx - e^{2x}\sinx & 2e^{2x}\sinx + e^{2x}\cosx \end{pmatrix} = e^{4x} \neq 0
    \intertext{Find $y_P = uy_1 + vy_2$}
    u(x) = -\int\frac{y_2r}{W}\d x = -\int\frac{e^{2x}\sinx\frac{2e^{2x}} {\sinx}}{e^{4x}}\d x = -2\int1\ \d x = -2x \\
    v(x) = \int\frac{y_1r}{W}\d x = \int\frac{e^{2x}\cosx\frac{2e^{2x}}{\sin x}}{e^{4x}}\d x = 2\int\cotx\ \d x = 2\ln\abs{\sinx} \\
    \implies y_P = 2\ln\abs{\sinx}e^{2x}\sin x - 2xe^{2x}\cos x \\
    \implies y = Ae^{2x}\cos x + Be^{2x}\sin x + 2\ln\abs{\sinx}e^{2x}\sin x - 2xe^{2x}\cos x
  \end{gather*}
}

\ex{}{
  Solve
  $$
    y'' - 4y' + 5y = \frac{2e^{2x}}{\sin x}
  $$
  using reduction of order.
  \begin{gather*}
    y = y_H + y_P
  \end{gather*}
}





\end{document}
