\documentclass[a4paper,12pt]{report}

\input{../../../../latex_template/preamble}
%From M275 "Topology" at SJSU
\newcommand{\id}{\mathrm{id}}
\newcommand{\taking}[1]{\xrightarrow{#1}}
\newcommand{\inv}{^{-1}}

%From M170 "Introduction to Graph Theory" at SJSU
\DeclareMathOperator{\diam}{diam}
\DeclareMathOperator{\ord}{ord}
\newcommand{\defeq}{\overset{\mathrm{def}}{=}}

%From the USAMO .tex files
\newcommand{\ts}{\textsuperscript}
\newcommand{\dg}{^\circ}
\newcommand{\ii}{\item}

% % From Math 55 and Math 145 at Harvard
% \newenvironment{subproof}[1][Proof]{%
% \begin{proof}[#1] \renewcommand{\qedsymbol}{$\blacksquare$}}%
% {\end{proof}}

\newcommand{\liff}{\leftrightarrow}
\newcommand{\lthen}{\rightarrow}
\newcommand{\opname}{\operatorname}
\newcommand{\surjto}{\twoheadrightarrow}
\newcommand{\injto}{\hookrightarrow}
\newcommand{\On}{\mathrm{On}} % ordinals
\DeclareMathOperator{\img}{im} % Image
\DeclareMathOperator{\Img}{Im} % Image
\DeclareMathOperator{\coker}{coker} % Cokernel
\DeclareMathOperator{\Coker}{Coker} % Cokernel
\DeclareMathOperator{\Ker}{Ker} % Kernel
\DeclareMathOperator{\rank}{rank}
\DeclareMathOperator{\Spec}{Spec} % spectrum
\DeclareMathOperator{\Tr}{Tr} % trace
\DeclareMathOperator{\pr}{pr} % projection
\DeclareMathOperator{\ext}{ext} % extension
\DeclareMathOperator{\pred}{pred} % predecessor
\DeclareMathOperator{\dom}{dom} % domain
\DeclareMathOperator{\ran}{ran} % range
\DeclareMathOperator{\Hom}{Hom} % homomorphism
\DeclareMathOperator{\Mor}{Mor} % morphisms
\DeclareMathOperator{\End}{End} % endomorphism

\newcommand{\eps}{\epsilon}
\newcommand{\veps}{\varepsilon}
\newcommand{\ol}{\overline}
\newcommand{\ul}{\underline}
\newcommand{\wt}{\widetilde}
\newcommand{\wh}{\widehat}
\newcommand{\vocab}[1]{\textbf{\color{blue} #1}}
\providecommand{\half}{\frac{1}{2}}
\newcommand{\dang}{\measuredangle} %% Directed angle
\newcommand{\ray}[1]{\overrightarrow{#1}}
\newcommand{\seg}[1]{\overline{#1}}
\newcommand{\arc}[1]{\wideparen{#1}}
\DeclareMathOperator{\cis}{cis}
\DeclareMathOperator*{\lcm}{lcm}
\DeclareMathOperator*{\argmin}{arg min}
\DeclareMathOperator*{\argmax}{arg max}
\newcommand{\cycsum}{\sum_{\mathrm{cyc}}}
\newcommand{\symsum}{\sum_{\mathrm{sym}}}
\newcommand{\cycprod}{\prod_{\mathrm{cyc}}}
\newcommand{\symprod}{\prod_{\mathrm{sym}}}
\newcommand{\Qed}{\begin{flushright}\qed\end{flushright}}
\newcommand{\parinn}{\setlength{\parindent}{1cm}}
\newcommand{\parinf}{\setlength{\parindent}{0cm}}
% \newcommand{\norm}{\|\cdot\|}
\newcommand{\inorm}{\norm_{\infty}}
\newcommand{\opensets}{\{V_{\alpha}\}_{\alpha\in I}}
\newcommand{\oset}{V_{\alpha}}
\newcommand{\opset}[1]{V_{\alpha_{#1}}}
\newcommand{\lub}{\text{lub}}
\newcommand{\del}[2]{\frac{\partial #1}{\partial #2}}
\newcommand{\Del}[3]{\frac{\partial^{#1} #2}{\partial^{#1} #3}}
\newcommand{\deld}[2]{\dfrac{\partial #1}{\partial #2}}
\newcommand{\Deld}[3]{\dfrac{\partial^{#1} #2}{\partial^{#1} #3}}
\newcommand{\lm}{\lambda}
\newcommand{\uin}{\mathbin{\rotatebox[origin=c]{90}{$\in$}}}
\newcommand{\usubset}{\mathbin{\rotatebox[origin=c]{90}{$\subset$}}}
\newcommand{\lt}{\left}
\newcommand{\rt}{\right}
\newcommand{\bs}[1]{\boldsymbol{#1}}
\newcommand{\exs}{\exists}
\newcommand{\st}{\strut}
\newcommand{\dps}[1]{\displaystyle{#1}}

\newcommand{\sol}{\setlength{\parindent}{0cm}\textbf{\textit{Solution:}}\setlength{\parindent}{1cm} }
\newcommand{\solve}[1]{\setlength{\parindent}{0cm}\textbf{\textit{Solution: }}\setlength{\parindent}{1cm}#1 \Qed}

\DeclareMathOperator{\sech}{sech}
\DeclareMathOperator{\csch}{csch}
\DeclareMathOperator{\arcsec}{arcsec}
\DeclareMathOperator{\arccsc}{arccsc}
\DeclareMathOperator{\arccot}{arccot}
\DeclareMathOperator{\arsinh}{arsinh}
\DeclareMathOperator{\arcosh}{arcosh}
\DeclareMathOperator{\artanh}{artanh}
\DeclareMathOperator{\arcsch}{arcsch}
\DeclareMathOperator{\arsech}{arsech}
\DeclareMathOperator{\arcoth}{arcoth}

\newcommand{\sinx}{\sin x}          \newcommand{\arcsinx}{\arcsin x}    
\newcommand{\cosx}{\cos x}          \newcommand{\arccosx}{\arccosx}
\newcommand{\tanx}{\tan x}          \newcommand{\arctanx}{\arctan x}
\newcommand{\cscx}{\csc x}          \newcommand{\arccscx}{\arccsc x}
\newcommand{\secx}{\sec x}          \newcommand{\arcsecx}{\arcsec x}
\newcommand{\cotx}{\cot x}          \newcommand{\arccotx}{\arccot x}
\newcommand{\sinhx}{\sinh x}          \newcommand{\arsinhx}{\arsinh x}
\newcommand{\coshx}{\cosh x}          \newcommand{\arcoshx}{\arcosh x}
\newcommand{\tanhx}{\tanh x}          \newcommand{\artanhx}{\artanh x}
\newcommand{\cschx}{\csch x}          \newcommand{\arcschx}{\arcsch x}
\newcommand{\sechx}{\sech x}          \newcommand{\arsechx}{\arsech x}
\newcommand{\cothx}{\coth x}          \newcommand{\arcothx}{\arcoth x}
\newcommand{\lnx}{\ln x}
\newcommand{\expx}{\exp x}

\newcommand{\bba}{\mathbb{A}}   \newcommand{\bbn}{\mathbb{N}}
\newcommand{\bbb}{\mathbb{B}}   \newcommand{\bbo}{\mathbb{O}}
\newcommand{\bbc}{\mathbb{C}}   \newcommand{\bbp}{\mathbb{P}}
\newcommand{\bbd}{\mathbb{D}}   \newcommand{\bbq}{\mathbb{Q}}
\newcommand{\bbe}{\mathbb{E}}   \newcommand{\bbr}{\mathbb{R}}
\newcommand{\bbf}{\mathbb{F}}   \newcommand{\bbs}{\mathbb{S}}
\newcommand{\bbg}{\mathbb{G}}   \newcommand{\bbt}{\mathbb{T}}
\newcommand{\bbh}{\mathbb{H}}   \newcommand{\bbu}{\mathbb{U}}
\newcommand{\bbi}{\mathbb{I}}    \newcommand{\bbv}{\mathbb{V}}
\newcommand{\bbj}{\mathbb{J}}   \newcommand{\bbw}{\mathbb{W}}
\newcommand{\bbk}{\mathbb{K}}   \newcommand{\bbx}{\mathbb{X}}
\newcommand{\bbl}{\mathbb{L}}    \newcommand{\bby}{\mathbb{Y}}
\newcommand{\bbm}{\mathbb{M}}   \newcommand{\bbz}{\mathbb{Z}}

\newcommand{\lb}{\left(}
\newcommand{\rb}{\right)}
\newcommand{\lbr}{\left\lbrace}
\newcommand{\rbr}{\right\rbrace}
\newcommand{\lsb}{\left[}
\newcommand{\rsb}{\right]}
\newcommand{\suchthat}{\medspace\middle|\medspace}
\newcommand{\bracks}[1]{\lb #1 \rb}
\newcommand{\braces}[1]{\lbr #1 \rbr}
\newcommand{\sqbracks}[1]{\lsb #1 \rsb}

\renewcommand{\floor}[1]{\lfloor #1 \rfloor}
\renewcommand{\ceil}[1]{\lceil #1 \rceil}

\newcommand{\cd}{\cdot}
\newcommand{\tf}{\therefore}
\newcommand{\Let}{\text{Let }}
\newcommand{\Given}{\text{Given }}
\newcommand{\Suppose}{\text{Suppose }}
\newcommand{\WeSee}{\text{We see }}
\newcommand{\So}{\text{So }}

\newcommand{\QED}{\hfill \qed}

\renewcommand{\dd}[1]{\frac{d}{d#1}}
\newcommand{\dyd}[2][y]{\frac{d#1}{d#2}}

\newcommand{\ddx}{\dd{x}}       \newcommand{\ddxsq}{\dyd[^2]{x^2}}
\newcommand{\ddy}{\dd{y}}       \newcommand{\ddysq}{\dyd[^2]{y^2}}
\newcommand{\ddu}{\dd{u}}       \newcommand{\ddusq}{\dyd[^2]{u^2}}
\newcommand{\ddv}{\dd{v}}       \newcommand{\ddvsq}{\dyd[^2]{v^2}}

\newcommand{\dydx}{\dyd{x}}     \newcommand{\dydxsq}{\dyd[^2y]{x^2}}
\newcommand{\dfdx}{\dyd[f]{x}}  \newcommand{\dfdxsq}{\dyd[^2f]{x^2}}
\newcommand{\dudx}{\dyd[u]{x}}  \newcommand{\dudxsq}{\dyd[^2u]{x^2}}
\newcommand{\dvdx}{\dyd[v]{x}}  \newcommand{\dvdxsq}{\dyd[^2v]{x^2}}

% Mathfrak primes
\newcommand{\km}{\mathfrak{m}}
\newcommand{\kp}{\mathfrak{p}}
\newcommand{\kq}{\mathfrak{q}}

%---------------------------------------
% Blackboard Math Fonts :-
%---------------------------------------
\newcommand{\bba}{\mathbb{A}}   \newcommand{\bbn}{\mathbb{N}}
\newcommand{\bbb}{\mathbb{B}}   \newcommand{\bbo}{\mathbb{O}}
\newcommand{\bbc}{\mathbb{C}}   \newcommand{\bbp}{\mathbb{P}}
\newcommand{\bbd}{\mathbb{D}}   \newcommand{\bbq}{\mathbb{Q}}
\newcommand{\bbe}{\mathbb{E}}   \newcommand{\bbr}{\mathbb{R}}
\newcommand{\bbf}{\mathbb{F}}   \newcommand{\bbs}{\mathbb{S}}
\newcommand{\bbg}{\mathbb{G}}   \newcommand{\bbt}{\mathbb{T}}
\newcommand{\bbh}{\mathbb{H}}   \newcommand{\bbu}{\mathbb{U}}
\newcommand{\bbi}{\mathbb{I}}   \newcommand{\bbv}{\mathbb{V}}
\newcommand{\bbj}{\mathbb{J}}   \newcommand{\bbw}{\mathbb{W}}
\newcommand{\bbk}{\mathbb{K}}   \newcommand{\bbx}{\mathbb{X}}
\newcommand{\bbl}{\mathbb{L}}   \newcommand{\bby}{\mathbb{Y}}
\newcommand{\bbm}{\mathbb{M}}   \newcommand{\bbz}{\mathbb{Z}}

%---------------------------------------
% Roman Math Fonts :-
%---------------------------------------
\newcommand{\rma}{\mathrm{A}}   \newcommand{\rmn}{\mathrm{N}}
\newcommand{\rmb}{\mathrm{B}}   \newcommand{\rmo}{\mathrm{O}}
\newcommand{\rmc}{\mathrm{C}}   \newcommand{\rmp}{\mathrm{P}}
\newcommand{\rmd}{\mathrm{D}}   \newcommand{\rmq}{\mathrm{Q}}
\newcommand{\rme}{\mathrm{E}}   \newcommand{\rmr}{\mathrm{R}}
\newcommand{\rmf}{\mathrm{F}}   \newcommand{\rms}{\mathrm{S}}
\newcommand{\rmg}{\mathrm{G}}   \newcommand{\rmt}{\mathrm{T}}
\newcommand{\rmh}{\mathrm{H}}   \newcommand{\rmu}{\mathrm{U}}
\newcommand{\rmi}{\mathrm{I}}   \newcommand{\rmv}{\mathrm{V}}
\newcommand{\rmj}{\mathrm{J}}   \newcommand{\rmw}{\mathrm{W}}
\newcommand{\rmk}{\mathrm{K}}   \newcommand{\rmx}{\mathrm{X}}
\newcommand{\rml}{\mathrm{L}}   \newcommand{\rmy}{\mathrm{Y}}
\newcommand{\rmm}{\mathrm{M}}   \newcommand{\rmz}{\mathrm{Z}}

%---------------------------------------
% Calligraphic Math Fonts :-
%---------------------------------------
\newcommand{\cla}{\mathcal{A}}   \newcommand{\cln}{\mathcal{N}}
\newcommand{\clb}{\mathcal{B}}   \newcommand{\clo}{\mathcal{O}}
\newcommand{\clc}{\mathcal{C}}   \newcommand{\clp}{\mathcal{P}}
\newcommand{\cld}{\mathcal{D}}   \newcommand{\clq}{\mathcal{Q}}
\newcommand{\cle}{\mathcal{E}}   \newcommand{\clr}{\mathcal{R}}
\newcommand{\clf}{\mathcal{F}}   \newcommand{\cls}{\mathcal{S}}
\newcommand{\clg}{\mathcal{G}}   \newcommand{\clt}{\mathcal{T}}
\newcommand{\clh}{\mathcal{H}}   \newcommand{\clu}{\mathcal{U}}
\newcommand{\cli}{\mathcal{I}}   \newcommand{\clv}{\mathcal{V}}
\newcommand{\clj}{\mathcal{J}}   \newcommand{\clw}{\mathcal{W}}
\newcommand{\clk}{\mathcal{K}}   \newcommand{\clx}{\mathcal{X}}
\newcommand{\cll}{\mathcal{L}}   \newcommand{\cly}{\mathcal{Y}}
\newcommand{\calm}{\mathcal{M}}  \newcommand{\clz}{\mathcal{Z}}

%---------------------------------------
% Fraktur  Math Fonts :-
%---------------------------------------
\newcommand{\fka}{\mathfrak{A}}   \newcommand{\fkn}{\mathfrak{N}}
\newcommand{\fkb}{\mathfrak{B}}   \newcommand{\fko}{\mathfrak{O}}
\newcommand{\fkc}{\mathfrak{C}}   \newcommand{\fkp}{\mathfrak{P}}
\newcommand{\fkd}{\mathfrak{D}}   \newcommand{\fkq}{\mathfrak{Q}}
\newcommand{\fke}{\mathfrak{E}}   \newcommand{\fkr}{\mathfrak{R}}
\newcommand{\fkf}{\mathfrak{F}}   \newcommand{\fks}{\mathfrak{S}}
\newcommand{\fkg}{\mathfrak{G}}   \newcommand{\fkt}{\mathfrak{T}}
\newcommand{\fkh}{\mathfrak{H}}   \newcommand{\fku}{\mathfrak{U}}
\newcommand{\fki}{\mathfrak{I}}   \newcommand{\fkv}{\mathfrak{V}}
\newcommand{\fkj}{\mathfrak{J}}   \newcommand{\fkw}{\mathfrak{W}}
\newcommand{\fkk}{\mathfrak{K}}   \newcommand{\fkx}{\mathfrak{X}}
\newcommand{\fkl}{\mathfrak{L}}   \newcommand{\fky}{\mathfrak{Y}}
\newcommand{\fkm}{\mathfrak{M}}   \newcommand{\fkz}{\mathfrak{Z}}


\begin{document}
\begin{center}
{\bf School of Mathematics and Physics, UQ}
\end{center}
\begin{center}
	{\large\bf MATH1071 Advanced Calculus \& Linear Algebra I \\ Semester 1 2025 \\ Problem Set 2} \\ \vspace{1em}
	Michael Kasumagic, 44302669 \\
	Tutorial Group \#8 \\
	Due 5pm Monday 31 March 2025
\end{center}

\qs{5 marks}{	
	Use the definition of limits, show that
	$$
    \lim_{n\to \infty} \frac{1}{n^3}=0
	$$
}
\sol \\
\Definn{1.1}{Limit of a Sequence} Let $\bracks{a_n}_{n=1}^{\infty}$ be a sequence of real numbers. The limit of $\bracks{a_n}_{n=1}^{\infty}$ equals $a\in\bbr$, written $\lim_{n\to\infty} a_n = a$, if $\forall\veps>0, \exists N\in\bbn: n\geq N\implies \abs{a_n - a} < \veps$. \\
\Lemma $\dps{\lim_{n\to\infty} \frac{1}{n^3} = 0}$ \\
\Proof Suppose $\bracks{a_n}_{n=1}^{\infty} := \bracks{\frac{1}{n^3}}_{n=1}^{\infty}$ is a sequence of real numbers, with a limit $a:=0\in\bbr$. Suppose $\veps > 0$. Let's consider
\begin{align*}
	\clap{$\dps{\abs{a_n - a} = \abs{\frac{1}{n^3} - 0} = \abs{\frac{1}{n^3}} = \frac{1}{n^3} < \veps}$} \tag*{($n\in\bbn$)}
	\intertext{When $n\geq N$. Solving for $N$ now,}
	\clap{$\dps{n^3 > \frac{1}{\veps} \iff n > \sqrt[3]{\frac{1}{\veps}}}$} \\
	\Choose N &= \ceil{\sqrt[3]{\frac{1}{\veps}}}
	\intertext{Therefore, $\forall\veps>0, \exists N\in\bbn, N = \operatorname{ceil}\bracks{\sqrt[3]{1/\veps}}$ we have}
	\clap{$\dps{n \geq N = \ceil{\sqrt[3]{\frac{1}{\veps}}}}$} \\
	\clap{$\dps{n \geq N \geq \sqrt[3]{\frac{1}{\veps}}}$} \\
	\clap{$\dps{n^3 \geq N^3 \geq \frac{1}{\veps}}$} \\
	\clap{$\dps{\frac{1}{n^3} \leq \frac{1}{N^3} \leq \veps}$} \\
	\longintertext{Verifying our choice of $N$, and completing the proof. \\ Therefore, by the $\veps{-}N$ definition of the limit, $\dps{\lim_{n\to\infty}\frac{1}{n^3} = 0. \QED}$}
\end{align*}


\newpage
\qs{10 marks}{
	Use suitable limit laws, find the limits for the following sequences. Please cite which laws you've used.

	\begin{enumerate}[label=(\alph*)]
		\item $\dps{\lim_{n\to\infty} \frac{2n^3 + 4n}{7n^4 + 5n^2 - 1}}$
		\item $\dps{\lim_{n\to\infty} \frac{\cos n + \sin n}{n}}$
	\end{enumerate}
}
\Theomm{2.1} (Sequence Limit Properties) Suppose $n\in\bbn$, $\lim_{n\to\infty}a_n = a$, $\lim_{n\to\infty}b_n = b$, and $\lm\in\bbr$ is fixed, then
\begin{multicols}{2}
	\begin{enumerate}[label=(\alph*), itemsep=0pt]
		\item $\dps{\lim_{n\to\infty}(a_n + b_n) = a + b}$
		\item $\dps{\lim_{n\to\infty}\lm a_n = \lm a}$
		\item $\dps{\lim_{n\to\infty}a_nb_n = ab}$
		\item $\dps{\lim_{n\to\infty}\frac{a_n}{b_n} = \frac{a_n}{b_n},\ \text{given}\ b\neq0,\ b_n\neq0,\forall n}$
	\end{enumerate}	
\end{multicols}
\Theomm{2.2} (Squeeze Theorem) Suppose we have three sequences $\bracks{a_n}_{n=1}^{\infty}, \bracks{b_n}_{n=1}^{\infty}, \bracks{c_n}_{n=1}^{\infty}$ such that $a_n\leq b_n\leq c_n$, and $a_n = c_n = L$. Then $b_n = L$. \\

\sol(a) \\
Just for fun, and for no particular reason, we'll divide through every term by $n^4$.
\begin{align*}
	\clap{$\dps{\Let L := \lim_{n\to\infty} \frac{2n^3 + 4n}{7n^4 + 5n^2 - 1} = \lim_{n\to\infty} \frac{\frac{2n^3}{n^4} + \frac{4n}{n^4}}{\frac{7n^4}{n^4} + \frac{5n^2}{n^4} - \frac{1}{n^4}} = \lim_{n\to\infty} \frac{\frac{2}{n} + \frac{4}{n^3}}{7 + \frac{5}{n^2} - \frac{1}{n^4}}}$}
	\longintertext{Let $a_n := \frac{2}{n} + \frac{4}{n^3}$ and $a:=\lim_{n\to\infty} a_n $ \\ Let $b_n := 7 + \frac{5}{n^2} - \frac{1}{n^4}$ and $b:=\lim_{n\to\infty}b_n$. \\ We need to make sure that $b_n \neq 0$, $\forall n\in\bbn$}
	n &> 0 \\
	n^4 &> 0 \\
	-n^4 &< 0 \\
	\frac{n^2}{5}-n^4 &< 0 \\
	\frac{5}{n^2} - \frac{1}{n^4} &> 0 \\
	7 + \frac{5}{n^2} - \frac{1}{n^4} &> 0 \\
	\tf b_n &> 0,\ \forall n\in\bbn
	\intertext{Since, $b_n\neq0,\ \forall n\in\bbn,\ \lim_{n\to\infty} b_n \neq 0$. Therefore, we can apply Theorem 2.1(d)}
	L &= \frac{a}{b}
	\intertext{Let's start by finding $a$}
	a &= \lim_{n\to\infty} \frac{2}{n} + \frac{4}{n^3} \\
	\longintertext{But this is just the sum of two other sequences! \\ Let $\alpha_n := \frac{2}{n}$ and $\alpha = \lim_{n\to\infty} \alpha_n$. \\ Let $\beta_n := \frac{4}{n^3}$ and $\beta:=\lim_{n\to\infty} \beta_n$}
	\Then a_n &= \alpha_n + \beta_n
	\intertext{So we can apply Theorem 2.1(a)}
	\clap{$\dps{a = \alpha + \beta = \lim_{n\to\infty} \frac{2}{n} + \lim_{n\to\infty} \frac{4}{n^3}}$}
	\intertext{So, let's find $\alpha$.}
	\alpha &= \lim_{n\to\infty} \frac{2}{n} 
	\intertext{2 is a fixed constant, so we can apply Theorem 2.1(b)}
		&= 2\lim_{n\to\infty} \frac{1}{n}
	\intertext{And $\lim_{n\to\infty} 1/n$ is trivially equal to 0. If I need further justification, I would direct you to Question 1, where the same argument holds, except choosing $N=\ceil{1/\veps}$.}
	\ignorealign{\tf \alpha = \lim_{n\to\infty} \frac{2}{n} = 2\lim_{n\to\infty} \frac{1}{n} = 2\cd0 = 0}
	\intertext{Next, we'll find $\beta$.}
	\beta &= \lim_{n\to\infty} \frac{4}{n^3}
	\intertext{4 is a fixed constant, so we can apply Theorem 2.1(b)}
		&= 4\lim_{n\to\infty} \frac{1}{n^3}
	\intertext{We proved in question 1 that $\lim_{n\to\infty}1/n^3 = 0$, so}
	\ignorealign{\tf \beta = \lim_{n\to\infty} \frac{4}{n^3} = 4\lim_{n\to\infty} \frac{1}{n^3} = 4\cd0 = 0}
	\intertext{Hence, we've found $a$,}
	\ignorealign{a = \alpha + \beta = 0 + 0 = 0} \\
	\longintertext{Next we'll find $b$ \\ $b$ is also the sum of three sequencess, so we can apply Theorem 2.1(a) \\ Let $\gamma_n := 7$ and $\gamma := \lim_{n
	\to\infty} \gamma_n$ \\ Let $\delta_n := \frac{5}{n^2}$ and $\delta := \lim_{n\to\infty} \delta_n$ \\ Let $\veps_n := \frac{1}{n^4}$ and $\veps := \lim_{n\to\infty} \veps_n$}
	\Then b_n &= \gamma_n + \delta_n + \veps_n 
	\intertext{And we can apply Theorem 2.1(a)}
	\ignorealign{b = \gamma + \delta + \veps = \lim_{n
	\to\infty} 7 + \lim_{n\to\infty} \frac{5}{n^2} + \lim_{n\to\infty} \frac{1}{n^4}}
	\intertext{We'll start by computing $\gamma$}
	\ignorealign{\gamma = \lim_{n\to\infty} 7 = 7\lim_{n\to\infty} 1 = 7\cd1 = 7}
	\longintertext{In the first step, we applied Theorem 2.1(b), and in the second we note that $\lim_{n\to\infty}1$ is trivially 1. \\ Next, we'll find $\delta$}
	\ignorealign{\delta = \lim_{n\to\infty} \frac{5}{n^2} = 5\lim_{n\to\infty} \frac{1}{n^2} = 5\cd0 = 0}
	\longintertext{In the first step, we apply Theorem 2.1(b). In the second, we note that $\lim_{n\to\infty}1/n^2$ is trivially equal to 0. If you're not convinced, apply Theorem 2.1(c) to $\lim_{n\to\infty}\frac{1}{n^2}=\lim_{n\to\infty}\frac{1}{n}\cd\lim_{n\to\infty}\frac{1}{n}=0\cd0=0$. \\ Finally, let's find $\veps$}
	\ignorealign{\veps = \lim_{n\to\infty}\frac{1}{n^4} = 0}
	\longintertext{This is trivial again. You can either apply Theorem 2.1(c) twice, to find the limit is equal to $0\cd0\cd0$, or you can repeat my argument from question, but choosing $N=\ceil{\sqrt[4]{1/n}}$. \\ Thus, we can compute $b$ using Theorem 2.1(a)}
	\ignorealign{b = \gamma + \delta + \veps = 7 + 0 + 0 = 0}
	\intertext{and we can proceed to find the limit we were looking for!}
	\ignorealign{L = \frac{a}{b} = \frac{\alpha + \beta}{\gamma + \delta + \veps} = \frac{0 + 0}{7 + 0 + 0} = \frac{0}{7} = 0}
	\intertext{and conclude that}
	\lim_{n\to\infty} \frac{2n^3 + 4n}{7n^4 + 5n^2 - 1} &= 0
\end{align*}

\sol (b) \\
We'll start by applying Theorem 2.1(a) to break up the limit into two limits
\begin{align*}
	\ignorealign{\Let L := \lim_{n\to\infty}\frac{\cos n + \sin n}{n} = \lim_{n\to\infty}\frac{\cos n}{n} + \lim_{n\to\infty}\frac{\sin n}{n}}
	\longintertext{Let $a_n := \frac{\cos n}{n},\ a := \lim_{n\to\infty}a_n$. \\ Let $b_n := \frac{\sin n}{n},\ b := \lim_{n\to\infty}b_n$. \\ $\tf L = a + b$. \\ We'll work out $a$ and $b$ by using Theorem 2.2, and finding some sequences that may squeeze $a$ and $b$, respectively.}
	\ignorealign{\begin{array}{ccc}
		-1 \leq \cos n \leq 1 & \qquad\qquad & -1 \leq \sin n \leq 1 \\
		\\
		\dfrac{-1}{n} \leq \dfrac{1}{\cos n} \leq \dfrac{1}{n} & \qquad\qquad & \dfrac{-1}{n} \leq \dfrac{1}{\sin n} \leq \dfrac{1}{n} \\
		\\
		\dfrac{-1}{n} \leq a_n \leq \dfrac{1}{n} & \qquad\qquad & \dfrac{-1}{n} \leq b_n \leq \dfrac{1}{n} \\
	\end{array}}
	\longintertext{Note that, since $n\in\bbn$, $n$ is strictly positive, so we don't have to flip the equalities. Let's now find the limit of these sequences, and see if the sqeeuze $a_n$ and $b_n$}
	\lim_{n\to\infty} \frac{1}{n} &= 0 
	\intertext{That is a trivial limit we've already identified and worked with in previously.}
	\lim_{n\to\infty} \frac{-1}{n} &= -1 \lim_{n\to\infty} \frac{1}{n} \\
		&= -1\cd0 \\ 
		&= 0
	\intertext{We apply Theorem 2.1(b) to pull the constant fixed factor out, then compute the trivial limit again. As we can see, $\lim_{n\to\infty}1/n = \lim_{n\to\infty}-1/n = 0$. Also, $-1/n \leq a_n \leq 1/n$ and $-1/n \leq b_n \leq 1/n$. Therefore, by Theorem 2.2, the sqeeze theorem,}
	\ignorealign{a=0, \quad\qquad b=0} \\
	\intertext{Now, we can calculate the limit of interest,}
	\tf L = a + b &= 0 + 0 = 0
	\intertext{Therefore, we can conlude that the limit}
	\ignorealign{\lim_{n\to\infty} \frac{\cos n + \sin n}{n} = 0}
\end{align*}

\newpage
\qs{10 marks}{
	Suppose $(b_n)_{n=0}^{\infty}$ and $(c_n)_{n=0}^{\infty}$ are two convergent sequences with
	$$
		\lim_{n\to \infty}b_n=\lim_{n\to \infty}c_n=L.
	$$
	Suppose there's another sequence $(a_n)_{n=0}^{\infty}$ where $b_n=a_{2n}$ and $c_n=a_{2n+1}$ for all $n\in\mathbb{Z}_{\geq 0}$. Use the definition of limits, show that $\lim_{n\to \infty}a_n = L$.
}
\sol \\
Since $b_n$ and $c_n$ are convergent sequences,
\begin{gather*}
	\tf\forall \veps > 0, \exists N_1\in\bbz_{\geq0}: n\geq N_1 \implies \abs{b_n - L} < \veps, \\
	\forall \veps > 0, \exists N_2\in\bbz_{\geq0}: n\geq N_2 \implies \abs{c_n - L} < \veps.
\end{gather*}
$a_n$ is made up of two subsequences. Even $n$'s take $b_n$'s value, while odd $n$'s take $c_n$'s value. \\
We must show that
$$
	\forall\veps > 0, \exists N\in\bbz_{\geq0}: n\geq N \implies \abs{a_n - L} < \veps
	$$
	Therefore, choose $n=2\max\braces{N_1,N_2}$. \\
Case $n=2k,\ k\in\bbz$. $\abs{a_{2N_1} - L} = \abs{b_{N_1} - L} < \veps$ \\
Case $n=2k+1,\ k\in\bbz$. $\abs{a_{2N_2} - L} = \abs{c_{N_2} - L} < \veps$ \\

Therefore, with this choice of $N$, in either case, $a_n$ converges to $L$.

\newpage
\qs{15 marks}{
	In class we studied the sequence $(a_n)_{n=0}^{\infty}$ where $a_0=1$ and $a_{n+1}=\frac{1}{a_n+1}$ for all $n$. We showed that the  subsequence with  even terms $(b_n)_{n=0}^{\infty}$ where $b_n=a_{2n}$ forms a bounded monotone decreasing sequence and concluded that it converges to the number $\phi=\frac{-1+\sqrt{5}}{2}$. The purpose of this exercise is to repeat this process for the  subsequence with odd  terms.

	\begin{enumerate}[label=(\alph*)]
		\item Write out the first five terms of $(c_n)_{n=0}^{\infty}$ where $c_n=a_{2n+1}$.
		\item Find a recursion between the terms of $c_n$. (Hint: use the recursion for $a_n$ twice!)
		\item Show that $c_n\leq\phi$ for all $n$.
		\item Show that $c_n$ is monotone increasing.
		\item Find the limit of $\lim_{n\to \infty}c_n$.
	\end{enumerate}
}
\sol (a) \\
The first 5 terms of $c_n$:
$$
	\begin{array}{ccccccccl}
		c_0 &=& a_1 &=& \dfrac{1}{a_0 + 1} &=& \dfrac{1}{2} &\approx& 0.5 \\ \\
		c_1 &=& a_3 &=& \dfrac{1}{a_2 + 1} &=& \dfrac{3}{5} &\approx& 0.6 \\ \\
		c_2 &=& a_5 &=& \dfrac{1}{a_2 + 1} &=& \dfrac{8}{13} &\approx& 0.615385 \\ \\
		c_3 &=& a_7 &=& \dfrac{1}{a_2 + 1} &=& \dfrac{21}{34} &\approx& 0.617647 \\ \\
		c_4 &=& a_9 &=& \dfrac{1}{a_8 + 1} &=& \dfrac{55}{89} &\approx& 0.617978 \\ \\		
	\end{array}
$$
\sol (b) \\
The easy part of developing the recursion relation is setting the starting point:
$$
	c_0 = 0.5
$$
is clear from the term list above. Next, the relation itself,
\begin{gather*}
	c_n = a_{2n+1} = \frac{1}{a_{2n} + 1} = \frac{1}{\dfrac{1}{a_{2n-1}+1}+1} = \frac{1}{\dfrac{1}{c_{n-1}+1} + 1} = \frac{1}{\dfrac{1}{c_{n-1}+1} + \dfrac{c_{n-1}+1}{c_{n-1}+1}} = \frac{c_{n-1}+1}{c_{n-1}+2}
\end{gather*}
Therfore,
$$
	c_0 = \frac{1}{2},\qquad c_{n+1} = \frac{c_n+1}{c_n+2}
$$

\newpage
\sol (c) \\
Let's consider the function $f:\bbr\to\bbr$ defined by
$$
	f(x) = \frac{x+1}{x+2}
$$
We're looking for a point which maps back to itself. I.e. if the reccurance relation $c_n$ ever reached this value, it would repeatedly map back onto itself. We'll call this point $\alpha$.
$$
	\alpha = f(\alpha) = \frac{\alpha + 1}{\alpha + 2} \iff \alpha(\alpha+2) = \alpha + 1 \iff \alpha^2 + 2\alpha = \alpha + 1 \iff \alpha^2 + \alpha - 1 = 0
$$
Applying the quadratic formula, to solve for $\alpha$,
$$
	\alpha = \frac{-(1)\pm\sqrt{1^2 - 4(1)(-1)}}{2(1)} = \frac{-1 \pm \sqrt{5}}{2}
$$
and we'll take the positive square root, which is the larger number. Therefore, $\alpha = \frac{-1+\sqrt{5}}{2} = \phi$. \\

So, since $c_n$ starts below $\phi$, and $c_n$ is increasing, and if $c_n$ ever \textit{reached} $\phi$, it would remain at $\phi$ forever, we can conclude
$$
	c_n \leq \phi
$$

\sol (d) \\
For $c_n$ to be monotone increasing, we must show that $c_n\leq c_{n+1},\ \forall n\geq 0$. i.e.,
\begin{gather}
	c_n \leq c_{n+1} = \frac{c_n + 1}{c_n + 2} 
\end{gather}
Let's work with the inequality, see if we can find find a fact which we certainly know is true
\begin{align*}
	c_n &\leq \frac{c_n + 1}{c_n + 2} \\
	c_n\bracks{c_n + 2} &\leq c_n + 1 \\
	c_n^2 + 2c_n &\leq c_n + 1 \\
	c_n^2 + c_n - 1 &\leq 0 \\
	\intertext{We've already solved this quadratic!}
	\tf c_n &\leq \frac{-1+\sqrt{5}}{2} \tag*{(2)}
\end{align*}
And we know that this is true, we proved this fact in the previous part. In other words, $(1)\iff (2)$. $(2)$. Therefore $(1)$.	Since the equality holds, we've proven that the sequence $c_n$ is monotone increasing. \\

\sol (e) \\
\Theomm{4.1} (Monotone Convergence Theorem)	A monotone sequence converges if and only if it is bounded. \\

In (d) we proved that $c_n$ is a monotone increasing sequence. In (c) we proved that $c_n$ is bounded, i.e. $c_n\leq\phi$. From these two facts, along with Theorem 4.1, it follows that
$$
	\lim_{n\to\infty} c_n = \phi = \frac{-1+\sqrt{5}}{2}
$$




\newpage
\qs{10 marks}{
	Show that a convergent sequence is always bounded. In other words, given a sequence $(a_n)_{n=0}^{\infty}$ and assume that $\lim_{n\to\infty}a_n=L$. Show that there exists a number $M$ such that $|a_n|<M$ for all $n$.
}
\sol 
\begin{proof}
	Directly, by construction.
	\begin{list}{}{\setlength{\leftmargin}{1in}\setlength{\topsep}{0pt}}\item
		Suppose $\bracks{a_n}_{n=0}^{\infty}$ is a convergent sequence with $\lim_{n\to\infty} a_n = L$. \\
		We will construct a global bounding value, $A$.

		Then, $\forall \veps > 0,\ \exists N\in\bbz_{\geq0}: n\geq N\implies \abs{a_n - L} < \veps$. \\
		Therefore, $\forall n\geq N$, $\abs{L}-\veps < \abs{a_n} < \abs{L}+\veps$. \\
		So, the ``tail'' of the sequence is bounded.

		Cosnider the sequence $\bracks{a_0, a_1,\dots,a_{N-1}}$. In other words, the subsequence made up of $a_n$'s terms up-to, but not including $a_N$. \\
		This subsequence is finite, since $N$ is an integer, \\
		therefore, $A_0 = \max\braces{\abs{a_0}, \abs{a_1}, \dots, \abs{a_{N-1}}}$ is well-defined.

		Take $A = \max\braces{A_0,\ \abs{L}+\veps}$. \\
		Our construction guarantees that $a_n \leq A,\ \forall n\in\bbz_{\geq0}$. \\
	\end{list}
	Therefore, a convergent seuqnce, $a_n$, is bounded.
\end{proof}

Note that you can arbitrarily choose $\veps>0$, and the constrution holds.


\end{document}
