\documentclass[a4paper,12pt]{report}

\input{../../../latex_template/preamble}
%From M275 "Topology" at SJSU
\newcommand{\id}{\mathrm{id}}
\newcommand{\taking}[1]{\xrightarrow{#1}}
\newcommand{\inv}{^{-1}}

%From M170 "Introduction to Graph Theory" at SJSU
\DeclareMathOperator{\diam}{diam}
\DeclareMathOperator{\ord}{ord}
\newcommand{\defeq}{\overset{\mathrm{def}}{=}}

%From the USAMO .tex files
\newcommand{\ts}{\textsuperscript}
\newcommand{\dg}{^\circ}
\newcommand{\ii}{\item}

% % From Math 55 and Math 145 at Harvard
% \newenvironment{subproof}[1][Proof]{%
% \begin{proof}[#1] \renewcommand{\qedsymbol}{$\blacksquare$}}%
% {\end{proof}}

\newcommand{\liff}{\leftrightarrow}
\newcommand{\lthen}{\rightarrow}
\newcommand{\opname}{\operatorname}
\newcommand{\surjto}{\twoheadrightarrow}
\newcommand{\injto}{\hookrightarrow}
\newcommand{\On}{\mathrm{On}} % ordinals
\DeclareMathOperator{\img}{im} % Image
\DeclareMathOperator{\Img}{Im} % Image
\DeclareMathOperator{\coker}{coker} % Cokernel
\DeclareMathOperator{\Coker}{Coker} % Cokernel
\DeclareMathOperator{\Ker}{Ker} % Kernel
\DeclareMathOperator{\rank}{rank}
\DeclareMathOperator{\Spec}{Spec} % spectrum
\DeclareMathOperator{\Tr}{Tr} % trace
\DeclareMathOperator{\pr}{pr} % projection
\DeclareMathOperator{\ext}{ext} % extension
\DeclareMathOperator{\pred}{pred} % predecessor
\DeclareMathOperator{\dom}{dom} % domain
\DeclareMathOperator{\ran}{ran} % range
\DeclareMathOperator{\Hom}{Hom} % homomorphism
\DeclareMathOperator{\Mor}{Mor} % morphisms
\DeclareMathOperator{\End}{End} % endomorphism

\newcommand{\eps}{\epsilon}
\newcommand{\veps}{\varepsilon}
\newcommand{\ol}{\overline}
\newcommand{\ul}{\underline}
\newcommand{\wt}{\widetilde}
\newcommand{\wh}{\widehat}
\newcommand{\vocab}[1]{\textbf{\color{blue} #1}}
\providecommand{\half}{\frac{1}{2}}
\newcommand{\dang}{\measuredangle} %% Directed angle
\newcommand{\ray}[1]{\overrightarrow{#1}}
\newcommand{\seg}[1]{\overline{#1}}
\newcommand{\arc}[1]{\wideparen{#1}}
\DeclareMathOperator{\cis}{cis}
\DeclareMathOperator*{\lcm}{lcm}
\DeclareMathOperator*{\argmin}{arg min}
\DeclareMathOperator*{\argmax}{arg max}
\newcommand{\cycsum}{\sum_{\mathrm{cyc}}}
\newcommand{\symsum}{\sum_{\mathrm{sym}}}
\newcommand{\cycprod}{\prod_{\mathrm{cyc}}}
\newcommand{\symprod}{\prod_{\mathrm{sym}}}
\newcommand{\Qed}{\begin{flushright}\qed\end{flushright}}
\newcommand{\parinn}{\setlength{\parindent}{1cm}}
\newcommand{\parinf}{\setlength{\parindent}{0cm}}
% \newcommand{\norm}{\|\cdot\|}
\newcommand{\inorm}{\norm_{\infty}}
\newcommand{\opensets}{\{V_{\alpha}\}_{\alpha\in I}}
\newcommand{\oset}{V_{\alpha}}
\newcommand{\opset}[1]{V_{\alpha_{#1}}}
\newcommand{\lub}{\text{lub}}
\newcommand{\del}[2]{\frac{\partial #1}{\partial #2}}
\newcommand{\Del}[3]{\frac{\partial^{#1} #2}{\partial^{#1} #3}}
\newcommand{\deld}[2]{\dfrac{\partial #1}{\partial #2}}
\newcommand{\Deld}[3]{\dfrac{\partial^{#1} #2}{\partial^{#1} #3}}
\newcommand{\lm}{\lambda}
\newcommand{\uin}{\mathbin{\rotatebox[origin=c]{90}{$\in$}}}
\newcommand{\usubset}{\mathbin{\rotatebox[origin=c]{90}{$\subset$}}}
\newcommand{\lt}{\left}
\newcommand{\rt}{\right}
\newcommand{\bs}[1]{\boldsymbol{#1}}
\newcommand{\exs}{\exists}
\newcommand{\st}{\strut}
\newcommand{\dps}[1]{\displaystyle{#1}}

\newcommand{\sol}{\setlength{\parindent}{0cm}\textbf{\textit{Solution:}}\setlength{\parindent}{1cm} }
\newcommand{\solve}[1]{\setlength{\parindent}{0cm}\textbf{\textit{Solution: }}\setlength{\parindent}{1cm}#1 \Qed}

\DeclareMathOperator{\sech}{sech}
\DeclareMathOperator{\csch}{csch}
\DeclareMathOperator{\arcsec}{arcsec}
\DeclareMathOperator{\arccsc}{arccsc}
\DeclareMathOperator{\arccot}{arccot}
\DeclareMathOperator{\arsinh}{arsinh}
\DeclareMathOperator{\arcosh}{arcosh}
\DeclareMathOperator{\artanh}{artanh}
\DeclareMathOperator{\arcsch}{arcsch}
\DeclareMathOperator{\arsech}{arsech}
\DeclareMathOperator{\arcoth}{arcoth}

\newcommand{\sinx}{\sin x}          \newcommand{\arcsinx}{\arcsin x}    
\newcommand{\cosx}{\cos x}          \newcommand{\arccosx}{\arccosx}
\newcommand{\tanx}{\tan x}          \newcommand{\arctanx}{\arctan x}
\newcommand{\cscx}{\csc x}          \newcommand{\arccscx}{\arccsc x}
\newcommand{\secx}{\sec x}          \newcommand{\arcsecx}{\arcsec x}
\newcommand{\cotx}{\cot x}          \newcommand{\arccotx}{\arccot x}
\newcommand{\sinhx}{\sinh x}          \newcommand{\arsinhx}{\arsinh x}
\newcommand{\coshx}{\cosh x}          \newcommand{\arcoshx}{\arcosh x}
\newcommand{\tanhx}{\tanh x}          \newcommand{\artanhx}{\artanh x}
\newcommand{\cschx}{\csch x}          \newcommand{\arcschx}{\arcsch x}
\newcommand{\sechx}{\sech x}          \newcommand{\arsechx}{\arsech x}
\newcommand{\cothx}{\coth x}          \newcommand{\arcothx}{\arcoth x}
\newcommand{\lnx}{\ln x}
\newcommand{\expx}{\exp x}

\newcommand{\bba}{\mathbb{A}}   \newcommand{\bbn}{\mathbb{N}}
\newcommand{\bbb}{\mathbb{B}}   \newcommand{\bbo}{\mathbb{O}}
\newcommand{\bbc}{\mathbb{C}}   \newcommand{\bbp}{\mathbb{P}}
\newcommand{\bbd}{\mathbb{D}}   \newcommand{\bbq}{\mathbb{Q}}
\newcommand{\bbe}{\mathbb{E}}   \newcommand{\bbr}{\mathbb{R}}
\newcommand{\bbf}{\mathbb{F}}   \newcommand{\bbs}{\mathbb{S}}
\newcommand{\bbg}{\mathbb{G}}   \newcommand{\bbt}{\mathbb{T}}
\newcommand{\bbh}{\mathbb{H}}   \newcommand{\bbu}{\mathbb{U}}
\newcommand{\bbi}{\mathbb{I}}    \newcommand{\bbv}{\mathbb{V}}
\newcommand{\bbj}{\mathbb{J}}   \newcommand{\bbw}{\mathbb{W}}
\newcommand{\bbk}{\mathbb{K}}   \newcommand{\bbx}{\mathbb{X}}
\newcommand{\bbl}{\mathbb{L}}    \newcommand{\bby}{\mathbb{Y}}
\newcommand{\bbm}{\mathbb{M}}   \newcommand{\bbz}{\mathbb{Z}}

\newcommand{\lb}{\left(}
\newcommand{\rb}{\right)}
\newcommand{\lbr}{\left\lbrace}
\newcommand{\rbr}{\right\rbrace}
\newcommand{\lsb}{\left[}
\newcommand{\rsb}{\right]}
\newcommand{\suchthat}{\medspace\middle|\medspace}
\newcommand{\bracks}[1]{\lb #1 \rb}
\newcommand{\braces}[1]{\lbr #1 \rbr}
\newcommand{\sqbracks}[1]{\lsb #1 \rsb}

\renewcommand{\floor}[1]{\lfloor #1 \rfloor}
\renewcommand{\ceil}[1]{\lceil #1 \rceil}

\newcommand{\cd}{\cdot}
\newcommand{\tf}{\therefore}
\newcommand{\Let}{\text{Let }}
\newcommand{\Given}{\text{Given }}
\newcommand{\Suppose}{\text{Suppose }}
\newcommand{\WeSee}{\text{We see }}
\newcommand{\So}{\text{So }}

\newcommand{\QED}{\hfill \qed}

\renewcommand{\dd}[1]{\frac{d}{d#1}}
\newcommand{\dyd}[2][y]{\frac{d#1}{d#2}}

\newcommand{\ddx}{\dd{x}}       \newcommand{\ddxsq}{\dyd[^2]{x^2}}
\newcommand{\ddy}{\dd{y}}       \newcommand{\ddysq}{\dyd[^2]{y^2}}
\newcommand{\ddu}{\dd{u}}       \newcommand{\ddusq}{\dyd[^2]{u^2}}
\newcommand{\ddv}{\dd{v}}       \newcommand{\ddvsq}{\dyd[^2]{v^2}}

\newcommand{\dydx}{\dyd{x}}     \newcommand{\dydxsq}{\dyd[^2y]{x^2}}
\newcommand{\dfdx}{\dyd[f]{x}}  \newcommand{\dfdxsq}{\dyd[^2f]{x^2}}
\newcommand{\dudx}{\dyd[u]{x}}  \newcommand{\dudxsq}{\dyd[^2u]{x^2}}
\newcommand{\dvdx}{\dyd[v]{x}}  \newcommand{\dvdxsq}{\dyd[^2v]{x^2}}

% Mathfrak primes
\newcommand{\km}{\mathfrak{m}}
\newcommand{\kp}{\mathfrak{p}}
\newcommand{\kq}{\mathfrak{q}}

%---------------------------------------
% Blackboard Math Fonts :-
%---------------------------------------
\newcommand{\bba}{\mathbb{A}}   \newcommand{\bbn}{\mathbb{N}}
\newcommand{\bbb}{\mathbb{B}}   \newcommand{\bbo}{\mathbb{O}}
\newcommand{\bbc}{\mathbb{C}}   \newcommand{\bbp}{\mathbb{P}}
\newcommand{\bbd}{\mathbb{D}}   \newcommand{\bbq}{\mathbb{Q}}
\newcommand{\bbe}{\mathbb{E}}   \newcommand{\bbr}{\mathbb{R}}
\newcommand{\bbf}{\mathbb{F}}   \newcommand{\bbs}{\mathbb{S}}
\newcommand{\bbg}{\mathbb{G}}   \newcommand{\bbt}{\mathbb{T}}
\newcommand{\bbh}{\mathbb{H}}   \newcommand{\bbu}{\mathbb{U}}
\newcommand{\bbi}{\mathbb{I}}   \newcommand{\bbv}{\mathbb{V}}
\newcommand{\bbj}{\mathbb{J}}   \newcommand{\bbw}{\mathbb{W}}
\newcommand{\bbk}{\mathbb{K}}   \newcommand{\bbx}{\mathbb{X}}
\newcommand{\bbl}{\mathbb{L}}   \newcommand{\bby}{\mathbb{Y}}
\newcommand{\bbm}{\mathbb{M}}   \newcommand{\bbz}{\mathbb{Z}}

%---------------------------------------
% Roman Math Fonts :-
%---------------------------------------
\newcommand{\rma}{\mathrm{A}}   \newcommand{\rmn}{\mathrm{N}}
\newcommand{\rmb}{\mathrm{B}}   \newcommand{\rmo}{\mathrm{O}}
\newcommand{\rmc}{\mathrm{C}}   \newcommand{\rmp}{\mathrm{P}}
\newcommand{\rmd}{\mathrm{D}}   \newcommand{\rmq}{\mathrm{Q}}
\newcommand{\rme}{\mathrm{E}}   \newcommand{\rmr}{\mathrm{R}}
\newcommand{\rmf}{\mathrm{F}}   \newcommand{\rms}{\mathrm{S}}
\newcommand{\rmg}{\mathrm{G}}   \newcommand{\rmt}{\mathrm{T}}
\newcommand{\rmh}{\mathrm{H}}   \newcommand{\rmu}{\mathrm{U}}
\newcommand{\rmi}{\mathrm{I}}   \newcommand{\rmv}{\mathrm{V}}
\newcommand{\rmj}{\mathrm{J}}   \newcommand{\rmw}{\mathrm{W}}
\newcommand{\rmk}{\mathrm{K}}   \newcommand{\rmx}{\mathrm{X}}
\newcommand{\rml}{\mathrm{L}}   \newcommand{\rmy}{\mathrm{Y}}
\newcommand{\rmm}{\mathrm{M}}   \newcommand{\rmz}{\mathrm{Z}}

%---------------------------------------
% Calligraphic Math Fonts :-
%---------------------------------------
\newcommand{\cla}{\mathcal{A}}   \newcommand{\cln}{\mathcal{N}}
\newcommand{\clb}{\mathcal{B}}   \newcommand{\clo}{\mathcal{O}}
\newcommand{\clc}{\mathcal{C}}   \newcommand{\clp}{\mathcal{P}}
\newcommand{\cld}{\mathcal{D}}   \newcommand{\clq}{\mathcal{Q}}
\newcommand{\cle}{\mathcal{E}}   \newcommand{\clr}{\mathcal{R}}
\newcommand{\clf}{\mathcal{F}}   \newcommand{\cls}{\mathcal{S}}
\newcommand{\clg}{\mathcal{G}}   \newcommand{\clt}{\mathcal{T}}
\newcommand{\clh}{\mathcal{H}}   \newcommand{\clu}{\mathcal{U}}
\newcommand{\cli}{\mathcal{I}}   \newcommand{\clv}{\mathcal{V}}
\newcommand{\clj}{\mathcal{J}}   \newcommand{\clw}{\mathcal{W}}
\newcommand{\clk}{\mathcal{K}}   \newcommand{\clx}{\mathcal{X}}
\newcommand{\cll}{\mathcal{L}}   \newcommand{\cly}{\mathcal{Y}}
\newcommand{\calm}{\mathcal{M}}  \newcommand{\clz}{\mathcal{Z}}

%---------------------------------------
% Fraktur  Math Fonts :-
%---------------------------------------
\newcommand{\fka}{\mathfrak{A}}   \newcommand{\fkn}{\mathfrak{N}}
\newcommand{\fkb}{\mathfrak{B}}   \newcommand{\fko}{\mathfrak{O}}
\newcommand{\fkc}{\mathfrak{C}}   \newcommand{\fkp}{\mathfrak{P}}
\newcommand{\fkd}{\mathfrak{D}}   \newcommand{\fkq}{\mathfrak{Q}}
\newcommand{\fke}{\mathfrak{E}}   \newcommand{\fkr}{\mathfrak{R}}
\newcommand{\fkf}{\mathfrak{F}}   \newcommand{\fks}{\mathfrak{S}}
\newcommand{\fkg}{\mathfrak{G}}   \newcommand{\fkt}{\mathfrak{T}}
\newcommand{\fkh}{\mathfrak{H}}   \newcommand{\fku}{\mathfrak{U}}
\newcommand{\fki}{\mathfrak{I}}   \newcommand{\fkv}{\mathfrak{V}}
\newcommand{\fkj}{\mathfrak{J}}   \newcommand{\fkw}{\mathfrak{W}}
\newcommand{\fkk}{\mathfrak{K}}   \newcommand{\fkx}{\mathfrak{X}}
\newcommand{\fkl}{\mathfrak{L}}   \newcommand{\fky}{\mathfrak{Y}}
\newcommand{\fkm}{\mathfrak{M}}   \newcommand{\fkz}{\mathfrak{Z}}

\usepackage{wasysym}

\begin{document}
\begin{center}
{\bf School of Mathematics and Physics, UQ}
\end{center}
\begin{center}
	{\large\bf MATH1081 Advanced Discrete Mathematics \\ Semester 1 2025 \\ Problem Set 1} \\ \vspace{1em}
	Michael Kasumagic, 44302669 \\
	Tutorial Group \#3 \\
	Due 5pm Friday 28 March 2025
\end{center}

\qs{10 marks}{
	Prove that \textsf{XOR} satisfies the associative law; that is:
	\[ p \oplus (q \oplus r) \equiv (p \oplus q) \oplus r. \]
}
\sol \\
\Definn{1.1}{\textsf{XOR}} For two predicates, $p$ and $q$, the \textsf{XOR} of them, denoted $p\lxor q \equiv (p\lor q) \land \lnot(p\land q)$. \\

\Theomm{1.1} \textsf{XOR} satisfies the associative law; that is: $p\lxor (q\lxor r) \equiv (p\lxor q)\lxor r$.
\begin{proof}
	By considering all cases with a truth table. 
	\begin{center}\begin{tabular}{|ccc|cc|cc|c|}
		\hline
		$p$ & $q$ & $r$ & $p\lxor q$ & $q\lxor r$ & $p\lxor(q\lxor r)$ & $(p\lxor q)\lxor r$ & $p\lxor(q\lxor r)\liff(p\lxor q)\lxor r$ \\ \hline 
		T & T & T & F & F & T & T	& T \\
		T & T & F & F & T & F & F	& T \\
		T & F & T & T & T & F & F	& T \\
		T & F & F & T & F & T & T	& T \\
		F & T & T & T & F & F & F	& T \\
		F & T & F & T & T & T & T	& T \\
		F & F & T & F & T & T & T	& T \\
		F & F & F & F & F & F & F	& T \\ \hline
	\end{tabular}\end{center}
	As we can see in the final column, $p\lxor(q\lxor r)$ is logically equivalent to $(p\lxor q)\lxor r$, for all possible value combinations of $(p,q,r)$. \\
	Therefore \textsf{XOR} satisfies the associative law.
\end{proof}

Which makes sense! Since the group $(\braces{T,F},\lxor)$ is isomorphic to $(\braces{0,1}, {+})$ (which I won't prove here $\smiley$).

\newpage
\qs{10 marks}{
	Using the laws of logical equivalence, prove that for any fixed $n \in \bbn$ and statement variables $p,q_1,q_2,\ldots,q_n$:
	\[
		p \land (q_1 \oplus q_2 \oplus \ldots \oplus q_n) \equiv
		(p \land q_1) \oplus (p \land q_2) \oplus \ldots \oplus (p \land q_n)
	\]
}
\sol Let's first make sure that $\land$ distributes over $\lxor$. \\

\Theomm{2.1} For three statement variables, $p,q_1,q_2$, $p\land(q_1 \lxor q_2) \equiv (p\land q_1) \lxor (p\land q_2)$.
\begin{proof}
	By considering all cases with a truth table. 
	\begin{center}\begin{tabular}{|ccc|ccc|cc|c|}
		\hline
		$p$ & $q_1$ & $q_2$ & $q_1\lxor q_2$ & $p\land q_1$ & $p\land q_2$ & $\cll := p\land(q_1\lxor q_2)$ & $\clr := (p\land q_1) \lxor (p\land q_2)$ & $\cll\liff\clr$ \\ \hline 
		T & T & T & F & T & T & F & F & T \\
		T & T & F & T & T & F & T & T & T \\
		T & F & T & T & F & T & T & T & T \\
		T & F & F & F & F & F & F & F & T \\
		F & T & T & F & F & F & F & F & T \\
		F & T & F & T & F & F & F & F & T \\
		F & F & T & F & F & F & F & F & T \\
		F & F & F & F & F & F & F & F & T \\ \hline
	\end{tabular}\end{center}
	As we can see in the final column, $p\land(q_1 \lxor q_2)$ is logically equivalent to $(p\land q_1) \lxor (p\land q_2)$, for all possible value combinations of $(p,q,r)$. \\
	Therefore $\land$ distributes over $\lxor$.
\end{proof}

\Theomm{2.2} For a fixed $n\in\bbn$, and statement variables $p,q_1,q_2,\dots,q_n,$ \\ $p \land (q_1 \lxor q_2 \lxor \ldots \lxor q_n) \equiv (p \land q_1) \lxor (p \land q_2) \lxor \ldots \lxor (p \land q_n).$ \\

Note: I will express $\dps{\bracks{q_1 \lxor q_2 \lxor \ldots \lxor q_n} \equiv \bigoplus_{i=1}^n q_i}$.

\begin{proof}
	Suppose $n\in\bbn$ is fixed and $p,q_1,q_2,\dots,q_n$ are statement variables.
	\begin{list}{}{\setlength{\leftmargin}{1in}\setlength{\topsep}{0pt}}\item
		Let's consider $\dps{p \land \bracks{\bigoplus_{i=1}^{n} q_i} \equiv p \land \bracks{q_1 \lxor \bigoplus_{i=2}^n q_i}}$.

		Let's define a statement variable $\dps{r_2 = \bigoplus_{i=2}^n q_i}$.

		Then we can rewrite the statement $\dps{p \land \bracks{\bigoplus_{i=1}^{n} q_i} \equiv p \land \bracks{q_1 \lxor r_2}}$. 

		We can apply Theorem 2.1, $\dps{p \land \bracks{\bigoplus_{i=1}^{n} q_i} \equiv (p\land q_1) \lxor \bracks{p\land\bigoplus_{i=2}^{n} q_i}}$.

		We can repeat this for $\dps{r_3 = \bigoplus_{i=3}^{n} q_i}$, and applying Theorem 2.1,

		$\dps{p\land\bracks{\bigoplus_{i=1}^n q_i} \equiv (p\land q_1) \lxor (p\land q_2) \lxor (p\land r_3) \equiv (p\land q_1) \lxor (p\land q_2) \lxor \bracks{p\land \bigoplus_{i=3}^{n} q_i}}$	

		We can continue this process, repeatedly taking $\dps{r_k = \bigoplus_{i=k}^n q_i,\ k \leq n}$, and then distributing $p\land$, according to Theorem 2.1.

		Eventually, when $\dps{k=n,\ r_k \equiv r_n \equiv \bigoplus_{i=k=n}^n q_i \equiv q_n}$, 
		
		and $\dps{p \land \bracks{\bigoplus_{i=1}^{n} q_i} \equiv \bigoplus_{i=1}^{n-1} \bracks{p\land q_i}} \lxor (p \land r_n) \equiv \bigoplus_{i=1}^{n} (p\land q_i)$. 

		which is equivalent to $(p\land q_1)\lxor\dots\lxor(p\land q_n)$, which is what we wanted to show. \\
	\end{list}
	Therefore, $p \land (q_1 \lxor q_2 \lxor \ldots \lxor q_n) \equiv
	(p \land q_1) \lxor (p \land q_2) \lxor \ldots \lxor (p \land q_n)$
\end{proof}


\newpage
\qs{10 marks}{
	Show that the following argument is valid, using the rules of inference and/or logical equivalences. Clearly label which rule you used in each step.
	$$
		\argument{
			1. & r \lthen \lnot a \\
			2. & \lnot r \lthen \lnot b \\
			3. & \lnot c \lthen a \\
			4. & \lnot c \lthen b \\
		}{c}
	$$
}
\sol
$$
	\argument{
		1. & r \lthen \lnot a \\
		2. & \lnot r \lthen \lnot b \\
		3. & \lnot c \lthen a \\
		4. & \lnot c \lthen b \\
		5. & a \lthen \lnot r & \text{(Contrapositive of 1.)} \\
		6. & a \lthen \lnot b & \text{(Transitivity of 5. and 2.)} \\
		7. & \lnot b \lthen c & \text{(Contrapositive of 4.)} \\
		8. & a \lthen c & \text{(Transitivity of 6. and 7.)} \\
		9. & \lnot c \lthen \lnot a & \text{(Contrapositive of 8.)} \\
		10. & \bracks{\lnot c \lthen a}\land\bracks{\lnot c \lthen \lnot a} & \text{(Conjunction of 3. and 9.)} \\
		11. & \bracks{c \lor a}\land\bracks{c \lor \lnot a} & \text{(Logically Equivalent to 10. (Def. of $\lthen$))} \\
		12. & c \lor \bracks{a\land\lnot a} & \text{(Logically Equivalent to 10. (Distributivity))} \\
		13. & c \lor \bot & \text{(Logically Equivalent to 10. (Negation))} \\
		14. & c & \text{(Logically Equivalent to 10. (Identity))} \\
	}{c}
$$


\newpage
\qs{10 marks}{
	Let $D$ be some domain, and let $p(x)$ and $q(x)$ be predicates in the variable $x \in D$. Write the following English sentences symbolically, i.e., using logical symbols, logical operations, and/or quantifiers. Your answers should not contain any English other than possibly the phrase ``such that''.
	\begin{enumerate}[label=(\alph*)]
		\item $p(x)$ is never true.
		\item $p(x)$ is a necessary condition for $q(x)$.
		\item It is impossible for $p(x)$ and $q(x)$ to both be true for the same value of $x$.
		\item Every $x$ satisfies exactly one of $p(x)$ or $q(x)$ (not both).
		\item There is exactly one value of $x$ (no more, no less) for which $p(x)$ is true.
	\end{enumerate}
}
\sol
\begin{enumerate}[label=(\alph*)]
	\item $\forall x\in D, \lnot p(x)$
	\item $\forall x\in D, q(x) \lthen p(x)$
	\item $\forall x\in D, \lnot(q(x) \land p(x))$
	\item $\forall x\in D, (p(x)\land\lnot q(x)) \lor (q(x)\land\lnot p(x))$
	\item $\exists x\in D: p(x) \land \forall y\in D,\ p(y) \lthen (y=x)$
\end{enumerate}

\newpage
\qs{10 marks}{
	\begin{enumerate}[label=(\alph*)]
		\item Prove that for all integers $n \in \bbn$, if $n$ is prime and $n > 2$ then $n$ is odd.
		\item Prove that for all integers $n \in \bbn$, if $n^2+3$ is prime then $n$ is even.
		\item Prove that for all integers $n \in \bbn$, if $n^2-1$ is prime then $n^2+1$ is also prime.
	\end{enumerate}
}
\sol (a)
\begin{proof}
	The statement's contrapositive is $\forall n\in\bbn, n>2,\ n\text{ is even} \lthen n\text{ is composite}$.
	\begin{list}{}{\setlength{\leftmargin}{1in}\setlength{\topsep}{0pt}}\item
		Suppose $n\in\bbn$, $n>2$ and $n$ is even. \\
		Then $n = 2k,\ k\in\bbz$. \\
		Hence, $2\divs 2k \iff 2\divs n$. \\
		Therefore $n$ is composite. \\
		Therefore, $\forall n\in\bbn, n>2,\ n\text{ is even} \lthen n\text{ is composite}$. \\
	\end{list}
	Therefore, $\forall n\in\bbn, n>2,\ n\text{ is prime} \lthen n\text{ is odd}$. \\
\end{proof}

\sol (b)
\begin{proof}
	The statement's contrapositive is $\forall n\in\bbn,\ n \text{ is odd} \lthen n^2+3 \text{ is composite}.$
	\begin{list}{}{\setlength{\leftmargin}{1in}\setlength{\topsep}{0pt}}\item
		Suppose $n\in\bbn$ and $n$ is odd. \\
		Then, $n = 2k+1,\ k\in\bbz$. \\
		Hence, $n^2+3 = (2k+1)^2+3 = 4k^2 + 4k + 1 + 3 = 4k^2 + 4k + 4$. \\
		So, $n^2 + 3 = 2(2k^2 + 2k + 2) \iff 2\divs n^2 + 3$. \\
		Therefore, $n$ is even.
		Therefore, $\forall n\in\bbn,\ n \text{ is odd} \lthen n^2+3 \text{ is composite}.$ \\
	\end{list}
	Therefore, $\forall n\in\bbn,\ n^2+3 \text{ is prime} \lthen n \text{ is even}.$ \\
\end{proof}

\sol (c)
\begin{proof}
	The statement's contrapositive is $\forall n\in\bbn,\ n^2 + 1 \text{ is composite} \lthen n^2 - 1 \text{ is composite}$
	\begin{list}{}{\setlength{\leftmargin}{1in}\setlength{\topsep}{0pt}}\item
		Suppose $n\in\bbn$, $n^2 + 1$ is composite. \\
		We note that $n^2 - 1$ can be factorised into $(n+1)(n-1)$.

		For $n{=}1$, $(n+1)(n-1) = 2\cd0 = 0 \notin \bbn$, so we can discard this case.

		For $n{=}2$, $(n+1)(n-1) = 3\cd1 = 3$, is not composite! \\
		However, $n^2 + 1 = 5$, is not composite. Since the premise of the implication is not true, we can actually discard this case.

		For $n{>}2$,\ $n+1 > 2$,\ $n-1 \geq 2 $. \\
		Which means, $n^2 - 1$ can be factorised into at least two factors.

		This is the definition of composite, hence, $n^2 - 1$ is composite.

		Therefore $\forall n\in\bbn,\ n^2 + 1 \text{ is composite} \lthen n^2 - 1 \text{ is composite}$.\\
	\end{list}
	Therefore $\forall n\in\bbn,\ n^2 - 1 \text{ is prime} \lthen n^2 + 1 \text{ is prime}$. \\
\end{proof}


\newpage
\qs{10 marks}{
	\begin{enumerate}[label=(\alph*)]
		\item Prove that there are infinitely many odd integers $n\in\bbn$ for which $n$ and $n+100$ are both composite.
		\item Prove that there are infinitely many odd integers $n\in\bbn$ for which $n$, $n+2$, $n+4$, $n+6$, \ldots, $n+1000$ are all composite.
	\end{enumerate}
}
\sol (a)
\begin{proof}
	Directly.
	\begin{list}{}{\setlength{\leftmargin}{1in}\setlength{\topsep}{0pt}}\item
		Suppose $k \in \bbn$. \\
		Choose $n=5(2k+1)\in\bbn$.

		Then $5\divs n$, since 5 is trivially a factor. \\
		Hence $n$ is composite. \\
		Consider, $n = 5(2k+1) = 10k + 5 = 2(5k+2) + 1 \iff 2\ndivs n$. \\
		Hence, $n$ is odd. 
		
		Consider $n + 100 = 5(2k+1) + 100 = 25k+5+100 = 25k+105 = 5(5k+21)$. \\
		Since $5\divs(n + 100)$, then $n+100$ is composite. \\
		Consider $n + 100 = 5(2k+1) + 100 = 25k+5+100 = 25k+105 = 2(12k+52)+(k+1) \iff 2\ndivs(n+100)$. \\
		Hence, $n+100$ is odd.

		Since there are infinite natural numbers $k$, there are infinite $n=5(2k+1)$s.
	\end{list}
	Therefore, there are infinitely many odd integers, $n$ for which $n$ and $n+100$ are composite.	
\end{proof}

\sol (b)
\begin{proof}
	Directly.
	\begin{list}{}{\setlength{\leftmargin}{1in}\setlength{\topsep}{0pt}}\item
		Suppose $k\in\bbn$. \\
		Choose $n = 2(1001!k + 1)+1$.
		Therefore, $n$ is odd.
		\begin{align*}
			n &= 2(1001!k + 1) + 1 \\
				&= 2\cd1001!k + 2 + 1 \\
				&= 2\cd1001!k + 3 \\
				&= 3\bracks{\frac{2\cd1001!k}{3} + 1}
		\end{align*}
		Therefore, $n$ is composite.

		Next, we'll consider $n+2$
		\begin{align*}
			n + 2 &= 2(1001!k + 1) + 3 \\
				&= 2\cd1001!k + 2 + 3 \\
				&= 2\cd1001!k + 5 \\
				&= 5\bracks{\frac{2\cd1001!k}{5} + 1}
		\end{align*}
		Therefore, $n+2$ is composite.

		Finally, we'll consider $n+1000$
		\begin{align*}
			n + 1000 &= 2(1001!k + 1) + 1001 \\
				&= 2\cd1001!k + 2 + 1001 \\
				&= 2\cd1001!k + 1003 \\
				&= 1001\bracks{2\cd1000!k + 1}
		\end{align*}
		Therefore, $n$ is composite.

		In general, for every $m\in\braces{0,2,4,\dots,1000}$, we can always factorise the expression
		$$
			n + m = 2(1001!k + 1)+1+m = (m+1)\bracks{\frac{2\cd1001!k}{m+1}+1}
		$$
		which shows that all of these numbers we've found are composite. 

		Since there are infinite natural numbers $k$, there are infinite $n\text{'s} = 2(1001!k + 1)+1$ with $n+m,\ \forall m\in\braces{0,2,\dots,1000}$ are all composite.\\
	\end{list}
	Therefore, there are infinitely many odd integers, $n$ for which $n+m,\ \forall m\in\braces{0,2,\dots,1000}$ are composite.	\\
\end{proof}


\end{document}
